\documentclass[9pt,a4paper,cleardoubleempty]{scrbook}

\usepackage{}
\usepackage[]{classicthesis-ldpkg}
\usepackage[parts,linedheaders,pdfspacing]{classicthesis}

\usepackage[T1]{fontenc}    % imposta la codifica dei font
\usepackage[utf8]{inputenc} % lettere accentate da tastiera
\usepackage[italian]{babel} % per scrivere in italiano
% \usepackage{layaureo}       % imposta i margini di pagina
\usepackage{lipsum}         % genera testo fittizio
\usepackage{indentfirst}
\usepackage{url}            % per scrivere gli indirizzi Internet
\hypersetup{pdfborder={0 0 0}}
\usepackage{allrunes}
\usepackage{phoenician}

\usepackage{color}
\definecolor{Brown}{cmyk}{0,0.81,1,0.60}
\definecolor{OliveGreen}{cmyk}{0.64,0,0.95,0.40}
\definecolor{CadetBlue}{cmyk}{0.70,0.57,0.23,0}

\usepackage[final]{pdfpages}

% args: image path
%       caption
%       width
%       label
\newcommand{\image}[4]{%
    \begin{figure}[hbt]%
    \centering%
    \includegraphics[width=#3\columnwidth]{#1}%
    \caption{\emph{#2}}%
    \label{#4}%
    \end{figure}%
}

% args: image path
%       caption
%       label
\newcommand{\imageFullWidth}[3]{\image{#1}{#2}{0.8}{#3}}

\newcommand{\picref}[1]{figura \ref{#1}}

\newcommand{\website}[2]{Sito web di \techname{#1}: \\ \url{#2}}
\newcommand{\paper}[3]{#1: \emph{#2} \\ \url{#3}}

\newcommand{\techname}[1]{{\sc #1}}
\newcommand{\cypher}[1]{{\tt #1}}
\newcommand{\syntax}[1]{{\tt #1}}
\newcommand{\codeconst}[1]{\lstinline{#1}}
\newcommand{\newterm}[1]{\emph{#1}}
\newcommand{\foreignword}[1]{\emph{#1}}

% Vari dettagli
\newcommand{\BW}{\emph{Burning Wheel}}

% Entità, NPC/PC e Luoghi importanti del gioco
\newcommand{\sciencegroup}{\techname{Athena Collective}}
\newcommand{\magicgroup}{\techname{Burning Universities}}
\newcommand{\theogroup}{\techname{Credo dei Dodici Divini}}

\newcommand{\theworld}{\emph{Xeresia}}
\newcommand{\theincident}{\emph{Shynthaian Pathfinder Incident}}
\newcommand{\firstvalley}{\emph{Valle di Ribia}}

\newcommand{\weapongoddess}{\emph{La II: Farcaster}}

\newcommand{\dacav}{\emph{Il ``Giovane'' Vecchio Dacav}}
\newcommand{\gods}{\emph{Nonoriko}}
\newcommand{\mitch}{\emph{Niyu-Rin}}
\newcommand{\ranger}{\emph{Alastor}}

\usepackage{}
\author{Michele "Jazzinghen" Bianchi}
\title{Project Xeresia: Characters}
\begin{document}
   
    \chapter{Motivazioni politiche}

    L'idea che c'è dietro alla mossa di bloccare l'entrata della gente
    anche durante l'estate è una ``stepping stone'' per il colpo di stato.
    Con gli accessi controllati la distribuzione delle informazioni è
    notevolmente rallentata, se non bloccata completamente. Il massimo per
    della gente che vuole far partire un golpe stealth.

    Inoltre, dall'esterno, questa mossa sembra più alimentata da ideologie
    razziste che non da controlli mediatici, perciò  rischia di ``trarre in
    inganno'' le persone che pensano in maniera superficiale.

    \chapter{Idee per scene}

    Scena che descrive come si sente The Fixer prima e durante un attacco di
    Ira, ovvero con le pupillle che si restringono subito prima, attacchi
    di panico, tutto sembra leggero, ecc.

    Scena di quando Hax tira fuori la tastiera per uccidere quello che ha
    trappolato con la mente di Celty. Tutta la mena di come sparare
    correttamente, vista al rallentatore. I consigli di Lissa, per poi
    capire che il tizio ha qualche tipo di campo di distorsione, visto che
    neanche Lissa è riuscito a colpirlo, e allora capisce che la tastiera,
    in materiali amagnetici, è l'unica arma che ha contro di lui.
    Situazioni disperate richiedono mosse disperate. Perfetto per il
    personaggio.

    Scena dove Hax e Lissa sono nella stanza di Xeresia, per parlarle.
    Lissa esplode e capisce che il gruppo è stato incastrato. Non sa
    perchè, ma ha capito di essere in mezzo ad un casino più grosso di
    loro.

\end{document}
