\documentclass[9pt,a4paper,cleardoubleempty]{scrbook}

\usepackage{}
\usepackage[]{classicthesis-ldpkg}
\usepackage[parts,linedheaders,pdfspacing]{classicthesis}

\usepackage[T1]{fontenc}    % imposta la codifica dei font
\usepackage[utf8]{inputenc} % lettere accentate da tastiera
\usepackage[italian]{babel} % per scrivere in italiano
% \usepackage{layaureo}       % imposta i margini di pagina
\usepackage{lipsum}         % genera testo fittizio
\usepackage{indentfirst}
\usepackage{url}            % per scrivere gli indirizzi Internet
\hypersetup{pdfborder={0 0 0}}
\usepackage{allrunes}
\usepackage{phoenician}

\usepackage{color}
\definecolor{Brown}{cmyk}{0,0.81,1,0.60}
\definecolor{OliveGreen}{cmyk}{0.64,0,0.95,0.40}
\definecolor{CadetBlue}{cmyk}{0.70,0.57,0.23,0}

\usepackage[final]{pdfpages}

% args: image path
%       caption
%       width
%       label
\newcommand{\image}[4]{%
    \begin{figure}[hbt]%
    \centering%
    \includegraphics[width=#3\columnwidth]{#1}%
    \caption{\emph{#2}}%
    \label{#4}%
    \end{figure}%
}

% args: image path
%       caption
%       label
\newcommand{\imageFullWidth}[3]{\image{#1}{#2}{0.8}{#3}}

\newcommand{\picref}[1]{figura \ref{#1}}

\newcommand{\website}[2]{Sito web di \techname{#1}: \\ \url{#2}}
\newcommand{\paper}[3]{#1: \emph{#2} \\ \url{#3}}

\newcommand{\techname}[1]{{\sc #1}}
\newcommand{\cypher}[1]{{\tt #1}}
\newcommand{\syntax}[1]{{\tt #1}}
\newcommand{\codeconst}[1]{\lstinline{#1}}
\newcommand{\newterm}[1]{\emph{#1}}
\newcommand{\foreignword}[1]{\emph{#1}}

% Vari dettagli
\newcommand{\BW}{\emph{Burning Wheel}}

% Entità, NPC/PC e Luoghi importanti del gioco
\newcommand{\sciencegroup}{\techname{Athena Collective}}
\newcommand{\magicgroup}{\techname{Burning Universities}}
\newcommand{\theogroup}{\techname{Credo dei Dodici Divini}}

\newcommand{\theworld}{\emph{Xeresia}}
\newcommand{\theincident}{\emph{Shynthaian Pathfinder Incident}}
\newcommand{\firstvalley}{\emph{Valle di Ribia}}

\newcommand{\weapongoddess}{\emph{La II: Farcaster}}

\newcommand{\dacav}{\emph{Il ``Giovane'' Vecchio Dacav}}
\newcommand{\gods}{\emph{Nonoriko}}
\newcommand{\mitch}{\emph{Niyu-Rin}}
\newcommand{\ranger}{\emph{Alastor}}

\usepackage{}
\author{Michele "Jazzinghen" Bianchi}
\title{Project Xeresia: Characters}
\begin{document}
    
    Listerò quì tutti i personaggi che utilizzerò in questo progetto,
    non so se questo è il metodo corretto di affrontare quest'impresa, però
    proviamoci, al massimo è fuffa. :3

    Intanto iniziamo con...

    \chapter{The Rollers - I protagonisti}

    Questa è la squadra di avventurieri che sopravvivono facendo lavoretti
    vari. Sia scorta, recupero di artefatti, eliminazione di qualche
    mostro, basta che chi lo richiede li paghi e loro lo faranno.
    Ovviamente per sopravvivere in questo business seguono delle condootte
    morali, quindi non vengono mai fatti lavori ``Sporchi''. \'E indubbio,
    comunque, che la squadra utilizzi dei metodi \emph{poco convenzionali}
    per risolvere i problemi. Gruppi che cercano soldi in questa maniera ce
    ne sono, ma non hanno metodi così ``particolari''.

    Ognuno di loro è un esperto scientifico in un qualche campo, quindi
    ognuno ha delle capacità arcane. Non che porti sempre vantaggi questa
    cosa, visto che ogni capacità ha dei drawback assurdi.

    Il nucleo iniziale de I Rollers era composto dei soli Hax, Elythia e
    Lissa. Dopo si aggiungeranno anche The Fixer e Lady Jam.

    \section{The Rollers - Hax - Nato a Jokula}
        Hax è un ragazzo venticinquenne, capelli corti scuri e disordinati,
        volto come se si fosse sempre svegliato in quel momento.
        \'E un ragazzo disinvolto con gli amici, meno con le persone che
        non conosce. Dannatamente impulsivo, tende a farsi delle pessime
        idee nei confronti della gente a primo acchitto, per poi scoprire
        di aver torto e dover riparare, facendo anche figure orride. Crede
        in una giustizia più alta di quella che non
        viene fatta seguire. Tenta di non uccidere nessuno, anche se,
        effettivamente, sta covando vendetta nei confronti della persona
        che ha fatto il lavaggio del cervello a Celty, amante di Lissa
        (Ovviamente non si sa questa cosa, nel senso, non si sa che Celty
        abbia subito un lavaggio del cervello).

        Ha studiato programmazione nella stessa accademia di Lissa, dove si
        sono incontrati per la prima volta (O forse no?). \'E finito là
        dopo una storia intricata fatta di poco studio, una parte da
        autodidatta, ecc. Per entrare nell'accademia è necessario passare
        una prova nella quale si dimostrava di avere le competenze base
        attraverso un test ed una dimostrazione pratica dove si dimostra
        di essere migliori di un altro potenziale studente. Dopo aver a
        malapena passato il test teorico non ci si sarebbe aspettati di
        vederlo passare quello pratico contro un altro studente molto più
        compentente. Il fatto è che Hax non ha passato il suo tempo solo a
        studiare, cosa che non gli ha permesso di avere una grande
        competenza teorica di algoritmica, imparata tutta attraverso il
        Retro-Engineering di altri programmi. Tutto il resto del tempo l'ha
        passato vivendo di espedienti, imparando a cavarsela in una gran
        quantità di situazioni. \'E uno smanettone, in pratica. Grazie a
        questo è riuscito a passare il test attraverso l'utilizzo delle
        capacità da programmatore, che non sarebbero bastate, e di quelle
        più ``fisiche'', con le quali ha steso l'avversario in corpo a
        corpo. 

        All'inizio lui e Lissa non andavano proprio d'accordo proprio per
        come lui è entrato nell'accademia, però ora sono come fratello e
        sorella e lui farebbe di tutto pur di proteggerla, anche se non lo
        ammetterebbe mai. Comportamento tipico, eh?

        Proprio per il suo passato accademico Hax sa utilizzare le tecniche
        arcane derivate dall'utilizzo della programmazione, solo che
        rischia sempre di fare casini in quanto non è preciso. Per quanto
        riguarda il combattimento, non importa cosa usa, l'importante è che
        si possa utilizzare per difendersi e per difendere ciò che è
        importante per lui, siano i compagni di party o delle ideologie. Ha
        comunque un'arma preferita: il suo piede di porco. Uno strumento
        utile in tutte le occasioni, veramente.

    \section{The Rollers - Lissa}
        Lissa è una ragazza anche lei venticinquenne dal fisico atletico,
        alta quanto Hax, con
        capelli corti ed un carattere forte. Capelli corti castani ed
        occhi azzurri che sembrano sempre vedere qualcosa oltre ciò che si
        vede. Cinica, con la battuta pronta e la lingua più tagliente di
        una spada. La sua vita è stata sconvolta dalla perdita di Celty, la
        quale, di punto in bianco, ha deciso di lasciarla per andare a
        lavorare per un'agenzia nazionale come agente segreto. Questo ha
        rinforzato la sua idea, nata durante l'infanzia, che nel mondo la
        giustizia sia semplicemente un fatto de ``Il Fine Giustifica i
        Mezzi''.

        Lissa ha imparato le arti della programmazione durante tutta la sua
        vita, dalle elementari in poi. Del suo passato ricorda di aver
        vissuto la prima parte della sua vita in una specie di centro di
        ricerca installato su di un'isola, dove venivano condotte ricerche
        su bambine, dalla quale è fuggita grazie ad una persona che, in
        poche ore, era arrivata, aveva fatto saltare in aria parte del
        complesso, e se l'era portata via. Una volta fuggita con questo
        sconosciuto, questo lo portò al primo villaggio costiero sicuro e
        la lasciò lì per fare in modo che l'accudissero. \'E stata
        fortunata, in quanto trovò una famiglia che l'accudì e le permise
        di studiare nelle migliori accademie, fino ad arrivare alla scuola
        superiore (Una specie d'università) dove ha studiato con Hax. La
        sua voglia di studiare informatica le è nata guardando lo
        sconosciuto mentre li portava fuori dall'inferno nel quale aveva
        vissuto per sei anni. Durante i primi due anni della Scuola
        Superiore odiò Hax per come fosse riuscito ad entrare nella Scuola
        senza le competenze necessarie per entrare in quell'accademia. Poi,
        però, Hax, senza motivo, la protesse da varie accuse che la
        vedevano implicata in crimini perpetruati nella scuola.
        
        In combattimento Lissa è forse la più preparata del nucleo base dei
        Rollers. Nella regione dalla quale viene la programmazione è
        solitamente
        accompagnata da un rigido allenamento di tipo militare che prevede
        l'utilizzo di pistola e pugnale, così da essere pronti in qualunque
        evenienza. Non sarebbe stato necessario, in quanto non vengono
        obbligate le ragazze a seguire questo tipo di allenamento, ma lei
        ha insistito in quanto lo straniero, andandosene, le ha lasciato un
        piede di porco, dicendole come fosse necessario per proteggere chi
        c'è d'importante, \emph{Yadda} \emph{Yadda}

        La ``magia'' praticata da Lissa è sì la programmazione, come Hax,
        solo che, essendo lei più preparata in quasi tutti gli aspetti, è
        molto più stabile e può controllarne meglio gli effetti, i quali,
        però, sono meno potenti. A differenza di Hax lei non ha studiato
        molto Retro-Engineering, non potendo copiare con efficacia i
        programmi di altra gente, però può correggere al volo gli errori
        nei programmi di Hax, evitando catastrofi.

    \section{The Rollers - Elythia}
        \graffito{Inizierò con descrizioni più brevi, altrimenti non
        finisco più XD}Medico del gruppo, trentenne, grazie alle competenze
        mediche può evocare spiriti degli Hazards scientifici (Tipo EMP,
        Polvere, Biohazard, Radiazioni). Corpo dannatamente sexy, più alta
        di Hax e Lissa. \'E la parte sociale del gruppo, se non ci fosse
        lei Hax e Lissa avrebbero causato ben più di un incidente
        diplomatico. Allenata nelle arti marziali della sua regione. Ha
        passato una vita tutto sommato felice, anche se suo padre faceva da
        medico per la criminalità organizzata, il quale le ha insegnato
        tutto. Tra i venti ed i venticinque anni ha studiato in una scuola
        specializzata, per poi mettersi a lavorare come ``mercenaria''. \'E
        da due anni che lavora con Hax e Lissa, ovvero da quando sono
        usciti dalla loro accademia.

        Seguita da tre spiriti: EMP, Mercury e Dust. Esiste anche un quarto
        spirito: Radiation, ma non si fa quasi mai vedere. \'E troppo
        pericolosa.

    \section{The Rollers - Lady Jam - Sanpan}
        La prima ``nuova aggiunta'' della squadra. Lady Jam è una
        combattente che ha imparato sia le arti della spada che della
        matematica, limitate solo agli uomini della sua zona. \'E riuscita
        ad impararle grazie ad un guru che viveva nelle foreste delle sue
        zone.

        Dopo essere partita si è promessa che sarebbe tornata a casa a
        dimostrare cosa era diventata ed a farsi accettare come ciò che
        era.

        A causa dell'addestramento che ha ricevuto è leggermente staccata
        dalla realtà.

    \section{The Rollers - The Fixer}
        L'ultimo arrivato (Per storia). Un personaggio strano, non parla
        moltissimo. \'Possiede un'arma che può venir cambiata in qualsiasi
        altra arma da corpo a corpo, in pratica. Ha capito come funziona la
        mente umana. Quando viene colpito da una crisi di panico non si
        spaventa, invece la incanala inconsciamente.

        Spesso capita che compaia da qualche parte con gli oggetti
        corretti, mangiando qualcosa o cose simili. Sembra che non gliene
        freghi un cazzo.

        Perchè è un vero uomo.

    \chapter{Alleati}
        
        Nel mondo non esiste solo la squadra dei ``Rollers''. Perchè questa
        storia esiste ci sono anche altri personaggi, tipo. Chi ha
        commissionato il lavoro? Chi è questa tizia che devono salvare? Chi
        è Celty?

    \section{Bard}
        Bard è l'amante di Xeresia, il quale ha presentato la richiesta
        fasulla che attira i protagonisti. \'E un suonatore, un buono
        standard da libri fantasy.

    \section{Xeresia}
        Xeresia è la principessa della nazione di Phaion, alla quale è
        stato dato il nome del mondo, una tradizione della sua famiglia. 

        Si è trovata all'interno di un complotto più grande di lei,
        rimanendo bloccata nel suo palazzo a lungo. Bard non ha proprio
        capito per quale motivo è che non potevano contattarsi, perciò ha
        deciso di emanare un contratto per gruppi di Tuttofare per vedere
        che le è successo, buttando se stesso ed il gruppo in un casino
        peggiore di quanto non indicasse il compenso del lavoro.
    
    \section{Celty Anderson}
      La ragazza di Lissa. \'E quella controllata da uno dei cattivi, che
      la sta utilizzando come agente. Ha rischiato di uccidere Lissa
      durante una delle missioni dei Rollers. Lissa sta ancora tentando di
      capire cosa le sia accaduto ed un modo per farla ritornare quella di
      una volta.

      Il lavaggio del cervello che le è stato fatto non prevedeva
      l'``installazione'' di qualche ordine, bensì è partito come se fosse
      una di quelle minacce alla ``se non lo fai uccidiamo chi ti sta
      caro''. Ovviamente, poi, sono state aggiunte parti che le facevano
      credere che alcune imprese fossero assolutamente necessarie.
      Un'operazione per evitare quel feeling da ``c'è qualcosa in me che
      non va'' e così via.

      Hax ha notato che c'era qualcosa che non andava il giorno che lei se
      n'è andata. \'E da quel giorno che sta provando a capire che diavolo
      ci fosse che non andava nella ragazza, visto che non credeva alle
      scuse che lei aveva tirato fuori quando se n'era andata.

    \section{Erika Walker}
      Dottoressa, compagna d'accademia di Elythia. Lei è quella della penna
      che va a fuoco.

    \section{Aya Gauss}

    \section{Reyleigh}

    \section{Il Maestro}
      \'E colui che ha insegnato a Jam le arti del combattimento e la
      matematica. \'E un innovatore, crede che la storia di non far
      scendere in battaglia le donne ed addirittura non permettere loro
      neppure di utilizzare le armi sia una cazzata dei vecchi tempi.

    \chapter{Chi tira le fila}
      
      \section{Il Culto del XII ~ Versione Fuffa}
        
        \subsection{Philip III Duca di Shien}
          Fa parte del culto dei XII ed è, fondamentalmente, una delle
          persone più influenti in quel gruppo. Ha deciso che farà partire
          un colpo di stato ai danni di Phaion con il pretesto, per il
          Culto, di far ritornare le tradizioni, ecc.

          In realtà questa mossa è stata appoggiata da BLAH (Non ho ancora
          un nome), il quale non ha interessi nello stato di Phaion, ma
          vuole utilizzarlo come base d'operazioni.

      
      \section{Altra gente malefica o, semplicemente, stronza}
        
        \subsection{Seysill Mann}
          Parlamentare ``mercenario'' all'interno del governo di Phaion,
          lavora per il culto del XII. Non lo fa per l'ideoologia, che vuoi
          che gliene freghi. Lo fa per il denaro. Non a caso pensa di
          andarsene una volta fatta passare la legge che blocca le entrate
          alla nazione.

          Non so se lo si rivedrà molto, potrebbe addirittura diventare un
          aiuto, visto che poi il culto dimostrerà che non sarà più
          necessario un parlamento. E allora lui a cosa serve?

          Certo, potrebbe anche andarsene, lo ha sempre fatto. Ha
          avvelenato governi dall'interno per anni. Non è che cambiare
          stato fosse una cosa nuova per lui, è solo che il governo di
          Phaion... \'E un sandbox. UN OTTIMO sandbox. E lui odia chi gli
          ruba le cose che gli piacciono.

          Poi, però, gli viene in mente che, se lo beccassero, potrebbe
          succedere la merda, perciò prova ad andarsene. Ma non può.
          Problemi. L'unico moodo è aiutare i Rollers. Bella merda.

          Anche se non sembra, non sempre fa lavori di sabotaggio dei
          governi. Ogni tanto ha fatto del bene, soprattutto a Durga,
          salvando il fratello dalla pena di morte. Accusato ingiustamente,
          Seysill è riuscito a far fare altre indagini, le quali hanno
          scagionato il ragazzo. Per Dio: è un gentleman, le ingiustizie
          non verranno fatte passare.
        
        \subsection{Durga}
          
          Durga è il braccio violento del Culto. Non ha nessuna
          connessione con Philip, però le è stata affidata la sua
          protezione. Non è comunque d'accordo con questa scelta, in quanto
          il tizio gli puzza dannatamente.

          Il culto nel quale crede, in realtà, è quello dei XIII, la
          versione originale. \'E per questo che lei è un personaggio che,
          secondo me, potrebbe passare dall'altra parte (Non proprio buoni,
          parliamo di stronzi XD)

          \'E un'amica di Seysill. Per quanto lui faccia dei lavori sporchi
          a lei non importa. \'E vero che ha un senso della giustizia che
          la porterebbe a picchiarlo, ma è anche vero che il credo di
          Umbra, una delle divinità del Culto dei XIII, dice che anche
          nelle azioni che paiono malvage si nasconde il bene.

\end{document}
