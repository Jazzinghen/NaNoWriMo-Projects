\chapter{I soldi, alla fine, servono a tutti}

  \section{Rollers - New Metredeth - Luglio 2635}

    Erano passate tre settimane dall'utltimo lavoro a Shynthia. Elythia era
    riuscita a far capire alla polizia che Hax era stato preso dai
    criminali che stavano seminando il panico nella città da qualche mese
    e che era riuscito a liberarsi a causa di un'esplosione o una cosa
    simile. Non si ricordava già più come fosse riuscita a levarlo da
    quella situazione, in realtà. Non importava. Alla fine erano riusciti a
    portare fuori da quel palazzo tutta la gente dentro le capsule
    criogeniche. Teconologia avanzata per dei criminali da quattro soldi.

    I Rollers avevano passato una settimana a tentare di capire perchè
    quelle strumentazioni fossero lì e a che diavlo servissero per poi
    arrivare ad una conclusione: Non sono affari nostri. Alla fine il loro
    lavoro era già un casino, non avevano tempo da passare in questioni che
    riguardassero la sicurezza del paese. Quello era il compito della
    polizia locale. Hax sembrava comunque un po' in pensiero per quella
    questione, ma gli passò in un paio di settimane.

    Ora avrebbero dovuto trovare un altro lavoro. Stare lì a non fare nulla
    per tutto il giorno li infiacchiva. Certo, non si poteva dire lo stesso
    di The Fixer che, nel tempo libero, metteva a posto armi in giro per
    città. Il problema era che lo faceva gratis, in quanto lo vedeva come
    allenamento.

    Quando non erano al lavoro vivevano tutti in una casa a due piani che
    erano riusciti a comprare con un lavoro molto ben retribuito durante il
    quale non avevano fatto casini tali da dover ripagare i danni causati.
    Lei viveva al piano terra assieme a Jam mentre gli altri tre vivevano
    di sopra, dove avevano i loro laboratori, oltre che una stanza dove Hax
    e Lissa passavano le serate davanti alle console da gioco.

    Era una splendida giornata, quindi Elythia decise di andare a fare un
    giro a vedere se erano disponibili dei lavori alle bacheche per i
    ``Tuttofare''. Fortunatamente non ce n'erano moltissimi, di gruppi come
    loro, in questa città, per quanto centrale. Evidentemente avevano
    deciso che quella zona non era remunerativa. Non avevano tutti i torti,
    ma vivere a New Metredeth, città universitaria con una popolazione
    di circa cinque milioni di persone, era molto stimolante a livello
    scientifico. Vivere da quelle parti significava avere accesso diretto a
    tutte le innovazioni tecnologiche ed alle nuove scoperte scientifiche.
    Era una cosa estremamente importante per il gruppo, in quanto tutti
    erano degli esperti in un campo scientifico differente.

    Elythia entrò in camera sua e si cambiò. Indossò uno dei suoi Qipao
    corti, mise ai piedi un paio di scarpe da KungFu ed uscì sul corridoio.
    Passando davanti alla cucina vide (e sentì) Lissa e Hax bisticciare mentre
    stavano finendo di pulire le stoviglie utilizzate per il pranzo di quel
    giorno. Non aveva capito bene quale fosse il discorso, ma aveva colto
    alcune parti nelle quali stavano parlando di prendere l'energia
    elettrica dai cavi della corrente che arrivavano in casa per poi
    cambiare la direzione dell'acqua per accelerare il lavoro di pulzia, in
    quanto era particolarmente noioso.

    Ad Elythia non sembrava proprio che si annoiassero. Ogni volta che
    toccava a loro mettere a posto o fare altri lavori di casa assieme si
    mettevano a discutere su come avrebbero potuto migliorare le operazioni
    casalinghe con degli algoritmi, cosa che duplicava il tempo necessario
    per i lavori che, comunque, venivano realizzati a mano, come facevano
    tutti.

    Elythia aprì la porta di casa e fuori vide The Fixer, che stava
    tornando con un paio di spade ed un fucile di precisione da riparare.
    ``Hei, lavoratore! Come va?'' fece all'uomo, tenendogli la porta aperta
    ``Mah, mi godo la giornata.'' ``Eh, vedo. Per quando devi ripararle?''
    ``Ripararle?'' rispose lui, perplesso, entrando in casa. ``Ah, no,
    nulla. Beh, allora buona giornata. Vado a farmi un giro.'' continuò
    Elythia, raggiante, anche se non aveva capito bene la risposta di Fixer
    ``Ah, senti, Elly, potresti prendermi del rhum invecchiato e, se fosse
    possibile, dei sassi levigati?'' ``Ok. Dove li prendo?'' ``Beh, i sassi
    li puoi trovare al fiume. Il Ruhm\dots{} boh. Pensavo che tu conoscessi
    tutti i negozi di cose costose.'' ``Hem. Beh, più o meno.'' ``Ottimo!
    Grazie. Buona giornata, mia cara.'' ``Nessun problema. Buona giornata a
    te.'' e si allontanò mentre dietro le sue spalle The Fixer chiudeva la
    porta con il piede.

    \emph{Rhum?} pensò, perplessa, Elythia, camminando lungo la strada
    verso il centro.

  \section{Gabriel - Dera - Luglio 2635}
    
    Era tornato a casa due settimane prima e, appena arrivato, aveva
    diramato un contratto di lavoro per i circoli dei ``Tuttofare''.
    Sperava che qualcuno accettasse il suo lavoro. Non era niente di
    difficile. Voleva soltanto chiedere ad uno di questi gruppi se poteva
    andare a Phaion per lui per vedere come stava Xeresia. Purtoppo erano
    passate due settimane senza che ci fossero state risposte. \'E vero che
    avrebbero potuto metterci molto ma quei circoli avevano un ottimo
    sistema di comunicazione, utilizzando corrieri, air ships, navi a
    vapore e così via.

    Ora doveva solo aspettare, non sarebbe passato molto prima che qualcuno
    accettasse la sua richiesa.

  \section{Celty - Shien - Luglio 2635}

    Celty era tornata a Shien dopo l'incidente a Shynthia. Le ci era voluto
    un po'. Aveva dovuto farsi curare da un dottore del mercato nero e poi
    tornare con mezzi improvvisati, visto che quasi tutti i soldi erano bruciati
    nella base di quei subdoli criminali e che i pochi che le erano rimasti
    aveva dovuto utilizzarli per farsi mettere a posto.

    Che stupido che era Hax. Evidentemente non aveva ancora imparato a
    mettere a posto il suo codice. Era da quando l'aveva conosciuto che era
    così. Altro che Lissa. Lei era un genio. Oltre ad essere un bel po'
    tsundere. Celty non trattenne una risata, pensando alla ragazza. Poi,
    però, fu colta dalla tristezza. Era un anno e mezzo che non la vedeva,
    che non le parlava. Era successo tutto così in fretta. 

    Duante l'accademia Celty faceva parte della squadra di atletica leggera
    e, se doveva ammetterlo, era la più forte dei suoi anni per quanto
    riguardava la corsa urbana. Era una competizione che prevedeva correre
    all'interno di un ambiente urbano, facendo il tragitto tra due o più
    punti in meno tempo possibile. La parte interessante era considerare
    tutte le strade possibili, il che comprendeva saltare tra tetti di
    costruzioni, superare grate, muri. Uno sport che la faceva sentire
    libera. Sentire il vento contro il volto quando correva oppure quando
    si trovava in quel momento del salto quando la forza di gravità
    annullava la spinta iniziale e si sentiva come sospesa in aria le
    faceva pensare di star volando. Era magnifico.

    Era per questo che aveva iniziato ad andare a fare gare internazionali
    anche dopo aver fatto l'accademia, posizionandosi spesso nei primi posti. Purtroppo però,
    l'ultima volta che aveva partecipato, qualcuno stava osservando le sue
    performance. Alla fine della competizione, alla quale era arrivata
    prima, le venne presentata una persona di nome Philip, la quale voleva
    offrirle un lavoro. A quanto le avevano detto era duca di una qualche
    nazione un bel po' lontana, Phaion. Sarebbe stata un'occasione d'oro
    se non fosse stato che lei non aveva bisogno di un lavoro e doveva
    ancora finire il suo libro. Comunque andò a parlargli di persona per
    rifiutare. Una volta ci furono una serie di convenevoli poi, quando lui
    le offrì il lavoro lei rifiutò cortesesemente. Si sarebbe aspettata
    dell'insistenza, ma non quello che le venne detto dopo.

    ``Capisco che lei non abbia intenzione di lavorare per noi ma, vede, io
    penso che lei debba anche considerare questa cosa: a me non piace venir
    rifiutato.'' fece l'uomo ``Scusi, però non posso proprio. Potrebbe
    chiedere a qualcun'altro. Immagino che ci siano molti altri
    partecipanti che farebbero volentieri il lavoro di corriere.'' rispose
    lei, garbata ``Ne sono sicuro ma, vede, io ho bisogno del MIGLIORE, non
    di uno scarto. Comunque sono sicuro che questo le possa far cambiare
    idea.'' schioccò le dita e quella che sembrava una guardia del corpo
    gli passò una busta. Celty sussultò quando vide scivolare fuori dalla
    busta una serie di ritratti scattati a Lissa, Hax, Elythia e Jam.
    ``Come può vedere conosco benissimo le persone a lei più care e, come
    le ho già detto, mi irrita particolarmente sentirmi dire di no.
    Questa\dots{}'' fece, tirando su un'immagine che ritraeva Lissa
    ``\dots{}è la sua ragazza? \'E carina. Sarebbe un problema se qualcuno
    le facesse del male.'' Celty saltò dalla sedia, impaurita ``Cosa sta
    dicendo?'' ``Sto dicendo che io voglio che lei lavori per me,
    altrimenti potrebbe accadere qualcosa di \emph{spiacevole}, sia a lei
    che ai suoi amici.'' ``M\dots{}Ma\dots{}'' sentì il mondo crollarle
    addosso. Era finita nella trappola di un ricattatore, un criminale
    della più bassa specie.

    Non potè far altro che accettare. Non voleva che venisse fatto a
    nessuno a causa sua. Non potè dire nulla a nessuno, quindi, dopo alcuni
    giorni, se ne andò lasciando tutti alle spalle. Probabilmente quel
    supplizio sarebbe finito con uno o due lavori. Il problema fu che Hax
    era un uomo tenace e la seguì per un po'. Fortunatamente dopo qualche
    giorno riuscì a seminarlo, potendo continuare il viaggio senza nessuno
    che la legasse al passato.

    Le aveva fatto piacere sapere che Hax stava bene. Probabilmente anche
    gli altri stavano bene, altrimenti giel'avrebbe detto. Certo, le aveva
    fatto un po' meno piacere aver dovuto combattere con lui. Non avrebbe
    mai pensato d'incontrarlo da quelle parti. Che ci faceva là? Comunque
    adesso sarebbe dovuta andare dal suo ``datore di lavoro'' a riferirgli
    che era successo a Shyntia. Non l'avrebbe presa bene. Quel bastardo era
    un cattivo come quello dei quali si leggeva nelle storie pubblicate sui
    settimanali.

    Entrò in una stanza esageratamente sfarzosa, non avrebbe esagerato a
    dire barocca. Il ``Duca'' era dentro, seduto su una poltrona a parlare
    con una donna. Probabilmente era Durga, una di quel culto del quale
    lui faceva parte. Una tizia strana. Parlava spesso di tradizioni, di
    giustizia ed altro ma poi lavorava per Philip. Celty non riusciva a
    capire perchè diavolo lavorasse per uno così. Era stata costretta anche
    lei oppure c'era qualche altro motivo?

    Comunque, dopo poco smisero di discutere e la donna si alzò. Avrà avuto
    una trentina d'anni. Era alta, molto. Aveva dei lunghi capelli bianchi,
    liscissimi, occhi blu profondo ed un fisico tonico, da combattente. Le
    passò vicino dopo averla fissata con interesse, per poi uscire dalla
    stanza, chiudendosi le porte dietro di lei. 

    ``Allora.'' fece il Duca dalla poltrona, senza neppure girarsi ``Ho
    sentito strane notizie venire da Shynthia. Che accade?'' Dannazione.
    Aveva già ricevuto informazioni. ``Ci sono stati problemi, Duca.''
    fece, impassibile, Celty ``Un esplosione ha fatto saltare la base
    d'operazione dei \emph{suoi} uomini. Dopo i controlli della polizia
    molti sono stati arrestati e le persone tenute in criostasi per il
    trasporto sono state liberate.'' ci tenne a sottolineare la parola
    ``suoi''. Philip battè il pugno destro sul bracciolo della sua poltrona
    ``Dannazione. Cosa l'ha causata?'' urlò, stizzito ``Non lo so, signore.
    Ero fuori dal complesso quando è successo.'' ``Maledizione. Proprio
    adesso doveva capitare? Abbiamo bisogno del denaro per quello che ci
    stiamo avviando a fare'' \emph{HAI bisogno, vecchio stronzo} pensò
    Celty ``Vabbeh. Ora và. Ti contatterò quando avrò bisogno dei tuoi
    servigi.'' concluse lui. Lei, senza battere ciglio, rispose ``Sì.'' e,
    girandosi, se ne andò.

  \section{Seysill - Metra - Luglio 2635}
    
    La legge che avevano votato aveva causato non pochi dissensi tra i
    commercianti. Effettivamente non permettere l'accesso agli esterni per
    più di qualche giorno significava mancanza di contrattazione, nessuna
    possibilità di aggiungere alla merce acquistata quel qualcosa in più
    che può farti arrotondare lo stipendio.

    Seysill era in piedi, davanti alla finestra del suo studio che dava sulla piazza
    centrale di Metra, con le braccia dietro la schiena. Osservava quella
    città che, probabilmente, non avrebbe più visto per un bel po' di
    tempo. Aveva visto fin troppe situazioni simili. Leggi che gli venivano
    richieste da un gruppo che pensava di fare il bene per una nazione che
    portavano inevitabilmente a qualche colpo di stato. In questo caso non
    era tanto differente. Per quanto la popolazione o, come aveva fatto
    credere loro, gli altri parlamentari credessero che questa legge era a
    fine della protezione della ``purità della razza Phaioniana'' e
    stupidaggini simili, in realtà era puntata a qualcosa di ben più
    subdolo.

    Cosa faceva fatica a circolare se l'accesso era limitato e richiedeva
    un certo tempo per venir concesso? I soldi? Assolutamente. Quello
    veniva permesso dalle dogane con un sistema fatto apposta per il
    commercio. No, quello che veniva bloccato era la comunicazione di
    notizie all'esterno. Ci voleva un mucchio di tempo prima che una
    qualche notizia di un cambiamento all'interno dei confini di una
    nazione con quelle leggi raggiungesse un qualche altro stato. Era
    incredibile come alla gente, però, sembrasse non importare di questa
    cosa. Vedevano la pagliuzza nell'occhio dell'altro e non vedevano la
    trave nel loro. Questo blocco, con tutta probabilità, serviva al gruppo
    che l'aveva ingaggiato per portare a termine un qualche colpo di stato
    ``silenzioso''. Una di quelle cose che porta ad un cambiamento di
    regime senza che la gente se ne accorga troppo e senza che le nazioni
    vicine possano fare nulla, in quanto, quando riceveranno la notizia,
    sarà troppo tardi.

    Era per quel motivo che, dopo qualche minuto che stava alla finestra,
    l'uomo si girò e si diresse verso la sua libreria. Era incredibile
    quanti libri fossero stati lasciati lì dal suo predecessore. Fare un
    lavoro come il suo richiedeva pianificazione ed un pizzico di fortuna.
    Ovviamente non poteva arrivare a mandato avviato. Doveva arrivare
    poco prima delle elezioni e, per un qualche motivo, sostituirsi ad un
    candidato, oppure scalare così rapidamente le fila di un partito per
    poi venir messo nel parlamento. Da quel momento in poi le cose
    diventavano quasi facili. Doveva aspettare il momento buono per
    iniziare a convincere quelli del suo schieramento di quanto fosse
    ``buona'' una certa idea. Si dava il caso che quella ``certa'' idea
    fosse consigliata dai suoi datori di lavoro. Poi il resto veniva da se.
    Proponeva la legge (anche ben scritta. Le sue capacità di legislatore
    erano a dir poco eccellenti, doveva ammettere) e questa veniva votata.
    Detto, fatto. Con questo metodo riusciva addirittura a portare a
    termine fino a due contratti all'anno. Sembreranno pochi, ma erano
    remunerativi e la vita del parlamentare spesso portava ad un gran
    numero di vantaggi.

    L'unico problema era che, per essere eletti nei parlameti di tutto il
    mondo, aveva bisogno di essere cittadino di quella nazione. Non era
    così problematico, in realtà. Aveva un gran numero di contatti per
    risolvere quel problema, oltre che una ottima collezione di documenti
    falsi.

    Aprì il cassetto della libreria che conteneva i suoi documenti privati
    e tirò fuori una serie di contratti assieme ai documenti d'identità.
    Lesse attentamente i contratti per scegliere il migliore tra di essi.
    Questa operazione richiese un po' di tempo però, alla fine, trovò un
    incarico che prevedeva di far passare al parlamento di Shinra la
    costruzione di un impianto di purificazione dell'acqua marina,
    collegata ad una centrale eolica. A quanto pare era da fin troppo
    tempo che in quella nazione il partito industriale stava vincendo,
    postponendo la costruzione a data da definirsi, e quindi il gruppo
    ambientalista lo aveva contattato per ``dare una spinta'' al progetto.
    Ottimo. Quello che la gente non capiva era che lui, fondamentalmente,
    faceva il bene quando poteva. Era ovvio che se non c'erano contratti
    come questo doveva accettare anche lavori eticamente meno corretti.
    Però, quando si presentava l'opportunità, sceglieva sempre quello che
    avrebbe portato più giovamento alla nazione interessata.

    Come facesse lui a sapere quale fosse il bene di una nazione non era
    importante. Forse ogni tanto non lo è stato. Però faceva un'analisi
    completa della situazione e decideva quello che, secondo lui, aveva il
    miglior impatto sulla qualità della vita della popolazione. Si può dire
    che molte nazioni di Xeresia, il mondo sul quale vivevano, avevano una
    parte di lui nella loro storia, senza che lo sapessero.

    Comunque era deciso. Ancora uno o due mesi e poi sarebbe partito per
    andare a vincere delle elezioni a Shinra. Era da un po' che voleva
    andare in un qualche posto esotico a farsi una vacanza.

  \section{Elythia - New Metredeth - Luglio 2635}
    
    La giornata era andata molto bene. Era riuscita a trovare una serie di
    contratti molto interessanti. Era anche riuscita a comprare il rhum per
    Fixer. Era passata da ``Al Tal De I Tali'', famosissima enoteca di New
    Metredeth, sita nella zona degli artisti. Una zona piena di accademie
    letterarie. Quante avventure aveva avuto da quelle parti quando
    studiava medicina?

    Il proprietario del negozio era sempre stato quello fin da quando se lo
    ricordava. I suoi genitori erano degli assidui frequentatori:
    apprezzavano moltissimo il buon mangiare, per non parlare dei liquori
    che potevi trovare in quella bottega.

    Ebbe dei problemi con i sassi per il chimico, comunque. Non proprio
    problemi gravi, se doveva essere sincera, più che altro non sapeva
    quali andassero bene. Si era fatta aiutare da dei ragazzini che stavano
    giocando dalle parti del fiume, i quali cercavano riparo dalla calura
    estiva. Alla fine se n'era andata con una decina di rocce con colori
    che andavano dal bianco all'antracite perfettamente levigati. E dieci
    guild in meno. Avidi mocciosi. Comunque ora poteva tornare a casa.
    Avrebbe dovuto cucinare lei, per non parlare poi del fatto che avrebbe
    dovuto discutere con gli altri del prossimo lavoro che avrebbero
    accettato.

    Arrivò a casa verso le sette di sera. Ottimo. Significava che poteva
    ancora ripassare una o due forme della sua arte marziale e poi avrebbe
    dovuto rispettare il turno settimanale per la cena. Entrò a casa,
    depositò i documenti sul tavolino del soggiorno, dove c'era Jam che
    stava leggendo un libro sulla teoria degli insiemi, sdraiata sul
    divano. ``Oh, ciao Elly!'' fece lei, distogliendo lo sguardo dal libro
    ``Com'è andata la giornata?'' ``Bene, grazie. Ho trovato quello che mi
    serviva.'' ``Ottimo! Hai preso qualcosa di buono per me?'' ``Ho del
    Rhum, ma è per The Fixer. Non so cosa voglia farne.'' ``Oh. Vabbeh.
    Allora ti lascio andare, che torno alla mia lettura.'' ``Ottimo. Buona
    lettura, Jam.'' Elythia se ne andò dal soggiorno, che era collegato
    alla porta d'entrata tramite un corridoio senza porte, lungo il quale
    c'erano una serie di porte che davano sulla cucina, alle camere del
    primo piano e così via. Salì le scale che portavano al piano superiore,
    passò davanti allo studio di Hax e Lissa, dal quale stavano uscendo dei
    rumori di videogames, ed arrivò davanti al laboratorio di Fixer.

    Bussò alla porta e fece ``Fixer! Ti ho portato quello che mi hai
    chiesto!'' Si sentirono rumori di pile di oggetti metallici crollare a
    terra ed i passi di una persona che si affrettava alla porta. Dopo poco
    l'uomo aprì la porta. Era in maglietta, sporco di olio e di componenti
    chimici che non riconobbe. ``Ciao.'' le disse, appoggiandosi allo
    stipite della porta ``Eccoti quello che mi avevi chiesto.'' prese la
    bottiglia di Rhum dalla sua borsa e gliela porse ``Il Rhum e\dots{}''
    frugò sempre nella borsa fino a tirarne fuori tutti i sassi ``\dots{}ed
    i sassi.'' Fixer appoggiò a terra la bottiglia ed osservò attentamente
    ad uno ad uno i sassi. Dopo un po' guardò Elythia e le fece
    ``Fantastico! Hai fatto proprio un ottimo lavoro, mia cara.'' e fece
    per entrare nella stanza. Poi si fermò, prese la bottiglia da terra e
    gliela ridiede indietro ``Ah, puoi portare questa alla nostra Lady? Sono
    sicuro che apprezzerà.'' ``Eh?'' fece Elythia, non capendo bene,
    prendendo la bottiglia ``Grazie! Ora scusami, ma devo tornare al
    lavoro.'' e chiuse la porta.

    Elythia guardò la bottiglia che aveva in mano, fece spallucce e tornò
    al piano di sotto. Non riusciva a capire bene The Fixer, però faceva
    spesso così. Sembrava sempre sapere di cosa avessero bisogno gli altri
    che gli stavano intorno. Tornò in soggiorno e, sollevando la bottiglia
    verso Jam, fece ``The Fixer mi ha detto di consegnarle questo, Lady.''
    Jam distolse lo sguardo dal libro e, appena notò la bottiglia,
    s'illuminò in volto ``Rhum? Di Jamenade, tra l'altro? Lì lo fanno
    benissimo. Grazie Elythia!'' ``Ma, veramente, me l'ha detto Fixer.''
    ``Sì, però sei andata a comprarlo te, immagino?'' ``Eh, in effetti. Me
    l'ha consigliato un amico.'' ``Mmmmh.'' fece contenta la ragazza, come
    se si stesse già pregustando l'alcoolico ``Non vedo l'ora di poterlo
    bere assieme a voi.'' ``Grazie. Bene, ora vado, che voglio fare una o
    due forme prima di cena.'' ``Ok!'' Jam appoggiò la bottiglia sul
    tavolino e poi tornò a leggere.

    Le cose buone da mangiare erano la debolezza della ragazza, assieme ai
    libri. Non era una che avrebbe speso moltissimi soldi in vestiti o cose
    così, ma quando si trattava di mangiar bene o di coccolarsi con un
    liquore ricercato non si faceva remore. Sapeva anche, però, che
    troppo avrebbe significato far perdere d'importanza quella sua
    passione, perciò cercava di limitarlo. Cosa che non si applicava ai
    libri: contenevano quello per cui lei viveva. Fossero nuove teorie
    matematiche o storie epiche.

    Elythia passò più o meno mezz'ora per ripassare due forme della sua arte
    marziale, la quale richiedeva più controllo e lentezza che non velocità
    e forza. L'idea era che, dal controllo delle energie interne contenute
    all'interno di ognuno e con l'utilizzo completo del corpo sia proprio
    che dell'avversario, si potevano fare cose molto più d'impatto che con
    arti marziali che basavano tutto sulla propria forza fisica. Quando
    qualcuno si sbilanciava. Era a terra, non c'erano alternative. Ed
    essere a terra significava spesso la sconfitta. Perciò studiare come si
    comporta il corpo e come fare per sfruttare anche un minimo squilibro
    per vincere era la strategia vincente. Ed era anche molto vicino al
    modo di pensare di Elythia, in quanto era una dottoressa. Finite le
    forme, poi, decise di utilizzare un'altra mezz'ora per coltivare il
    rapporto che aveva con i suoi spiriti protettori. Studiare medicina
    significava due cose: imparare come funzionava il corpo umano per
    poterlo rimettere in sesto e riuscire a comunicare con esseri che erano
    legati ad energie o ad aspetti della realtà. Questo avveniva perchè,
    fondamentalmente, il dedicare la propria vita a salvare le vite degli
    altri permetteva al medico di contattare creature arcane che
    condividevano una visione della vita con loro. O forse era il
    contrario. Il fatto era che, durante l'accademia, uno dei pasaggi era
    legarsi ad uno o più spiriti. Spesso i medici consideravano questi
    spiriti come dei ``servi'', ai quali far fare ciò che non si riusciva o
    ciò che non si voleva fare. 
    
    Per lei era un'altra cosa. Lei le (perchè , alla fine, erano tre
    ragazze nell'aspetto) trattava come delle amiche. Loro facevano
    qualcosa per lei, lei faceva qualcosa per loro. Non erano su piani
    differenti, alla fine. Inoltre aveva scoperto essere collegate
    direttamente alla divinità \emph{Eclipse}, dea di scienza, segreti e
    della vita. Era per questo che lei credeva nei XIII. Non era
    una cosa molto diffusa. Si sapeva che esistevano delle creature con
    poteri inimmaginabili, ma siccome non avevano mai agito su Xeresia
    allora le si considerava come delle semplici icone, niente di più. Il
    fatto che questi tre spiriti fossero venute ad aiutarla le aveva dato
    un motivo in più per seguire quella ``religione'', se così si poteva
    chiamare.

    Quando le richiamò arrivarono dopo un paio di minuti, alla fine anche
    loro avevano altro da fare. La prima ad arrivare fu EMR. Era sempre la
    più reattiva. Lei riusciva a controllare le onde elettromagnetiche.
    Elythia non capiva bene quel concetto. Però un giorno lo spirito le
    spiegò che tutto quello che riusciva a vedere era grazie a quelle onde.
    I colori erano onde elettromagnetiche con una certa frequenza mentre,
    ad esempio, le comunicazioni delle radio erano onde ad un'altra
    frequenza e così via. Quando combatteva utilizzava tre strumenti: uno
    era una grandissima aureola fatta di metallo alla quale erano collegate
    delle sfere nelle quali riluceva della luce azzurra. Questo poteva
    emettere delle onde che facevano bollire l'acqua a distanza. Un'altra
    era una lama che stordiva gli avversari al contatto con la pelle
    attraverso l'utilizzo di un impulso elettrico ad alta potenza. L'ultima
    arma era un fucile molto strano che emetteva gas ad alta temperatura,
    che EMR chiamava ``plasma''. Diceva che riusciva a scaldarlo a quelle
    temperature grazie alla ionizzazione forzata dell'idrogeno attraverso
    l'eccitazione elettromagnetica dello stesso.

    Non capiva bene tutta quella teoria, non faceva parte del suo campo di
    ricerca. Capiva molto più facilmente il funzionamento delle armi della
    seconda ad arrivare: Mercury. Lei utilizzava il mercurio come arma.
    Aveva una spada formata da un blocco in acciaio collegato ad un
    serbatoio pieno del suo elemento, il quale fluiva a comando lungo un
    solco formando una lama. Oltre a quella aveva un'arma a distanza che
    lanciava chakram sempre in mercurio. Era ovvio il funzionamento. Se non
    riusciva ad uccidere i suoi avversari tagliandoli sarebbero morti
    d'avvelenamento da mercurio entro breve.

    La terza era Dust. Fondamentalmente controllava tutte le polveri
    possibili. Fossero esse la polvere che si forma in casa quando non si
    pulisce o polvere chimica esplosiva, non importava, riusciva a farle
    muovere a piacimento o aspirarle nel suo serbatoio volante per poi
    ritirarle fuori a necessità. Anche se non sembra anche lei le ha
    salvato la vita più di una volta, magari bloccando completamente dei
    macchinari facendo fluire della polvere all'interno dei componenti più
    fragili.

    Passò un po' di tempo a chiacchierare del più e del meno con loro
    finchè non divenne ora di andare a preparare la cena. Le invitò a cena
    ma loro, al solito, rifiutarono. Si sentivano un po' a disagio con gli
    altri, anche perchè non capivano bene di cosa parlassero, visto che
    avevano modi di vedere la realtà molto differenti dai loro. Non che non
    avessero provato a fare amicizia. Era solo che si sentivano fuori
    posto. Dopo aver salutato scomparirono ognuna col proprio effetto
    visivo: un lampo azzurro, uno sbuffo di fumo grigio scuro,
    un'esplosione di metallo liquido.

    Tornò in casa e fece da cena. Una volta che tutti ebbero finito e Hax e
    Lissa stavano pulendo i piatti (facendo, al solito, i soliti discorsi
    su come avrebbero potuto rendere più veloce il loro lavoro) Elythia
    andò a prendere i documenti che aveva messo in soggiorno, seguita da
    Jam, che prese la bottiglia di Rhum. ``Bene,'' fece la dottoressa,
    mentre Jam stava distribuendo dei bicchieri a tutti ``allora, oggi sono
    andata a fare un giro in città, quindi sono passata dai soliti posti
    per recuperare dei contratti papabili come nostro prossimo lavoro.''
    Elythia prese ad uno ad uno i contratti e li distribuì sul tavolo.
    ``Ne ho scelti una serie in base alla quantità di soldi che ci vengono
    dati ed un'altra parte in base al rapporto difficoltà prezzo.
    Ovviamente molti di questi contratti fanno parte sia di un gruppo che
    dell'altro, ma che ci volete fare.'' The Fixer e Jam passarono un po'
    di tempo a leggere i contratti, senza però riuscire a sceglierne uno in
    specifico. Dopo un po' arrivarono anche Lissa ed Hax, ma non prima di
    prendersi un bicchiere di Rhum. ``Hum.'' fece Hax, guardando poco
    convinto un contratto ``Pagano un sacco per questo contratto. Non sono
    un grande fan di ribaltare governi per darli in mano a \emph{Salvatori
    della nazione}'' ``Ma come no? Da quando non ribalti governi?'' fece
    Lissa, guardando da dietro le sue spalle il contratto ``Da quando a
    richiedermelo sono delle persone che, probabilmente, porterebbero la
    nazione allo sfacelo. Dai, alla fine non mi pare che a Nerta stiano
    così male, no?'' ``In effetti la nazione ha un governo democratico. Uno
    che vorrebbe una rivoluzione in quel posto è uno che vuole conquistarlo
    con la forza.'' aggiunse The Fixer ``Perchè diavolo lasciano ancora
    certi contratti nelle bacheche?'' ``Beh, Fixer. Perchè non tutti hanno
    una condotta morale tipo la nostra, lo sai. \'E anche per quello che
    abbiamo sempre pochi soldi a disposizione.'' ``Dici che dovremmo
    iniziare ad accettare lavori come questo?'' rispose Hax polemico, dopo
    aver preso un sorso di Rhum ``No, Hax, sai come la penso. Meglio pochi
    soldi che dei lavori sporchi sulla coscienza. Temo che quel lavoro sia
    arrivato all'interno della lista visto che ho richiesto tutti i
    contratti in base ad un range di paga.'' ``Vabbeh,'' fece Lissa, per
    evitare che la discussione si protrasse troppo. Hax era prono ai litigi
    sulla natura del loro lavoro e voleva evitare che continuasse:
    normalmente portava solo ad una serie di discussioni senza obiettivo,
    tranne quello di perdere tempo. ``passiamo ad altro. Sentite un po'
    questo: recupero di un generatore tecno-arcano. Pagano abbastanza.''
    ``Hum, dove dovremmo andare?'' ``Hem, fammi leggere\dots{} Ah, ecco.
    Oh.'' il volto di Lissa lasciava trasparire del disappunto ``Jango.
    Fantastico.'' ``Le Jungle di Jango?'' chiese Hax, stupito ''Sì.'' ``Ma
    cavolo. \'E imppossibile trasportare un generatore di quelle dimensioni
    in quella zona!'' fece Jam ``Mi ricordo di quando ero piccola. Quando
    il maestro mi ci ha portata per un allenamento era difficile anche solo
    camminarci attraverso. Ci sono stati dei cambiamenti da quando me ne
    sono andata?'' ``No.'' rispose Lissa, quasi infastidita dalla cosa
    ``Non parlarmene. L'ultima volta che sono andata a fare un lavoro là mi
    hanno quasi mangiato le sanguisughe. Scartato.'' Lissa era una che,
    effettivamente, aveva accettato lavori singoli in giro per i posti
    peggiori del globo. Ogni tanto lo facenvano di dividersi per fare un
    po' più di soldi. Lissa si era addestrata per sopravvivere in
    situazioni anche abbastanza complicate. Mai quanto Hax, il quale, non a
    caso, veniva mandato a fare lavori in posti dimenticati dagli dei nei
    quali non poteva fare \emph{troppi} danni.

    Passarono a vagliare ipotesi una dopo l'altra, tra un bicchiere di Rhum
    e l'altro, senza però trovare un lavoro adatto, fino a che The Fixer
    non se ne uscì con ``Oh. Questo potrebbe andare.'' ``Di cosa si
    tratta?'' fece Elythia  ``Allora. Un certo Gabriel sta  cercando
    qualcuno che possa andare a Metra, capitale di Phaion, per vedere come
    sta Xeresia. Questa Xeresia penso possa essere la sua ragazza.'' ``E
    dove abita questo Gabriel?'' fece Elythia ``A Dera, pare.'' ``Scusa? Ma
    Dera sarà a non meno di due giorni di distanza da Phaion.'' ``Eh, lo
    so, ma lo sai che ci sono quelle regole per il numero limitato di
    giorni\dots{}'' ``Ma d'estate non c'è la fiera o quello che è?'' ``Non
    hai letto i giornali?'' ``No, non proprio.'' ``Beh, fondamentalmente
    hanno fatto passare una qualche legge che rimuoveva la fiera estiva e
    riduceva le giornate d'accesso annuali.'' ``Ah, quindi questo vuole che
    facciamo una visita alla sua amata al posto suo visto che abbiamo
    ancora tutte le giornate disponibili.'' ``Esatto.'' ``E quanto paga?''
    ``Hum. Diciamo un mese di lavoro.'' ``COSA?!? \'E troppo poco! Lo sai
    quanto costa arrivare a Phaion da quì?'' ``\'E proprio lì che ti volevo
    cara Elyhia. Questo ci paga il VIAGGIO.'' ``Scusa?'' ``Esatto. Hai
    capito bene.'' ``Ma questo vuol dire\dots{}'' ci fu un momento di
    silenzio nella cucina, che venne rotto da Jam che battè le mani sul
    tavolo e, alzandosi in piedi, urlò contenta ``VACANZE PAGATE!''

    Seguì una serie d'esclamazioni di felicità. Erano almeno tre anni che
    non facevano vacanza. E questo quì, in pratica, stava per pagarglielo.
    La loro fortuna, inoltre, era che la paga era così bassa che nessuno
    l'avrebbe accettato. Nessuno, tranne un gruppo di scienziati con pochi
    soldi che avevano bisogno di rilassarsi. Anche la deadline era molto
    leggera. Fondamentalmente era la fine dell'anno. Ovvio, anche perchè
    poi sarebbe potuto andare lui. ``Ok!'' fece Elythia, buttando giù tutto
    d'un fiato quello che rimaneva del suo drink ``Visto che mi pare che
    siamo tutti d'accordo domani vado a dire che accettiamo il lavoro.''

  \section{Xeresia - Metra - Luglio 2635}
    
    Era stato uno shock. Venne a scoprire che, a causa di una legge
    parlamentare adesso non c'era più il periodo della fiera estiva. Fu per
    quel motivo che non vide Gabriel all'inizio del mese. Non poteva
    entrare a causa di un numero ancora pià ristretto di giorni che erano
    stati resi disponibili ai visitatori esterni. Come avrebbero fatto?

    Tentò di parlarne con suo padre, il re, il quale, però, le rispose
    dandole anche tutti i motivi politici sul perchè non potesse rimuovere
    quella legge. Era stata votata dal parlamento il quale,
    fondamentalmente, faceva il volere del popolo. Ed ignorare il volere
    del popolo era un soppruso, un abuso di potere, e lui questo non lo
    voleva. Le disse che avrebbe dovuto aspettare finchè non arrivava
    l'anno nuovo, nel quale avrebbe potuto di nuovo vedere il suo amore.

    Ma lei sapeva che Gabriel avrebbe trovato il modo per andare da lei
    molto prima.

  \section{Seysill - Metra - Luglio 2635}

    Oh, delizioso tè di Phaion. Quanto gli sarebbe mancato una volta
    andatosene. Aveva deciso di godersi i piaceri di questa nazione ancora
    per un po', prima che accadesse l'irreparabile. Era seduto in un
    bistrot di una delle strade principali di Metra a godersi la città la
    sera, quando si fermò davanti al suo tavolino una donna. ``Seysill.''
    fece, abbassando lo sguardo per guardarlo in volto ``Oh, Durga.'' ``\'E
    libera?'' fece, lei, indicando una delle due sedie libere accostate al
    suo tavolino ``Ma certo. Sai che non rifiuterei mai la compagnia di una
    signorina.'' poi, appoggiandoo il gomito al tavolino ed avvicinandosi
    alla sua interlucotrice ``Soprattutto se è attraente come te.''

    Durga spostò la sedia, per poi sedersi sopra. ``Oh, smettila.
    Adulatore.'' ``Il signor Mann non scherza mai su una cosa così
    importante come quanto è attraente una signora.'' disse Seysill,
    ritornando appoggiato allo schienale della sua sedia, parlando in terza
    persona ``Allora. Come va? Era da un po' che non ti vedevo.'' ``Eh,
    guarda, non parlarmene.'' ``Oddio. Il tuo \emph{Culto} ti ha
    fatto fare qualcosa di sgradevole?'' ``Non dire queste cose. Ne abbiamo
    già discusso.'' ``Se lo dici te mi fido. Allora, che accade?'' ``Mi
    hanno messo a lavorare come guardia di un certo \emph{Duca di Shien}.
    Ma non mi convince molto.'' ``Oh. Da quando fai questo tipo di lavori?
    Comnque conosco il tizio. Si chiama Philip, sì?'' ``Sì, esatto.''
    ``Oddio. \'E quello che mi ha commissionato il lavoro che ho portato a
    termine il mese scorso. Viscido bastardo. Ti ha causato problemi? Potrei
    vedere se riesco a rovinarlo.'' Seysill stava già pregustandosi la
    scena. Philip che, a causa di uno scandalo perdeva il suo titolo,
    finiva per strada senza un soldo e falliva miseramente nel suo sogno di
    grandezza mentre lui lo osservava da una camera
    d'albergo a cinque stelle. Sarebbe stato magnifico. ``Nono, lascia
    stare. A quanto pare si è messo nelle peste da solo ed adesso dovrà
    trovare un altro modo per guadagnare denaro. Se prima era
    insopportabile adesso lo sarà ancora di più.'' ``Maledizione. Una bella
    gatta da pelare, eh? Senti, io fra due mesi me ne andrò da questo
    posto. Vuoi venire con me? Vado a Shinra. Spiagge bianche, centrali
    eoliche, estate quasi tutto l'anno\dots{}'' ``Hum. Vorrei proprio
    Seysill, ma purtroppo questo è quello in cui credo.'' ``Un aspirante
    dittatore al servizio di un culto religioso? Sicura? Diamine, neppure
    nei miei lavori peggiori sono arrivato a livelli così infimi nella
    scala sociale.'' ``No, scemo. \'E l'obiettivo che si è prefissato il
    Culto. Arrivare alla distribuzione in tutto il continente.'' ``Ma
    scusa, vuoi dire che lavoreresti per un Culto che è una degenerazione
    della tua religione? \'E un po' una contraddizione.'' ``Ma è l'unico
    modo.'' ``Capito. Ne riparleremo più avanti. Per il resto, dimmi, come
    va la vita?''

    Da lì in poi i due discussero del più e del meno, finchè non arrivò ora
    di andare a dormire ed i due si separarono. Seysill doveva ammettere
    che non gli sarebbe dispiaciuto passare la serata con la signorina
    Durga ma non era proprio il momento, pareva. Peccato. Sarebbe stato per
    un'altra volta.

  \section{Rollers - New Metredeth - Luglio 2635}

    Hax venne svegliato dai raggi del sole che entravano nella sua stanza.
    Si rivoltò nel letto per leggere l'ora dall'orologio per scoprire che
    erano le nove. Doveva uscire dal letto ed iniziare a fare cose. Odiava
    buttare via le giornate a letto. Cosa che Lissa, invece, preferiva tra
    tutto. Saltò fuori dal letto per scoprire che l'alcool dell'altra sera
    aveva avuto il solito effetto: sete, mal di testa e ipersensibilità
    a luce e suoni.

    ``Gh.'' grugnò, stringendo gli occhi per fermare la luce, grattandosi
    la testa ``Oddio. Ho la testa che scoppia.'' continuò, mormorando per
    non farsi male per la voce troppo alta. Era incredibile come, la
    mattina dopo una sbronza, sentiva mal di testa anche se alzava la voce
    lui stesso. Sarebbe finita entro pranzo, ma sapeva che quelle tre,
    quattro ore erano un calvario. O come avere sempre attivo il cambio di
    paradigma per la programmazione. Punti di vista. Barcollò fino al bagno
    e lì, dopo essersi guardato allo specchio per controllare esattamente
    quante occhiaie avesse, si buttò in faccia dell'acqua gelida per
    riprendersi. Ottimo. Già si sentiva più sveglio di prima. Il secondo
    passo sarebbe stato scendere le scale, molto lentamente, arrivare in
    cucina e prepararsi un Bloody Mary. Niente di meglio per il post
    sbronza. Quello avrebbe fermato il mal di testa ad un livello
    accettabile. Prese il coraggio e partì.

    \subsection{Due ore prima\dots{}}

    The Fixer quella mattina si era alzato presto al solito. Non gli
    serviva dormire molto. Aveva vissuto tutta la sua vita a dormire meno
    di otto ore per notte, perciò il suo fisico si era abituato alla
    mancanza di sonno. Era vero che ogni tanto aveva bisogno di recuperare
    le ore di sonno perse, ma era una volta al mese e dormiva semplicemente
    dieci ore ed aveva fatto. Quella mattina, al solito, aveva preparato
    il materiali per i suoi prossimi esperimenti sulla produzione di
    metalli con strati di compatibilità organica. Cose avanzate. Non sapeva
    neppure lui perchè ci stesse lavorando sopra, però parevano
    interessanti. Se fosse arrivato ad un qualche risultato sarebbe
    significato poter creare delle protesi in metallo che non venivano
    rigettate dal corpo. Il solito problema dato dal fatto che il corpo
    umano tende ad eliminare tutto ciò che non è stato creato da lui.
    Avrebbe dovuto chiedere ad Elythia. Il fatto era che la ragazza (era
    abbastanza giovane per lui per chiamarla ragazza) se n'era uscita
    presto a sua volta per andare a confermare il contratto di cui avevano
    parlato ieri sera. Appena tornato a casa dal suo giro per comprare il
    giornale per lui, pane per Jam e latte per Lissa, si sedette
    alla tavola della cucina, per poi, ancora prima di mettersi a leggere
    il quotidiano, rialzarsi, mettersi in tasca le chiavi di casa che aveva buttato
    sul tavolo, prendere una borsa in tela ed uscire di nuovo di casa.

    Seguì la strada fino al suo emporio di fiducia. Una volta dentro prese
    una bottiglia di Vodka, del sedano, carote, una bottiglia
    di succo di pomodoro ed altre verdure. Passando davanti al ripiano delle spezie si
    chiese se c'era ancora Tabasco a casa ma poi, per non sbagliarsi, ne
    mise una bottiglietta dentro nella borsa, assieme ad un tubetto di
    pasta di Wasabi. Prese anche un pezzo di carne di manzo, poi andò a pagare.

    Una volta a casa mise la Vodka ed il succo di pomodoro dentro nel
    refrigeratore ed iniziò a preparare un consommè di carne per poi
    mettersi a fare colazione, leggendo il suo giornale. Dopo un'ora
    passata tra la colazione, il giornale ed i vari passaggi per la
    preparazione dell'elaborata zuppa, questa fu pronta. Cucinare era un
    po' come il suo lavoro. Era scienza e creatività. Quando scoprivi o
    creavi una ricetta interessante capivi di aver le competenze necessarie
    per arrivare a livelli ancora più alti. Una ricerca senza fine della
    perfezione, della ricetta perfetta. Come per la chimica applicata alla
    metallurgia. Il suo campo era, se doveva essere sincero, un casino
    impressionante. Lavorare con composti inorganici, leghe metalliche e
    chimica molecolare contemporaneamente portava alla necessità di
    considerare un numero esponenzialmente alto di interazioni che,
    normalmente, quando si studiano quei tre campi a coppie, si potevano
    ignorare senza controindicazioni.

    Finito il tutto prese il consommè e lo mise da parte. Verso le nove,
    quando ebbe finito di leggere la parte più interessante del giornale,
    si alzò, recuperò il suo mixer ed iniziò ad infilarci dentro dei
    cubetti di ghiaccio, della Vodka, parte del consommè, del succo di
    pomodoro, del tabasco, pepe, sale, una spruzzata di limone e del
    Wasabi. Diede una robusta shakerata e versò tutto in un bicchiere
    cilindrico abbastanza grande. D'improvviso si sentirono dei rumori di
    qualcosa che rotola giù dalle scale provenire dal corridoio, seguiti da
    un ``Ma che cazzo!''. Era Hax, nel suo solito stato post-sbronza.
    ``Opporcaputtana.'' si sentì mormorare da fuori della cucina, mentre il ragazzo
    si rimetteva in piedi aiutandosi col muro. Fixer finì di guarnire il
    drink che aveva preparato con un gambo di sedano, un pezzo di carota,
    delle olive ed un cetriolo sott'aceto tagliato a fette. Prese il
    bicchiere e lo appoggiò sul tavolo davanti ad una sedia vuota, per poi
    sedersi sulla sua sedia e continuare con la lettura del quotidiano. Il
    ragazzo avrebbe avuto bisogno di quel drink. Se lo avesse lasciato fare
    si sarebbe versato vodka, succo di pomodoro e sale in un bicchiere. Un
    crimine. Fare cocktail, apppunto, è un'arte, non va sminuito. E
    svegliarsi con un Bloody Mary ben fatto era il massimo dopo una
    sbornia.

    Hax barcollò fino alla cucina, mentre in testa vorticavano insulti a
    tutto spiano che non voleva urlare perchè sapeva gli avrebbero fatto
    esplodere il cervello. ``Occazzo. Che male.'' disse a abssa voce,
    massaggiandosi la testa. Sperava di non aver svegliato gli altri col
    casino che aveva fatto cadendo dalle scale. Doveva aver mancato uno o
    due scalini. Entrò in cucina per vedere un bicchiere, sul quale si era
    formata della condensa, con dentro quello che sembrava un Bloody Mary.
    Solo che non era il solito drink che si preparava normalmente. Era più.
    Un'opera d'arte in onore del Bloody Mary. Alzò la testa per vedere chi
    avesse preparato il drink per vedere The Fixer che stava leggendo il
    giornale. Hax spostò la sedia per potersi seder sopra, poi la avvicinò
    al tavolo. Allungò lentamente la mano verso il bicchiere e poi, una
    volta preso, si girò verso The Fixer e gli fece, quasi commosso
    ``Grazie.'' L'uomo abbassò il giornale, si girò e, guardandolo, gli
    rispose ``Il massimo post sbronza.'' e poi tornò a leggere.

    Iniziò a bere e poi, dopo aver assaporato a fondo il drink, chiese a
    Fixer ``Sai dove sono le altre?'' ``Hum. Elly se n'è partita presto per
    andare a farsi un giro, oltre a dire alla gilda che accettiamo il
    lavoro. Le altre non so. Penso che stiano ancora dormendo assieme nella
    stanza di Jam.'' ``Hum. Capito. Allora mi sa che andrò fuori a lavorare
    un po'. Non vorrei svegliare quelle due.'' e continuò a bere il suo
    Bloody Mary. Sentiva già il corpo che, grazie al drink, reagiva
    all'alcool presente nelle vene. Fantastico. ``Ah, a proposito, Hax.
    Tieni.'' fece Fixer, passandogli un pezzo di giornale ``Che è?'' ``C'è
    un articolo su degli scontri a Dyniar.'' ``Eh? Fammi leggere\dots{}''
    Hax prese il giornale. Dyniar era uno stato vicino a Jokula, dove era
    nato lui. In effetti aveva sempre avuto problemi con il proprio
    governo: prima o poi sarebbe successo di sicuro che una parte della
    popolazione si sarebbe ribellata. Poi notò la foto che avevano messo
    nell'articolo rappresentava un uomo con una bandana sul volto che gli
    copriva la parte inferiore, che stava per lanciare quella che sembrava
    una bottiglia con un fazzoletto incendiato dentro.

    ``Fixer,'' fece, girando il pezzo di giornale verso il chimico,
    indicando la foto ``scusa. Tu sai cos'è questa?'' ``Hm\dots{} Fammi
    vedere un po'\dots{}'' Fixer fissò attentamente l'immagine, piegandosi
    in avanti per vederla meglio poi, dopo essersi di nuovo appoggiato allo
    schienale della sedia, fece ``Oh, quello! \'E un tipo di bomba fatta in
    casa utilizzata nella guerriglia urbana, normalmente. Fondamentalmente
    riempi una bottiglia di una qualche mistura esplosiva, sia essa
    propellente, bevande ad alto grado alcoolico, composti chimici, ecc. E
    poi ci attacchi qualcosa che possa incendiare la mistura e voilà''
    Fixer mimò un'esplosione con le mani ``hai la tua granata incendiaria a
    costo quasi zero.'' Hax aveva gli occhi che brillavano. Per uno come
    lui che utilizzava armi improvvisate quasi sempre quella era un'idea
    geniale. Spesso si chiedeva come fare per fare danni ad area senza
    utilizzare un algooritmo che rischiava inesorabilmente di mandare un
    ottimo piano al vento. ``Fantastico!'' fece, buttando giù quello che
    rimaneva del Bloody Mary. Si alzò in piedi, lanciato da questa nuova
    scoperta, e poi, guardando Fixer, gli disse ``Grazie per il drink
    Fixer! Ora di fare scienza.'' e poi andò di sopra a cambiarsi.

    ``Non bere un Bloody Mary come questo tutto d'un fiato.'' mugugnò
    Fixer fissando il bicchiere vuoto, per poi continuare a leggere il
    giornale, soddisfatto che fosse stato apprezzato.

  \section{Gabriel - Dera - Luglio 2635}

    Era da un po' che Gabriel stava aspettando una risposta al suo
    contratto, era per quello che, praticamente ogni giorno, passava per la
    sede della gilda a Dera a vedere se avevano trovato qualcuno che
    l'avesse accettato. Aveva ricevuto risposte da tutte le parti di
    Xeresia, in realtà, ma contenevano commenti tipo che la paga era troppo bassa per
    il loro tempo, che non avrebbero accettato un lavoro così di basso
    livello ed altre cose del genere.

    Stava per perdere la speranza quando, un bel giorno, il responsabile
    locale, appena lo vide, gli disse ``Oh, Gabriel! Ho ottime notizie.
    Qualcuno ha risposto in maniera positiva alla tua richiesta.''
    ``Davvero?'' ``Sì, eccoti la lettera che ci hanno inviato.'' e gli
    consegnò il documento.

    Era una lettera inviata da New Metredeth, dall'altra parte del mondo.
    Gliel'aveva inviata un gruppo chiamato ``The Rollers''. A quanto pareva
    erano cinque scienziati che avevano accettato il lavoro una settimana
    prima. Non riusciva a crederci. Non solo qualcuno aveva accettato il
    lavoro, ma era anche un gruppo di gente competente che si prendeva la
    briga di farsi un viaggio di quasi un mese per andare a completare il
    lavoro. Che persone magnifiche.

  \section{Rollers - Il Mare Infinito - Agosto 2635}
    
    Elythia era sul ponte principale, in bikini, che stava prendendosi il
    sole. Era da moltissimo che voleva farsi una vacanza del genere.
    Prendere il sole, bere cocktail, fare festa la sera il tutto pagato da
    qualcun'altro. Non doveva correre per andare a raccattare contratti,
    non rischiava che qualcuno degli altri venisse ferito in maniera grave,
    non doveva preoccuparsi che Hax venisse arrestato. Poteva prendersi il
    suo tempo per ripassare le forme, per discutere con EMR e le altre.
    Queste cose che non riusciva a fare a New Metredeth, insomma.

    Prese un sorso del suo Sex On The Beach e poi tornò a godersi il caldo
    del sole d'agosto, pensando a cosa avrebbe dovuto mettersi quella sera
    per andare al ballo che si sarebbe tenuto sulla nave, la sua unica
    preoccupazione, in pratica. Sorrise al solo pensiero.

    Jam e Lissa stavano godendosi la loro nuotata giornaliera dentro la
    piscina di acqua dolce che quella nave poteva permettersi, mentre The
    Fixer era nella palestra allestita appositamente per i viaggiatori che
    volevano tenersi in forma che faceva pesi.

    Hax, invece, era in T-Shirt che stava girando all'interno della nave
    per tentare di comprendere il funzionamento di tutto l'apparato. Era
    una settimana che erano su quella nave. Fortunatamente entro cinque
    giorni avrebbero iniziato a fare tappa tra le varie isole
    dell'arcipelago di Pangaia, dove c'era anche il Paese dove era nata
    Jam: Sanpan. Quello avrebbe voluto dire scendere a terra
    e, finalmente, vedere posti nuovi. Non è che ad Hax dispiacesse non
    doversi preoccupare di nulla e riposarsi per un po', però stava
    iniziando ad averne le palle un po' piene. Mare in qualunque
    direzione, sempre la solita nave, attività ricreative per decerebrati.
    Non faceva per lui. Era anche vero che quella sera ci sarebbe stata una
    serata un po' ricercata con signorine ben vestite e cocktail fatti da
    gente che ci sapeva fare e non dei baristi part-time. Gli dava fastido
    doversi vestire bene, però lo spettacolo che si sarebbe previsto ne
    sarebbe valso il sacrificio. Finì di seguire una delle linee per il
    trasporto energetico dai motori al generatore e poi decise di salire al
    ponte superiore per farsi un bagno.

    La giornata passò così, tra alcuni momenti di tedio, fino a che non
    scese la sera e fu ora di andare alla festa. Il cielo sembrava
    promettere incazzature non indifferenti, però la nave era resistente e
    l'\emph{Happening} si sarebbe tenuto all'interno, perciò non c'era da
    preoccuparsi. Hax finì di prepararsi, mettendosi addosso quello che
    aveva di più elegante, che consisteva in un paio di pantaloni neri,
    camicia viola, cravatta nera e gilet da sopravvivenza che, togliendo le
    placche anti perforazione, le tasche aggiuntive e le altre cose poco
    ortodosse, poteva sembrare un abito solo un po' sportivo.

    S'incontrò con gli altri, che erano evidentemente meglio preparati per
    una serata simile. Elythia aveva un lungo qipao nero con gli inserti in
    oro con uno spacco che lasciava intravedere il pizzo alla fine delle
    autoreggenti, Lissa aveva un tailleur grigio, The Fixer aveva un
    completo viola, un cappello viola a tesa larga con una piuma, degli
    anelli in platino ed un bastone con in fondo un pomello simile alla
    palla otto da biliardo. Hax dovette fermarsi un attimo per processare
    quest'ultima visione. Era fin troppo clichè per essere vera. Aveva
    dello stile e del fegato, doveva ammetterlo. Jam, infine, era vestita
    con uno yukata corto.

    La serata passò tra cocktails, musica lounge e discorsi vari.
    Ovviamente i più strani erano quelli che gli vennero fatti da The
    Fixer.

    Fu verso la fine della serata che l'uomo iniziò un discorso anche
    abbastanza interessante. ``Ti ricordi Hax di quella missione durante la
    quale abbiamo avuto a che fare con i morti che tornavano a camminare?''
    ``Eh. Gli zombie? Certo.'' ``Hai presente che, per fermarli, ci abbiamo
    messo un casino di tempo?''

    \subsection{Un anno prima}

    ``Lavoro di merda.'' fece, Hax grondante un liquido verdastro, con un
    maglio da lavoro insanguinato in mano. Scavalcò alcuni cadaveri
    verdognoli. Qualche stronzo di biologo aveva ben pensato di testare
    qualche nuova teoria per la quale si poteva accelerare la rigenerazione
    dei tessuti ed altre cose. E per testarne gli effetti a lungo termine
    che fai? Lo testatii su tutti i cadaveri di un cimitero. E che succede
    quando fai questo? Che non funziona niente?

    No, OVVIAMENTE! Che qualche gruppo di idioti deve andare a fermare un
    invasione di zombie. Gli altri erano allenati nell'uso di qualche arma,
    mentre lui aveva dovuto utilizzare tutto quello che gli era passato
    per le mani. Solo che non morivano tanto facilmente. Aveva dovuto
    ridurre il corpo in poltiglia, in pratica, per fermarli. Ma che
    diavolo.

    Mentre stava provando a togliersi la poltiglia di dosso gli piombarono
    addosso altri due o tre zombie, buttandolo a terra. ``MA
    EPPORCAPUTTANA!''

    \subsection{Di nuovo sulla nave}

    ``Come fosse ieri.'' fece Hax sconsolato, lasciando cadere le braccia
    ``Eh, allora ho fatto delle ricerche, ho chiesto ad Elythia, sentito
    altri esperti e sono arrivato ad una conclusione.'' ``Ah, sì? Bene,
    ottimo. Anche perchè non è che non ci siano creature risvegliate in
    giro per Xeresia. Sentiamo.'' ``Eh, allora,'' The Fixer non riuscì a
    continuare perchè fu fermato dallo sbattere di una delle porte della
    sala ed un ragazzo che urlò ``Aiuto! \'E caduta una ragazza in mare!'' Ci fu una sopita
    agitazione e la musica si fermò per un attimo. Subito venirono mandati
    dei marinai a controllare la situazione. La gente aveva notato che c'erano tutti quelli che
    conoscevano, a quanto pare, oltre ad essere confidenti che il personale
    della nave stesse già provvedendo. Qualcuno, interessato, uscì dal
    salone. Hax, invece, reagì in maniera
    differente. Non riusciva a calmarsi in situazioni simili, era troppo
    paranoico. Non gli ci volle molto per notare che sia Lissa che Elythia
    si stavano guardando in giro ansiosamente. Stava ancora tentando di
    capire cosa ci fosse che non andava che Lissa arrivò da lui, gli
    strinse le braccia e gli fece ``Hax! Hai visto Jam?'' ``N-No.'' rispose
    lui, scombussolato ``Non è andata ai servizi?'' ``Prima si è
    allontanata dicendo che un ragazzo l'aveva chiamata per parlarle. Non
    vorrei che le fosse capitato qualcosa di brutto.''

    Ci fu come un fulmine che colpì Hax. ``Dannazione.'' fece, liberandosi
    dalla stretta di Lissa, per poi correre fuori dalla stanza a sua volta. Il
    tizio poteva averla buttata giù. Si trovava nel corridoio di quel piano
    della nave, corse verso le scale. Oppure poteva aver pensato di fare il
    galante e di farle fare un giro sul ponte principale. Salì le scale
    fino al piano della sua stanza. Aprì la prota, sbattendola contro il
    muro, e recuperò lo zaino, mettendoselo in spalla. Un'idea idiota, ovviamente, visto il tempo.
    Bastava una qualunque onda di una certa magnitudine per buttare a terra
    qualcuno. Ancora scale. E, ovviamente, cadere sul ponte principale di
    una nave significava rischiare di cadere in mare. Un mucchio di gente
    davanti ad un boccaporto, con dei marinai che li tenevano dentro.
    Oppure poteva non essere Jam. Ma non importava. Una vita era una vita,
    dannazione. Non poteva andarsene senza fare nulla. I marinai
    intimidivano alla gente di stare dentro perchè fuori era troppo
    pericoloso.

    Scavalcò la gente di fronte a se, spintonò a terra uno dei marinai e
    corse fuori. C'era una tempesa. Il vento e la pioggia stavano
    buttandolo a terra. Cambiò paradigma, iniziò a guardarsi intorno.
    Ognuno aveva una traccia differente. Se fosse riuscito a vedere
    qualcosa di differente dalle energie del mare avrebbe trovato la
    ``ragazza in mare'' come aveva detto il tizio. Il suo campo visivo,
    però, era saturato dalla quantità di energie in gioco. Fulmini,
    cavalloni, pioggia, vento, qualunque cosa contribuiva ad inquinare il
    suo campo visivo. Ad un certo punto una serie di dati con un colore
    differente attirò il suo sguardo. Aguzzò la vista in quella direzione
    ed, in effetti, vide delle tracce di una persona in mare. Nessuno era
    riuscito a fare niente, gli stronzi. Corse avanti, sfondò con un calcio
    una delle cassette di sicurezza e prese una di quelle borse
    impermeabili per la sopravvivenza. Raggiunse quindi una ciambella di
    salvataggio e poi, senza lasciarsi il tempo per ripensarci, si buttò a
    sua volta in mare.

    Che cazzo. Lui sapeva nuotare dannatamente bene.

  \section{Rollers senza Hax e Jam - Il Mare Infinito - Agosto 2635}

    Il giorno dopo la tempesta Elythia e gli altri erano ancora scossi per
    quello che era successo. La sera stessa Elythia passò due ore a capire
    che cosa fosse successo, per venire a scoprire che uno scemo, pensando
    di fare colpo su Jam, l'aveva invitata a farsi un giro e, ignorando
    quanto fosse pericolosa una tempesta in mezzo al mare, aveva voluto
    fare il grosso invitandola sul ponte principale, cacciando anche frasi
    tipo ``Non ti preoccupare ci sono quì io'' e così via.

    La tempesta ed un ondata hanno fatto il resto. Jam ha perso
    l'equilibrio ed è caduta in mare. Per quanto riguarda Hax, invece,
    Lissa gli aveva chiesto se lui avesse visto la ragazza e questo, senza
    neppure dire una parola, era corso fuori dal salone, aveva preso il suo
    zaino e poi si era buttato in mare. Non c'era modo che sapesse che
    quella era Jam. Ma, in effetti, si stava parlando di Hax, paladino
    della giustizia. E delle stupidaggini.

    Quello che preoccupava più il gruppo non era tanto se il ragazzo fosse
    riuscito a raggiungere Jam, ma quanto dove sarebbero potuti finire.
    Hax aveva le competenze per sopravvivere in isole ricoperte di giungla
    senza troppi problemi, il fatto era su quale isola sarebbero arrivati,
    se fossero arrivati e come avrebbero fatto gli altri dei Rollers a
    trovarli.

    Il vero problema, però, era un altro. Per altri cinque giorni la nave
    non sarebbe attraccata in quanto non c'erano porti che potessero
    supportare il suo arrivo. Il capitano aveva detto a The Fixer che, se
    avesse potuto, sarebbe attraccato alla prima isoletta disponibile, però
    il rischio di distruggere la barca in quella situazione era troppo alto
    ed andare a salvare della gente per poi aspettare di venir salvati a
    loro volta era troppo stupido. Quindi quello che potevano fare era sperare.

  \section{Hax e Jam - Isola Sconosciuta - Agosto 2635}

    Hax si svegliò con tutto il corpo che faceva male. Era riuscito a
    raggiungere Jam, farla aggrappare alla ciambella e quindi nuotare a più
    non posso verso una direzione verso la quale aveva visto arrivare dei
    dati che parevano interessanti al fine dela loro sopravvivenza. Diciamo
    che, però, la maggior parte del lavoro l'aveva fatta la tempesta,
    facendoli arenare sulla spiaggia bianchissima di un'isola di cui non
    conosceva nulla.

    Si girò in posizione supina e fece, socchiudendo gli occhi ``Che festa
    del cazzo.''

\cleardoublepage{}
