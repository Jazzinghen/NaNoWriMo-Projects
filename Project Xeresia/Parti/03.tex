\chapter{Voi e i vostri paradisi tropicali}

  \section{Hax e Jam - Isola Sconosciuta - Luglio 2635}

    A Hax ci volle un po' prima d'iniziare a considerare la propria
    situazione. Passò almeno cinque minuti a vedere se tutti i muscoli
    funzionavano ancora. Chiuse entrambe le mani a pugno due o tre volte.
    Poi mosse i piedi in circoli. Andava tutto, fantastico. Si mise seduto
    ed iniziò a tastarsi. Non gli pareva ci fossero ferite anche se, molto
    probabilmente, con tutta la sabbia ed il sale con cui era venuto a
    contatto, se fosse stato ferito avrebbe sofferto come un cane. Si
    scrocchiò i muscoli del collo facendo fare alla testa delle rotazioni
    in senso orario ed anti orario.

    Appena fu confidente di potersi tenere in piedi si alzò, barcollando un
    attimo. Fantastico. Se non fosse stato per la fame, la sete ed il fatto
    che si trovava su di un isola sconosciuta, stava bene. Si diede
    un'occhiata in giro. Si trovava su quella che le agenzie di viaggio
    vendevano come ``isole paradisiache'': spiaggia bianchissima, mare
    azzurro, jungla. A quanto riusciva a capire doveva essere un atollo o
    un arcipelago e lui si trovava su una delle isole secondarie. Riusciva
    ad intravedere, abbastanza lontano dalla sua posizione, un'isola molto
    molto più grande con una montagna, addirittura.

    Iniziò subito a fare piani su come arrivare a quell'isola quando, però,
    due cose balzarono nella sua mente. ``Lo zaino!'' fece, guardandosi in
    giro agitato. Poi notò Jam con i vestiti sgualciti a terra, ancora con
    gli occhi chiusi ``Occazzo, Jam!'' corse verso la ragazza e si chinò su
    di essa. ``Allora.'' iniziò, tentando di ricordarsi i passaggi da
    seguire in casi di emergenza medica che gli aveva insegnato Elythia
    ``Battito cardiaco.'' appoggiò indice e medio sulla giugulare della
    ragazza e sentì che c'era battito cardiaco ``Ottimo. Respiro.''
    avvicinò l'oreccio alla bocca della ragazza, la quale stava respirando.
    Non era sicuro si facessero proprio in quest'ordine, però per ora era
    sicuro che fosse ancora viva. Adesso doveva provare a svegliarla, per
    vedere se poteva spostarla all'ombra. Non voleva rischiare danni
    permanenti. ``Jam!'' urlò, stringendola alle braccia ``Oi! Svegliati!''

    La ragazza mugugnò qualcosa, stringe gli occhi e si girò sul lato.
    ``Eh?'' fece il ragazzo, guardandola stranito. Avviccinò l'orecchio
    alla bocca della ragazza per sentire cosa stesse dicendo. ``Hax,
    lasciami dormire, è vacanza.'' disse, di nuovo, per poi fare un verso
    tipo ``Mnya'' e tornò, appunto, a dormire.

    Hax si alzò in piedi lasciando cadere le braccia, abbattuto. Era
    possibile che fosse così pippa lui? Jam stava dormendo senza problemi,
    mentre lui aveva tutto il corpo indolenzito. Probabilmente doveva
    allenarsi di più. Tirò un sospiro di sollievo e sorrise guardando la
    ragazza dormire beatamente. La sollevò e la portò all'ombra di una
    delle palme al limitare della spiaggia, così non avrebbe preso
    un'insolazione. Adesso doveva preoccuparsi di un'altra cosa. Lo zaino.
    Il suo FOTTUTISSIMO zaino. E la borsa di sopravvivenza. Gli sarebbero
    serviti se volevano uscirne vivi.

    Passò un'ora ad andare avanti ed indietro per la spiagga a controllare
    dove fosse finito il suo equipaggiamento. Trovò il kit di sopravvivenza
    mezzo sepolto dalla sabbia, mentre lo zaino era attaccato su una palma.
    Doveva essere stato sbalzato là dai cavalloni. ``Ma che diavolo?''
    fece, guardando lo zaino. Appoggiò sotto la palma il kit ed iniziò a
    scalarla dal lato convesso. Gli ci vollero più di dieci minuti per
    arrivare in cima. Non era molto preparato allo scalare alberi,
    soprattutto se senza rami intermedi. Una volta in cima allungò il
    braccio per prendere lo zaino, attaccato per una delle fasce alla parte
    finale della palma. Scivolò dal tronco, in quanto provò ad allungarsi
    ulteriormente per prendere lo zaino, con l'unico effetto di cadere di
    schiena sulla sabbia. Espulse quasi tutta l'aria che aveva nei polmoni,
    poi rimase dei secondi a listare mentalmente gran parte delle
    imprecazioni che conosceva. Arrivato alla M di ``Maitresse di Alto
    Borgo'' dovette ricominciare quando gli caddero addosso due noci di
    cocco e lo zaino.

    Finita la lista si mise seduto ed aprì lo zaino per controllare che
    tutto fosse a posto. Pareva non ci fossero problemi. Anche il piede di
    porco che aveva legato su uno dei due lati della borsa era ancora al
    suo posto. Chiuse lo zaino e, dopo essersi alzato in piedi, se lo mise
    in spalla. Assicurò le fasce ai fianchi, raccolse il kit di
    sopravvivenza e quindi si reincamminò verso dove aveva lasciato Jam.

    Dopo una decina di minuti di camminata sulla spiaggia arrivò dalla
    ragazza, la quale si era svegliata da poco. ``Ah.'' fece lei, dopo
    essersi stirata il volto con la mano destra ``Hax. Che cavolo?'' ``Eh,
    cara. Ben alzata.'' ``Siamo arrivati all'arcipelago, finalmente?''
    ``Jam, non ti ricordi nulla di ieri sera?'' ``Non proprio. Però ho
    sognato di essere finita in mare e te sei venuto a salvarmi\dots{}
    Aspetta.'' la ragazza diede un'occhiata in giro, per poi guardare Hax
    con sguardo stranito ``Mi vuoi dire che non era un sogno?''
    ``Evidentemente.'' fece lui, per poi prendersi un sorso di acqua dalla
    sua borraccia in alluminio ancora piena, attaccata allo zaino. ``Tieni,'' fece,
    lanciandola a Jam, seduta a terra a gambe incrociate ``bevi un po',
    avrai sete.'' Jam prese un sorso d'acqua e poi richiuse la borraccia.

    ``Che è successo, alla fine, ieri sera?'' chiese il ragazzo. ``Beh,
    hem.'' lei era evidentemente imbarazzata ``In pratica un ragazzo mi
    aveva chiesto se volevamo andare a parlare da qualche parte di più
    calmo. Era simpatico, perciò l'ho seguito.'' Hax aveva la sua solita
    faccia da  \emph{L'avevo detto} ``Lui mi ha portato verso il ponte
    superiore dicendomi che voleva andare a prendere un po' di aria, ma
    quando gli ho fatto presente che di fuori c'era una tempesta mi ha
    rassicurando dicendo che mi avrebbe protetto lui. E così, da vera
    stupida l'ho seguito.'' \emph{WAT} pensò il ragazzo, ascoltando la
    storia come se l'avesse già sentita ``Poi, però, un'ondata mi ha
    trascinata in mare e il resto\dots{} Lo conosci già, no?'' poi,
    guardandolo imbarazzata fece ``Grazie.'' ``Di che?'' rispose lui ``Beh,
    sei venuto a salvarmi sapendo che non saresti riuscito a risalire sulla
    nave.'' ``Figurati. Cosa facevo? Ti lasciavo là? So che non sai nuotare
    proprio bene.'' ``Beh, grazie comunque, Hax.'' Hax non era bravo a fare
    l'uomo duro.

    ``Dove siamo, dunque?'' continuò lei ``Eh. Boh? Una di quelle isole che
    vedi nelle cartoline, immagino.'' rispose lui, ironico ``Oh, beh,
    allora vorrà dire che arriveranno gli altri tra poco, no?'' ``Certo.
    Sempre che prima non moriamo disidratati.'' rispose lui. Notò che la
    ragazza era sull'orlo di piangere ``Nono, aspetta.'' disse Hax per
    mettere una pezza a quello che aveva detto prima, non considerando di
    non star parlando con Lissa, agitanto le mani di fronte a se ``Quello
    che intendevo è che, sicuramente verranno a prenderci, però se vogliamo
    sopravvivere fino a quando arriveranno dobbiamo andare da qualche parte
    dove possiamo recuperare dell'acqua dolce.'' Jam, sentite queste
    parole, inspirò profondamente per trattenere le lacrime ``Capito.''
    poi, dopo essersi alzata in piedi ed aver dato un'occhiata intorno,
    fece ``Però, questo posto mi ricorda un sacco dove ho vissuto io.''
    ``Quindi potremmo essere vicini alla civiltà?'' ``Beh, potremmo. Ma da
    queste parti le isole si somigliano moltissimo. Dovrebbero esserci
    delle barche, comunque. Ne hai viste?'' ``No, sfortunatamente. Però ho
    notato che c'è un'isola molto grande in quella direzione. Ha
    addirittura delle formazioni montane, quindi potrebbe non essere
    semplicemente un'isola da atollo. Potrebbe avere delle sorgenti
    d'acqua. C'è un problema, ovviamente. Dovremo andarci a nuoto passando
    da un'isola all'altra, ce la fai?'' ``Scherzi?'' Non scherzava. Lui era
    uno straccio. Pensare di farsi una cosa come tre isole a piedi più
    tutte le varie parti a nuoto lo stava spaventando non poco.
    Probabilmente Jam non era stanca viste le ore di allenamento alle quali
    si sottoponeva ogni giorno, mentre lui era una pippa. Ma come
    diceva The Fixer? \emph{Vai e fallo, come un vero uomo}

    ``Già.'' rispose Hax, iniziando a camminare lungo la spiaggia, in
    direzione dell'isola maggiore. ``Ottimo!'' rispose la ragazza
    ``Comunque, se quì è come da me, probabilmente tra le isole il fondo
    del mare è così basso che si può camminare per la maggior parte del
    tragitto.'' ``Davvero?'' Hax si sentì improvvisamente sollevato ''Sì,
    anche se probabilmente è comunque più veloce nuotare.'' ``Oh. Beh,
    vedremo. Vuoi qualcosa da mangiare?'' ``Magari. Hai qualcosa?''
    ``Dentro il Kit di sopravvivenza dovrebbero esserci delle razioni.''
    rispose Hax, aprendo la borsa del kit ``Bleach. Teniamocele per quando
    non abbiamo alternativa, cittadino.'' lo schernì Jam e quindi,
    indicando una palma, continuò ``Per adessoo godiamoci i frutti di
    questa terra.'' e corse verso l'albero. Lo scalò con una facilità
    disarmante e quindi, usando dei calci, fece cadere quattro noci di
    cocco, quindi saltò a terra senza farsi nulla. ``Ta-dah!'' intonò,
    indicando i frutti. ``Hum. Ok.'' rispose Hax, avvicinandosi e
    prendendone una. Dopo aver mangiato ripresero il viaggio. Ci vollero
    una giornata e mezza per arrivare all'isola centrale. Fortunatamente
    all'interno del kit di sopravvivenza c'era un telo impermeabile con
    quale Hax riuscì a fare un riparo provvisorio. Diede la sua coperta
    termica a Jam mentre lui dormì senza nessuna copertura, da vero
    gentleman.

    Arrivati all'isola, però, le scorte d'acqua finirono. Era stato anche
    abbastanza facile fino a quel momento perchè utilizzarono il latte di
    cocco per andare avanti, però adesso avrebbero dovuto raggiungere una
    qualche fonte o un qualche fiume per recuperare i liquidi che
    servivano. Tutta questa vegetazione non poteva andare avanti senza un
    qualche apporto di acqua dolce.
    
    Utilizzando il cambio di paradigma di Hax e delle funzioni applicate al
    suo corpo da Jam riuscirono a sentire lo scorrere dell'acqua in
    distanza. Fu seguendolo che, mentre camminavano nella jungla
    dell'entroterra di quell'isola, caddero all'interno di un crepaccio
    alto una decina di metri il quale nascondeva un laghetto di acqua
    dolce, alimentato da un torrente che formava una cascata su uno dei
    lati, per poi andare a scaricarsi lungo un secondo corso che usciva
    dalla parte opposta. Appena si accorsero di dove si trovavano
    d'acqua ci bevvero a piene mani e poi, una volta tolti i vestiti, con
    un certo imbarazzo da parte d'enrambe, si lavarono.

    ``Senti, Hax.'' fece Jam, mentre erano seduti dentro l'acuqa a
    riposare ``Secondo te ce la faremo ad andarcene, seriamente?'' ``Beh.''
    iniziò il ragazzo, schoccandosi il collo, per poi tornare a sedere in
    maniera rilassata ``Sarà dura, non posso nasconderlo. Ma lo sai come la
    penso, no? Si può sempre fare tutto. Soprattutto se siamo in due. Ho passato la maggior parte della mia gioventù a
    vivere d'espedienti. Sopravvivere in un posto del genere per me è veramente
    come essere in vacanza.'' e si girò verso di lei ``E poi ho te, no?''
    La ragazza non riuscì a trattenere l'imbarazzo ``Ah, eh?'' fece, non
    riuscendo a mettere le parole una dietro l'altra ``Ah, no, intendevo
    dire che con te che conosci così bene la natura da queste parti e che
    sai combattere così bene allora non c'è di cui preoccuparsi.'' ``Oh.''
    fece lei, un po' disappunta ``Certo. Il problema più grande da
    risolvere sarà come faremo ad ANDARCENE da quì. Senza un'imbarcazione
    possiamo solo sperare di trovare del materiale per creare un qualche
    tipo di\dots{} catapulta, immagino.'' ``Anche senza Lissa?'' ``Beh, se
    sto quì e ci penso per molto tempo e faccio molti test, dovrei riuscire
    a non far saltare in aria qualcosa'' rispose lui, facendole
    l'occhiolino. La ragazza rise e poi disse ``Immagino di sì.''

    Passarono un'altra ventina di minuti a discutere del più e del meno,
    poi uscirono dall'acqua per asciugarsi. Dopo essersi messi i primi
    vestiti asciutti riempirono le borracce ed ogni altro recipente
    utilizzabile di acqua. Fu allora che sentirono una serie di alberi buttati a terra
    come se fossero stati sradicati a forza. Finirono di vestirsi in fretta
    e furia per andare a vedere cosa stesse accadendo ma non riuscirono a
    capire molto, visto che la jungla era particolarmente fitta.

    ``Di là.'' fece la ragazza, indicando in una direzione. Hax osservò ed,
    in effetti, entrava più luce da quella parte. Iniziarono così ad andare
    verso quella parte di foresta. Dopo meno di cinque minuti arrivarono ad
    una area dove gli alberi erano stati come buttati a terra da una
    qualche creatura gigantesca. ``Ma che cazzo?'' fece Hax, guardando lo
    stato degli alberi. ``Hax!'' lo richiamò la ragazza, strattonandolo per
    il gilet. Lui, mentre si girò per vedere cosa stava succedendo, fece
    ``Che?'' Poi capì. Erano in un punto abbastanza rialzato per vedere in
    distanza una delle baie nascoste dalle montagne dell'isola, nella quale
    c'erano delle navi. ``Navi! Cazzo sì!'' fece il ragazzo, abbassando il
    pugno al fianco ``No, Hax, quell'altro.'' rispose Jam, indicando verso
    una grande città, che si sviluppava intorno alla baia, verso la quale,
    però, si stava dirigendo una gigantesca creatura a forma di uomo
    formata da pietre, sulle quali erano addirittura cresciuti degli
    alberi, e navi mezze distrutte, causa della scia di
    alberi abbattuti in mezzo alla foresta. ``Eh, no, cazzo, dai. I golem
    giganteschi no.'' Poi, alzandosi, fece sbuffando ``Ok, capito. Jam,
    andiamo. Se quella \emph{Cosa} distrugge quel villagio addio viaggio di
    ritorno per un po'.'' e s'incamminò lungo la radura creata da quella
    creatura ``M-ma.'' ``Nono, andiamo, Jam.''

    Jam lo seguì, finchè non arrivarono alla città, nella quale stava
    impazzando una battaglia che vedeva molti dei cittadini maschi
    combattere contro la creatura. Tutti parevano ben allenati ed
    utilizzavano una spada molto simile a quella che utilizzava Jam ``Eh?''
    fece Hax ``Era quello che ti stavo provando a dire, Hax.'' fece Jam
    ``L'ho notato prima. Questa è Sanpan.'' ``Davvero? Allora ve\dots{}'' e
    si fermò ``\dots{}Ma che c'entra?'' ``Quì la maggior parte degli uomini
    sa combattere.'' e, mentre Jam li guardava con ammirazione ``Sono tutti
    così bravi, eh? Sono allenati. Cose del genere capitano spesso.''

    Hax non riuscì a capire come il fatto che la città stesse venendo distrutta a suon di calci
    dalla gigantesca statua formata di pietre fluttuanti tenute attaccate
    da non sapeva neppure lui quale forza fosse una cosa di cui non
    preoccuparsi. ``Davvero? Perchè a me sembra che mi stiano distruggendo
    la città che può fornirmi il viaggio via da questo posto\dots{}''
    fece, appoggiando a terra lo zaino. ``Beh,'' iniziò Jam ``anche fosse
    non so cosa potremmo fare noi due. I combattenti più bravi del posto
    non lo stanno fermando. Se non ce la fanno loro\dots{}'' ``Aspetta''
    fece il ragazzo, tirando fuori dallo zaino un rampino metallico con una
    molla per permettere l'apertura all'ultimo momento ``mi vuoi dire che
    dei tizi con delle spade sanno fermare un costrutto alto venti metri
    meglio di quanto non lo sappiamo fare noi?'' continuò, tirando fuori
    dalla borsa una corda nera di seta rinforzata ``No, stavo dicendo che
    ce l'hanno sempre fatta. E rischiare la vita per una cosa del genere
    non mi sembra necessario.'' Hax legò uno dei capi della corda al gancio
    del rampino ``Capisco, ma quanto tempo passerà prima che il golem là''
    ed indicò il costrutto gigantesco che si stava muovendo sempre di più
    verso la baia ``arrivi alle barche? Con quanti danni ne uscivate,
    normalmente, da un attacco?'' ``Per una cosa così?'' Hax testò quanto
    bene avesse fatto il nodo dando uno strattone alla corda e poi si alzò
    in piedi ``Beh, comunque. Sai come la penso. Probabilmente là c'è gente
    che sta perdendo la casa, oppure sta venendo ferita. Non posso
    possibilmente lasciar andare.'' Allacciò lo zaino ai fianchi e
    corse dentro la città.

    Quel comportamento, per Jam, era poco comprensibile. Era capitato
    spesso, quando lei viveva quì, che ci fossero attacchi del genere. Ne
    erano sempre usciti. Non senza problemi, è vero, ma avrebbero potuto
    rimanersene lì ed aspettare. Ma questa era una delle cose belle di Hax.
    Nascosto dietro quella persona sempre stanca che non aveva quasi mai voglia
    di fare nulla, con un sacco di incertezze, c'erano dei principi morali
    che lo spingevano a portare a termine imprese coraggiose, nascoste come
    azioni per interesse personale. Seppur folli
    e senza piani, in realtà. La ragazza tirò un sospiro di rassegnazione e
    lo seguì camminando.

    Hax non aveva esattamente idea di cosa avrebbe dovuto fare per tirare
    giù quel coso, però non poteva permettere che qualche stronzo di
    gigantesco mucchio di sassi andasse in giro per quella città a provare
    a schiacciare gente o far perdere loro la casa. Incominciò a correre
    per le strade della città fatta di costruzioni in legno. Superò della
    gente armata che gli urlò qualcosa in una lingua che non conosceva.
    ``Scusate!'' urlò, senza neppure fermarsi ``Parliamo dopo, se mai
    capirò quello che mi dite.''

    Capendo che non avrebbe mai raggiunto la creatura dovendo seguire i
    vari vicoli formati dalle case decise che era ora di mettere in pratica
    quelle due o tre lezioni che gli aveva dato Celty. Aprì la porta di una
    delle case lungo la via che stava percorrendo e ci corse dentro. Mentre
    correva per il corridoio, andando verso le scale, urlò ``Prima che
    qualcuno mi attacchi, se mi capisce, lo sto facendo per salvarvi!''
    Ovviamente non ebbe l'effetto sperato. Una vecchietta si sporse da una
    porta e, dopo avergli urlato qualcosa, gli lanciò una specie di padella
    in ghisa. Ignorandola Hax corse su per le scale, fino al sottotetto.
    Aprì una delle finestre che davano all'esterno e la scavalcò,
    trovandosi al di sopra della maggior parte delle case nei dintorni. Ora
    doveva solo correre di tetto in tetto finchè non raggiungeva una delle
    due gambe della creatura.

    Ovviamente il ``solo'' non considerava il fatto che lui non fosse
    preparato per fare questa cosa. Corse lungo il tetto, facendo in modo
    di non scivolare in strada visto che era inclinato, poi saltò su quello
    della costruzione successiva. Il secondo palazzo era fortunatamente
    meno alto del primo, quindi fu un gioco da ragazzi, come saltare un
    piccolo fossato. Mentre correva lungo il tetto della seconda casa
    dovette schivare dei pezzi di costruzione proiettati inconsciamente
    nella sua direzione da un piede del golem che distrusse mezzo palazzo
    calpestandolo. Finito di rotolare per la schivata continuò a correre,
    prendendo la rincorsa per il prossimo salto. Questa volta era molto più
    difficile, si staccò da terra all'ultimo momento, convinto di arrivare
    senza problemi dall'altra parte ma evidentemente aveva fatto molto male
    i calcoli. Arrivò sull'altro
    tetto con solo il torso, perciò fu un grandissimo colpo di fortuna se
    riuscì ad aggrapparsi al cornicione prima di cadere in strada.

    ``Cazzo.'' imprecò, mentre si tirava su, aggrappandosi a tutto quello
    che trovava per darsi una mano ``Nei videogiochi non la fanno sembrare
    così difficile questa cosa. Come diavolo fa Celty?'' In effetti la
    ragazza era allenata, quindi oltre a saltare più lontano di lui sapeva
    anche quali tipi di ostacoli poteva superare. Dopo poco riuscì a
    ritirarsi sul tetto e riprese a correre. Questa costruzione era stata
    distrutta in parte dalla creatura, facendo in modo che il tetto fosse
    inclinato verso l'alto, creando una rampa naturale verso l'ultima
    costruzione prima di raggiungere il costrutto. A metà della
    costruzione, però, il golem, senza neppure averlo visto, si girò
    calciando un carretto il quale, lanciato in aria dal colpo, arrivò sul
    tetto dove si trovava Hax e lo sfondò, facendo finire il ragazzo al
    piano inferiore. Dopo aver toccato terra ed aver imprecato a pieni
    polmoni si alzò di nuovo, notando che non poteva ritornare sul tetto
    per raggiungere l'altra costruzione. Doveva trovare un'altra soluzione.
    Notò che uno dei muri della casa proseguiva anche oltre il perimetro
    della stessa, proprio a causa dei danni subiti. Fu così che decise di
    riprovare a vedere se le cose che gli aveva raccontato la runner e che
    aveva visto nei videogiochi erano applicabili alla realtà. Prese la
    rincorsa ed incominciò a correre verso il muro, traiettoria che lo
    avrebbe portato contro di esso dove finiva il pavimento (e la casa in
    sè). Arrivato ad un passo dal muro si staccò da terra, iniziando a
    correre lungo la parete verticale, facendo scivolare la mano sinistra
    contro il muro per non arrivarci addosso con tutto il resto del corpo.
    Non era una corsa, in realtà, quando un aiuto
    per il salto, sfruttanto l'impulso iniziale. Arrivato al litime della
    spinta iniziale sfruttò il muro per compiere un altro salto. Se avesse
    fallito quella ``stunt'' sarebbe finito per strada, probabilmente con
    una o due gambe rotte.

    Jam, la quale era più pronta fisicamente, era arrivata al palazzo,
    seguendolo da terra. Ora era per strada mentre lo vide compiere
    quell'acrobazia. Sapeva che Hax provava delle cose che aveva anche solo
    visto per una sola volta di sfuggita, vista anche la sua
    specializzazione in campo scientifico, ma quello era troppo anche per
    lui. Infatti lo vide puntare ad un cornicione per non arrivare così in
    alto e finire dentro una finestra un metro più sotto.

    Hax sfondò il tavolo di una cucina, dopo aver distrutto la finestra che
    dava sulla strada. Ce l'aveva fatta. Incredibile. ``Ha!'' fece, rivolto
    al cielo, dopo essersi alzato da terra e tolto di dosso tutte le
    macerie ``Hai visto, Celty, Faith? Lo so fare anche io.'' Poi,
    ricordandosi del perchè avesse fatto una tal stupidaggine, corse fuori
    dalla cucina, si guardò intorno in maniera frettolosa, e quindi corse
    attraverso il soggiorno di quell'appartamento, quindi fuori di esso.
    Salì le scale di corsa.

    Arrivato sul tetto era ormai sotto il golem. Fortunatamente non si era
    mosso, altrimenti avrebbe rischiato la vita dovendo fare chissà quale
    altra prodezza da runner professionista. Si trovava proprio sotto il
    bacino di quella creatura, formato da un enorme masso ricoperto di vongole
    ed altri gusci. Iniziò a far ruotare la corda col rampino collegato.
    L'obiettivo era di lanciarlo abbastanza in alto per farlo legare alla
    barca  diroccata che formava il petto. La corda era abbastanza lunga.
    Appena il rampino ruotò abbastanza in fretta lasciò la presa sulla
    corda, facendo in modo che questo volasse verso un albero della nave.
    Quando fu il momento di dare lo strattone per aprire il rampino, però,
    il golem spostò il braccio, facendolo impigliare su un albero rimasto
    su una delle roccie che formavano gli arti superiori.

    \emph{Non è giornata} pensò Hax, mentre venne trascinato via dal
    movimento rotatorio del corpo del golem. Ora doveva riuscire a salire
    su una parte della creatura che continuava a muoversi. Se fosse stato
    il petto avrebbe ``solamente'' dovuto scalare un muro che si muoveva,
    ma ora era attaccato al braccio di quella cosa! Proprio a causa di
    questo venne lanciato in giro dai movimenti del golem. Bastava che
    questo facesse un passo che lui, a causa del movimento delle braccia,
    veniva mandato d'ovunque. La parte più dolorosa fu quando uno di questi
    movimenti lo mandarono contro un magazzino, sfondandone due pareti.
  
    Jam intanto, stava seguendo la scena dal basso. Seguì Hax, o almeno
    provò a seguire la sua traiettoria in volo. Ad un certo punto, però,
    si fermò vicina ad uno dei combattenti del villaggio e gli fece, nella
    sua lingua ``\emph{Ho bisogno della tua arma. ORA!}'' lui, ritraendo
    l'arma, le rispose, irritato ``\emph{Questa non è fatta per le donne.
    Ora sparisci!}'' Jam, già preoccupata per la sorte del suo amico, prese
    per il kimono la persona con cui stava parlando e, arrabbiata, fece
    ``\emph{Senti, non ho tempo nè voglia di stare quì a discutere con te
    delle nostre tradizioni. Voi non potete tirarlo giù. Dammi. Quell'.
    Arma.}'' Il tizio, però, oppose resistenza. Fu così che lei gli scattò
    oltre e gli tirò un colpo sulla nuca con la mano a taglio, facendolo
    cadere a terra stordito. ``Odio fare queste cose.'' commentò,
    raccogliendo la spada da terra.

    Jam raggiunse di corsa il ragazzo, sdraiato a terra dopo l'impatto, che stava
    mormorando costantemente qualcosa. Si chinò su di lui per controllare
    le sue condizioni. Notò che stava perdendo un po' di sangue da un
    graffio alla testa. Niente di pericoloso, sembrava. ``Tutto ok, Hax?''
    chiese, agitata ``Hmm\dots{}'' fece, lui, mentre si alzava dandosi una
    mano con il braccio ``Sì, sì. Lo zaino deve aver attutito il colpo con
    i muri.'' E si alzò in piedi ``Senti, Hax, se vogliamo batterlo non
    puoi caricarlo così a testa bassa, rischi di morire senza compiere
    niente!'' ``Ma il mio piano era perfetto. Non ci fosse stato quel colpo
    di sfiga'' fece, lui, ansimando. Era stanco, in effetti, aveva corso
    come un disperato ed ora era rimasto attaccato ad un golem impazzito.
    ``Ok. Fammi fare qualcosa.'' Jam si mise in modalità \emph{analisi},
    che era una condizione simile al cambio di paradigma di Hax, solo che
    era più affine al suo modo di vedere la realtà.

    Quando analizzava il mondo tutto appariva più scuro, un piano
    trasparente reticolato in maniera regolare con delle linee viola
    luminose che seguiva la linea dell'orizzonte, mentre un reticolato
    seguiva alla perfezione il terreno. 
    Guardare il mondo in quella maniera le permetteva di focalizzarsi su
    un oggetto od una creatura, evidenziandone i contorni, facendo comparire vicino a questa una serie
    di dati matematici e biometrici, tra cui la distanza, vettori
    delle forze in azione su di essa, traccia termica, traiettoria che più
    probabilmente sarebbe stata percorsa, campo visivo, segnali immaginari
    che indicavano l'influenza elettromagnetica della persona sulla realtà
    circostante e così via. Poteva anche
    selezionare più oggetti, ricevendo dei dati delle relazioni che
    intercorrevano tra di loro, quanto distavano l'una dall'altra, la
    quantità di spazio che occupavano e così via. Grazie a questi dati ed
    a tutti gli studi che aveva effettuato sulla matematica poteva
    modificare il comportamento di questi corpi utilizzando delle formule.
    Ovviamente non poteva, ad esempio, uccidere una persona oppure renderla
    immortale, oppure creare energia infinita. Quello sarebbe andato contro
    le regole della conservazione energetica, farlo avrebbe significato
    incappare in paradossi della realtà. Quelle cose, normalmente,
    portavano alla rapida distruzione dell'oggetto sul quale erano state
    applicate le formule con una distorsione della realtà atta al riportare
    lo stato delle cose a come doveva essere, con un rischio proporzionato
    per chi aveva utilizzato la formula.
    
    Un esempio era provare a far
    ruotare all'infinito una trottola (una delle cose che venivano
    insegnate alle prime lezioni pratiche, anche perchè era una cosa così
    piccola che non causava danni agli studenti). Dopo non molto tempo la
    trottola implodeva, scomparendo nel nulla, lasciando una distorsione
    dello spazio, il quale tornava normale dopo un po' e, se qualcosa era
    stato fatto cadere a terra dalla trottola, questo tornava in piedi. Era
    per quello che utilizzare formule che andavano contro le regole era un
    taboo. C'erano comunque dei matematici che dicevano che utilizzare
    formule o tesi di questo tipo potevano venir utilizzate sfruttando
    questi loro ``effetti collaterali'' in maniera intelligente per
    raggiungere degli scopi altresì irraggiungibili.
    
    Naturalmente per avere degli effetti interessanti, usando questo
    espediente, andavano scritte
    delle tesi, che erano come i programmi per gli esperti di algoritmica:
    formule estese molto complesse che venivano valutate utilizzando
    come medium l'oggetto sulla quale venivano inscritte. Normalmente
    quando venivano valutate delle funzioni veniva utilizzato il corpo
    del matematico che le utilizzava oppure quello sul quale venivano
    applicate, se fosse stata una creatura vivente. Alla fine, comunque, l'effetto era
    molto simile a quello degli algoritmi per i programmatori. Ogni tanto
    si rischiavano anche ferite, visto che si andava ad operare a livelli
    molto più bassi. Ma, comunque, era
    un'imposizione morale quella di non utilizzare metodi che andassero
    contro le leggi fondamentali dell'universo. 

    Jam utilizzò le sue capacità per diminuire gli effetti della
    stanchezza sul ragazzo, oltre ad aumentare la sua capacità polmonare e
    quanto ossigeno riusciva a portare il suo sangue. ``Ecco.'' fece lei.
    Hax tirò un sospiro di sollievo, come se il suo corpo respirasse
    meglio. Gli sembrava di aver corso molto meno di quanto non avesse
    fatto. ``Grazie, ok. Allora, com\dots{}'' non riuscì a finire la frase
    che venne trascinato da una forza inarrestabile al piede destro. La
    corda attaccata al rampino gli si era legata intorno alla caviglia ed
    era bastato che il golem facesse uno scatto in una direzione per
    trascinarlo via. Jam, istintivamente, prese il braccio di Hax per
    provare a fermarlo, solo per venir lanciata via assieme a lui. In un
    istante si schiantarono nel petto della creatura. Hax riuscì a mettersi
    tra la ragazza e la finestra del castello, abbracciandola, attutendo l'impatto col suo
    corpo. Rotolarono a terra all'itnerno della stanza, fermandosi qualche
    metro di distanza dalle finestre. Hax continò a stringere la ragazza
    anche da fermi, teso per lo shock dello sbalzo, fino a che lei,
    agitata, non gli urlò ``Hax! Lasciami andare!'' lui la lasciò, non si
    aspettava una reazione del genere, preoccupato di averle fatto del
    male o che lei non volesse questo genere di contatto. Lei scattò via,
    estraendo la katana, quando Hax venne nuovamente trascinato via dalla
    corda. Fortunatamente, però, prima che venisse lanciato fuori dalla
    finestra, Jam riuscì a tagliare la corda. Hax riuscì a fermarsi
    aggrappandosi a terra, prima di finire fuori dal castello verso una
    caduta di una quindicina di metri. Si tirò su a fatica e, una volta in
    piedi, fece, evidentemente imbarazzato ``S-Scusa. Non volevo.'' ``Eh?''
    ``Dico, l'averti abbracciato e tutto il resto.'' ``Ma no, stupido.
    Dovevo tagliare la corda altrimenti saremmo stati trascinati di nuovo
    via. Siamo stati fortunati a finire quì.'' Hax non ne era proprio
    sicuro a sentire il male che aveva a causa di tutte le botte che aveva
    preso in quella giornata ``Anzi, al contrario, grazie.'' ``Figurati,
    dovere. Allora. Come si fa a tirare giù una cosa del genere?''
    ``Normalmente la distruggevano a suon di trappole. \'E per quello che
    ci voleva così tanto. Spaccavano le creature come questa pezzo a
    pezzo.'' \emph{Come facevamo con gli Zombie?} pensò Hax ``Però il mio
    maestro mi ha sempre detto che, da qualche parte deve risiedere lo
    spirito che li anima, altrimenti come fanno a muoversi?'' ``Hum. Ottimo
    ragionamento.'' ``Solo che non so come vederlo.'' ``Beh, non sarò bravo
    come Lissa, ma se ti fornisco dei da\dots{}'' vennero buttati a terra
    da uno scossone. Il golem stava muovendosi, probabilmente. Hax si
    aggrappò ad una colonna in legno e si tirò in piedi. Tutto si muoveva
    come se il golem stesse camminando. ``Dicevo. Se ti
    recupero dei dati sulla creatura riesci ad applicarci un filtro o
    qualcosa del genere per scoprire dove risiede? Se è veramente
    alimentato da uno spirito dovrebbe esserci una traccia diffusa
    d'ovunque.'' si fermò per pensare ``Almeno, questo era quello che mi
    diceva Elly su come funzionavano gli spiriti.'' ``Hum, quindi dici che
    se applico un filtro basandomi su una funzione caratteristica dovrei
    riuscire ad isolare i nuclei?'' ``Sì, tipo. Poi non so come funziona
    perchè non sono un matematico, lo sai.'' ``Dovresti studiare altro
    oltre il Retro-Engineering. Penso che Lissa lo saprebbe fare.'' ``Eh.
    Ma lei è più brava e non abbiamo tempo di stare quì a discutere.''
    ``Ok. Proviamoci.''

    Hax cambiò paradigma e, dopo aver controllato in giro, estrasse il
    segnale emanato da una parete, il quale comparve vicino a lui,
    galleggiando nell'aria, come una linea luminosa complessa posizionata
    all'interno di un piano cartesiano. Applicò quindi un algoritmo di
    estrazione dello spettro, l'FFT, il quale gli restituì una funzione
    per rappresentare il segnale. Disse la funzione a Jam, che la ripulì
    dalle tracce di segnali comuni, come residui della vita degli alberi,
    che potevano rimanere all'interno delle fibre del legno, e poi la
    utilizzò per applicare un filtro visivo alla modalità di
    visualizzazione della realtà, permettendole di evidenziare solo i punti
    dove quella funzione era più forte. Riuscì a trovare due punti. Uno era
    al centro della nave diroccata, mentre l'altro si trovava dove doveva
    essere la testa. ``Ok. Ci siamo.'' fece lei, girandosi verso Hax,
    ancora aggrappato alla colonna ``Dobbiamo dividerci. Un pezzo dello
    spirito si trova quì da qualche parte. L'altro si trova sulla testa. Io
    andrò verso la testa visto che so scalare meglio di te.'' Hax si sentì
    ferito, ma aveva ragione la ragazza. Lui avrebbe solo rischiato di
    cadere a terra. ``Ok. Allora ci vediamo dopo. Buona fortuna.'' fece
    lui, per poi mollare la colonna ed uscire dalla stanza. Lei, invece, si
    applicò una funzione per migliorare la resa delle gambe
    e saltò fuori da un buco che c'era sul soffitto del castello.
    
    Hax si trovò sul ponte della nave, che era sovrastato da una roccia che
    formava le spalle del gigante. Intanto in giro il mondo si muoveva.
    Vedere dalla prospettiva dello sterno di una persona che cammina dava
    una strana sensazione, ma se ne fece una ragione in poco tempo. Il
    problema più grosso era cosa sarebbe accaduto quando entrambe i
    contenitori per lo spirito fossero andati distrutti. Mentre corse giù
    per le scale che portavano in coperta pensò a quali effetti sarebbero
    potuti capitare. Il primo poteva essere che le energie interne che
    tenevano assieme ma separati gli oggetti avrebbero perso il loro
    equilibrio facendo o implodere tutto oppure facendo schizzare via
    lontani tutti i pezzi. Il secondo era quello che, una volta perse le
    energie che tenevano tutta la baracca in piedi, le leggi della fisica
    sarebbero tornate a lavorare normalmente e quindi, dopo una caduta di
    quindici metri, gli sarebbero caduti addosso spalle e testa della
    creatura. In entrambe i casi sarebbe stato brutto trovarsi all'interno
    della barca. Fortunatamente la parte di prua della nave era stata
    staccata completamente, lasciando un buco per ogni piano verso
    l'esterno. Decise che, appena distrutto il nucleo, sarebbe corso fuori,
    buttandosi sul tetto di una casa. Un salto di probabilmente cinque, sei
    metri.

    Arrivato al terzo ponte vide un bagliore provenire da dentro la
    Polveriera della nave. Scese altri due piani, per poi arrivare davanti
    ad una porta a grata in legno chiusa. Prese il piede di porco che aveva
    attaccato allo zaino e, dopo averlo incastrato tra la porta e lo
    stipite, tirò fino a sfondare la serratura. In un lato della
    polveriera, dove c'erano una serie di barili, uno di questi brillava di
    una luce azzurrognola. Lo aprì per scoprire che era un barile
    d'esplosivo. Ci pensò due secondi e poi, pensando che fosse il modo
    migliore, prese una delle miccie che si trovavano a terra e la inserì
    nella polvere dentro nel recipiente. Non se lo faceva dire due volte di
    far saltare in aria qualcosa per fare un buon lavoro.

    Appena finì di mettere tutti vicini i barili tirò fuori il suo
    accendino a propellente e lo accese con uno scatto del polso. Si girò
    per vedere se, correndo verso l'esterno, sarebbe saltato sul tetto di
    una casa o si sarebbe dovuto buttare su un covone di paglia per strada
    o cose del genere. Cosa stupida, anche perchè, a quanto pare, in quella
    città non si mettevano in giro per strada covoni di fieno. Appena vide
    che c'erano dei tetti che si avvicinavano rapidamente (anche se,
    probabilmente, erano loro ad avvicinarsi) incendiò la miccia e scattò
    verso la fine della stanza. Arrivato al limite del pavimento si abbassò
    leggermente per caricare il salto e, appena la caduta era il più bassa
    possibile, si lanciò fuori. Mentre fu in volo la polvere prese fuoco,
    provocando un'esplosione che riempì tutta la polveriera, fuoriuscendo
    dal lato di quel che restava della nave. Ci fu come un urlo provenire
    dal golem, probabilmente causato dal rilascio di parte dello spirito
    che lo animava. L'onda d'urto provocata
    dall'esplosione lo spinse via, facendolo finire molto più vicino alla
    fine del tetto di quanto non avesse calcolato. Appena toccò la
    superficie iniziò a rotolare. Dopo pochissimo si trovò fuori dal tetto.
    Fu grazie alla funzione di Jam che riuscì ad allungare abbastanza in
    fretta il braccio per aggrapparsi al cornicione e fermarsi. Ora doveva
    scendere da là. Probabilmente il golem gli sarebbe caduto in testa
    entro pochissimo.

    Jam, mentre Hax scendeva nelle profondità del relitto, scalò il braccio
    destro della creatura. Era complicato ma non difficile. Essendo fatto
    di pietre queste non erano levigate, quindi era presente un gran numero
    di appigli che aiutavano di molto la salita, oltre che permetterle di
    tenersi quando la creatura si dimenava per distruggere qualcosa. Le ci
    volle un po' per arrivare alla spalla. Purtroppo, però, una volta là,
    ci fu un'esplosione all'interno del petto del golem. Questo barcollò
    urlando,
    facendole cadere la spada in strada, per poi mettere un piede di fronte
    a sè, per recuperare l'equilibrio. Il contraccolpo la spinse via, verso
    il braccio sinistro della creatura. Jam riuscì a ruotarsi in volo ed
    aggrapparsi alla mano sinistra. C'era un problema, però, non aveva
    l'arma. Anche fosse arrivata sulla testa, cosa molto più complicata
    ora, visto che il braccio sinistro era molto più distante dalle spalle
    che non quello destro., non sapeva come l'avrebbe distrutta. Avrebbe
    potuto potenziare i suoi muscoli abbastanza per distruggere una pietra,
    ma le ci sarebbe voluto un sacco di tempo per recuperare l'uso della
    mano destra a causa delle fratture. Le ci volle un po' per prendere il
    coraggio ed iniziare a scalare il braccio.

    Hax era arrivato in strada, proprio sotto le gambe della creatura, per
    fuggire. Era strano, comunque, che Jam non fosse ancora scesa. Guardò
    in alto e notò che la ragazza era ancora sulla creatura che stava
    salendo lungo il braccio sinistro. ``Ma non aveva iniziato a salire il
    braccio destro? Che cazzo?'' fece lui, guardando in alto, per poi
    abbassare lo sguardo, notando che c'era un ragazzino ancora in mezzo
    alla strada, fermo ``Hei! Tu!'' urlò lui, facendo girare il ragazzo
    ``Vattene, non vedi che è pericoloso?'' Quello scappò via, lasciando a
    terra quello che aveva raccolto. Era una katana in un fodero, che ad
    Hax parve di aver già visto. Si girò per guardare Jam attaccata alla
    mano della creatura, per notare che non aveva più l'arma. Doveva
    esserle caduta quando lui aveva fatto saltare in aria il primo
    ``contenitore'' dello spirito. ``Jam!'' urlò.

    La ragazza sentì l'urlo e girò la testa verso il basso per guardare chi
    fosse stato e vide Hax il strada. ``Hax!'' rispose, lei. Notò che ai
    piedi del ragazzo c'era la spada che aveva perso prima. Forse lui
    riusciva a lancargliela. Ma doveva ancora trovare un modo per salire
    fino alla testa ``Aspetta!'' Hax raccolse la spada da terra. Jam si
    guardò in giro per vedere se c'era qualcosa che poteva aiutarla quando
    vide il rampino di Hax ancora attaccato al braccio sinistro della
    creatura con quello che rimaneva della corda. Iniziò a scalare più
    rapidamente, fino a raggiungere l'attrezzo in metallo. Lo slegò da dove
    si trovava e, tenendosi con la mano sinistra, lo fece ruotare.

    Hax, che stava seguendo la cosa da terra, capì cosa volesse fare la
    ragazza, perciò iniziò a correre lontano dal golem. Appena la ragazza
    lanciò il rampino, il quale andò ad agganciarsi su una protuberanza
    della testa della creatura, cambiò paradigma. Non ce l'avrebbe mai
    fatta a lanciare la spada così in alto, a meno di non utilizzare un
    impulso istantaneo, generato da un algoritmo abbastanza corto da avere
    un rischio molto basso di errore. Generò due dischi bianchi, ai quali
    collegò delle scritte fondamentalmente uguali e ne collegò uno alla
    mano ed uno alla spada. Avrebbero creato un impulso elettromagnetico
    di uguale intensità e polarità. Quello, assieme alla forza iniziale
    data dal braccio, avrebbe dovuto far arrivare la spada abbastanza
    lontana. Caricò il lancio e, quando vide che Jam stava per saltare,
    scagliò l'arma, la quale, appena lasciata dal braccio, schizzò in
    avanti. Le cariche elettromagnetiche crearono una distorsione dell'aria
    tra mano ed elsa dell'arma, creando un'onda d'urto che spinse a terra
    il ragazzo.

    Jam era in modalità analisi quando Hax lanciò l'arma. Aveva focalizzato
    la katana, perciò vide dove sarebbe finita se non ci fossero state
    influenze esterne. Era un po' fuori traiettoria, perciò saltò,
    tenendosi con la mano destra alla corda, in direzione dell'arma.
    Con quella traiettoria, però, sarebbe dovuta tornare, grazie alla
    corda, direttamente sul contenitore della seconda parte di spirito.
    Volò in direzione dell'arma e, come aveva calcolato, riuscì a prenderla
    per l'elsa. Con un movimento del polso lanciò lontano il fodero,
    estraendola. A quel puntò iniziò a cadere verso la testa. Si passò
    l'arma sulla mano destra, lasciando la corda, e la impugnò come un
    pugnale. Dopo pochi secondi impattò con la testa della creatura.
    Affondò la spada aiutandosi con la mano sinistra, mettendoci tutta la
    forza che aveva in corpo. Si formò una crepa sul contenitore a partire
    dal punto d'impatto dalla quale, dopo poco, fuoriuscì un fumo
    azzurrognolo, facendo scomparire il suo alone luminoso. A quel punto il
    golem iniziò a collassare su se stesso. Prima la testa s'incassò sulle
    spalle, poi caddero a terra le braccia, una alla volta, finendo di
    distruggere due case.

    Hax era a terra, che guardava il golem crollare a terra. Era, in
    effetti, il risultato migliore che poteva capitar loro, almeno non
    avrebbero rischiato così tanto la morte. Notò che Jam stava aspettando
    il momento giusto per buttarsi su una delle case. Ad un certo punto,
    però, la testa del golem crollò parzialmente, facendole perdere
    l'equilibrio. Il ragazzo scattò così in avanti e, dopo pochi secondi,
    riuscì a prenderla al volo. Per Hax quella era l'ultima prodezza del
    giorno. Sentiva il corpo che non gli stava più dietro, strenuato
    dall'uso di algoritmi, funzioni, corse lungo le pareti, venir
    utilizzato come palla da demolizione ed altro. Si alzò in piedi, con
    ancora la ragazza in braccio, e, mentre dietro di loro stava crollando
    l'ultima parte del golem, la appoggiò a terra in piedi. ``Visto?'' fece
    lui, focalizzandosi sul rimanere coscente ``\'E così che lavoriamo,
    no?'' continò, alzando il pugno destro verso di lei ``Sei matto.''
    rispose lei, appoggiando a sua volta il pugno destro su quello del
    ragazzo.

    Appena il polverone creato dal crollo della creatura si levò in aria si
    sentirono delle persone urlare nella loro direzione. Hax disse a Jam,
    abbattuto ``Ecco, ci siamo. Ma cazzo. Adesso vedrai quello che mi
    succede sempre.'' e lei, di tutta risposta ``Non pensare di essere
    l'unico ad esserci passato.'' Il ragazzo non capì nulla di quello che
    si urlarono Jam e gli abitanti del villaggio, anche se sapeva che
    sarebbe finito in una cella. Una sola cosa gli passò per la mente prima
    di crollare per la fatica\dots{}

  \section{Seysill e Durga - Metra - Agosto 2635}

    ``BIOPARCO!'' fece Seysill levando la penna in aria. ``Ah, no,
    aspetta.'' continuò, mordicchiando la penna, leggendo meglio il
    cruciverba che aveva di fronte ``Uno, due\dots{} Quattordici. No,
    niente.'' Non c'era niente di meglio che fare un cruciverba mentre si
    beveva dell'ottimo tè quando pioveva. L'uomo era nel suo ufficio dopo
    una seduta del parlamento. Era una delle sue ultime votazioni prima di
    andarsene.

    ``Bioparco che, Seysill?'' fece Durga, togliendo lo sguardo dalla
    libreria dell'uomo. ``Eh, vedi, quattordici parole, inizia con la B.
    \emph{Vi si trovano specie di piante protette}.'' ``Eh? Sicuro?'' fece lei, tornando a
    controllare i libri che possedeva l'amico ``Beh, sì.'' e poi,
    dopo aver letto meglio ``Ah. No, perbacco. Ho letto la definizione
    sbagliata. Allora. \emph{Lo è quell'elemento meccanico che può
    effettuare rotazioni lungo un asse}. Con la B? Mah.'' ``Brandeggiabile.
    Per Eclisse, sei davvero ignorante.'' rispose lei, con sufficienza,
    senza neppure girarsi ``Huh. Scusa, eh. Ma è una parola vera? Ah. Ci
    sta. Ottimo! Ma scusa, Eclisse pensavo fosse stata rimossa come
    divinità dal culto per il quale \emph{lavori}'' ``Hmph.'' rispose lei,
    sbuffando ``Tu lo sai che seguo i XIII. Lavoro per il culto dei XII
    solo perchè c'è una vera possibilità che la religione raggiunga più
    gente grazie a loro.'' ``Oooh. Quindi mi stai dicendo che lavorare per
    dei criminali che vogliono rovesciare un paese è corretto ai fini della
    distribuzione su ampia scala di una religione?'' ``Che c'entra?''
    ``Beh, quel tuo \emph{Philip} non è un pezzo grosso del culto?'' ``Sì,
    ma questo non vuol dire che quello che fa gli segue i piani del
    culto. \'E vero che il suo apporto monetario post colpo di stato
    sarebbe un'ottima cosa per l'estensione del culto, ma se non mi fosse
    stato di proteggerlo lo ucciderei seduta stante.'' ``Hm. Sento sempre
    un brivido lungo la schiena quando fai la cattiva. Che sia eccitazione?
    Oppure è paura?'' ``Ma perchè sto ancora quì a parlare con te?''
    
    Seysill rise e poi, dopo aver preseo un altro sorso di tè, disse, più
    serio ``Senti, Durga. Tra poco in questa città ci sarà qualcosa simile
    ad un colpo di stato. Sei sicura di non volertene andare via con me?''
    ``No, te l'ho già detto. Mi hanno detto di tenere d'occhio quel
    tizio.'' ``Voglio dire, devi controllare quello che porterà avanti un
    cambiamento forzato di governo?'' ``Sì.'' ``E tutto questo perchè te
    l'hanno detto quelli del culto?'' ``Esatto.'' ``Ma non è contrario a
    qualche regola aiutare della gente che ammazzerà delle persone?''
    ``Finchè questo lo fa con il pretesto di aiutare il culto temo che vada
    bene.'' ``Cioè, finchè la gente viene uccisa per il culto va tutto
    bene?'' ``Io non ho mai ucciso a meno di violazioni del codice dei
    XIII.'' ``Mah, vedremo.''

  \section{Rollers senza Hax e Jam - Mare Infinito - Agosto 2635}

    Elythia era al bancone del bar della nave, chinata sul suo drink. Era
    lì da un'ora che stava guardando il bicchiere, preoccupata sul da
    farsi. Mancavano ancora alcuni giorni prima di arrivare a terra e poter
    iniziare le operzioni di ricerca dei due ragazzi del loro gruppo. Però
    sapeva anche una cosa. In caso di scomparsa le probabilità di ritrovare
    qualcuno diminuiva molto rapidamente fino a diventare zero dopo
    quarantotto ore. Però era anche vero che le persone scomparse erano due
    scienziati particolarmente competenti nei loro campi. Oddio. Hax.
    Competente. Era una parola grossa. Conosceva un sacco di trucchi, vero,
    ma non era proprio quello che si potrebbe chiamare un esperto teorico.
    Aspetta. Ma che serviva essere un esperto teorico se sei disperso? Le
    venne quasi da piangere.

    Lissa entrò nel bar, vuoto, nel quale trovò la dottoressa seduta sul
    bancone, mentre il barista stava pulendo alcuni dei bicchieri. Si
    avvicinò e prese uno sgabello sul quale sedersi, lo allontanò dal
    bancone e ci si accomodò sopra. Agganciando il poggiapiedi trasciò se e
    la sedia vicina al bancone. Fece un cenno al barista con la mano e gli
    disse ``Un cosmopolitan, grazie.'' ``Subito, signora.'' rispose
    immediatamente il barista che, appenà finì di asciugare il bicchiere si
    mise a preparare il drink ``Elly, ascolta. Non ti fa bene stare così
    aggrottata. Non fa bene al tuo viso, oltre che al tuo umore.'' Elythia
    alzò la testa per guardare Lissa in faccia ``Non sei preoccupata, te?''
    ``Guarda che la gente non è più ritrovabile dopo due giorni solo se
    sono stati rapiti.'' la dottoressa scattò in posizione eretta ``C-Come
    facevi a sapere che stavo pensando a quello?'' ``Lo fai ogni volta che
    scompare uno di noi, in pratica. Stai lì e pensi a tutte le
    implicazioni, a cosa avresti potuto fare, a come starà e,
    soprattutto, che se non lo troviamo entro quarantotto ore allora la
    probabilità di ritrovarlo sarà pari a zero.''

    Elythia era evidentemente imbarazzata. Quello che aveva detto Lissa era
    vero. Era successo più di una volta che si ritrovava in un bar, di
    sera, a fare tutte quelle considerazioni, sentendosi impotente di
    fronte alla scomparsa di uno della squadra. Era una prerogativa del suo
    lavoro, però. L'essere sensibile alle condizioni di chi le stava vicino
    era una benedizione ed una maledizione al contempo. Le permetteva di
    avere contatti con gli spiriti, ma la faceva sentire molto peggio degli
    altri quando qualcuno era in pericolo. Era in quei casi che arrivavano
    o Lissa o Hax a trovarla, senza che lei capisse come facessero, per
    rassicurarla. ``Hm.'' fece lei, mugugnando, come sovrappensiero
    ``Senti Elly.'' iniziò la programmatrice. In quel momento il barista,
    dopo aver steso un fazzoletto sul bancone, appoggiò il bicchiere da
    cocktail pieno del drink violaceo con dentro una cigliegia al
    maraschino e lo spinse vicino al braccio della ragazza, facendole
    ``Ecco a lei.'' ``Grazie'' rispose, per poi continuare il discorso con
    l'amica ``Lo so che sei preoccupata. Lo sono anche io. Sapere Hax ed
    una ragazza carina come Jam da soli su un'isola deserta mi mette in
    agitazione. Cosa ne sarà del nostro eterno single?'' Elythia non riuscì
    a trattenere una risata, seguita da Lissa subito dopo ``Ecco, molto
    meglio. Sei più bella quando ridi. Asciugati le lacrime, dai.'' ``Sai che non funziona con me,
    giusto? Non mi farebbe effetto neanche se tu fossi un uomo.'' ``Oh,
    cavolo.'' rispose, facendo a finta di essere dispiaciuta ``Comunque,
    scherzi a parte, Hax e Jam non sono i primi due scemi di questo
    pianeta.'' ``Ma non sappiamo neppure se sono finiti su un'isola oppure
    se sono ancora al largo.'' ``Ti ricordi che ha detto il capitano? Che da
    quelle parti è pieno di isole che sono irraggiungibili con questa nave.
    Probabilmente sono stati trasportati là dalle onde. Magari Hax è anche
    riuscito a spingere lui e Jam verso la terra ferma più vicina.''
    ``Hum.'' ``Inoltre Hax è sopravvissuto in posti peggiori. Ti ricordi di
    quella volta che l'abbiamo mandato in quella città del Mare di Sabbia?''
    Lissa iniziò a sghignazzare, appoggiando il bicchiere per non
    rovesciarne il contenuto. Elythia rise pensando a come era tornato il
    ragazzo ``Sì, certo. Quando era tornato gli ci era voluta una settimana
    per togliere tutta la sabbia dallo zaino. Ricordi quanto era
    incazzato?'' Lissa iniziò ad imitare l'amico ``\emph{Ma cazzo. Almeno
    potevate dirmelo che mi sarebbe servita una tenda!} Era riuscito a
    sopravvivere tra deserto ed Oasi per una settimana.'' ``E ti ricordi di
    quel cratere di vetro che ha lasciato?''

    Ci fu un momento di silenzio imbarazzante, dove la ragazza e la
    dottoressa si fermarono a pensare a quello che aveva fatto Hax oltre a
    portare a termine il lavoro. ``Hem.'' ruppe il ghiaccio Lissa,
    schiarendosi la voce ``Comunque
    sono sicura che riusciranno a sopravvivere in una condizione ambientale
    così \emph{gentile}, soprattutto visto che Jam ha passato quasi tutta
    la sua vita in posti simili. Inoltre ho notato che qualcuno aveva
    sfondato a calci una cassetta d'emergenza sul ponte principale e manca
    lo zaino dello scemo dalla sua stanza. Penso potrebbe sopravvivere anche
    in quelle dannate jungle piene di sanguisughe dove volevi andare te.''
    ``Dici?'' ``Eddai, eh?'' rispose Lissa, facendo un cenno con la testa,
    per poi riprendere a sorseggiare ``Comunque non ti preoccupare. Appena
    arriveremo a terra recupereremo quale nave veloce o airship e andremo a
    cercarli. Penso di poter utilizzare una qualche traccia energetica di
    Hax.'' ``Ok. Tra l'altro, stavo pensando\dots{}'' ``Dimmi.'' ``Non
    dobbiamo neppure preoccuparci che distrugga città oppure venga
    incarcerato se è su un'isola disabitata, no?'' Le due si misero a
    ridere.

  \section{Hax e Jam - Sanpan - Agosto 2635}

    Era in una jungla. Sapeva di avere un lavoro da compiere, un qualche
    ricercatore da salvare, qualche macchinario da distruggere. Lo avevano mandato là buttandolo giù da un
    aereo, probabilmente, anche se non si ricordava bene quella parte.
    Nascosto dietro ad un albero attendeva che le due guardie che
    controllavano l'entrata del complesso militare si girassero per
    stordirle e nasconderle da qualche parte. Era una missione segreta, una
    di quelle nelle quali, se ti beccano, sei finito. Anche perchè non
    aveva molto. Aveva il suo piede di porco ed una pistola
    tranquillizzante. Si sporse per controllare lo stato. Ottimo. Le due
    guardie si erano girate. Colpì la prima alla nuca con la pistola,
    facendola cadere addormentata. Rapidamente cambiò mira, sparando un
    colpo al sedere della seconda, facendola cadere addormentata a sua
    volta. Uscì rapidamente dal nascondiglio e, prendendo sotto le ascelle
    una delle due, la trascinò nell'erba alta lì vicino. Questa lasciò
    cadere una scatola di munizioni tranquillizzanti. Non aveva capito come
    mai da quelle parti la gente avesse delle munizioni che non c'entravano
    nulla con le armi che portavano in giro. Trascinò anche la seconda
    guardia e poi iniziò ad infiltrarsi nella base. C'erano molte più
    guardie lì che non in giro per la foresa. Si appoggiò su un angolo
    coperto della costruzione. La sua tenuta mimetica era improvvisamente
    cambiata in una urbana. \emph{Ma che?} pensò. Doveva continuare se
    voleva riuscire a distruggere il Me**l Gear in tempo, prima che
    iniziasse un'ennesima guerra atomica! Fece per scattare in avanti per
    andare ad interrogare una guardia riguardo alla posizione dello
    scienziato quando il suo comunicatore suonò. Tornò in copertura per
    rispondere ``Sì, che c'è?'' fece  ``Sna\dots{}Hax! Elythia si trova in
    quel complesso!'' rispose una voce femminile al comunicatore, agitata
    ``S-Sì, grazie Lissa, lo so.'' ``A proposito! Ti ho mai parlato di
    007: Dalla Russia con Amore?'' ``Eh? Che c'entra?'' ``\'E un film
    fantastico, sai? Forse dovremmo recuperarti dei gadget da spia anche a
    te\dots{}'' ``Guarda, ho da fare, Lissa, ci sentiamo dopo.'' ``Ah, non
    t'interessa, eh? Allora potrebbe darti una mano per capire come
    trattare con le donne\dots{}'' ``Lissa! CIAO!'' urlò, mettendo giù.
    Capì che quella era stata una stupidaggine. In pochissimo tempo tutta
    la base lo stava cercando, era pure comparso un countdown che gli
    indicava per quanto tempo ancora l'avrebbero cercato. Non c'era via di
    fuga.

    A quel punto Hax si svegliò di scatto e la prima cosa che vide fu un soffitto in legno. Era
    giorno. ``Ancora quel sogno\dots{}'' mormorò. Faceva sogni come quello
    ogni volta che giocava ad uno dei suoi videogame preferiti mentre Lissa
    stava là a commentare. Lissa non era una giocatrice, non giocava da
    sola, almeno, per quanto questo potrebbe creare un'enorme lista di
    doppi sensi. Più e più volte lui e la ragazza avevano finito giochi
    assieme, ma quando uno non prevedeva di giocare la campagna in
    cooperazione tra due giocatori lei preferiva lasciar stare e far
    giocare il ragazzo.
    
    Mentre pensava a come diavolo avrebbe finito per la seconda volta il
    videogame in difficoltà estrema battè le palpebre una decina di volte prima di sentirsi pronto
    ad alzarsi. Com'era ragionevole aveva tutto il corpo che gli faceva
    male. Quella sensazione che si prova quando uno non riesce a dormire
    tutta la notte. Eppure aveva dormito, lo aveva dimostrato il sogno di
    prima. Prima di alzarsi, però, si guardò intorno. Vide dietro di sè una grata in
    legno che dava su una veranda e quindi su un cortile interno di una
    struttura quadrata e davanti a sè una finestra con delle sbarre in
    legno. ``Hm.'' fece, irritato ``Chissà perchè non ne sono per niente
    stupito.'' Quando si alzò in piedi lo sguardò si scurì per un secondo
    per poi ritornare normale un po' alla volta, mentre dappertutto
    giravano dei puntini luminosi multicolorati. Era la differenza di
    pressione data dall'alzarsi così in fretta assieme alla fame, di cui si
    accorse solo dopo essersi messo in posizione eretta. Aveva bisogno di
    mangiare qualcosa. Notò comunque che c'era un vassoio con sopra una
    ciotola piena di riso bianco ed una zuppa fatta con\dots{} Boh. Beh,
    sicuramente non poteva stare lì a fare lo schizzinoso.

    Jam era nella cella vicina. Era stata messa là dentro dopo la sua
    impresa con Hax. Li avevano sbattuti al fresco senza fornir loro cure
    mediche o altro. Anche se fossero morti a chi sarebbe importato di una
    donna e di un tizio che veniva da fuori? Aveva paura che il ragazzo
    fosse svenuto e che non riuscisse a riprendersi, poi però sentì
    qualcuno scucchiaiare dalla cella vicina. Si precipitò alle sbarre
    della cella per guardare fuori, sperando di vedere se fosse Hax che
    aveva ripreso conoscenza. Dopo poco sentì il ragazzo urlare ``Ah!
    Oporcaputtana! Acquacquacquacquacquacqua.'' Sì, era definitivamente
    sveglio ``Hax!'' fece, lei, euforica ``Sei vivo!'' ``Non ancora a lungo
    se non riesco a fermare il piccante. Questa cosa mi corroderà
    dall'interno!'' ``Aspetta un secondo. Finisce subito.'' Ci fu un
    momento di silenzio e poi il ragazzo disse ``Oh. Hei! Quella cosa che
    mi stava trapassando il cervello è scomparsa. Ma che diavolo?'' ``Sono
    delle nostre spezie. Riadici, tipo.'' ``Ah.'' ``Hanno un effetto del
    genere se ne sono state usate troppe o se ne mangi troppo
    tutto in un colpo.'' ``Hum. Non era male, comunque!'' fece Hax e poi
    riprese a mangiare.

    Dopo poco tempo ed alcune imprecazioni date dall'effetto delle radici
    ebbe finito il suo pasto. ``Allora.'' iniziò lui, dopo essersi seduto
    sul suo letto appoggiando la testa al muro che lo separava dalla
    ragazza ``Dimmi, come mai siamo in galera?'' Prima che lei potè
    cominciare aggiunse ``No, aspetta, anzi. Come mai TE sei in galera?''
    ``Hem. \'E una storia lunga. Diciamo che, per riassumere, tu sei in
    galera perchè sei un esterno che si è messo a distruggere un golem e
    pensano che tutti i danni alla città si sarebbero evitati se avessero
    fatto come al solito.'' ``Sè, certo.'' ``Io, invece, sono dentro per un
    motivo per il quale ho visto la gattabuia altre volte. Da noi le donne
    non possono brandire armi. \'E contro le tradizioni.'' ``Cosa? Cioè.
    Non sono mai stato uno che pensava che combattere per fare in modo che
    non dovessero farlo le donne fosse un'idea molto galante. Ma se una
    vuole imparare qualche arte marziale non mi pare un grande problema.
    Voglio dire, basta guardare te, Lissa ed Elythia. Sicuramente siete più
    brave di me.'' ``\'E un modo di pensare differente, Hax. Da queste
    parti la donna ha dei diritti differenti. \'E una società maschilista e
    fossilizzata sulle vecchie usanze. Pensa che mi hanno messo dentro
    anche perchè ho i capelli lunghi.'' ``Scusa? Cos'hanno che non vanno?
    Non sono abbastanza belli?'' ``\'E un'altro comportameto della gente da
    queste parti. Hai notato donne coi capelli lunghi?'' ``In effetti.''
    ``Portare i capelli lunghi è prerogativa degli uomini, un altro modo
    per far capire chi comanda.'' ``Hurg. Che brutta cosa.'' ``A chi lo
    dici.'' ``Ma se la società è così estremista sulle sue tradizioni come
    mai sei riuscita ad imparare l'arte della spada e la matematica?''
    ``Beh, come in ogni società con regole così ferree c'è sempre qualcuno
    che vorrebbe cambiare le cose. Il mio Maestro era uno di quelli. Sono
    stata fortunata di averlo incontrato quando iniziai ad esprimere il mio
    interesse per l'arte della spada ai miei.'' ``Perchè? Che è successo?''
    ``Mi hanno cacciato di casa.'' ``Scusa, come? Quanti anni avevi?''
    ``Otto, forse?'' \emph{Ma che diavolo di gente vive da queste parti?}
    pensò Hax, sentendo quella risposta ``Fatto sta che mi prese come sua
    allieva nel suo Dojo in mezzo alla foresta.'' \emph{Gh. Può essere
    più clichè di così la cosa?} ``\'E una persona un po' strana però è la
    persona più competente che conosca nei suoi campi.'' ``Immagino. Viste
    le tue capacità significa che il tuo maestro era uno dannatamente
    bravo. Ora però non mi spiego come mai tu te ne sia andata.'' ``\'E
    stato lui a dirmi di farlo. Voleva che andassi in giro per il mondo a
    farmi una nomea, per poi tornare quì a dimostrare di cosa possono
    essere capaci le donne combattenti.'' Ci fu una pausa, come di
    riflessione ``Ma non pensavo di tornare così presto!'' ``Beh, ok. Il
    nostro problema, comunque, rimane quello di come andarcene via di
    quì.'' ``Vedrai che, prima o poi, verranno a chiamarci per interrogarci
    e cose del genere. In qualche modo ce la faremo. Probabilmente se
    spiego che le cose là fuori si fanno in maniera differente, che non
    pensavo di essere quì, che li abbiamo salvati da una distruzione ben
    più grande di quella che non è stata causata e così via potrebbero
    cacciarci dall'isola.''
    ``Hum. OTTIMO. Proprio quello che ci serviva. Venir cacciati da
    un'isola tropicale.''

    Dopo qualche ora, come predisse la ragazza, due guardie vennero a
    prendere i due per portarli alla sala del consiglio, una sala con le
    pareti in legno a cassettoni, con un braciere verso l'entrata ed un
    tavolino a forma di ferro di cavallo dalla parte opposta, dove ad aspettarli
    c'erano una serie di anziani seduti a terra. Ogni anziano che li stava
    aspettando aveva dei vestiti con dei colori differenti, probabilmente
    facevano parte di famiglie differenti, oppure, semplicemente era la
    moda del luogo. Il ragazzo non capiva bene le usanze locali, in effetti. Vennero fatti
    inginocchiare a terra. Questo pareva non dar fastidio alla ragazza,
    mentre per Hax era una posizione di uno scomodo indescrivibile. L'uomo
    di fronte a loro iniziò a parlare. Il ragazzo era pronto a ribattere
    qualunque accusa. Non fosse stato che non capiva nulla.

    ``\emph{Intanto iniziate col dirmi chi siete}'' fece l'uomo
    ``\emph{Siamo Jam di Sanpan ed Hax di Jokula}'' rispose la ragazza,
    mentre Hax fece un cenno con la testa per confermare. Era vero che non
    capiva niente ma non poteva fare una figura del genere, perciò decise
    di dimostrare di seguire il discorso facendo dei cenni mentre parlava
    Jam ``\emph{Hum. Mi pareva di averti già vista.}'' fece un altro
    ``\emph{Sì. \'E stata l'ultima volta che l'abbiamo arrestata per essere
    andata in giro con i capelli lunghi.}'' aggiunse un terzo ``\emph{Però
    adesso l'accusa è differente. Questa volta tu hai usato delle armi
    mentre il tuo compagno di viaggi ha interferito nelle operazioni di
    allontanamento della creatura che ha attaccato il villaggio, ieri,
    creando un gran numero di danni.}'' ``\emph{\'E vero che abbiamo creato
    dei danni, però li abbiamo tenuti sotto un certo limite. Ce ne sono
    stati meno di quanti non ne avrebbe causati il golem continuando ad
    andare in giro. Avrebbe distrutto la flotta.}'' Vedendo che Jam rispose
    in maniera concitata Hax annuì in maniera più energica di prima. Uno degli anziani che
    non aveva ancora parlato si grattò il mento e commentò
    ``\emph{Effettivamente bisogna prendere in considerazione questa
    cosa.}'' ``\emph{Lo abbiamo sempre fatto di fermare creature di quelle
    dimensioni, perchè questa volta è differente?}'' ``\emph{Le altre volte
    ci sono stati danni molto maggiori, alla fine, nevvero?}''
    ``\emph{Hmph.}'' rispose, scocciato, il primo uomo che aveva parlato,
    per poi girarsi verso la ragazza ``\emph{Anche togliendo l'aggravante
    dei danni alle costruzioni rimane il fatto che tu ti sei macchiata di
    andare in giro con delle armi.}'' ``\emph{Questo perchè adesso che vivo
    al di fuori di quì posso farlo. Sono allenata e competente.}'' Hax
    annuì ancora ``\emph{Sì, certo. Come se una donna potesse diventare
    competente nel combattimento.}'' Jam aveva capito che quel discorso non
    avrebbe portato da alcuna parte, perciò decise di utilizzare un metodo
    un po' estremo. Per aumentare l'enfasi del discorso lasciò il ginocchio
    sinistro appoggiato a terra e mise il piede destro davanti a
    sè,``\emph{Sì, esatto. Adesso al di fuori di queste isole
    mi chiamano \emph{Lady} Jam, in segno di rispetto, e questo ragazzo è il
    mio protettore, solo che è differente che da noi. Di fuori il
    protettore di una ragazza nobile è un sottoposto e non il contrario.}''

    Hax non capì bene cosa fosse successo, ma la ragazza sembrava lanciata,
    perciò aspettò un attimo per annuire ``\emph{Oh, capisco. Perciò dovrei
    far fare ad entrambe la prova per l'innocenza?}'' fece il vecchio, con
    gli occhi socchiusi ``\emph{No, asp\dots{}}'' provò a controbattere
    Jam, venendo però bloccata da Hax che, inaspettatamente, le mise la
    mano destra sulla spalla e fece di sì con la testa. Quando tutti gli
    uomini iniziarono a mormorare e Jam ritornò inginocchiata Hax le
    sussurrò ``Allora, ti ho dato una mano?'' ``Hax,'' rispose lei,
    massaggiandosi le tempie con il pollice ed il medio ``ti sei creato dei
    problemi da solo.'' ``Eh?'' ``Certo, non posso dire che tu non mi
    abbia dato una mano, in realtà.'' ``Aspetta, che intendi dire?''
    ''\emph{Bene, allora.}'' disse, solenne, il primo anziano ''\emph{\'E
    stato decretato che voi due, domani, combatterete assieme per la vostra
    innocenza, come decretano le leggi.}'' quindi, rivolgendosi alle
    guardie, fece ``\emph{Potete portarli via.}'' Sentito questo Jam si
    alzò e fece un inchino, seguita da Hax che la imitò.

    Quando furono nelle loro celle il ragazzo si mise di nuovo sul suo
    letto e fece ``Allora. Dimmi un po', quali problemi mi sono creato?''
    ``Per farla corta gli ho detto che tu sei un mio protettore.'' ``Gah.
    Nel senso di\dots{}'' ``No, tipo maggiordomo.'' ``Ah, per fortuna.
    Avremmo fatto una pessima figura. Aspetta, maggiordomo?'' ``Sì.''
    ``Hum, vabbeh. E questo come mi comporta dei problemi?'' ``In realtà
    non sono solo problemi. Diciamo che ci sono due modi per risolvere le
    accuse giudiziarie. La prima è che, semplicemente, la persona deve
    accettare quello che è stato deciso dagli anziani e fine. La seconda è
    quella nella quale il consiglio non è convinto dell'innocenza o meno
    (o, nel nostro caso, se condannarci o meno per quello che abbiamo
    fatto) e allora decidono con una lotta.'' ``Oh.'' ``La parte positiva
    di tutto questo è che possiamo ancora uscirne sconfiggendo qualcosa in
    combattimento, la
    parte negativa è che ti sei tirato dentro.'' \emph{Ah, cazzo. No.
    Aspetta. Se non andassi poi come farei a mostrare la mia faccia a The
    Fixer?} ``Beh. Meglio no? Con le nostre abilità saremo liberi prima
    ancora di accorgercene.'' ``Bilanceranno lo scontro per due,
    immagino.'' ``Ma cazzo.'' ``Non ti preoccupare. Alla fine dovrai farmi
    da supporto, alla fine.''
    
    Hax si sdraiò sul letto, buttando la testa sul cuscino, a fissare il soffitto, tanto non avrebbe avuto
    molto altro da fare se non concordare delle tattiche da battaglia con
    la ragazza. ``Ai suoi ordini, Lady.'' Jam arrossì e rispose ``Grazie,
    Protettore.''

  \section{Celty - Udemia - Agosto 2635}
    
    Celty era felice di poter andare a fare un lavoro che non
    prevedesse fare qualcosa di illegale o di moralmente discutibile. Il
    lavoro consisteva nell'andare fino ad Udemia, una delle città più
    importanti vicina ai confini di Phaion, per andare a controllare lo
    stato dei vari contratti per gruppi come quello di Lissa e degli altri.
    Lo faceva ogni mese, più o meno, a meno che non fosse lontana dalla
    città.
    Servivano allo \emph{Stronzo} per vedere se era possibile richiedere un
    loro aiuto o dove ci fossero possibilità di profitto. Lei sapeva che il
    gruppo, comunque, non avrebbe mai accettato di portare a termine un
    qualunque di quei lavori sporchi che ha proposto Philip in passato. In
    effetti non li ha mai visti accettare uno di quei contratti. Non sono
    un gruppo di gente che farebbe di tutto per i soldi. Gliel'hanno sempre
    dimostrato.

    Doveva ammettere che ogni tanto aveva pensato di fuggire sfruttando
    quel tempo che le veniva lasciato per andare, ma sapeva che andarsene
    sarebbe significato il peggio per Lissa e gli altri. Ogni tanto questa
    cosa sembrava non avere senso, però quando si fermava a ragionare le
    risultava ovvio, visto il numero di contatti che aveva il Conte, le
    influenze, il potere economico e sociale. Non era sicuramente una buona
    idea, non con le possibilità attuali della ragazza.

    Decise comunque di andare a piedi fino ad Udemia, aveva bisogno di
    schiarirsi le idee. Stava troppo tempo dentro quella villa a non fare
    nulla se non allenarsi. Non serviva a nulla. Ogni tanto decise di
    prendere delle strade alternative per sgranchirsi un po'. Ovviamente
    era pià pericoloso, quindi non si spinse troppo, però saltare oltre
    crepacci anche di una quindicina di metri le faceva ricordare i vecchi
    tempi, dove correva per passione e non per dovere.

    Era già in viaggio da qualche giorno ed arrivò ad Udemia per sera.
    Andò a recuperarsi un letto alla taverna \emph{Le tre spighe} e poi si
    diresse verso l'ufficio della gilda che organizzava il lavoro dei
    gruppi di Tuttofare. Era abbastanza tardi, ma quei locali rimanevano
    aperti fino ad ore piccole, non si poteva mai sapere quando un lavoro
    veniva completato. Una volta dentro salutò la responsabile, che
    conosceva già viste le volte che era passata, e, dopo essersi
    avvicinata al bancone, le disse ``Ciao, senti, sono venuta per vedere i
    vari contratti, posso?'' ``Sì, certo Celty. Ecco.'' e le porse una
    cartella chiusa con dello spago contenente un gran numero di fogli.
    ``Posso?'' fece la ragazza, indicando un tavolino con una sedia più in
    là ``Sì, ovviamente.'' ``Ah, grazie.'' prese il plico e lo appoggiò sul
    tavolino.

    Vide di tutto, si appuntò i contratti già presenti, se erano stati
    presi, chi li aveva accettati, quanto venivano pagati ed altre cose che
    sapeva interessavano a Philip. Se non avesse fatto quel lavoro bene non
    l'avrebbe mai più mandata per quell'incarico e allora addio giornate
    libere. Dopo una mezz'oretta vide un contratto strano. Pagavano poco,
    ma il viaggio veniva fornito dal committente. Prevedeva di andare a
    Metra a parlare con ``una certa ragazza''. Che diavolo di lavoro era?
    Troppo poco per tutti i problemi che c'erano in quel periodo. Comunque
    pareva che qualcuno l'avesse accettato. Celty incominciò a scorrere i
    documenti per scoprire quale gruppo poteva accettare paghe così basse.

    Trovò il documento. Citava: ``Caro Gabriel. Grazie per la sua proposta
    di contratto. Tra pochi giorni salperemo in direzione di Phaion.
    Arriveremo in un mese o giù di lì. Con la speranza di un rapporto
    lavorativo fruttuoso per entrambe, Elythia. Gruppo Rollers.''

    ``No.'' mormorò Celty, lasciando cadere il foglio sulla pila.

  \section{Hax e Jam - Sanpan - Agosto 2635}

    Jam ed Hax vennero portati davanti ad un enorme arco fatto in legno
    laccato di rosso. Oltre di esso, ad una cinquantina di metri, c'era
    un'enorme complesso, formato da vari palazzi, costruiti su palafitte e
    collegati tra di loro da dei ponti in legno. Questo complesso era
    posizionato sul pendio di una collina, che limitava con la jungla.
    ``Ottima architettura.'' commentò Hax, guardandolo da lontano ``Già.''
    rispose Jam ``Peccato che lo usino come terreno di prova.'' ``Ma come,
    scusa? Non hanno paura di rovinarlo?'' ``Si aspettano tagli, Hax, non
    esplosioni.'' ``Davvero? E se uno non sa usare la spada?'' ``Eh. Si
    dovrà adattare, mi sa.'' ``Vabbeh, ho capito.'' Arrivati vennero
    intimati di smettere di parlare (almeno, questo è quello che sussurrò
    la ragazza ad Hax), dopo di che l'anziano che li aveva costretti ad
    andare iniziò a parlare con voce solenne

    ``\emph{In base alle nostre leggi oggi queste due persone, Hax e Jam, verranno
    sottoposte alla prova dell'innocenza. Se qualcuno ha delle obiezioni
    le faccia subito, altrimenti dovrà andare a scusarsi sulle loro tombe
    in caso di fallimento. Se ne usciranno vincitori allora significherà
    che le accuse a loro carico non erano fondate, in quanto sono stati
    trovati innocenti dagli spiriti stessi. Bene, siete pronti?}''
    ``\emph{Sì.}'' rispose la ragazza, seguita da Hax al quale era stato
    istruito di rispondere uguale a lei in quel caso ``\emph{Bene.
    Accompagnateli all'entrata.}''
    
    Vennero portati fino oltre al cancello e quindi vennero tolte loro le
    manette per i pollici. ``\emph{Da quì potete proseguire da soli.}'' fece una delle
    due guardie, che poi buttò a terra ai due delle spade rovinate ``\emph{Eccovi le
    vostre armi. Ora andate. E non pensate di tornare indietro.}'' Hax
    mosse i pollici, doloranti ``Amate proprio il sadomaso, eh?'' fece lui,
    chinando il capo verso la ragazza ``Sei incorreggibile.'' fece lei,
    raccogliendo una delle due spade ``Oh, ora si parla!'' continuò,
    tagliente, lui, mentre fece ruotare leggermente l'arma che aveva
    raccolto, per saggiarla ``Proprio una delle mie armi. Sembra un tubo di
    metallo piuttosto che una katana.'' ``Non si parla così di queste armi.
    Potranno buttarle a terra e trattarle come cose da abbandonare, ma
    hanno lo spirito del costruttore e di chi le ha usate. Sono come
    guerriere veterane.'' rispose Jam, seria come non lo era mai stata
    ``Hem. Ok.'' il ragazzo non sapeva esattamente come rispondere o che
    tono usare.

    Dopo qualche minuto di camminata arrivarono al complesso. Alla fine del
    ponte c'era il primo palazzo. Si sentivano dei mugugni venire da tutto
    intorno, come se fosse pieno di creature che lasciavano uscire quel
    verso. \emph{Hum. \'E familiare questo rumore.} pensò il ragazzo,
    guardandosi intorno. Jam sembrava particolarmente focalizzata.
    Impugnava l'arma con due mani e la teneva di fronte a lei. Normalmente
    aveva delle posture da battaglia molto più mobili. ``Jam. Che hai?''
    fece Hax, sottovoce e lei, nervosa ``Hax. In queste prove gli avversari
    sono formidabili. Non hai paura?'' ``Con tutto quello che abbiamo
    passato? Abbiamo fermato un drago meccanico.'' fece lui, alzando le
    braccia, tenendo la spada con la destra ``Eravamo in cinque.'' ``Eh. Al
    massimo scrivo due codici e voilà.'' ``Sono ancora più preoccupata
    ora.'' ``Hum. Però dobbiamo farlo. Su.'' Hax allungò la punta della
    spada, agganciò la porta in legno a scorrimento e la fece scorrere di
    scatto.

    ``Ma che cazzo di isola è questa?'' chiese, sconsolato, Hax. Jam, in
    tutta risposta, urlò dal fondo dei polmoni e cadde a terra in
    ginocchio. La stanza pullulava di cadaveri deambulanti, che
    ciondolavano. Quando sentirono la ragazza iniziarono a dirigersi
    lentamente verso i due. ``N-Non è possibile.'' balbettò terrorizzata la
    ragazza, che si teneva la testa con le mani. ``Che hai Jam? Non ti ho
    mai vista così.'' fece lui, avvicinando la mano sinistra alla ragazza.
    ``Que-Questi sono gli \emph{Spiriti che tornano a camminare}. \'E
    sacrilego solo toccarli. E sono immortali! Come fa un morto a morire
    ancora?'' ``Ma scherzi? Zombie? Ma, allora, quella volta coi
    ricercatori?'' ``Eh?''

    Durante quel discorso uno era riuscito ad avvicinarsi abbastanza da
    poter toccare la ragazza. Appena quello alzò la mano destra per
    sfiorarla lei urlò ancora dal fondo, contraendosi. Prima che arrivasse,
    però, Hax menò un colpo dall'alto al basso che fece volare via il
    braccio della creatura ``Scusa. Basta picchiarli finchè non si muovono
    più.'' continuò lui, impassibile. Non riusciva a capire la paura per
    quei cosi. Erano lenti, senza forze, non combattevano. Era il massimo.
    Si sentiva il re dei combattenti contro quelle creature, poteva fare
    delle figate senza preoccuparsi che si spostassero, al massimo cadevano
    l'uno sull'altro. Il massimo per fare bella figura. CON TUTTI. Continuò
    a colpire lo zombie a terra finchè non si muovette più. Tirò un sospiro
    di sollievo. ``Visto?'' si girò verso la ragazza, tendendole la mano,
    leggermente sporca di \emph{roba} verde, sfoggiando un sorriso ``Eh?''
    fece lei, guardandolo con occhi sbarrati, come se avesse assistito ad
    un evento impossibile ``S-Sei così forte?'' ``Che diavolo dici? Sono
    sempre stato così abile!'' ``Comunque non li posso combattere. \'E
    contrario al mio credo. Non tocco i morti.'' ``Cosa?!?'' fece, stupito,
    il ragazzo e poi, girandosi per guardare l'intero complesso che, adesso
    che guardava meglio, brulicava di zombie, urlò ``Devo ucciderli tutti
    IO?!?''

    ADESSO rimpianse il non aver sentito il discorso di The Fixer fino in
    fondo. Comunque, non era quello in tempo per piangere sul latte
    versato. Guardò la stanza dentro la quale stava per buttarsi. C'erano
    un sacco di modi per eliminare quegli stronzi in maniera creativa e lui
    non voleva rischiare di farsi del male con la katana. Si era già fatto
    del male da solo con delle spade, non voleva ripetere l'esperienza. Si
    sentiva stupido. Corse avanti, saltò, mise il piede sul limite di uno dei pannelli
    che formava il pavimento in quella stanza, il quale fece perno al
    centro e, facendo leva, staccò la testa di uno zombie col lato opposto,
    il quale smise immediatamente di muoversi.
    Hax si girò verso sinistra, chinandosi. Caricò il colpo con la spada,
    per poi rilasciarlo contro il bacino di un altro cadavere, sfondandogli
    le ultime ossa che rimanevano, lasciandolo a terra. Girando la testa
    notò una decorazione appesa al soffitto che sembrava molto pesante. Con
    uno scatto lanciò la katana, facendola roteare, verso il cavo che la
    teneva attaccata al soffitto, tranciandolo di netto. L'oggetto, che
    sembrava un vaso in ghisa, cadde su un gruppo di zombie, schiacciandoli
    all'istante. Corse verso il vuoto che si era fermato, saltò sul vaso ed
    aggrappò un bastone ornamentale attaccato al muro. Dovette far forza
    con le gambe sul muro per staccarlo, in quanto pareva non venir via.
    Quando lo strappò via cadde all'indietro e buttò a terra tre zombie col
    suo corpo ed il bastone. Scattò in piedi, tirò un pestone al volto
    della creatura che aveva direttamente sotto e poi, con due colpi
    piazzati, sfondò i crani degli altri due con gli estremi dell'asta.
    Iniziò a far roteare sopra la testa il bastone come gli aveva insegnato
    Elythia. Un giro e mezzo per senso, poi attacco. Prima a destra, poi a
    sinistra, con tutto il corpo. Ne sfontì un gran numero prima che il
    bastone si spaccò. ``Stupide armi d'esposizione.'' fece, tenendo in
    mano metà del bastone, che tirò contro uno zombie. Quello non fece una
    piega. Dovette arrendersi quando Hax gli tirò un colpo in faccia con
    una teiera in ghisa. ``Questo resiste.'' osservò, notando che non si
    era fatta neanche un graffio. Passò dieci minuti a picchiare creature,
    buttandole anche giù dalla struttura, sfasciando le teste contro le
    colonne, sfondando pareti, utilizzando mobili, soprammobili, tutto. Poi
    finirono.

    Uscì dalla stanza affaticato ma baldanzoso e si diresse verso Jam.
    ``Fiu'' fece, asciugandosi il sudore dalla fronte ``Finiti. Allora,
    andiamo?'' e le allungò di nuovo la mano. Lei si alzò aiutandosi con la
    mano del ragazzo e poi rispose ``Cavolo. Certo. Comunque temo ce ne
    saranno altri.'' ``Altri?!?'' ``Cosa pensi servano tutte quelle
    stanze?'' ``Hem. Ristoro?'' ``Alcune. Temo che dovrò chiederti di
    picchiare te tutti gli \emph{zombie} che troveremo.'' Il ragazzo
    squadrò il complesso, per quello che poteva vedere da dove si trovava e
    rispose, alzando il ciglio destro ``Ah, certo. Figuriamoci se non ce la
    faccio.''

  \section{Hax e Jam - Sanpan - Agosto 2635}
    
    Era ormai quasi sue ore che andavano in giro per il complesso a visitare
    stanze e svuotarle dagli zombie. Anche se, ormai, si poteva dire che
    fosse solo Hax ad ammazzarli. Aveva anche trovato un ottimo modo. Aveva
    capito che, se staccava loro la testa o gliela rompeva, si fermavano
    subito. Erano arrivati alle ultime stanze, ormai.

    ``Fiu.'' fece il ragazzo, pulendosi la fronte dal sudore ``Dai, ce
    l'abbiamo quasi fatta. Ora andiamo in quelle stanze, ammazziamo qualche
    altro zombie.'' ``No, aspetta.'' ``Già, hai ragione. Magari sveniamo,
    prima, eh? Sono rovinato. Ne avrò ammazzati a migliaia. Avrei dovuto
    contarli. Chissà se ci danno qualcosa di più per averli liberati da
    tutta questa feccia.'' ``Non penso proprio che ci siano \emph{bonus di
    produttività}.'' ``Hum, dici? Bene. Non voglio più stare su questa
    struttura, mi sto rompendo le palle.''

    Il ragazzo raccolse la spada che, incredibilmente, era riuscito a
    tenersi dall'inizio della prova senza averla rotta, buttata giù da una
    delle piattaforme oppure incastrata da qualche parte senza poi
    riuscire più a tirarla fuori, per quindi dirigersi verso la passerella
    che li avrebbe portati alla, probabilmente, ultima costruzione della
    struttura. ``Aspetta, Hax.'' fece Jam, rincorrendolo. Arrivarono
    davanti alla porta dopo una lunga passerella, quì Hax allungò la mano
    e, con una spinta, aprì la porta in uno scatto, per non avere secondi
    pensieri.

    ``Hax? Che succede?'' chiese Jam, mentre notò che il ragazzo aveva
    ancora la mano sinistra nell'incavo per aprire la porta. Il ragazzo era
    bloccato, sembrava avere dei tic all'occhio destro. Subito chiuse la
    porta, sbattendola contro l'anta che non aveva neppure mosso. Senza
    muoversi da quella posizione chiese, sull'orlo di una crisi di nervi
    ``Jam. Che isola è questa?'' ``Sanpan.'' rispose lei, impassibile ``Ok.
    Domanda scema, in effetti. Senti, perchè avete tutti questi problemi?''
    ``Perchè? Che c'è lì dentro?'' ``Voglio dire, io non ho
    grand\emph{issimi} problemi con l'ammazzare degli zombie, ok, ma ora mi
    pare scorretto.'' ``Hax? Che vuoi dire?'' ``Voglio dire che c'è un
    FOTTUTO zombie gigante là dentro. Sarà grande come l'intera struttura,
    cazzo. Che facciamo?'' ``Lo ammazzi?'' ``Ah, grazie, JAM. Oh, 'fanculo.
    Ho capito. Poi non lamentarti se faccio esplodere questa struttura. Me
    l'avete chiesto voi.''

    Hax cambio paradigma. Vedeva chiaramente la creatura al di là delle
    porte. Troppe \emph{suture} matematiche per venir fermate da una porta
    in legno, alla fine. Iniziò a ragionare su che algoritmo utilizzare, se
    uno per fare danni diretti o per accelerare la spada, o per sperare di
    sfondare il pavimento, bloccandoci dentro il gigante di carne. Iniziò a
    creare delle righe di codice raccogliendo quà e la dalla memoria quali
    fossero le regole ed i blocchi per generare campi elettromagnetici.
    Voleva evitare di manipolare energie che, normalmente, portavano a
    risultati un po' troppo \emph{pirotecnici}, anche perchè aveva notato
    che era pià facile fare errori con la scrittura di algoritmi che ne
    facevano uso. Certo. Era anche vero che modificando troppo i lagami
    elettromagnetici degli oggetti nella realtà rischiava di far partire
    reazioni a catena e cose simili, ma la realtà aveva una certa
    \emph{inerzia} per queste cose e quindi non si rischiavano
    annichilazioni a cascata. Inoltre aveva visto come lavorare con le
    forze elettromagnetiche solo un mese e qualcosa fa e l'ultima volta che
    l'aveva utilizzato come algoritmo, contro il golem, non aveva avuto
    nessun effetto collaterale. Questa volta, però, decise di fare una cosa
    differente. La tecnica era simile a quella della catapulta
    elettromagnetica usata con la macchina, appunto. Creò un disco bianco
    brillante, il quale era collegato ad una serie di codici per la
    generazione istantanea di un campo elettromagnetico con un'intensità
    abbastanza grande per attirare un oggetto metallico pesante come la
    spada. L'idea era far muovere la spada ad una velocità di una decina di
    furlong al secondo. Con quello avrebbe ammazzato la creatura in un
    istante. \emph{Best plan EVER!} pensò Hax mentre completò il disco
    bianco applicandoci delle patch e delle porte di connessione. Aveva il
    suo protipo. Avrebbe dovuto testarlo, ma non aveva tempo. Creò un
    secondo disco, nero che emetteva un alone viola, che faceva la stessa
    cosa di quello bianco, ma con una polarità invertita. Li collegò tra
    loro per fare in modo che si attivassero allo stesso momento. Se non
    l'avesse fatto la spada sarebbe uscita dal percorso. E mandare l'unico
    proiettile che aveva in orbita non gli interessava. Ovviamente era
    l'unico. Se avesse voluto creare un'arma multi-purpose sarebbe
    incappato in una serie di problemi che avrebbero aumentato la
    probabilità di bug. Replicò la coppia di dischi una decina di volte.
    Quattro per circondare l'arma ed altre sei per creare una specie di
    tunnel. Più lungo era più precisione avrebbe avuto. Almeno, frase di
    Fixer per quanto riguardava le armi da tiro. L'ultima cosa che fece fu
    creare una routine che faceva ruotare lungo l'asse di tiro le coppie di
    dischi, collegate tra di loro da dei timers. L'idea era questa. La
    rotazione avrebbe mantenuto l'arma esattamente sulla linea d'attacco,
    mentre la temporizzazione avrebbe permesso di avere un'accelerazione
    uniforme della spada. Sarebbe dovuto rimanere in paradigm-shift per
    tutto il tempo se voleva vedere dove stava puntando. Non aveva molto
    tempo, una volta riaperta la porta. La creatura sapeva che lui era là.
    Non avrebbe aspettato stavolta. Niente discorsi tra personaggi fighi
    delle storie. Hax odiava quelle storie.
    
    Collegò la direzione dell'arma
    rispetto alla direzione del braccio, quindi ora girava con la katana
    che fluttuava lungo il braccio destro, come se fosse un'estensione del
    suo corpo. ``Ok, Jam. Pronta per i botti?'' fece lui, tirandosela, con
    la mano sinistra appoggiata al fianco e il braccio destro appoggiato
    alla spalla. La
    ragazza lo aveva guardato durante tutto il procedimento di sviluppo del
    sistema di attacco. Si ricordò delle parole di Lissa, guardandolo,
    mentre negli occhi del ragazzo fluttuavano dei pallini di vari colori e
    scie luminose che scorrevano da un lato all'altro del suo sguardo.
    ``Hax non è proprio uno preciso nel suo lavoro'' le disse ``ma se mai
    si trovasse nei casini potrebbe venirsene fuori con un piano che lo
    tirerà fuori dalle peste. Piano che a me non passerebbe neanche per
    l'anticamera del cervello. \'E il problema di essere troppo precisi e
    di conoscere troppo a fondo gli algoritmi. Sai che una cosa è sbagliata
    perciò non ci provi neppure.'' Jam guardò interessata il risultato del
    lavoro del ragazzo e pensò \emph{Provare cose che sono pericolose,
    eh?} ``Sicuro Hax che quella cosa non ti staccherà il braccio?'' chiese
    lei, preoccupata ``Hum. Ecco, penso che questo proprio no. \'E l'arma
    che è collegata al braccio, non il contrario. Se sposto l'arma questa
    ritorna al suo posto, vedi?'' fece, lui, dando una spinta con la
    sinistra alla spada, la quale galleggiò via dal braccio per ritornare
    al suo posto dopo poco tempo, come se fosse collegata ad una molla
    ``\'E un collegamento monodirezionale.'' ``S-Se lo dici te, Hax. Io
    l'avrei ucciso normalmente.'' ``\'E un gigante. Io faccio fatica con
    quelli piccoli.'' Rispose, per poi avviarsi verso il portone. Il piano
    era facile. Apriva di scatto, mirava alla testa, che sapeva già dove si
    trovava, a spanne, e poi spediva nell'oblio quella creatura a suon di
    acciaio accelerato elettromagneticamente.

    ``Ha!'' fece, come se avesse spiegato il piano a qualcuno. Corse verso
    la porta, tenendo lo sguardo verso la nube di formule matematiche che
    dovevano tenere sotto controllo la testa della creatura. Infilò la mano
    nella fessura che serviva per aprire la porta, l'aprì di scatto,
    s'inginocciò, per avere più mira, alzò il braccio destro, tenendolo
    fermo con la mano sinistra. La creatura si accorse di lui
    immediatamente e si girò nella sua direzione, camminando verso di lui.
    Fortunatamente la sua mole non gli permetteva di muoversi molto in
    fretta. Hax direzionò il braccio in modo che le serie di anelli che
    aveva programmato in modo che indicavassero il percorso iniziale
    dell'arma attraversassero la testa della creatura. Se funzionava come
    tutte le altre sarebbe morta in un colpo e poteva mandare a 'fanculo
    quella prova del cazzo e quel covo di matti. Prese un respiro per
    rendere ancora più precisa la mira ed eseguì l'algoritmo.

    Il tempo sembrò come rallentare. Mentre i dischi iniziarono a ruotare
    ed ad attivarsi assieme ai timers, la creatura allungò il braccio per
    raggiungere il ragazzo, mettendolo in mezzo alla traiettoria del colpo.
    Uno alla volta i dischi s'illuminarono, la spada si mosse dalla sua
    posizione, appena rilasciata dal vincolo che la teneva vicina al
    braccio di Hax. I primi tre dischi non diedero problemi mentre quando
    si attivò il quarto s'illuminnarono delle scritte che il ragazzo aveva
    abilitato apposta come variabili per la monitorizzazione di fattori
    fisici, tra cui l'energia applicata sull'arma. Se i valori non fossero
    andati oltre i dieci furlong al secondo e l'energia d'impatto non
    avesse ecceduto gli otto MegaJoule allora le scritte sarebbero dovute
    rimanere azzurrine, se questi valori fossero scesi troppo in basso o
    troppo in alto sarebbe dovuto esser stato avvisato da quelle scritte.
    Non ci volle molto prima che la velocità superasse i quindici furlong
    al secondo e che l'energia d'impatto oltrepassasse i dieci MJ. Si
    formarono delle distorsioni dell'aria dove passava l'arma, aveva
    sfondado poco prima il muro del suono, probabilmente l'avrebbe superato
    di una decina di volte. Doveva
    aver mantenuto dei dati utilizzati per la macchina. Era molto più
    pesante ed aveva bisogno di un'accelerazione differente da quella
    necessaria per una spada. Purtroppo tutta quell'energia avrebbe
    scaturito una grande quantità di plasma. Hax vide l'arma perforare il
    braccio della creatura come se fosse burro, mai esistito, mentre questa
    si lasciava dietro di se una nube di fiamme verdi date dal gas
    ionizzato dall'energia. Mentre le fiamme gli si avvicinavano si sentì
    trascinare dal collo del gilet, mentre cadeva all'indietro vide Jam
    che, evidentemente, aveva superato il terrore degli zombie e lo aveva
    preso per buttarlo a terra. Si vide passare sopra la nube di gas
    ionizzato, la quale incendiò il soffitto ed i muri della struttura,
    andando verso l'alto.
    
    Improvvisamente tutto tornò alla velocità normale, sentì la schiena
    colpire il pavimento, Jam urlò mentre era a terra con le mani sulla
    testa per evitare il fuoco, mentre c'erano rumori di legno che andava
    in frantumi sempre più in distanza, mentre il fuoco stava mangiando la
    struttura. Ci sentiva ancora. Fortunatamente l'architettura di questa
    zona non prevedeva spazi chiusi, altrimenti ci avrebbe rimesso l'udito
    quasi definitivamente. Alzò la testa per vedere che fosse successo, per
    scoprire che la spada aveva attraversato il braccio del gigante, gli
    aveva sfondato il cranio, aveva attraversato la costruzione,
    continuando a perforare muri, ignorando qualunque cosa ci fosse sulla
    sua traiettoria, lasciando una scia infuocata e buchi perfettamente
    circolari, fino ad impiantarsi nelle rocce ad uno, due Furlong di
    distanza. Mentre stava guardando il corpo senza metà del petto e tutta
    la testa del gigante sentì un forte rumore di rottura sopra di sè e,
    senza neanche pensare, si buttò sopra Jam per proteggerla in caso fosse
    loro caduto addosso una trave che formava il soffitto. Sentì cedere il
    tetto. Una trave gli cadde sulla gamba destra. Grugnì per il dolore,
    aggiungendo un ``Cazzo!'' che veniva direttamente dal cuore. Era da un
    po' che voleva dirlo. Fortunatamente non sentì
    nessun rumore di ossa rotte, però non poteva muoversi. Si rannicchiò pensando che, diminuendo lo spazio occupato la
    probabiltià di venir schiacciato sarebbe stato più basso possibile. Si
    sentirono dei rumori di estrazione di katana e di fendenti tirati ad
    una velocità impressionante e poi legno che cadeva intorno ai due.

    Hax stava ancora aspettando che qualcosa gli cadesse in testa quando
    una voce gli chiese, con un accento stranissimo ``Yo, 'sup, BRO. Tutto a
    posto là sotto, bello?'' Hax si interrogò, prima di aprire gli occhi,
    sul chi parlasse in questa maniera credendo di essere  figo. ``S-Sì.
    Grazie.'' rispose lui, affaticato, mentre si girò per guardare in volto
    chi gli avesse salvati. Si trovò a guardare dal basso verso l'alto un
    uomo che avrà stato sulla sessantina, scuro di pelle come tutti gli
    abitanti di quelle isole, con un afro bianca, il quale stava fissando
    il panorama al di fuori della costruzione facendo in modo che il suo paio d'occhiali da
    sole riflettessero il sole direttamente in faccia ad Hax.
    Teneva una katana appoggiata alla spalla destra con il suo braccio
    destro completamente tatuato con un pattern tribale complessissimo.
    ``Ho visto che ci sai fare, Coso.'' continuò il vecchietto, rimettendo
    nel fodero la spada con un movimento fluido ``Eh, g-grazie.'' rispose,
    di nuovo, il ragazzo, spostandosi per lasciare spazio a Jam ``Allora ho
    fatto bene a far girare il mondo alla piccola, pare.'' ``Scusi?'' Jam,
    che doveva aver perso temporaneamente i sensi, si riprese e, dopo aver
    rimesso a fuoco la scena, notò l'uomo ``Ah! Maestro!'' disse, sorpresa
    ``'Sup.'' rispose lui ``S-Sono tornata.'' ``L'ho notato, bella. E ti
    sei portata dietro un tizio che non si trattiene dal distruggere il
    male, noto.'' ``Se è per la costruzione, posso spiegare.'' ``Non è
    mia, non ti preoccupare, BRO.'' ``Come mai è quì, Maestro?'' ``Perchè
    non potevo lasciarvi morire per una trave, Duh.'' ``Vuol dire che ci
    hai seguito per tutto il tempo, vecchio?!?'' ``Beh, sì.'' ``E non hai
    fatto niente.'' ``Ovvio, ragazzo, che ti credi? \'E una prova
    ufficiale!'' ``Ma che prova della minchia, scusa? Se fossimo morti per
    gli Zombie?'' ``Oh, figurati, non sei morto neanche per quel gigante
    là. Per poco non ti ammazzavi da solo. Sentite, andiamocene prima che
    crolli tutto e prima che quelli mi scoprano.''

    Hax non se lo fece ripetere due volte, si tirò in piedi aiutandosi con
    i muri e, tenendosi in piedi a forza, aiutò Jam a rialzarsi, anche se
    pareva non aver bisogno di una mano, visto che sembrava più pimpante di 
    lui, svenimento a parte. Alla fine venne aiutato dalla ragazza la quale
    commentò che quello era il minimo che poteva fare, visto che, in
    pratica, l'aveva scagionata dall'accusa a suon di botte sugli zombie
    per più di due ore. Una volta usciti dall'altra parte del complesso là
    si trovavano delle guardie ed uno degli anziani che li osservò uscire
    con gli occhi sbarrati. L'uomo era scomparso, nessuna traccia. Aveva
    detto loro che li avrebbe reincontrati più tardi, dopo il loro
    rilascio. ``\emph{S-Siete usciti vivi?}'' chiese, incredulo, l'anziano del
    villaggio ``\emph{Certo.}'' rispose, gonfiando il petto, Jam
    ``\emph{Grazie al mio Protettore.}'' L'anziano guardò il ragazzo che
    si trascinava rovinato e poi guardò le guardie, le quali alzarono le
    mani, poco convinte a loro volta. L'anziano, continuando a non
    crederci, iniziò a recuperare dei documenti e continuò a parlare con
    Jam ``\emph{Ma come avete fatto con tutti gli Spiriti che Tornano a
    Camminare? Inoltre avete distrutto metà della struttura superiore.}''
    ``\emph{Ve l'avevo detto che il mio Protettore era abile
    al livello di una Lady come la sottoscritta. Inoltre l'ultima creatura
    che avevate predisposto era fuori da qualunque livello comprensivo.
    Ragionate sui vostri errori.}'' L'uomo la guardò ancora
    più straniato e, dopo aver completato i documenti per il rilascio e la
    dichiarazione d'innocenza tramite prova fece, allungando le carte alla
    ragazza ``\emph{\emph{Lady} Jam, ecco a Lei. Non riesco a crederci.
    Siete i primi che superano questa prova quando vengono impiegati gli
    Spiriti.}'' ``\emph{Grazie mille, Signore. Siamo onorati.}'' rispose
    lei, prendendo i documenti con due mani, per poi recuperare il braccio
    destro del ragazzo ed allontanarsi.

    Quando furono abbastanza lontani Hax alzò la testa e disse ``Ha! Hai
    visto la faccia del tizio quando siamo usciti?'' ``Hm-hm.'' rispose
    lei ``Dovevi dirgli \emph{\'E per questo che siamo dei Rollers!}. Sai
    quanto rosicava?'' Hax rise per poi tossire. Evidentemente aveva male
    da qualche parte. Dopo un po' che stavano camminando verso la città
    sbucò dalla foresta il Maestro ``Il ragazzo non sta bene?'' fece,
    camminando a fianco di Jam, dalla parte dove non c'era Hax ``Eh.''
    rispose il ragazzo, ansimando ``Devo essermi spinto un po' troppo visto
    che \emph{qualcuno} non si è degnato di darci una mano.'' ``Oh,
    smettila. Dovevo testare per vedere che tu fossi davvero adatto come
    \emph{Protettore} per la mia bambina.'' ``Ma che stronzata! E se fossi
    morto?'' ``L'avrei salvata io.'' ``Oh.'' ``Tutto a posto, comunque,
    noto, se riesci a litigare con me.'' ``Eh, non pro\dots{}'' il ragazzo
    non riuscì a finire la frase che svenne e si accasciò contro Jam
    ``Hax!'' urlò lei, appoggiandolo a terra. ``Oh. Ho capito. Lo prendo
    io, cara.'' fece il Maestro, abbassandosi e lo caricò sulle spalle.
    ``Si riprenderà?'' chiese, in apprensione, la ragazza ``Ma sì, certo.
    Diavolo. \'E sopravvissuto a quella\dots{} cosa che ha lanciato.''
    ``Era una katana, Maestro.'' ``Oh. Dovrò andare a recuperarla. Se è
    ancora intera è diventata una qualche reliquia epica per quanto ha
    visto.'' Il Maestro era uno che credeva che nelle spade ci fossero le
    esperienze di chi le aveva usate per il combattimento, quindi qualunque
    cosa strana fosse capitata ad un'arma formava la sua ``epicità''.
    ``Comunque adesso andiamo a casa mia, che mi devi raccontare un sacco
    di cose e mettiamo a posto il tuo amico, visto che ci siamo.'' e si
    avviò in mezzo alla foresta. ``Sì, grazie.'' rispose Jam, rincuorata
    dall'invito.

  \section{Rollers senza Hax e Jam - Qemfom - Agosto 2635}

    Erano finalmente arrivati ad un'isola abbastanza grande da poter
    accettare la nave da crociera che avevano preso. Appena arrivati
    Elythia e Fixer erano andati in giro per cercare informazioni per un
    team di ricerca o solo per l'affitto di una nave che potesse andare da
    là alla zona dove si erano dispersi i due del gruppo.

    Lissa era rimasta ad un bar del porto a fare quello che sapeva far
    meglio: bere ed ascoltare. Dopo un'enorme quantità di discorsi
    assolutamente inutili al suo scopo notò una persona che, una volta
    entrato, si lamentò col barista delle sue \emph{solite sfighe}. Il
    discorso suonava un po' tipo ``$[$..$]$il solito, dunque?'' ``Sì.''
    ``Trovato lavoro?'' ``Macchè? Ormai pare che diano lavoro ai Tuttofare
    o ai Riparatori.'' ``Ma smettila di dire stronzate.'' ``No, è vero.
    Dai, guarda le bacheche!'' ``Da quanto ti ha lasciato quella che
    lavorava per un gruppo di Tuttofare sei sempre nelle sale della gilda.
    \'E ovvio che non trovi lavori differenti da quelli.'' ``Sì, ma che
    vuoi che faccia uno come me da queste parti se non lavorare per gente
    così?'' ``Eh, tra l'altro mantenere un'airship non dev'essere cosa da
    poco.'' Lissa fermò il discorso immediatamente intromettendosi. ``Ha!
    Aspettate. Non ho potuto fare a meno di sentire il vostro discorso.''
    ``Scusi? Ma era dall'altra parte della sala, signorina.'' rispose il
    barista, che intanto aveva portato il drink all'avventore ``Non
    importa. Allora, signore. Lei ha bisogno di un lavoro.'' ``Sì?''
    chiese lui, preso alla sprovvista ``Ottimo!'' rispose lei, euforica
    ``Perchè io ed i miei compagni abbiamo giusto bisogno di un trasporto
    per la ricerca di due dispersi nelle isole del sud di
    quest'arcipelago!'' ``Ma che scherza, signorina?'' rispose il barista,
    immediatamente ``Le isole di Sanpan sono impossibili da sorvolare,
    soprattutto visto che non hanno spazi abbastanza grandi per
    l'atterraggio.'' ``Oh, davvero?'' rispose Lissa, seriamente abbattuta.
    Non sapeva che ci fossero così pochi spazi per l'atterraggio ``Beh,
    allora, addio possibilità. Vabbeh. Grazie lo stesso.'' e si alzò dallo
    sgabello sul quale si era seduta per intrattenere quella discussione.

    Mentre si allontanava la mano dell'aviatore sconosciuto la prese per la
    spalla destra e questo le disse ``Aspetti. Chi le ha detto che io debba
    atterrare sulla terra?''

  \section{Rollers senza Hax e Jam - Cieli dell'arcipelago di Sanpan -
    Agosto 2635}
    
    ``Hahahaha!'' rise Lissa, mentre guardava l'alba fuori dal finestrino laterale
    della airship del tizio che avevano ingaggiato per portarli da quelle
    parti ``Vedrai appena diremo ad Hax che dovrà salire con noi su
    quest'airship! Muore!''

    La airship era un veicolo molto piccolo, da trasporto passeggeri e
    merci piccole. Sembrava fatto apposta per il trasporto di archeologi,
    mercanti e cose simili. Aveva delle slitte per l'acqua, così poteva
    atterrare anche in laghi, nel mare e cose così. Aveva quattro eliche,
    due per ala, ed aveva un generatore di vapore che occupava i due lati
    della sezione principale dove erano collegate le ali, lasciando un
    corridoio che collegava la parte per i passeggeri alla cabina di
    pilotaggio. Lissa rimaneva sempre stupita da quanto bello fosse vedere
    il mondo da uno di quegli affari. Le isole che scorrevano sotto di loro
    ad una velocità mozzafiato, le nuvole che sembravano dei cumuli di
    cotone, lo scenario che s'illumina con il sole che sorge. Aveva
    utilizzato più volte gli aerei, soprattutto per il trasporto in zone
    dove doveva compiere lavori abbastanza pericolosi. Non li aveva mai
    usati per questo tipo di \emph{turismo}.

    The Fixer ogni tanto guardava giù, per vedere dove si trovavano, ma non
    gli faceva nè caldo nè freddo. Non che non gli sembrasse una bella
    esperienza, ovvio, ma non da fare così tanto casino come la ragazza.
    Elythia, invece, era in parte preoccupata, in parte presa dal viaggio.
    Anche lei voleva goderselo come Lissa, però non riusciva a pensare che,
    forse, qualcosa era capitato ad Hax ed a Jam. Avevano deciso che, come
    miglior corso d'azione, sarebbero arrivati a Sanpan assieme al pilota,
    il quale diceva di saper parlare la lingua del luogo, e lì chiedere se
    avessero visto oppure solo sentito parlare dei loro amici.

    Nel mentre Lissa aveva iniziato anche a scandagliare con lo sguardo
    un'isola alla volta. Se Hax avesse utilizzato degli algoritmi ci
    sarebbero dovuti essere dei suoi fingerprint in paradigm-shift.
    Ovvimente anche solo la traccia vitale poteva essere un fingerprint, ma
    questa la poteva vedere solo da abbastanza vicino.
    
    Ci vollero tre ore d'airship per arrivare nelle vicinanze di Sanpan.
    Era una grande isola oblunga, con una serie di montagne, circondata da
    una moltitudine di isolette. Le barche parevano muoversi senza problemi
    tra le isole. La vegetazione sembrava essere quella tipica di una
    jungla tropicale, come molte altre isole in questa zona. ``Opporc!''
    esclamò Lissa, guardando in direzione dell'isola. ``Hei, hei, Elly! Che
    vedi in quella direzione?'' continuò, indicando in direzione dell'isola
    ``Hem, vedo delle case distrutte da delle rocce e da mezzo galeone.''
    ``E da quella?'' Elythia osservò nell'altro punto indicato ``Ma che?
    C'è una serie di costruzioni su delle palafitte che sono state
    perforate in linea da una cosa che sembra aver continuato il
    percorso\dots{} Per finire dentro la roccia.'' ``HAHA!'' Lissa scoppiò
    a ridere ``Quel dannato non si da tregua, eh?'' ``Come?'' ``Lì ci sono
    tracce dei Fingerprints di Hax. Incredibile. Deve aver sbagliato lì
    dove c'è quella fila di distruzione in mezzo ai palazzi.'' ``Ed\dots{}
    \'E morto?'' ``Ma figurati! Certo. Ha fatto un sacco di danni, quindi
    non so che ne avranno fatto. Potremmo doverlo tirare fuori dalla
    galera. Vabbeh. Sentiremo Jam una volta giù.''

    Elythia fu subito sollevata di sentire quelle parole. Significava che
    non solo si erano salvati, ma che erano arrivati direttamente
    sull'isola corretta. Sperava solo che Hax non avesse fatto arrabbiare
    troppo la popolazione locale. Fortunatamente non capiva la lingua
    quindi non avrebbe potuto prendersela per la condizione di chi viveva
    là e quindi far scoppiare una rivolta, inoltre aveva Jam che parlava,
    quindi lei avrebbe mitigato i discorsi. Era una brava ragazza, aveva
    capito alla perfezione i discorsi che Elythia le aveva fatto quando
    discutevano sul come trattare col ragazzo.

    Riuscirono ad atterrare con qualche scossone in mezzo alla baia della
    capitale di Sanpan, che pareva essere nel bel mezzo di qualche
    riparazione. Evidentemente i danni si erano verificati non molti giorni
    prima. Un motivo in più per ben sperare per la sorte dei due. Una volta
    giù il pilota fece scivolare in acqua una scialuppa dal retro
    dell'airship, che si apriva come fosse una bocca, formando una rampa
    verso il retro dell'aereo per quello che si trovava all'interno.
    Salirono sulla scialuppa ed arrivarono ai moli. Il pilota salutò la
    gente con la lingua locale, poi chiacchierò un po' con i marinai che
    erano presenti ai moli. Dopo un po' arrivò e commentò ``Cavolo, se
    questo è stato veramente il vostro amico. Beh. C'è da fargli il
    cappello se non l'hanno ammazzato. Mi hanno detto di andare a parlare
    con gli anziani del villaggio per quanto riguarda la fine che ha fatto
    il ragazzo con l'amica che pareva venisse da queste parti.''
    Proseguirono verso la sala del consiglio, dove erano presenti alcuni
    degli anziani del villaggio.

    Fatte le debite presentazioni il pilota si propose come traduttore per
    il gruppo e così iniziarono a discutere. ``Chiedono come mai siete
    venuti quì.'' ``Siamo quì perchè dobbiamo recuperare Hax e Lady Jam,
    due amici che sono stati dispersi in mare una cosa come una settimana
    fa.'' rispose concisa Elythia. Sapeva come si trattava con gli anziani
    dei villaggi. Dopo che il pilota tradusse la frase agli anziani, questi
    ebbero un sussulto e quindi risposero ``Beh, a quanto pare non vogliono
    più averci a che fare con quei due.'' ``Chiedigli che hanno fatto, se
    possiamo rimediare.'' ``Hm. A quanto pare hanno bloccato l'avanzata di
    un golem facendolo saltare in aria da dentro e quindi, dopo che li
    avevano arrestati, hanno superato la \emph{Prova dell'Innocenza},
    distruggendo un terzo del tempio che l'ospitava da tempi immemori.
    Però, grazie a quello, pare siano stati liberati dalle loro colpe.''
    ``Chiedigli se sa dove si trovino.'' ``Ha detto che, dopo la
    distruzione causata alla prova e dopo aver preso la dichiarazione
    d'innocenza sono scomparsi.'' ``Hm. Ma dannazione. Non possono starsene
    fermi quei due? Vabbeh. Ringraziali e digli che ce ne andremo appena li
    avremo trovati.'' ``Dice che è meglio se ce ne andiamo il prima
    possibile. Non vuole più vedere donne coi capelli lunghi o uomini coi
    capelli corti.'' ``Non gli piacciono?'' ``Beh, almeno i miei gli vanno
    bene.'' aggiunse, scherzosamente, Lissa ``Hm. Digli che lo faremo.''

    Dopo essersene andati dalla sala del consiglio decisero di raggiungere
    il luogo dove Hax aveva creato la scia di distruzione. Una volta là
    notarono che doveva aver creato qualche tipo di fascio energetico
    direzionato. Il fatto è che la quantità di danni era tale che l'avrebbe
    ucciso se fosse stato un qualche elemento standard. Seguirono con lo
    sguardo i danni in fila, fino ad arrivare al solco lasciato sul fianco
    della collina sul quale era stato costruito il complesso. Finiva al
    centro di una specie di cratera. Doveva aver spostato o distrutto una
    gran quantità di roccia per poi esser stato estinto. Arrivarono sopra
    il cratere, dal quale uscivano le imprecazione di un uomo, che pareva
    star facendo qualche grosso sforzo. ``Cazzo, BRO. Certo che l'hai
    cacciato dentro proprio duro.'' faceva, tra i versi dati dagli sforzi.
    Lissa prese l'iniziativa e scese lungo la corda che il signore doveva
    aver utilizzato per scendere. Arrivata giù vide un uomo abbastanza
    anziano con un afro bianco. ``Hem. Signore, scusi? Capisce quello che
    dico?'' fece lei, cortese ``Ma certo, squinzia. Che ti credi? Non vedi
    che sono impegnato?'' rispose lui, scortese ``S-squinzia?'' si chiese
    Lissa, stringendo il pugno destro ``Senti, Vecchio, ma con chi cazzo
    ti credi di parlare, eh?'' ``Con una giovinastra che non mi lascia fare
    il mio lavoro, ecco con chi credo di parlare.'' Lissa si avvicinò,
    irritata, pronta a far rissa con un vecchietto appena incontrato ``Che
    stai facendo, vecchio? Forse posso aiutarti, visto che non mi pari
    capace.'' ``Oh? Davvero? Parli proprio come un tizio che conosco. Bene.
    Tirami fuori questa spada.'' e si spostò, mostrando l'impugnatura di
    una katana impiantata fino alla guardia nella pietra ``Qualche idiota
    ha ben pensato d'impiantarla quì dentro per ammazzare una creatura.''
    ``Si dia il caso che conosca l'idiota.'' fece, Lissa, cambiando
    paradigma, ponendo due dischi, uno sopra l'altro, tra guardia e muro.
    Con un impulso fece muovere la spada e poi l'estrasse. ``Ecco a lei
    l'arma. Anche se non so che se ne farà. Avrà perso tutto il filo.''
    ``Hmph. Uguali. Quest'arma, mia cara, avrà perso il filo, ma non lo
    spirito.'' poi, guardandola in faccia, ``Mentre noto che qualcuno ha
    qualche crepa nel proprio.'' ``Cosa?!?'' ``Eri preoccupata, sì?''
    Lissa fu colpita nel profondo. Aveva nascosto la preoccupazione dietro
    alle storie del fatto che Hax se l'era sempre cavata, però non poteva
    mostrarsi debole, altrimenti Elythia sarebbe crollata ``Hem. Uff. A voi
    Vecchi con l'afro non si può proprio nascondere
    nulla, eh?'' ``Ne conosci molti?'' ``Ovviamente no.'' ``Ovvio, vorrei
    vedere. Io sono l'unico e più figo Maestro.'' ``Non lo metto in
    dubbio.'' ``Dovrai risolvere anche le altre crepe, signorina, comunque.
    Venite, te ed i tuoi amici lì sopra, vi porto dai due scapestrati.''

    Dopo essere salito ed aver fatto le dovute presentazioni, il vecchio
    Maestro accompagnò il gruppo alla sua casa in mezzo alla foresta.
    ``$[$..$]$ed è stato così che Jam mi disse \emph{Maestro, voglio
    diventare una maid.}.'' finì il vecchio Maestro, mentre entrò il casa,
    seguito dagli sghignazzi dei tre che lo seguivano. ``Bene.'' fece
    l'uomo, una volta dentro ``Aspettate quì, vado a chiamarveli.'' I tre
    aspettarono per cinque minuti nella veranda della costruzione in mezzo
    alla jungla, palpitanti. Ad un certo punto si sentì un rumore di passi
    che si avvicinavano a gran velocità nella loro direzione e poi,
    dall'entrata, uscì di corsa Jam che saltò al collo di Elythia
    ``Signorina Elythia!'' urlò, contenta, con le lacrime agli occhi ``Sono
    così contenta di vederla!'' poi, fece cadere le braccia dal collo della
    dottoressa e, dopo averle lasciate a penzoloni lungo i fianchi,
    continuò mesta ``Scusatemi per quello che ho fatto. Se non fosse stato
    per me ci saremmo goduti la vacanza. Non volevo. Non so che mi ha
    preso.'' Elythia la guardò, severa, e poi, dolce, fece, abbracciandola ``No, non è
    colpa tua. Avevamo paura di non vederti più. Sono così contenta di
    vedere che stai bene. Oddio. Non sai quanto ero preoccupata.'' Durante
    quella discussione c'era stato un rumore più basso, ritmico, di una
    stampella, venire dalla veranda. Da uno degli angoli della casa uscì
    Hax, fasciato, con una stampella per la gamba destra. Faccia incazzata
    come al solito, stanca. ``Meh. Eravamo così preoccupati per te, Hax.''
    commentò, acido, a bassa voce. Lissa appena lo vide gli si avvicinò. A
    lei non importava particolarmente di Jam. Era una ragazzina simpatica,
    ma il suo fratello di battaglia era insostituibile. Lo guardò
    direttamente negli occhi, lui ricambiò lo sguardo. Lo sapeva di aver
    detto una cazzata. Se ne accorse solo dopo, però, al suo solito ``Hei,
    distruttore di palazzi.'' ``Hei, carpet muncher.'' rispose, sorridendo
    ``Oh, cazzo. Sorridi solo quando dici battute del cazzo, ragazzino?''
    ``Ma taci, che hai la mia stessa età.'' Risero assieme per un attimo e
    poi lei alzò il pugno sinistro, lui fece lo stesso e poi li
    appoggiarono uno contro l'altro ``Bentornato, fratello.'' ``Sono
    contento di poter tornare a lavorare con te, sorella.'' e quindi si
    abbracciarono ``Allora, ce la siamo portata a letto la Lady?'' lo
    punzecchiò Lissa, mentre erano ancora abbracciati ``Ma che scherzi?
    Avevo da salvare una città, IO. Inoltre te l'ho già detto che non è il
    mio tipo.'' ``Ah. Vedo che la permanenza non ti ha bevuto il cervello.
    Bravo fratellone!'' fece, lei, battendogli sulla spalla ``Senti, ora ti
    porto via dal vecchio bavoso e poi mi racconti cosa è successo, ok?''
    ``Haha. Allora vedi che è bavoso?''

    Quando Elythia ebbe finito con Jam si accorse della presenza del
    ragazzo. Appena notò che andava in giro con le stampelle corse da lui
    ``Hax!'' fece, in apprensione ``Che ti è successo?'' ``Eh,'' rispose
    lui, come se niente fosse ``nulla. Ho distrutto cose, mi hanno messo in
    galera, ho distrutto altre cose\dots{} Il solito.'' ``Hm. se non ti
    lamenti vuol dire che non ti sei fatto troppo male, eh?'' ``EH?''
    ``Cavolo. Ed io che ero così in pensiero per te, Hax. E ti trovo quì a
    goderti il paradiso tropicale?'' ``Vuoi il tuo Paradiso? Prenditelo. A
    me portatemi via.'' ``Ok, capito. Mi racconterai mentre ce ne andiamo.
    Comunque, Hax.'' ``Sì?'' ``Sono davvero felice che tu non ti sia fatto
    niente di grave.'' ``Grazie, Elythia. Anche io sono davvero felice di
    vedervi di nuovo.''

    Dopo un po' anche Fixer si avvicinò ad Hax e gli fece ``Bel lavoro, mio
    caro.'' ``Grazie, Fixer.'' rispose lui ``Ah, a proposito. Gli zombie
    si ammazzano staccando loro la testa o spaccandogliela.'' ``Sì, esatto!
    Hai fatto ricerche anche te da queste parti?'' ``Hem, in un certo senso
    ho fatto esperimenti sul campo.'' ``Scienziati?'' ``No, vecchietti
    impazziti. Quì sono tutti fuori.'' ``Vai a capirli. Tipo quella cosa
    del non rubare dai negozi.'' ``No, Fixer. Quella è una legge che vale
    anche da noi.'' ``Appunto.'' ``Eh.'' Hax non capiva, spesso, cosa
    intendesse The Fixer con quei discorsi.

    Dopo un po' decisero di ringraziare il vecchio Maestro di Jam, che fu
    ben felice di vederli andare, anche perchè voleva che la sua pupilla
    sperimentasse ancora un po' cosa fosse il mondo fuori di là. Il Mestro,
    prima che se ne andassero tutti, però, decise di consegnare la katana
    distrutta a The Fixer, in quanto lui sembrava degno di ripararla.
    ``Riparala. E fai in modo di darla a chi di dovere, quando sarà
    necessario, mi raccomando.'' fece, consegnandogliela in un fodero.
    ``Sì, Maestro.'' rispose Fixer, per poi seguire il gruppo.

    ``Bene. Vacanza FATTA.'' commentò Hax sulla strada verso la città,
    zoppicando sulla stampella ``Sono più stanco di prima, ma mi sono
    divertito, alla fine.'' ``Eh.'' rispose Lissa ``E non è ancora
    finita.'' sghignazzò sotto i baffi ``Eh, sì, dobbiamo arrivare fino a
    Phaion. Ci vorrà ancora un casino di tempo. Ah. A proposito. Come avete
    fatto ad arrivare quì così in fretta?'' ``Eh, vedi.'' ``Occazzo no.''

  \section{Gabriel - Dera - Agosto 2635}

    Gabriel guardò verso l'orizzonte sul quale c'era uno splendido
    tramonto, mentre si trovava in veranda durante una delle pause che si
    prendeva durante la stesura o la riordinazione delle sue storie.

    ``Chissà dove sono?'' si chiese, per poi prendere un sorso di vino.

\cleardoublepage{}
