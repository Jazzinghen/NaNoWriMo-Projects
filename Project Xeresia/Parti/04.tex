\chapter{C'è sempre bisogno di un principe che salva una donzella}

  \section{Rollers - Dera - Settembre 2635}

    ``Aaah, Dera: casa di artisti, ottimo vino, fervente comunità
    letteraria, la cremè della cremè culturale.'' quella a parlare come una
    guida per i viaggi era Elythia. ``Sì. La cremè della cremè della
    noia.'' apostrofò Lissa. Lissa non era una grande fan della
    letteratura ``Immagino che non abbiano neppure una facoltà
    scientifica.'' ``Sì che ce l'hanno, Lissa. Solo che è di scienze
    cognitive. \'E una delle più rinomate di Xeresia, tra l'altro.'' ``Eh?
    \'E una scienza quella?'' ``Beh, Lissa. \'E una scienza molto
    interessante.'' s'intromise Hax ``Se si riuscisse ad interfacciare
    meglio la mente umana con la tecnologia l'algoritmica potrebbe entrare
    a far parte della vita della gente di tutti i giorni senza ostacoli.
    Non sarebbe magnifico?'' il ragazzo era un visionario per quanto
    riguardava collaborazioni tra campi di ricerca scientifica.
    ``Inoltre.'' aggiunse Fixer, con già in mano una borsa con dentro una bottiglia di vino
    rosso locale e dei dolcetti tipici ``Hanno dell'ottima cucina quì.''
    Poi, guardando Lissa, le porse una specie di biscotto glassato molto
    sottile ``Lingua di demone?'' Lissa lo prese e rispose ``Oh.
    Gra\dots{}'' per poi accorgersi che stava facendo il gioco dell'uomo,
    però ormai non poteva rifiutare una cosa che sembrava così succulente
    ``Vabbeh. Comunque è quì che vive quel tizio? Hem\dots{} Gabriel,
    giusto?'' ``Sì, esatto. Nel contratto c'era scritto di andare a casa
    sua prima del lavoro per ricvere gli ultimi dettagli ed una lettera che
    non poteva permettere che leggessero tutti.'' ``Oh, per gli Dei, no.''
    commentò, acida, Lissa, mettendosi la mano sulla faccia ``Una lettera
    sdolcinata da innamorati? Ma chi siamo?'' ``Ma se non sappiamo ancora
    che cosa ci sia contenuto, Lissa. Perchè ti fasci la testa ancora prima
    di essertela rotta? E poi tanto m'immagino che non potremo neppure
    leggerla, su.'' ``Ma io non voglio fare da Spiritello dell'Amore per
    qualche matto.'' ``Senti, è il lavoro per il quale ci pagano, lascia
    perdere questi dettagli minimi.'' ``Hm.'' sbuffò la ragazza ``Ok.
    Diamoci dentro.'' completò la frase, facendo a finta di essere eccitata
    per quel lavoro, tirando un buffetto all'aria.

    I Rollers camminarono lungo le strade da cartolina di Dera, dove dietro
    ogni angolo c'era un qualche parchetto popolato d'artisti, oppure un
    qualche Bistrot al quale sono seduti uomini col baschetto. Una città in
    piena età d'oro culturale. Ricordava moltissimo New Metredeth, solo che
    andavano sostituiti workshop meccanici ai Bistrot e gente che scriveva
    su lavagne sotto tettoie dei parchi al posto dei suonatori. Era
    evidente che le accademie cambiassero la popolazione della città,
    certo, non sempre per il meglio. Elythia si trovò a fare questi
    ragionamenti mentre camminava distrattamente al fianco di Lissa. Le
    venne da pensare alla città che ospitava la sua accademia di medicina.
    L'alta densità di spiriti protettori da quelle parti portava ad una
    gran quantità di problemi in tutta la città, visto che le persone con
    le quali erano collegati li usavano in maniera anche indiscriminata.
    Matti. ``Vabbeh.'' mugugnò, mentre Lissa stava discutendo con Hax della
    possibilità che visioni non scientifiche del mondo potessero manipolare
    la realtà come facevano loro con algoritmi o funzioni. ``Vabbeh,
    Elly?'' chiese Lissa, pensando che la dottoressa stesse parlando con
    loro ``Eh? Ah, nono.'' rispose lei, distrattamente ``Comunque, non per
    criticare, ma la casa di questo tizio dove si trova?'' continuò Lissa
    ``Dalle indicazioni che mi hanno dato\dots{}'' rispose Jam, che era
    quella che si era adattata di più alla città ``\dots{}dovrebbe essere un
    po' fuori dal centro. Diciamo che ci vorranno ancora una ventina di
    minuti a piedi.'' ``Altri venti minuti di discussione, Hax.'' ``Già''
    rispose lui, pronto per continuare con l'argomentare i suoi punti di
    vista.

  \section{Seysill - Metra - Settembre 2635}
    
    Era arrivato settembre. Ora di mettere le cose dentro le valige e di
    andarsene da quel posto. Entro non moltissimo la \emph{Macchina per il
    colpo di stato}, come piaceva chiamarla a Seysill in questi casi,
    avrebbe iniziato a muoversi e lui sarebbe stato in pericolo qualunque
    fosse stato il risultato di quella azione. Quella sera aveva finito il
    lavoro al solito, aveva raccolto tutti i documenti riguardanti quel
    contratto, prese tutti i documenti d'identità che teneva nascosto
    assieme alle proposte di contratto e mise tutto dentro una
    ventiquattr'ore. Prese la sua borsa in pelle e lì ci mise dei sacchetti
    di tè che voleva portarsi via in ricordo di Phaion, dei sigari ed
    alcuni libri per il viaggio. Sarebbe stato noioso. Era sempre noioso
    viaggiare tanto da solo.
    
    ``Oh, beh.'' si disse, guardando i libri che
    aveva in mano, prima di metterli dentro nella borsa ``Sarebbe stato
    molto più divertente con te.'' si riferiva a Durga, la quale sarebbe
    rimasta là a fare da tirapiedi a quel finto culto. Rise malinconico e
    poi ripose i libri. Si guardò intorno per vedere se aveva preso tutto.
    Era un po' riluttante al pensiero di lasciare tutto quello a
    qualcun'altro. ``Se lo goda, mio successore.'' disse all'aria,
    ironico. Si mise la borsa a tracolla, prese con la mano sinistra la
    valigetta e quindi uscì dalla stanza. Chiuse la porta a chiave e quindi
    si avviò per uscire dal palazzo. ``Buona serata, signor Mann.'' lo
    salutò la sua segretaria ``Va a fare un viaggio, signore?'' ``Hm.
    Diciamo di sì, signorina Rais. E dica che sono via se mi cercano.''
    ``Devo anche dire dove?'' ``Preferirei di no.'' ``Va bene signor Mann.
    Buon viaggio, allora.'' ``Grazie. Buona giornata. E si cerchi un nuovo
    lavoro, lei è sprecata quì.'' ``S-Sono licenziata?'' ``No, era un
    consiglio. Odio vedere signorine con potenziale sprecarlo in posizioni
    da segretaria.'' Seysill si mise in testa il suo fedora e se ne uscì.

    Arrivò alla casa che gli avevano fornito assieme alla posizione di
    parlamentare, entrò, salì al secondo piano e riempì una valigia con la
    ventiquattr'ore ed una serie di vesiti necessari per il viaggio e per i
    primi giorni di permanenza a Shinra. Il resto l'avrebbe comprato lì.
    Lui era uno di quei gentleman alla \emph{Spaziolino e soldi}. Un
    viaggiatore d'assalto. Avrebbe vissuto nella maniera nella quale si
    vive a Shinra. Se devi lavorare in un posto è meglio se ti abitui a
    vivere come la popolazione locale. Soprattutto se devi far passare una
    legge che è più a favore di popolo. Devi capire di che hanno bisogno,
    di come vivono, di come vedono il problema. Era un professionista lui,
    diamine.

    Dopo aver riempito la valiga uscì di casa, valigia in mano, borsa a
    tracolla e soldi in tasca. Gli ci sarebbe voluto un po' per arrivare
    fino al confine in automotiva, però non ci sarebbero dovuti essere
    problemi. Aprì il bagagliaio e ci mise dentro la valigia. Poi aprì la
    portiera dell'autista, buttò la borsa a tracolla sul sedile del
    passeggero, si sedette al posto di guida e poi si tirò dietro la
    portiera.

    ``Beh, allora, Metra\dots{} Addio. Chissà se tornerò mai per vedere
    com'è andata alla fine.'' disse, guardando la città fuori dal
    finestrino, per poi avviare l'auto.

  \section{Celty - Metra - Settembre 2635}

    Era particolarmente irritata da quello che le stava capitando da un po'
    di tempo a quella parte. Era stata tirata scema dallo Stronzo,
    nell'ultimo periodo, visto che, a quanto pare, aveva trovato una nuova
    fonte d'introiti, quindi aveva di nuovo bisogno del suo corriere di
    fiducia. Non che fosse \emph{troppo} triste per quel fatto. Dover
    andare in giro per portare comunicazioni a scagnozzi che facevano
    inconsapevolmente soldi per il Conte per Celty significava solo che il
    suo schiavista era interessato a quello che non a portare avanti
    progetti pericolosi o, peggio ancora, cercare i suoi amici.

    Il nuovo piano per fare soldi di Philip era quello di sfruttare dei
    trafficanti del mercato nero di Udemia, i quali commerciavano in
    alcoolici, droghe, componenti meccanici illegali e, probabilmente,
    organi umani, per far entrare merce proibita a Phaion. Era ovvio che
    era necessario un commercio illegale per cose come droghe o altro, ma
    quello che rasentava il ridicolo era che questo fosse necessario per
    merce come certo tipo di musica o certo tipo di narrativa.
    Incredibilmente Philip aveva bloccato con delle leggi alcuni canali
    solo per trarne profitto. Era un bastardo di prima categoria. Era da un
    po', perciò, che Celty era ritornata a fare da corriere tra Shien,
    Udemia ed alcuni villaggi di confine sia dentro che fuori Phaion.

    Quella volta, però, era differente. Il Conte le aveva chiesto di andare
    a Metra per consegnare una lettera ad alcuni amici. Le aveva detto di
    riferirsi a quello che c'era scritto sulla busta per conoscere il luogo
    esatto per la consegna. Era per questo che era particolarmente
    irritata, perchè questo andava oltre i soldi. Significava che stava
    pensando ad altro che avrebbe portato a casini. Grossi casini. Inoltre
    essere a Metra la preoccupava doppiamente. Non riusciva a non pensare
    al contratto di cui aveva letto nella sede della gilda di Udemia. Lissa
    e gli altri stavano venendo da quelle parti. E sarebbero dovuti
    arrivare in quei giorni.
    
    Cosa sarebbe successo se l'avessero vista? Hax l'avrebbe di nuovo
    pedinata, scoprendo dove lavorava in quel momento? Lissa si sarebbe
    incazzata? Lo Stronzo li avrebbe minacciati? Era preoccupata per
    questo. E per quello che stava facendo in quel momento il Conte. Se
    avesse scatenato qualche evento, se avesse iniziato a creare problemi a
    Metra, questi avrebbero raggiunto molto facilmente i Rollers. Anche
    perchè, a quanto aveva capito lei, dovevano andare ad incontrare una
    ragazza molto particolare in quella città. E là, di molto molto
    importante c'era la principessa Xeresia. Tentò di levarsi dalla testa
    quei pensieri. Iniziò a sfiorarle il pensiero di non consegnare la
    lettera, di fare a finta che le fosse andata persa, che le fosse stata
    rubata durante un attacco di briganti. Questi pensieri, però, vennero
    scacciati via quasi subito dal pensiero di cosa sarebbe potuto capitare
    ai suoi amici se non avesse portato a termine il lavoro. Era
    irrazionale, ma era là. Onnipresente.

    Fu proprio mentre era sovrappensiero che urtò un'automotiva ferma ad uno
    stop nel centro di Metra. Celty cadde a terra, emettendo un leggero
    gemito di dolore quando colpì terra con le natiche. Subito scese
    dall'auto il guidatore, un signore distinto con una fedora, la quale
    chiese immediatamente, in apprensione ``Signorina. Tutto a posto?''
    ``Mh.'' rispose lei, in maniera affermativa, mentre si massaggiava la
    natica destra ``Sì, grazie.'' ``Oh. Fantastico. Ecco, mi permetta di darle una mano ad
    alzarsi.'' rispose lui, porgendole la mano. Celty afferrò la mano e si
    tirò in piedi con l'aiuto del gentiluomo ``Mi perdoni, non volevo
    sbattere contro la sua auto'' si scusò, lei, pulendosi i pantaloni.
    ``No, si figuri. Aspetti. Le è caduto questo.'' rispose, cortese,
    l'uomo, per poi piegarsi sulle ginoccia a raccogliere la busta che il
    Conte le aveva chiesto di consegnare. Doveva esserle caduta quando
    aveva sbattuto contro l'auto. L'uomo ebbe un sussulto quando scorse ciò
    che c'era scritto sulla busta, però si rialzò come niente fosse e
    gliela porse. ``Grazie.'' fece, lei, prendendo la busta dalle mani del
    signore ``Si figuri, dovere, signorina.'' ``Bene, ora devo andare.
    Scusi ancora per l'inconveniente, non sapevo dove avevo la testa.''
    ``Non si preoccupi. Vada pure. E, per i divini, stia attenta, non si
    può mai sapere cosa capiti quando non si presta attenzione.'' ``Ah,
    s-sì, grazie.'' e la ragazza si avviò lungo il viale, lontana
    dall'auto.

  \section{Seysill - Metra - Settembre 2635}
    
    ``Dannazione.'' imprecò tra sè e sè Seysill, mentre guardava
    allontanarsi la ragazza che aveva sbattuto contro la sua auto. Dopo un
    secondo rientrò nel veicolo tirandosi dietro la portiera. ``Quella.
    Nessuno porta lettere a quell'indirizzo senza l'uso della posta. A meno
    che non voglia non essere scoperto. Maledetto Philip. Quando ha
    intenzione di far scoppiare il polverone?'' si chiese, avviando di
    nuovo l'auto, che intanto si era spenta. Non aveva tempo da perdere,
    doveva partire. Doveva andarsene da là se non voleva finire in mezzo a
    qualche golpe da quattro soldi. Allacciò la cintura, premette il piede
    sull'acceleratore e partì alla volta del confine.

  \section{Rollers - Dera - Settembre 2635}

    ``Va bene, lasciate parlare a me, almeno all'inizio.'' ammonì il gruppo
    Elythia. Non che non si fidasse del gruppo, ma non voleva fare brutta
    figura, soprattutto con un committente di una città così ricercata.
    Sapeva che avrebbe incontrato qualcuno con la puzza sotto il naso. Lei
    odiava quella gente. Lei era raffinata, adorava la gente raffinata ma
    odiava quelli con la puzza sotto il naso. Non c'era modo peggiore di
    pubblicizzare la propria ricchezza se non facendolo pesare agli altri.
    ``Ottimo.'' rispose, rapidamente, Hax ``Sono sicuro che sarà uno
    stronzo rompipalle dalle chiappe strette.'' Era sempre così con il
    ragazzo. Si costruiva sempre un gran numero di preconcetti sulla gente
    che poi doveva smontare faticando e dovendo scusarsi più e più volte.
    Non che non ne fosse capace, ovviamente, però gli richiedeva un'enorme
    quantità di tempo e di autostima.

    ``Bene, allora. Se avete capito adesso andiamo là e ci presentiamo a
    questo Gabriel.'' mise in chiaro lei, per poi avviarsi verso la porta
    della casa indicata come residenza del committente. Suonò il campanello
    e quindi si mise con le braccia incrociate al petto. Era una delle
    tecniche che usava ogni tanto. Non solo le risaltava il seno, che
    faceva sempre colpo, anche in maniera subconscia, negli uomini, ma le
    dava un'aria più seria e composta. Cosa che le serviva con il gruppo
    col quale si ritrovava. In queste situazioni rischiavano sempre di fare
    la figura degli straccioni, dei poco seri. Ma la gente doveva sapere
    con chi si trovavano a che fare. E lei doveva essere il baluardo di
    difesa della qualità del loro operato.

    Dopo poco arrivò qualcuno ad aprire la porta. Era un ragazzo più
    giovane di lei, vestito con un gilet, una camicia bianca, dei pantaloni
    rossi bordeaux di velluto. ``Salve.'' fece lui, come se fosse
    speranzoso di sentir dire qualcosa di specifico dalla dottoressa
    ``Salve,'' rispose Elythia, lasciando cadere le braccia lungo i fianchi
    e quindi alzare la mano destra per voler stringere la mano a chi le
    aveva aperto la porta ``sono Elythia, dei Rollers. \'E lei Gabriel?''
    Lui allungò rapidamente la mano ed iniziò a squoterla energicamente
    ``Sì, sì. Sono io!'' esclamò euforico ``Finalmente siete arrivati! Non
    sapete quanto ero impaziente di conoscervi! Entrate, entrate!'' e
    indicò loro l'interno della casa. ``Grazie.'' rispose lei, leggermente
    scossa dalla reazione del ragazzo. Entrò in casa, seguita subito dopo
    dagli altri del gruppo, che intanto si erano persi a guardare la casa
    ed il giardinetto.

    Una volta che furono tutti entrati Gabriel chiuse dietro di loro la
    porta d'entrata e fece ``Ma, ditemi, ditemi. Avete fatto un buon
    viaggio? Come mai ci avete messo tanto?'' ``Hem, è una storia lunga, se
    vuole le possiamo raccontare tutto. Comunque sappia che abbiamo
    apprezzato particolarmente il fatto che lei abbia pagato il viaggio.''
    rispose Elythia, mentre guardava il corridoio ed i soprammobili ed i
    quadri presenti in esso, attendendo che il ragazzo li guidasse in un
    punto specifico della casa o che dicesse loro dove andare. ``Ah, ma
    certo. Sapete ascoltare storie per poi riportarle è il mio lavoro.
    Seguitemi pure da questa parte, andiamo in veranda.'' continuò lui,
    mentre scivolò vicino ai Rollers, per poi mettersi davanti ad Elythia,
    prima della fila. Proseguì lungo il corridoio in un salotto che dava
    sull'esterno, dove c'era un tavolo che era posizionato a metà tra
    l'interno della casa ed una tettoia nel giardino interno della casa.
    ``Prego. Prendetevi pure delle sedie. Fate come se foste a casa
    vostra.''

    \emph{Oddio.} pensò Elythia, mettendosi una mano in faccia, mentre Jam
    e Lissa presero una sedia a testa e si sedettero al tavolo. Anche lei
    fece lo stesso. Hax e Fixer, però, erano più interessati alla parte del
    \emph{Fate come a casa vostra}, perciò iniziarono a guardare tutto
    quello che era presente nella stanza. Soprammobili, libri, bottiglie
    d'alcoolici, mappe, recipienti per dolci. ``Bene. Posso portarvi
    qualcosa, prima di metterci a discutere?'' chiese il ragazzo, mentre
    osservava divertito l'uomo ed il programmatore osseravare tutto con
    cura maniacale. ``Sì, certo, grazie. Potrei avere del tè?'' chiese
    Elythia, garbatamente, seguita poi da Jam e Lissa che aggiunsero un
    ``Anche a me, grazie.'' Fixer, invece, disse, senza neanche staccare lo
    sguardo da uno stocco custodito all'interno di una vetrinetta ``Scotch,
    doppio, on the rocks. Grazie'' per poi continuare ad analizzare
    l'oggetto che si trovava dinnanzi a se. Hax, con in mano una mappa
    della zona, si girò verso Gabriel e gli disse ``Un succo al mirtillo,
    se è possibile, altrimenti prendo anche io il tè.''

    ``Va bene. Vedrò che posso fare.'' rispose il ragazzo, per poi
    andarsene verso quella che doveva essere la cucina. Mentre era via
    Elythia rimproverò i due uomini del gruppo ``Ma che diavolo, ragazzi.
    Siate un po' più cortesi, per Eclisse.'' ``Sì, scusa.'' rispose Hax,
    evidentemente imbarazzato, mettendo a posto la mappa che aveva in mano,
    per poi continuare una volta presa una sedia tra lei e Lissa ``Ma ha detto di
    fare come a casa nostra, quindi mi pareva scortese non accettare questo
    tipo di accoglienza.'' ``Non ti preoccupare Hax. Comunque non sembra
    così tanto con la puzza sotto il naso, no?'' ``Hm. Non so. Ancora non
    mi pare una persona così affabile.'' ``Oh, dai fai così con tutti.''

    ``Fa così cosa?'' chiese il ragazzo, che aveva sentito solo l'ultima
    frase. Era rientrato nella sala portando un bicchiere con dello scotch
    e ghiaccio ed un bicchiere pieno di liquido viola. ``Ecco a lei.''
    fece, appoggiando il bicchiere con l'alcoolico sulla vetrinetta ``Se è
    interessato a quell'arma posso tirargliela fuori.'' ``Lo farebbe?''
    chiese Fixer, prendendo in mano il suo bicchiere ``Sì, mi dia un
    secondo.'' ``Grazie.'' ``Ed ecco a lei il suo succo di mirtillo.''
    continuò, porgendo la bevanda ad Hax. Tornò quindi da Fixer, il quale
    stava osservando la spada nell'interezza mentre sorseggiava il suo
    drink e, muovendo delle levette ai lati della vetrinetta, la aprì e ne
    tirò fuori l'arma. ``Eccola,'' fece, consegnando l'arma all'uomo ``\'E
    un'arma di famiglia. Veniva utilizzata durante parate ed eventi
    particolari. Spero la trovi di suo interesse.'' ``Indubbiamente.''
    rispose The Fixer, prendendo l'arma con entrambe le mani dopo aver
    appoggiato su di un tavolino il suo bicchiere.

    Il metallurgo aspettò finchè il ragazzo non uscì nuovamente dalla
    stanza per fare quello che gli risuciva meglio. Analizzare opere in
    metallo, composti chimici, creazioni alchemiche. Per farlo in fretta,
    perchè non voleva infastidire più di così il loro ospite, passò in
    modalità spettrometrica. La parola nascondeva tutte le sfaccettature
    di quella modalità. Era come vedere tutto quello di cui era formata la
    materia, in maniera facile da riordinare, facile da accedere. Questo
    non signigicava che fosse facile da comprendere o che tutta la chimica
    o le scienze dei materiali diventassero improvvisamente facili da
    comprendere. Era più un fatto di accesso ai dati su ciò che formava la
    fibra stessa della materia. In una frazione di secondo The Fixer poteva
    passare da una visione macroscopica ad una microscopica, vedere la
    struttura molecolare di un particolare materiale, vedere che
    struttura avessero i cristalli di un certo materiale (se si disponeva
    in maniera cristallina, ovviamente, quindi con ceramiche o vetri questo
    non funzionava), visualizzare dati spettroscopici del materiale.
    Questo gli permetteva di comprendere e di simulare il
    comportamento a temperature differenti, dove potevano formarsi i
    punti di rottura di un oggetto, quale fosse la conduttività
    dell'oggetto o del materiale in questione e di molti dati
    particolarmente importanti per lo sviluppo di nuovi materiali. Se erano
    già presenti dei punti di rottura, ovviamente, gli
    venivano indicati se lo richiedeva. Poteva scendere ancora più in profondità e conoscere i dati
    riguardanti a singoli atomi, come elettroni presenti, elettronegatività
    ed altri dati relativi. Grazie a quei dati poteva face calcoli
    on-the-fly su reazioni chimiche più probabili e cose così. E questo era
    quello che la gente sapeva degli scienziati dei materiali o dei
    chimici. Quello che non sapevano era che gli scienziati avevano anche
    una comprensione del funzionamento degli atomi e quindi potevano
    manipolarne il funzionamento. Normalmente questo significava un poter
    cambiare la forma, con più o meno precisione, di oggetti toccati.
    Richiedeva una grande quantità di energia, però si poteva cambiare
    l'aspetto, il materiale, quasi tutto, insomma, di un oggetto. L'unica
    regola da seguire, in pratica, era quella della conservazione della
    massa. Non tanto perchè era un'imposizione morale, quando era proprio
    impossibile trasformare un oggetto in un altro che avesse meno massa
    del primo. Era direttamente collegato alla regola dei paradossi di cui
    gli aveva parlato una volta Jam. Qualcosa a che fare con
    l'impossibilità di creare o distruggere la materia. Comunque non è che
    loro potessero creare la materia, la potevano solo plasmare, quindi non
    c'erano problemi di paradossi o di distruzione del mondo,
    fortunatamente.

    Osservò rapidamente l'arma, la fece ruotare nelle mani, per poi
    rimetterla con disappunto, nella vetrinetta. Prese il bicchiere col suo
    drink ed andò a sedersi al tavolo, dove c'erano tutti gli altri.
    ``Allora?'' gli chiese Elythia ``Perdita di tempo. \'E la solita arma
    da esposizione. Speravo avesse qualcosa di più interessante.'' rispose
    lui, amaro ``Meglio così. Non avrei voluto stare quì a contrattare per
    un'arma quando abbiamo meglio da fare.'' ``Tipo?'' ``Completare questo
    contratto in fretta, visto che è una cosa facile facile.'' Odiava dire
    queste cose, anche perchè poi si rivelavano completamente sbagliate, ma
    le era sfuggito. Improvvisamente sentì gli sguardi di tutta la squadra
    puntati su di lei. Ci fu qualche secondo di silenzio imbarazzato.
    ``Che?'' chiese, alzando le mani, rompenddo il ghiaccio ``Elythia.'' disse
    Hax, guardandola con occhi imploranti ``Perché? Ogni volta che dici
    così sucedono dei casini impossibili.'' ``Non crederai a queste cose?''
    ``Hem. Sono finito in galera già una volta per questo lavoro,
    preferisco credere che ci sia un modo per limitare la sfiga.''
    ``Vabbeh, che vuoi farci, ormai?'' ``Haha. Ma sì, che stavo
    scherzando.'' e scoppiarono in una fragorosa risata.

    Lei, in realtà, era quella che credeva più di tutti a quella storia.
    Ogni volta che credeva che il lavoro sarebbe stato facile succedeva
    qualcosa che andava contro le sue aspettative. Era per quello che,
    normalmente, considerava tutti i lavori pericolosi allo stesso modo
    così, qualunque cosa fosse successa, non si sarebbe fatta buttare a
    terra. Dopo un po' tornò il ragazzo con il tè.

    ``Ecco a voi.'' fece, appoggiando delle tazze con cucchiaino appoggiato
    sul piattino vicino alle ragazze. Appena finì di disporre la
    zuccheriera, un vassoio con dei biscotti, la teiera ed il brichetto del
    latte, si sedette nell'ultima sedia libera, tra Hax e Fixer. ``Bene,
    allora. Come vi ho già detto sono contento di vedervi. Vi stavo
    aspettando. Io sono Gabriel Bard. Posso sapere i vostri nomi?''
    ``Certo.'' rispose, cortese Elythia, dopo aver preso un sorso del suo
    tè ``Io sono Elythia, come le ho già detto. Sono la dottoressa e, in
    pratica, sono il responsabile burocratico del gruppo. Questi'' ed
    indicò Hax e Lissa, che erano seduti in un lato del tavolo ``sono Hax e
    Lissa.'' i due risposero alzando uno a breve distanza dell'altra la
    mano, salutandolo ``Sono i nostri due esperti di algoritmica. Lissa, inoltre, è
    esperta in operazioni sotto copertura, mentre Hax, quì, hem. \'E
    esperto in piani ben cognegnati, diciamo.'' ``Salve.'' li salutò,
    cortese, Gabriel ``Questa signorina quì è il nostro fiore
    all'occhiello.'' continuò con le presentazioni Elythia, indicando Jam
    ``Lei è Lady Jam, matematica e spadaccina provetta di Sanpan.''
    ``Salve.'' lo salutò la ragazza, inchinando la testa nella sua
    direzione. ``Ultimo ma non ultimo, il nostro esperto di scienze dei
    materiali e chimica, The Fixer.'' ``Hm.'' fece The Fixer, per salutare
    ``Ma, scusi, Fixer è un soprannome o è il nome, se non sono
    indiscreto?'' ``\'E indiscreto.'' rispose fredo The Fixer ``Ahah.''
    rise, sforzandosi, Elythia, e poi, per mettere pezza ``Deve
    vedere che non lo sappiamo neanche noi. Lo abbiamo sempre chiamato
    così, non conosciamo altro nome col quale riferirci a lui, ma non è un
    problema.''

    Elythia prese un altro sorso di tè, per poi continuare, tendando di
    recuperare il discorso ``Hm.'' si schiarì la voce ``Allora, volevamo
    avere qualche altro dettaglio riguardante il lavoro per il quale ci ha
    chiamati quì. \'E stato un po' nebuloso. Ma, per quanto abbiamo capito,
    dobbiamo andare ad incontrare una \emph{certa ragazza} a Metra,
    capitale di Phaion.'' ``Esatto.'' ``E che doveva darci una lettera che
    noi dovremmo consegnare.'' ``Precisamente.'' ``Ottimo. Allora, volevamo
    avere qualche dettaglio per quanto riguarda questo lavoro.'' ``Chieda
    pure.'' ``Perfetto. Allora, per iniziare. Come mai non può andare lei
    direttamente ad incontrare questa ragazza?'' ``Mh. Come sapete a Phaion
    sono sempre state in vigore delle leggi per il controllo degli accessi
    durante l'anno, che però venivano sollevate durante l'estate per la
    fiera estiva.'' ``Hum, sì.'' ``Anche quest'anno pensavo di andare a
    fare visita alla ragazza, però questa volta hanno fatto passare qualche
    legge razzista che rimuoveva la fiera estiva, facendo in modo che il
    controllo per gli accessi fosse in vigore per l'intero anno. Siccome io
    avevo finito le giornate disponibili per l'accesso a primavera di
    quest'anno non sono potuto entrare.'' ``Ah, capisco. Bene, quindi ha
    chiamato un gruppo di tuttofare per andare a discutere con la ragazza
    ed a consegnarle una lettera?'' ``Precisamente.'' ``Come mai non ha
    potuto inviarla la lettera, esattamente?'' ``Questo risponderà anche
    alla domanda \emph{Chi è la ragazza che dobbiamo incontrare},
    immagino.'' ``Davvero? Perchè, chi è questa ragazza?'' ``Xeresia,
    principessa del regno di Phaion.'' ``Oh.''

    \emph{Quella era la risposta alla frase \emph{questo è un lavoro facile
    facile}, scemo.} pensò Elythia, mentre sentiva che gli altri stavano
    connettendo cosa significasse veramente il significato di quella frase.
    ``O\dots{}Ottimo.'' continuò lei, tentando di mantenere la compostezza
    giusta ``E, immagino che basterà andare là e chiedere un incontro?''
    ``Non lo so, veramente. Probabilmente sarebbe meglio che voi facciate
    come faccio io. Che la visitiate la notte.'' ``Ah, e se ci fermano?''
    ``Secondo voi perchè ho chiesto ad un gruppo di Tuttofare e non a degli
    amici e basta?'' Elythia si fermò a pensare e rispose ``Ha senso, in
    effetti.'' ``Allora, quando pensate di partire?'' ``Domani, immagino,
    se non prima per arrivare prima a Metra.'' ``Capisco. Allora dovrò
    darvi i documenti. Aspettate che vado a recuperarli.''

    Quando il ragazzò uscì dalla stanza Hax aspettò qualche secondo e poi
    commentò ``Non ho ben capito, Elly, ma il lavoro è in qualche modo più
    tosto, eh?'' ``Già.'' aggiunse Lissa ``In pratica dobbiamo infiltrarci
    in un palazzo reale per consegnare una lettera d'amore.'' ``Beh, non
    dite che non abbiamo mai fatto una cosa simile.'' rispose la dottoressa
    ``Sì, beh.'' la supportò Jam ``E poi non scaricate la colpa su Elythia.
    Come avrebbe fatto a saperlo?'' ``Beh, no. Non le sto dando la colpa,
    in realtà. \'E solo che mi preoccupa non poco questa cosa, sapete?'' si
    difese Lissa ``In che senso?'' chiese Elythia ``Se sono veramente così
    razzisti sarà una buona idea andare là?'' ``Ma và, non scherzare. \'E
    vero che hanno fatto questa legge, ma sarà a difesa della produzione
    interna commerciale.'' ``Hm. Se lo dici te.'' ``Io, invece, sono
    scettico su questa cosa. Secondo me c'è un altro motivo ancora per
    l'approvazione di questa legge.'' s'intromise The Fixer ``Beh, ragazzi.
    Temo che sia inutile fasciarsi la testa prima d'essersela rotta,
    sapete? Quindi io dico che adesso prendiamo questa beneamata lettera e
    poi andiamo a Phaion a vedere com'è lo stato delle cose.''

    Continuarono a discutere su dettagli poco importanti finchè non tornò
    Gabriel con la lettera, sigillata con della cera lacca sulla quale era
    impresso un qualche logo. ``Ecco.'' fece, porgendo la lettera ad
    Elythia ``Questa è la lettera che dovrete consegnarle. E se avete
    tempo di parlarle, ditele che la amo.'' La dottoressa prese la busta e
    la infilò con cura nella propria borsa ``Sarà fatto, signor Bard.''
    prese l'ultimo sorso di tè dalla sua tazza e poi fece ``Bene, è stato
    un piacere, ma adesso dobbiamo andare. Grazie per la sua ospitalità.''
    e si alzò in piedi, seguita dagli altri. ``Grazie.'' ripetè Hax,
    provando ad essere cortese, per poi avviarsi verso l'uscita, senza
    neppure attendere un saluto. ``Ah, grazie a voi per essere venuti.
    Sinceramente non speravo che qualcuno avrebbe davvero accettato la mia
    proposta di lavoro, sapete?'' rispose lui, alzandosi ``Non si
    preoccupi. Vedrà che porteremo a termine il lavoro.'' commentò la
    dottoressa, avviandosi a sua volta verso la porta d'entrata.

    Dopo altri convenevoli furono tutti fuori. Tornarono sulla strada dopo
    aver attraversato tutto il giardinetto della casa del loro committente
    quando Elythia fece, rivolta al gruppo ``Bene. Andiamo, dunque?''

  \section{Seysill - Confini di Phaion (Lato interno) - Settembre 2635}

    Era finalmente arrivato al confine dopo una giornata e mezza di
    viaggio. Parcheggiò l'auto di fianco agli uffici del confine. Doveva
    solo sbrigare quella cosa dei documenti per l'uscita e quindi avviarsi
    verso le stanghe e poi addio Phaion.

    Entrò negli uffici, dove c'era un'impiegata dietro un bancone, senza
    nulla da fare, che stava leggendo un libro. In effetti a quell'ora non
    c'era molto da fare. Seysill si avviò verso il bancone e quindi la
    salutò con un energico ``Salve.'' ``Ah. Salve. Desidera?'' la ragazza
    appoggiò il libro dopo aver messo un segnalibro per indicare dove fosse
    arrivata ``Sì, dunque,'' rispose lui, porgendole i suoi documenti
    d'identità ``devo uscire da Phaion. Eccole i documenti.'' ``Ah.
    Vacanza?'' ``Diciamo di sì.'' La ragazza prese il plico ed iniziò a
    controllare documenti ``Ah, hm. Vediamo un po'\dots{}'' Passarono un
    paio di minuti mentre l'impiegata controllava tra plichi e scartoffie
    varie quando poi rimise i documenti d'identità sul bancone e fece ``Mi
    spiace, signor Mann, ma non posso lasciarla uscire.'' ``Ah, grazie. No,
    aspetti. Come scusi?'' ``Non ha i permessi per uscire dal paese, quindi
    non posso lasciarla passare.'' ``Come sarebbe a dire? Da quando non ho
    i permessi per uscire?'' ``Beh, signor Mann. Ho quì una serie di
    nominativi di persone che non possono lasciare il paese se non
    provvisti di un visto speciale del re.'' ``\'E perchè sono
    incriminato?'' ``No, non si preoccupi. A quanto pare è una legge che
    non permette ai parlamentare di usicre dal Paese senza visto. Dovrebbe
    saperlo, no?'' ``E-Ero malato in quel periodo, si vede.'' ``Beh, non
    dovrebbe essere difficile per lei recuperare un visto per l'uscita. I
    procedimenti non prevedono grossi controlli nè nulla. Prendono un po'
    di tempo, però, a quanto ho sentito.'' ``Ah. Capito, grazie.'' Seysill
    prese i documenti e se li reinfilò nella tasca interna della giacca.
    ``Bene, allora, signor Mann. Buona giornata.'' lo salutò cortese
    l'impiegata ``Buona giornata a lei.'' rispose Seysill, distratto,
    mentre usciva dall'ufficio.

    Era fregato. Gli avevano bloccato la via d'uscita. Non poteva dire che
    quella legge era stata fatta apposta per lui, perchè poteva essere
    utile per non far fuggire coloro che andavano \emph{purgati} dal golpe,
    però in quella maniera lui non poteva uscire. ``Dannazione.'' imprecò,
    entrando in macchina sbattendo la portiera. Doveva trovare un qualche
    modo per proteggersi finchè non fosse finito tutto mentre cercava una
    via di fuga da quello stato, prima che diventasse una trappola.

  \section{Rollers - Confini di Phaion - Settembre 2635}

    ``Bene, eccoci quì, finalmente.'' commentò Elythia, guardando gli
    uffici doganali di Phaion. Avevano viaggiato per una giornata e mezza,
    metà col treno, metà a piedi. Adesso dovevano farsi fare il visto per
    la visita e poi avrebbero potuto preoccuparsi di come avrebbero fatto
    ad incontrare la principessa Xeresia. Fortunatamente Gabriel aveva
    fornito loro una serie d'indicazioni per arrivare alla stanza della
    ragazza, il fatto era che la dottoressa avrebbe preferito metodi meno
    \emph{equivoci}. ``Bene, aspettatemi quì, vado a fare le carte
    necessarie.'' disse, girandosi verso il gruppo, che era seduto su una
    panchina a riposarsi, con le borse vicino ``Roger.'' rispose Hax, con
    la testa reclinata in dietro, sdravaccato.

    Entrò e si avvicinò al bancone dietro al quale c'era la responsabile
    per la documentazione di confine ``Salve!'' la salutò la ragazza
    ``Salve. Sono venuta per richiedere cinque permessi per l'entrata a
    Phaion.'' ``Ma certo! Posso chiedere di che natura è la visita?''
    ``Turismo.'' ``Ah, ma certo. Metra è stupenda in questo periodo
    dell'anno. Allora devo chiederle i suoi documenti d'identità?''
    ``Certo.'' rispose lei, consegnando una serie di documenti. C'era la
    sua carta d'identità, una specie di pergamena con i dati di Jam, le due
    tessere a risonanza arcana di Hax e Lissa ed un plico di documenti di
    Fixer, il quale pareva avere cittadinanza un po' dappertutto su
    Xeresia ``Ecco.''. ``Mi dia un attimo.'' fece la ragazza, prendendo
    tutti i documenti ed iniziando a controllarli attentamente uno ad uno.
    Passò un po' di tempo, nel quale Elythia i guardò intorno, cercando
    qualche informazione utile per quando sarebbe stata a Metra, come
    volantini, locandine pubblicitarie, però non trovò nulla, con suo
    disappunto. Dopo una decina di minuti la ragazza alla reception la
    chiamò. ``Signorina Elythia, sono spiacente d'informarla che non posso
    convalidarle i visti per la permanenza.'' ``Qualche problema?'' ``Sì,
    in effetti. Purtroppo non posso verificare l'identità per tre di questi
    documenti.'' ed indicò i documenti di Lissa, Hax e Jam ``Inoltre pare
    che lei e il signor Fixer siate già stati quì, lasciandovi una sola
    giornata d'accesso.'' ``Ma com'è possibile? Non sono mai venuta quì.''
    ``Non lo so, è quello che ho indicato all'interno dei documenti. Mi
    spiace.'' ``Dannazione. Non si può fare nulla?'' ``No, mi spiace.
    Guardi, da quando hanno cambiato le regole molte persone stanno avendo
    problemi. Prima o poi si ristabilirà. Secondo me non andrà avanti molto
    così. Se vuole ritornare l'anno prossimo forse avremmo cambiato questa
    stupida legge.'' Elythia sbuffò e prese i documenti ``Capito. Grazie
    comunque. Buona giornata.'' ``Mi spiace ancora. Buona giornata.''
    Mentre Elythia stava uscendo la ragazza aggiunse ``Ah, se vuole mi
    hanno detto che le isole di Shinra, in questo periodo, sono magnifiche.
    Potreste volerci passare sulla via del ritorno a casa.'' ``Hm. Temo che
    ci sia qualcuno nel mio gruppo di viaggiatori che ne ha avuto
    abbastanza di paradisi tropicali.'' Ed uscì.

    Tornò dal gruppo e consegnò loro i documenti. ``Com'è andata?'' chiese Jam,
    alzandosi ``Possiamo passare?'' ``No.'' rispose secca Elythia ``Eh?''
    ``Eh?'' disse subito dopo Lissa ``EEEEH?!?'' urlò infine Hax, alzando
    la testa. ``Hanno detto che i vostri documenti sono irriconoscibili e
    che io e Fixer siamo già stati quì e che, al massimo, possiamo
    permanere un giorno. Non arriviamo neanche a Metra in quel tempo.''
    ``Retrogradi.'' commentò Hax, ributtanto la testa indietro e tornando a
    guardare il cielo. ``Beh. Allora dovremo trovare un altro modo per
    entrare, no?'' disse Lissa ``Già.'' ``\'E per questo che, durante i
    tempi morti del viaggio ho analizzato la conformazione di Phaion.''
    ``Cosa? Che passatempo è?'' ``Non si può mai sapere.'' ``Ma non
    dobbiamo fare una missione sotto copertura o cose simili.'' ``Beh, tu
    non ti preoccupare.''

    Lissa tirò fuori dallo zaino tre mappe, una politica, una fisica ed una
    topografica di Phaion. Tutte avevano una serie di segni e commenti a penna di vari
    colori su di esse. ``Come ti ho detto, Elly\dots{}'' continuò il
    discorso Lissa ``\dots{}ho fatto dei controlli in casi come questo ed
    ho notato varie cose.'' Stese le mappe a terra e quindi tutti si misero
    intorno, inginocchiati, tranne Hax, che si mise seduto sulla panchina,
    con i gomiti appoggiati sulle ginocchia e la testa appoggiata alle
    mani. ``Dunque, vedete questi?'' chiese lei, indicando dei cerchi
    intorno a delle strade d'alta percorrenza, in concomitanza con i
    confini dello stato. ``Bene. Queste sono dogane. Ce ne sono un gran
    numero, anche quì, in questi villaggettini.'' ed indicò dei villaggi di
    confine. ``Ho controllato la conformazione del terreno e sembra
    \emph{inattaccabile} per quanto riguarda l'entrare bypassando le
    dogane. Anche se\dots{}'' ed indicò una riga tratteggiata segnata sia
    sulla carta fisica che su quella topografica ``\dots{}grazie alle
    capacità della vostra Lissa c'è ancora speranza. Vedete questa linea?
    \'E una strada non battuta che è percorribile con un po' di fatica.
    Porta direttamente al cuore di Phaion evitando qualunque posto di
    controllo. Una volta dentro questo perimetro\dots{}'' ed indicò una
    riga rossa tratteggiata che distava qualche decina di furlong in scala
    dal confine, seguendolo ``\dots{}saremo al sicuro dai controlli e
    potremmo andarcene in giro indisturbati. Certo. Potrebbero ancora
    chiederci i documenti per la permanenza, ma di quello ci dovremmo
    preoccupare solo se ci ferma la polizia. Se non facciamo nulla di
    pericoloso non dovrebbero esserci problemi. Poi. Non so come lavori la
    polizia in questo stato, in realtà.''

    Si fermarono tutti a pensaere al piano ed alle sue implicazioni. Dopo
    un momento Elythia prese parola e commentò ``Dunque, se ho capito bene,
    Lissa, stai proponendo di infrangere chissà quante leggi di uno stato
    per portare una lettera. E tutto questo solo dopo venti minuti dopo
    l'inizio del lavoro.'' ``Esatto.'' rispose Lissa, mettendosi le
    mani sui fianchi e gonfiando il petto ``Oddio. Per solo un mese di
    stipendio, in pratica.'' poi, guardando i membri della squadra, i quali
    sembravano tutti pronti a mettere in atto quel piano ``Maledetta me e
    la mia etica lavorativa. Ok. Facciamolo.''

  \section{Celty - Shien - Settembre 2635}

    Celty era al palazzo del Conte, tra una consegna ed un'altra. Stava
    pensando all'incontro che aveva avuto l'altra sera con quel signore. La
    reazione era quella di chi sa cosa c'era dentro la busta che avrebbe
    dovuto consegnare e chi l'avrebbe ricevuta. Quello che l'aveva
    preoccupata di più era la frase dove la metteva in guardia nei
    confronti di quello che sarebbe accaduto entro breve. Era per quello
    che aveva deciso, la sera nella quale aveva incontrato l'uomo, di
    andare agli uffici doganali per modificare alcuni dati. Era riuscita a
    fare in modo che i giorni disponibili per Lissa e gli altri del suo
    gruppo fossero ridotti ad uno, come se avessero già visitato Phaion. In
    quella maniera poteva essere sicura che non entrassero e non si
    mettessero nei casini.

    Grazie a quello adesso era più calma, anche se si stava preoccupando di
    quali fossero i piani di Philip e di quanta gente ne avrebbbe sofferto.
    Sapeva che era qualunquista ed egoista, ma finchè Lissa era al sicuro
    lei dormiva sonni più tranquilli.

  \section{Rollers - Boschi di Phaion - Settembre 2635}

    ``Oh, Hax. Ora non guardare giù.'' lo rassicurò Lissa, dall'altro lato
    della gola, mentre il ragazzo era appeso ad una corda che la
    attraversava da lato a lato. Avevano dovuto utilizzare una corda con
    rampino in quanto era uno dei punti che andavano attraversati per
    entrare a Phaion. ``HA! La fai facile, te!'' rispose Hax, urlando,
    mentre si teneva aggrappato alla corda con tutte le forze ``Ma che
    minchia di strada ci hai fatto prendere, Lissa?'' ``Senti, questa era
    l'unica strada possibile, Hax. Fattenene una ragione, per gli dei. Ora
    continua a proseguire verso questo lato, altrimenti te ne starai lì
    attaccato per sempre!'' ``Ok, ok. Porca puttana.'' Il ragazzo iniziò a
    mettere una mano dietro l'altra, seguendo la corda, mentre questa
    ballonzolava ad ogni suo movimento. Ci vollero dieci minuti perchè il
    ragazzo arrivasse dall'altra parte, tra un'imprecazione e l'altra.
    ``Visto? Non è stato così difficile.'' lo schernì la ragazza ``Già,
    certo. Stronza.'' rispose lui, scherzosamente. Hax non si sentiva le
    gambe, aveva passato gli ultimi minuti a tenere tutto il corpo in
    tensione per la paura ed adesso non sapeva neanche più se riusciva a
    camminare. Alla fine, in un'altra
    decina di minuti, furono tutti dall'altra parte. Recuperarono la corda,
    che era stata bloccata dall'altra parte del crepaccio, e quindi
    ripresero a camminare all'interno della foresta, anche perchè Hax aveva
    recuperato le energie ed era pronto per continuare nel viaggio.

    Era notte e il viaggio prevedeva di attraversare delle grandi porzioni
    di boschi dell'entroterra di Phaion. Il problema più grave era il fatto
    che non potevano utilizzare luci troppo potenti, altrimenti avrebbero
    reso inutili tutti gli sforzi per passare inosservati. A causa di
    quello rischiarono più volte di finire dentro fossati, cadere in
    trappole dei cacciatori dei villaggi di quelle zone, finire dentro
    roveti ed altre cose pericolose. Dopo tre ore finalmente arrivarono in
    una zona pianeggiante. ``Oh, finalmente'' commentò Hax, una volta
    fuori, togliendosi dei ramoscelli dai vestiti e da sotto le fasce dello
    zaino. ``Già.'' aggiunse Elythia, spostando una fronda ``Ora è meglio
    se troviamo da dormire, sono distrutta.'' ``Sono completamente
    d'accordo. Dici che in quel villaggio hanno da dormire?'' ``Beh, voglio
    dire, non è che siamo in una zona barbarica. Avranno una locanda,
    immagino.'' ``Ok. Ottimo, allora, anche perchè, a quanto ho capito,
    mancherà ancora una giornata prima di arrivare alla città principale?''
    ``Hm.'' mugugnò Lissa ``Sì, forse meno. Dipende da che tipo di
    trasporti hanno in questo posto. Se riuscissimo ad arrivare abbastanza
    presto potremmo addirittura entrare nel palazzo domani sera, appena
    arrivati, così ce lo leviamo direttamente dalle scatole.'' ``Dici?''
    chiese Hax, mentre il gruppo si avviò verso il paesino ``Beh, sì. Alla
    fine abbiamo i dati. Ce li ha forniti il tizio. Se prendiamo un
    carretto, un treno o qualcosa di simile allora possiamo analizzare la
    mappa che ci ha fornito e fare un piano. Se poi riusciamo in fretta nel
    nostro piano possiamo goderci la città per un po' e poi fuggire.''
    ``Se dici che è possibile, Lissa\dots{}'' commentò Elythia
    ``\dots{}allora direi che non ci sono problemi, se per gli altri va
    bene come piano.'' gli altri del gruppo risposero in maniera
    affermativa. Il lavoro era una fregatura, quindi toglierselo dalle
    scatole il prima possibile era il metodo migliore, anche perchè
    avrebbero diminuito le probabilità di venir beccati dalle forze
    dell'ordine locali. ``Ok. Allora è deciso. Mi metterò a studiare
    assieme ad Hax il piano per domani sera. Facciamo così. Non importa a
    che ora arriviamo. Se riusciamo ad arrivare domani sera proveremo ad
    entrare. Più tardi è meno guardie ci sono all'interno dei palazzi,
    normalmente.'' ``Lasciamo stare il \emph{normalmente}. Guarda che
    lavoro ci è capitato, voglio dire. Non c'è nulla di normale fin
    dall'inizio.'' commentò Hax, inacidito ``Beh, dai, alla fine non
    dobbiamo uccidere nessuno.'' ``Oddio, perfavore, non aggiungere altro,
    non vorrei che capitasse che, in realtà, nella lettera, ci siano degli
    ordini aggiuntivi tipo \emph{Cara Xeresia, queste sono le tue nuove
    guardie del corpo}.''

    Appena finì la frase lui, Lissa e The Fixer iniziarono a guardare
    Elythia, che stava camminando di fronte a loro. ``Che avete?'' chiese
    lei, quando si accorse dei tre che la fissavano ``Beh,'' iniziò Hax,
    fingendo d'essere innocente ``potremmo aver avuto un qualche incidente
    che ha rotto il sigillo della lettera, no?'' ``Cosa? Scherzate, vero? E
    che fine ha fatto l'etica lavorativa?'' ``Ma senti, non vorrai altre
    fregature, vero?'' ``No, questo non posso permettertelo, Hax.'' ``Dai,
    Elly. Non possiamo dare un'occhiata? E se poi questo ha veramente messo
    dentro qualche direttiva stupida?'' incalzò Lissa, preoccupata a sua
    volta per questa cosa ``Sentite, se ci sono \emph{direttive} strane,
    noi diremo che o ci paga per l'incarico aggiuntivo oppure il nostro
    contratto è terminato lì. Fine. Che problemi ci sono, ragazzi?'' ``Ah,
    in effetti non avevo pensato a questa evenienza. Beh, se la mettiamo
    così allora non ci sono problemi. Grazie Elythia.'' concluse Lissa.
    Dopo una ventina di minuti arrivarono al villaggio. Non ci misero molto
    per raggiungere l'unica locanda disponibile: ``Alla Botte Misteriosa''

    Dentro c'erano pochi avventori, il locandiere dietro il bancone, una
    cameriera che girava svogliata per i tavoli ed un uomo con una giacca
    leggera ed un Fedora in testa che stava chinato sul bancone, seduto su
    uno sgabello da bar. Si avviarono verso il bancone, schivando le sedie.
    Una volta arrivati appoggiarono gli zaini per terra e vennero salutati
    dal locandiere. ``Salve viaggiatori!'' fece, solare, mentre si alzarono
    dopo aver scaricato gli zaini. In effetti il loro aspetto tradiva la
    provenienza. Per non parlare del fatto che in un posto del genere si
    saranno conosciuti tutti. ``Ah. Salve!'' rispose Elythia ``Allora. Che
    posso fare per voi?'' ``Volevamo vedere se avevate posto per la notte.
    Basta anche una sola stanza per cinque persone, non ci facciamo
    problemi.'' ``Haha! Ma certo, ma certo. Se volete ho due stanze
    separate.'' ``Ah. Ottimo. Allora le prendiamo. Quanto vengono?''
    ``Vengono cinquanta matra per stanza.'' ``Hm.'' Elythia controllò nella
    sua borsa. Aprì il proprio portamonete e contò i soldi che aveva
    dietro ``Ottimo. Ecco a lei. La cena è compresa?'' e gli porse una
    banconota da cento metra ``Ma certo,
    ovviamente. Cena e colazione.'' ``Fantastico. Dove sono?''  Elythia
    alzò la mano, indicando verso l'alto ``Sono di sopra?'' ``Certo. Stanza
    203 e 204. Eccovi le chiavi.'' e il locandiere fece scivolare due
    chiavi attaccate con dello spago a dei pezzi di legno con sopra i
    numeri delle stanze ``Grazie.'' rispose lei, prendendo entrambe le
    chiavi ``Grazie a voi. Non si vedono più molto spesso dei visitatori da
    fuori da queste parti da un po' di tempo.'' ``Eh.'' ``Una volta quando
    c'era la fiera estiva quì era sempre pieno di gente nuova. Dannazione a
    quei damerini del parlamento.'' ``Non mi dica.'' ``Ma sa che le dico?
    Prima o poi il re se ne accorgerà e allora vedrà se non mettono tutto a
    posto. Così potremo tornare ad avere i visitatori di sempre. Per
    Gnome.'' Elythia socchiuse per un momento gli occhi quando sentì
    nominare Gnome. Era il dio del metallo del suo culto. Proteggeva, per
    estensione, anche quello di mercanti e di artigiani. Come mai quella
    persona conosceva il culto dei XIII? Era vero che lei non sapeva da
    dove veniva, visto che gli spiriti che la seguivano l'avevano scelta e
    poi si erano presentate come servitrici di Eclipse e quindi da lì
    lei ha iniziato a seguire i dettami di quella religione, però era
    strana come cosa. Aveva sempre sentito parlare di luoghi molto
    differenti da quelli da EMR e le altre. ``Haha. Ma certo, ne sono
    sicura. Ora scusi, ma preferirei andare su a darmi una rinfrescata. Sa,
    sono molto stanca dal viaggio.'' ``Ma certo. Non si preoccupi. Poi
    venite a mangiare. Questa sera abbiamo un brasato ottimo!'' Elythia
    prese la sua borsa a tracolla, se la mise in spalla e si avviò su per
    le scale. Gli altri si congedarono dal mercante a loro volta e quindi
    la seguirono. Erano tutti particolarmente stanchi, inoltre Lissa ed Hax
    dovevano mettersi a studiare il piano per il giorno dopo, quindi
    volevano mettersi subito all'opera.

  \section{Seysill - Alla Botte Misteriosa - Settembre 2635}

    \emph{Che strano gruppo} pensò Seysill, mentre guardava il gruppo di
    viaggiatori salire al piano superiore per andare nelle loro camere.
    Poi, quando anche l'ultimo scomparì lungo le scale, tornò a contemplare
    il bicchiere di vino che aveva ordinato. Avrebbe sperato di non vedere
    per un po' gli alcoolici di Phaion, però eccolo lì, in una locanda di
    chissà quale posto sperduto nelle campagne di Phaion, a sorseggiare il
    vino di quella zona. \emph{Dannazione.} imprecò mentalmente, mentre
    muoveva ritmicamente le dita della mano destra, appoggiata sul bancone.
    ``Allora, signore, le va bene il vino?'' iniziò il locandiere, per fare
    due chiacchiere ``Oh? Oh, sì, certo. Grazie mille. \'E ottimo,
    veramente.'' rispose, cortesemente. Non era proprio il massimo, però
    non poteva aspettarsi grandi cose dal vino non imbottigliato per le
    grandi cantine della capitale, in realtà ``Davvero? Lo dirò al
    produttore appena lo vedo. Ma allora, mi dica, come mai si trova da
    queste parti? Non mi sembra uno che si fa dei giri in campagna per
    sfizio.'' ``Eh, sono in vacanza.'' ``Ah. Ed ha deciso per il tour
    enogastronomico, eh? Haha!'' commentò il locandiere, per poi scoppiare
    in una fragorosa risata ``Bravo, lei. \'E un ottimo modo per passare le
    proprie vacanze, mio caro. Conosce qualcuno che la ospita o ha bisogno
    di un posto dove dormire anche lei come i viaggiatori di prima?'' ``Hm,
    in realtà\dots{}'' poi si fermò a pensare. Era vero. Cosa avrebbe fatto
    per nascondersi finchè non si sarebbero calmate le acque? Avrebbe
    affittato una casetta in campagna? Iniziò a vagliare una lunghissima
    serie di opzioni, per poi ricevere l'illuminazione ``\dots{}ho qualcuno
    che mi ospita, sì.'' ``Ah, peccato. Comunque si diverta.'' e si girò
    per rimettersi a lavorare dietro al bancone ``Ah, senta. Ci vuole degli
    stuzzichini col vino?'' ``Hm.'' rispose Seysill, col morale un po'
    ripreso ``Sì, grazie, se fosse così cortese.'' ``Ma certo.'' il
    locandiere gli porse un tagliere con sopra una serie di spuntini ``Ecco
    a lei.'' ``Grazie.''

    Dopo una ventina di minuti Seysill ebbe finito col vino e con gli
    stuzzichini, prese il suo fedora, che aveva appoggiato sul bancone, e
    se lo mise in testa. Lasciò sul tavolo una banconota da venti matra e
    si allontanò. ``Grazie!'' gli disse il locandiere ``E torni presto,
    signore. Buona vacanza.'' ``Grazie.'' rispose lui, senza girarsi,
    alzando la mano destra in segno di saluto ``Vedrò di seguire il suo
    consiglio.'' e quindi uscì dalla locanda. Mentre le porte si chiusero
    sentì i ragazzi del gruppo di prima che discutevano animatamente mentre
    scendevano le scale. Stava ancora cercando di capire cosa ci facesse un
    gruppo simile da quelle parti. Beh, non che importasse. Non riuscì a
    trattenere un sorriso, comunque. Era bello vedere che c'era ancora
    gente che girava il mondo così giovane solo per farsi un viaggio.

    Si avviò verso l'auto e poi, dopo aver sbloccato le serrature,
    considerò che andare da Durga a chiederle ospitalità sarebbe stato come
    andare in bocca al leone, però lei aveva un modo differente di vedere
    il mondo e, probabilmente, aveva una casa separata da quella delle
    teste alte del culto. Il massimo per nascondersi e per non vivere tra
    un albergo ed un altro.

  \section{Rollers - Metra - Settembre 2635}

    Erano appena arrivati a Metra, una giornata di viaggio dove Hax e Lissa
    ci avevano dato giù duro con la pianificazione per la serata. Le mappe
    erano incredibilmente dettagliate e le posizioni delle guardie in base
    alle ore erano molto precise ed erano state di grande aiuto per
    studiare il percorso da seguire. Il piano si basava su un concetto
    molto semplice. Esisteva un punto di scarico per i rifiuti della
    cucina che, ovviamente, non veniva utilizzato dopo una certa ora.
    Entrando da lì, poi, bisognava fare una parte di strada in giro per i
    corridoi del castello, quindi prendere un percorso segreto per i
    servitori che collegava una stanza particolare dietro la sala da
    pranzo, senza descrizione, probabilmente un ripostiglio o un posto dove
    appoggiare le pietanze prima di servirle. Grazie a quel passaggio
    segreto sarebbero arrivati direttamente alla camera della principessa.
    Una volta là avrebbero dovuto svegliarla senza farla urlare e quindi
    spiegarle la situazione. La lettera avrebbe fatto il resto. ``Ottimo,
    no?'' aggiunse Lissa, fiera del proprio lavoro ``Entriamo, facciamo il
    lavoro, usciamo. Bicchiere d'acuqa, molto probabilmente.'' non voleva
    dire che era facile di sicuro perchè aveva paura delle conseguenze
    ``Già. Allora, chi va?'' ``Io ed Hax. Abbiamo studiato questo piano
    appositamente in base alle nostre competenze. Siamo i due più
    silenziosi. Voi farete da pali fuori dallo scarico dei rifiuti della
    cucina. Così se succede qualcosa sarete pronti.'' ``Se succede
    qualcosa?'' ``Non si può mai sapere. In caso di problemi abbiamo già
    preparato una via di fuga alternativa. Se ci tagliano l'uscita dalle
    cucine dovremo calarci al di fuori di uno di questi balconi\dots{}'' ed indicò
    una delle sporgenze della struttura disegnate sulla mappa
    ``\dots{}lanciando un rampino su una delle case vicine al palazzo.
    Siamo fortunati che questo \emph{castello} non sia uno di quelli
    isolati completamente dal resto della città, così avremo molte più vie
    di fuga. In caso di questo scenario dovremo incontrarci da qualche
    parte di sicuro. Direi che\dots{}'' e Lissa indicò una locanda chiamata
    ``La Corona e la Mela'' ``\dots{}quella locanda va bene. Comunque sono
    tutti mezzi di sicurezza di prassi, in realtà. Spero che non accada
    niente di grave.'' ``Sì, infatti.'' commentò Elythia ``Bene, allora se
    non avete commenti io direi che potremmo avviarci verso il palazzo. Ci
    arriveremo là in un'oretta. Questa città è grande.'' continuò Lissa
    ``Vuol dire che ci arriveremo là circa all'una. Ora perfetta, no?''

    Arrivarono all'una meno cinque, come avevano predetto. La città era
    silenziosa, anche perchè era la notte di un giorno infrasettimanale,
    quindi era plausibile che nessuno girasse a quell'ora, se non dei
    poliziotti, che il gruppo aveva attentemente evitato passando per
    vicoletti e parchi. Risucirono a raggiungere l'entrata del palazzo
    senza farsi vedere da alcuna guardia, sfruttando l'ombra delle mura.
    Una volta là Lissa lasciò la sua borsa ad Elythia, Fixer e Jam,
    tenendosi solo il pugnale da sopravvivenza, la sua pistola silenziata,
    una corda rinforzata con rampino metallico e gli arnesi da scasso,
    tutto legato al corpo o dentro fondine aderenti per non fare rumore, mentre
    Hax si fissò meglio lo zaino alla schiena, per non farlo muovere.
    Agganciò meglio il piede di porco al lato della borsa per evitare che
    strisciasse sullo zaino, poi fece ``Ok, pronto.'' commentò, sottovoce, facendo qualche
    saltello per sentire che tutto fosse bloccato al meglio. ``Perfetto.
    Allora noi andiamo. Ci vediamo tra poco, ragazzi.'' aggiunse Lissa,
    sempre sottovoce, indicando la porta leggermente rialzata rispetto al
    terreno.

    Hax si appoggiò al muro, facendo da scaletta a Lissa, la quale salì in
    piedi sulle spalle del ragazzo, appoggiandosi alla parete, per arrivare
    alla serratura della porta. Estrasse alcuni dei grimandelli che aveva
    nel suo set ed iniziò a lavorare sulla serratura. Dovette fermarne
    alcuni in posizioni precise con del nastro adesivo in quanto della
    serratura era più complessa del normale. Non che si aspettasse niente
    di differente da una serratura per l'accesso ad un palazzo reale, in
    realtà, ma lavorare sulle spalle di una persona che si lamenta era una
    palla. ``La vuoi smettere di lamentarti, scemo?'' fece, sottovoce,
    dandogli un leggero colpo col piede. ``Sìsì, scusa.'' rispose lui,
    sempre sottovoce. Dopo qualche minuto riuscì ad aprire la porta con un
    leggero \emph{Clack}. Rimise apposto i grimandelli, anche quelli
    attaccati alla porta e quindi entrò nelle cucine. Da lì aiutò il
    ragazzo a salire e quindi accostò la porta nel caso non si aprisse
    dall'interno senza chiavi. Non voleva dover scassinare di nuovo quella
    porta, soprattutto se erano inseguiti da delle guardie che volevano
    metterli in galera. ``Ottimo.'' commentò, sottovoce, dopo che entrambe
    erano nelle cucine, poi indicò un'uscita ``Andiamo.'' fece, avviandosi
    nel buio delle cucine. Si appoggiarono ai lati delle scale subito dopo
    l'uscita per sfruttare meglio le ombre per nascondersi. Ora iniziava la
    parte dura. C'erano dei punti caldi nella loro salita verso la stanza
    della principessa, che era dall'altra parte rispetto a quella entrata.
    Lissa indicò Hax, per poi indicargli la cima delle scale, dopo essersi
    passata le dita vicino agli occhi. Il ragazzo annuì e quindi iniziò a
    salire le scale lentamente, strisciando contro il muro. Hax arrivò in
    cima alle scale e, dopo aver guardato a destra ed a sinistra, si
    accertò che non ci fosse nessuno in linea di vista, quindi si girò
    verso Lissa e le fece cenno di avvicinarsi con la mano. Lei fu molto
    più veloce a raggiungerlo. Aveva meno zavorra e, soprattutto, era
    sicura che non ci fosse nessuno. Ora dovevano superare un corridoio,
    quindi la sala del trono. Infine passare attraverso la sala da pranzo
    verso il passaggio segreto. Sarebbero passati in due punti caldi. Hax
    scivolò lungo il corridoio, tenendosi il più lontano possibile dalle
    poche torce presenti in quella zona, facendo da avanguardia per Lissa.
    
    Una volta arrivati alla sala del trono Lissa toccò la spalla ad Hax e,
    avvicinandosi al suo orecchio, gli sussurò ``Vado avanti io.'' lui
    annuì e poi rispose, avvicinandosi a sua volta ``Sai che mi fai venire
    i brividi quando fai così?'' lei, dopo avergli dato uno scappellotto,
    avanzò nella sala del trono. Lì era il posto dove si aspettavano più
    resistenza di tutto il palazzo. Dovevano stare attenti al massimo se
    non volevano guai. Lissa striciò lungo la parete che non era in luce
    rispetto all'illuminazione della luna che entrava dal lucernario. Fu in
    quel momento che le venne in mente che, se avesse saputo prima del
    lucernario, sarebbero entrati da quella parte. Pareva più facile.
    Fortunatamente non trovarono nessuna guardia che controllava la sala
    del trono, quindi Lissa indicò all'amico di proseguire velocemente
    verso la sua posizione per poi continuare all'interno della sala da
    pranzo. La sala da pranzo aveva delle credenze sui muri ed una tavolata
    a ferro di cavallo che riempiva quasi per intero tutto lo spazio
    libero. Lissa corse dall'altra parte, stando attenta alle due uscite,
    dalle quali, però, potevano uscire solo servitori, secondo la mappa che
    si erano preparati. Notando che non c'era nessuno arrivò all'entrata
    della stanza antecedente al passaggio segreto, dove fece cenno ad Hax
    di raggiungerla. Una volta lì chiusero la porta dietro di loro. La
    stanza era buia e fredda. Pareva essere una specie di cella refrigerata
    per mantenere alcune pietanze o bevande che poi venivano utilizzate
    direttamente per i pasti. ``Hei, Lissa.'' sussurrò Hax, mentre la
    ragazza stava cercando una qualche leva per far aprire il passaggio
    ``Hai notato?'' ``Sì.'' rispose rapidamente lei. Effettivamente non
    avevano trovato guardie. Era una cosa strana. Era vero che a quell'ora,
    normalmente, la sicurezza di un castello era al minimo, però anche non
    trovare nessuno era strano. ``Senti, cosa pensi che sia?'' ``Non lo so,
    seriamente. Dici che abbiamo beccato un qualche cambio della guardia o
    cose del genere?'' ``Mh. Non lo so.'' Hax pareva respirare un po' più
    affanosamente ``Hax. Calmati. Non è nulla, dai. Senti, questo posto non
    è zona di guerra, nè di tumulti. Lascia stare. So cosa stai pensando.''
    Lissa strisciò le mani vicino ad uno dei cassoni che contenevano le
    ventole che forzavano il circolo d'aria all'interno della stanza finchè
    non sentì una sporgenza. ``Bingo.'' fece, sorridendo, mentre tirò verso
    il basso quella che era una levetta. Con un leggero \emph{Clack} il
    murò s'incassò leggermente. Lissa spinse quella porzione verso il muro,
    la quale rivelò un cunicolo che dava verso l'interno delle mura del
    castello. ``Su, smettila di preoccuparti ed andiamo. O mi dici che mi
    sarei dovuta portare dietro Jam?'' ``Figurati. Andiamo.'' ``Ottimo.''

    Seguirono i cunicoli per alcuni minuti, salendo rampe di scale e
    girando ad incroci. Lì potevano parlare a voce un po' più alta e
    potevano utilizzare delle torce, almeno per leggere la mappa che si
    erano portati dietro. ``Senti, comunque adesso andiamo là, facciamo il
    lavoro e ci leviamo di torno il più in fretta possibile.'' ``Hai
    ragione.'' poi, Hax ebbe un sussulto. ``H-Hai visto?'' sussurrò,
    avvicinandosi alla ragazza ``Cosa?'' ``Mi pareva di aver visto un'ombra
    muoversi là.'' indicando un incrocio ``Dai, smettila di essere
    paranoico. Chi vuoi che giri a quest'ora? Sarà stato un topo.''
    ``Dici?'' ``Sì, smettila.'' Questo significava tenere a bada i
    comportamenti di Hax. Era una persona che aveva paura, ma combatteva
    queste paure in maniera caciarona. Era per quello che si ficcava sempre
    nei guai. Era per bloccare quelle paure al nascere. E Lissa era una
    maga nel farlo, con lei Hax sembrava reagire subito alle paure in
    maniera razionale. Dopo averlo calmato non ci volle molto per arrivare
    alla porta che dava sulla stanza della principessa Xeresia. Lissa
    attivò una leva che fece scattare la porta, permettendo loro di
    aprirla.
    
    Entrarono in una stanza enorme, con un letto a baldacchino da,
    probabilmente, tre piazze. Era arredata per ospitare un'adolescente.
    Lissa si sentiva già male. Hax l'esortò ad andare al letto e svegliare
    la principessa senza farla urlare. Lissa scivolò ai piedi del letto. Il
    problema era che la ragazza era al centro del letto. La programmatrice
    salì sul letto, tentando di far meno rumore possibile mentre spostava
    il velo del baldacchino e quindi di evitare che il materasso si
    muovesse troppo. Entro breve fu a portata del volto della principessa.
    Con un movimento fulmineo bloccò la bocca della ragazza, la quale si
    svegliò di soprassalto, incominciando a mugugnare, mentre provava ad
    urlare. Subito Lissa appoggiò l'indice sinistro alla
    bocca e le fece capire di non urlare. ``Sssh.'' sibilò la ragazza, per
    poi continuare a sussurrare ``Ok, calma signorina. Non ti vogliamo fare
    nulla. Siamo stati mandati quì da Gabriel. Gabriel Bard, sai di chi sto
    parlando?'' la ragazza annuì scuotendo la testa ``Ottimo. Guarda.''
    tirando fuori la lettera ``Abbiamo anche una lettera da parte sua.
    Seriamente. Senti, se prometti di non urlare ora tolgo la mano. Ho la
    tua parola?'' La ragazza annuì. Lissa spostò la mano pronta a
    rimetterla sulla bocca della principessa se avesse urlato. Questa,
    però, mantenne la promessa ed iniziò a parlare in fretta, a metà tra
    eccitamento e agitazione ``Vi ha mandato Gabriel? Come sta? Chi siete
    voi? Come mai non è potuto venire lui?'' ``Ok, ok.'' rispose Lissa,
    mettendo le mani avanti ``Allora. Io sono Lissa e lui è Hax. Siamo del
    gruppo di tuttofare Rollers. Siamo venuti noi perchè lui aveva finito i
    giorni a disposizione per entrare in questo stato o una cosa del
    genere. Che poi li avessimo finiti anche noi questa è un'altra storia.
    Gabriel sta bene, ma è preoccupato per te, ovviamente, altrimenti non
    ci avrebbe mandato. E, come ho detto prima, ci ha detto di portarti
    questa lettera.'' e le consegnò la lettera. ``Ora, se hai delle domande
    da fare o qualcosa da riportargli basta che ce lo dici e noi\dots{}''
    
    Lissa non riuscì a finire la frase che la porta della camera della
    ragazza venne sbattuta ed entrò una cameriera, in tenuta da notte,
    tutta affannata. Senza neanche guardare dentro disse, ad alta voce
    ``Signorina Xeresia, Signorina Xeresia! Presto, deve venire subito!''
    poi alzò la testa per guardare nella stanza e notò Hax in piedi che si
    stava facendo i fatti suoi e Lissa inginocchiata sul letto vicino alla
    ragazza, seduta sempre nel letto con in mano la lettera. ``Ma voi?''
    iniziò a chiedere la cameriera. Lissa fece un cenno con la testa verso
    la cameriera ed Hax scattò verso di questa, arrivandole alle spalle e
    bloccandole la bocca. Xeresia subito difese i due ragazzi, informando
    l'inserviente di chi fossero e che non volevano farle del male, quindi
    Hax lasciò la cameriera, scusandosi. ``Signorina. Non c'è tempo.'' la
    cameriera era evidentemente agitata ``Che succede?'' ``C'è qualcuno nel
    castello. Oltre ai suoi due amici, intendo.'' ``Cosa?'' Hax ebbe un
    sobbalzo ed iniziò a respirare profondamente. Lissa, invece, sentì una
    scarica d'adrenalina, iniziò a respirare più rapidamente. ``Non so bene
    che stia succedendo, ma ho visto una persona aggirarsi nel palazzo.
    Tutte le guardie sono scomparse. TUTTE.'' ``Che significa?'' chiese la
    principessa, scendendo dal letto e mettendosi una camicia da notte
    ``Non lo so, signorina, ma temo che possa capitarle qualcosa di
    brutto.''

    Lissa era scattata in piedi, stava guardandosi in giro nervosamente.
    Era vero che a tenere calmo Hax lei era il massimo, ma quando le teorie
    dell'amico si rivelavano corrette era lei quella che andava fuori di
    testa. Iniziò a progettare piani, calcolare piani di fuga, creare
    congetture su chi fossero, cosa avrebbero dovuto affrontare, cosa
    volessero, come fermarli. Hax, invece, sembrava stesse raggiungendo
    l'illuminazione. Ad un tratto un ombra entrò nella stanza, caricando un
    colpo di pugnale, diretta alla cameriera. Questa cacciò un urlo dal
    profondo dei polmoni. Hax estrasse il piede di porco dallo zaino con un
    movimento del polso, si girò verso l'avversario e, con un fendende
    dall'alto verso il basso, colpì il braccio destro dell'assalitore
    sconoscosciuto, espirando per darsi più forza, affondando la parte affilata dello strumento nella
    carne del suo braccio, facendolo urlare di dolore. Subito cambiò
    impugnatura dell'arma e, con un movimento fulmineo, lo colpì in volto
    con la parte arrotondata dell'arnese. Lo sconosciutò cadde a terra,
    svenuto. ``Oh, cazzo.'' disse subito il ragazzo, chinandosi sulla
    persona ``Non l'ho ucciso, vero?'' e tolse il cappuccio all'assalitore.
    ``Lo conoscete?'' chiese, rivolgendosi alla principessa ed alla
    cameriera ``N-No.'' rispose prontamente l'inserviente. ``Cazzo.''
    imprecò, alzandosi, dopo aver controllato che il cuore battesse ancora.
    A quel punto notò Lissa, che sembrava sul punto di una crisi di nervi.
    Chiuse la porta e le si avvicinò ``Ok,'' disse, rivolto a Xeresia ed
    alla cameriera ``aspettate un secondo.''

    ``Ok un cazzo, Hax'' grugnì la ragazza fuori dai denti, in preda a dei tic nervosi,
    mentre lo guardava fisso in volto con gli occchi fuori dalle orbite,
    con il fiato corto ``Quel Gabriel ci ha fottuti.'' ``No, aspetta.''
    ``Ci ha messi in una trappola. Ci ha incastrati. Dobbiamo andarcene.
    Ora.'' ``E che ne facciamo della principessa?'' ``Lavoro finito. Non
    possiamo stare quì. Ci ammazzeranno, oppure ci incastreranno per
    l'assassinio di qualcuno d'importante.'' poi si girò verso la
    principessa e le urlò ``Sei contenta del piano del tuo ragazzo, eh?
    Chissà cosa c'è scritto sulla lettera? Huh? Che farà ricadere la colpa
    su di noi mentre te puoi fuggire. Che nasconderà la tua scomparsa con
    un qualche rapimento/omicidio, eh? EH? Rispondi, maledizione!'' la
    ragazza era sull'orlo delle lacrime ``Lissa!'' urlò Hax, scuotendola
    per le braccia ``Ascolta. Non è possibile. C'è qualcos'altro.''
    ``Tipo?'' ``Non lo so. Un colpo di stato?'' ``Te e le tue teorie
    cospirazionali. Che cazzo dici?'' ``Senti, ascolta. Non possiamo
    lasciare quì la ragazza indifesa. Non possiamo neppure incolparla di
    qualche piano malefico nei nostri confronti!'' ``E allora cosa
    facciamo?'' ``Fuga?'' la ragazza sembrò aver visto una luce in fondo ad
    un tunnel e poi, mentre rallentava il respiro, rispose, guardandolo in
    volto ``Fuga.'' Hax le lasciò le braccia e, ritornandole lo sguardo,
    fece ``Ok, vediamo solo di non fare troppo casino.''

    In quel momento la porta venne aperta con forza da una coppia di
    persone incappucciate. Hax e Lissa, senza neppure pensare, si buttarono
    contro la libreria appoggiata vicino alla porta e la rovesciarono
    contro i due assalitori, i quali rimasero schiacciati dal mobile. Ne
    presero uno a testa e li fecero svenire con un calcio in testa. ``Ma
    cazzo. Quanti ce ne sono?'' chiese Hax a nessuno in particolare,
    guardando i due tizi svenuti a terra. ``Chi sono queste persone?''
    chiese ai due la principessa, evidentemente preoccupata. ``Ah, se non
    lo sapete voi chi sono le persone che stanno assaltando il vostro
    castello temo di non riuscire a rispondervi.'' rispose Lissa,
    estraendo con la mano destra il pugnale da sopravvivenza e con la
    sinistra, dopo aver sbloccato il bottone a pressione che l'assicurava
    nella fondina, la pistola silenziata. ``Che facciamo con la ragazza?''
    chiese quindi Lissa ad Hax. ``Per favore, salvatela!'' li implorò la
    cameriera, mettendosi in ginocchio ``Beh? Che vuoi? Lasciarla quì con
    la possibilità che la ammazzino?'' commentò Hax, ignorando
    completamente l'inserviente ``Ma non è come mettere il culo davanti
    alle pedate?'' ``Hm.'' ``Non starete discutendo veramente su questa
    cosa?'' chiese la principessa, evidentemente irritata ``CERTO!''
    risposero assieme i due ``Vedi?'' commentò Lissa, guardando Hax ``\'E
    per questo che non sopporto le missioni di salvataggio. Quando becchi
    un nobile è sempre una palla.'' ``Ma questo non significa che possiamo
    lasciarla quì. Scusa, metti che stia avvenendo un colpo di stato.''
    ``Se è veramente così c'è un motivo in più per lasciarla quì,
    maledizione.'' ``Un colpo di stato?'' chiese, agitata, Xeresia ``Cosa
    significa?'' Ignorando quelle parole i due continuarono a discutere
    ``Senti, a me non importa di rivolte, assassinii o altre stronzate,
    Lissa. Se lei rimane quì verrà sicuramente uccisa o peggio.'' ``E la
    cameriera?'' Hax si girò a guardare la cameriera, la quale fece un
    cenno al ragazzo, quindi tornò a fissare Lissa ``Immagino che non le
    faranno nulla.'' ``Mh.'' la ragazza mugugnò, tentando di pensare a
    tutto ciò che implicava il portarsi dietro una personalità così
    \emph{sensibile} di quella nazione, per poi rispondere ``Ok, hai vinto
    te questa volta. Lo sai che Elythia ci ammazzerà per questo vero?''
    ``Non se lei paga.'' ``Sè, vabbeh.'' Hax si girò verso la principessa
    ``Vero?'' ``Eh?''

    Sentendo quest'ultima uscita Lissa si mise la mano sulla faccia per
    poi sbuffare, per poi dire ``Ok, va bene, basta. Andiamo prima che arrivino altri di
    questi stronzi.'' ed avviarsi verso la porta che dava al passaggio
    segreto. ``Ci segua.'' fece Hax, rivolgendosi a Xeresia e poi,
    girandosi verso la cameriera e facendole un cenno con la testa, le
    chiese ``Lei ce la fa?'' ``Sì'' rispose questa,
    annuendo ``\dots{}non vi preoccupate, non dovrebbero fare molto ad una
    cameriera.''. Hax quindi andò dietro a Lissa dentro nel cunicolo, seguito a
    sua volta dalla principessa.

    Il ragazzo raggiunse Lissa, la quale aveva già estratto il suo pugnale
    da combattimento con la destra e la pistola silenziata con la sinistra.
    ``Dici che dovremmo combattere?'' chiese Hax, notando le armi ``Secondo
    te, Hax?'' chiese, indispettita, Lissa ``Voglio dire, non è che siamo
    all'interno di un castello, durante un attacco di qualche stronzo
    incappucciato, entrati di nascosto a nostra volta con al seguito una
    principessa, no?'' ``E su\dots{}'' ``Nono, infatti. Non penso ci sia
    nulla di preoccupante. Le ho tirate fuori solo perchè avevo voglia di
    giocherellarci.''

    Finita di dire questa cosa Lissa si fermò in mezzo al corridoio poco
    illuminato. Si girò verso Hax e gli fece un segno, appoggiandosi
    l'indice sinistro alle labbra, tenendo stretta l'arma da fuoco. 
    Hax si fermò a sua volta e si zittì.
    Erano vicini alla rampa di scale che li avrebbe portati ai piani
    inferiori, dai quali sarebbero potuti uscire. La ragazza si schiacciò
    contro l'angolo che dava sulla rampa, appoggiando la mano destra al
    muro ed alzando la pistola all'altezza del volto. C'erano
    effettivamente dei rumori molto soffusi in fondo alle scale. Hax si
    appoggiò al muro e incicò alla principessa di fare la stessa cosa alla
    sua destra, così da essere coperta. Lissa non poteva permettersi di
    combattere quasi alla cieca, perciò fece cambio di paradigma.
    
    Tutti gli spigoli vennero evidenziati. Non aveva voglia di stare lì più
    tempo del necessario, perciò, dopo aver messo a posto il pugnale nella
    fondina, avvicinò la mano destra alla pistola ed iniziò ad applicare
    degli algortimi all'arma. Selezionò alcuni pezzi di codice presenti
    nella sua libreria ed iniziò a combinarli. Recuperò quello per la
    connessione di oggetti, quello per la generazione del plasma, quello
    per l'attivazione condizionata degli algoritmi e quello
    per il raffreddamento degli oggetti. Posizionò in primis quello per la
    connessione ai lati dell'impugnatura dell'arma, selezionando i
    proiettili come uno dei due capi per la connessione. Questo fece
    comparire una serie di fili di luce bianca che collegò ogni proiettile
    ai dischi che rappresentavano l'algoritmo. Subito sopra
    i dischi di collegamento posizionò quelli per la generazione del
    plasma, i quali si posizionarono a qualche centimetro di distanza dai
    primi, con delle linee rosse che collegavano punti differenti dei due.
    Collocò quindi gli algoritmi di raffreddamento ai lati della canna,
    collegandoli alla parte interna della canna. Non poteva rischiare che
    alla sua \emph{Ace of Spades} si fondesse la canna. Temeva che non
    avrebbe avuto tempo per passare da un armaiolo per un bel po' di tempo.
    Come ultima cosa posizionò l'algoritmo di attivazione subito dopo la
    camera di scoppio, settandolo in modo che attivasse l'algoritmo per la
    generazione del plasma collegato ad un proiettile quando questo fosse
    passato attraverso di esso. La cosa aveva tre effetti. Oltre a quello
    ovvio di avvolgere il proiettile con del plasma, serviva per non far
    esplodere l'intero caricatore, oltre ad avere un effetto di espansione
    del gas maggiore, visto che il contatto del plasma con le pareti
    raffreddate tramite un algoritmo avrebbero spinto in avanti il
    proiettile ancora di più del solo gas prodotto dall'esplosione della
    polvere da sparo contenuta nella cartuccia.
    
    ``Hax.'' sussurrò la ragazza ``penso siano in due. Io
    prendo quello di destra. Tu riesci a fermare l'altro per qualche
    secondo?'' ``Hm, forse? E se è uno solo?'' ``Se è a destra lo prendo
    io, altrimen\dots{} Smettila di fare domande stupide, scemo. Il
    problema è se sono più di due.'' ``Dovrebbero essere più di tre?''
    ``Non riesco a contare la gente dal rumore che fanno i loro vestiti,
    HAX.'' ``Ok, capito. Dammi un secondo.''

    Hax fece a sua volta cambio di paradigma e si prese del tempo per
    prepararsi. Non poteva rischiare di far esplodere i passaggi del
    castello. Notò che anche Lissa aveva utilizzato degli algoritmi, solo
    che a lei bastava poco tempo per fare qualcosa di fatto bene, che non
    esplodesse a cazzo. A lui serviva un po' più di tempo\dots{} Decise di
    non utilizzare degli algoritmi per la \emph{proiezione a distanza} di
    energia. Sarebbe riuscito a controllare meglio e più in fretta le cose
    così. Inizò a comporre un algoritmo per la levitazione. Sarebbe stato
    stancante, ma non l'avrebbe utilizzato per molto. L'idea era di levarsi
    da terra
    quel poco che basta per poter scivolare lungo le scale a tutta velocità
    contro l'avversario. Una volta finito l'algoritmo, che posizionò lungo
    i punti cruciali del lato sinistro del proprio corpo, si mise a pensare
    all'algoritmo per l'impulso iniziale della scivolata. 

    ``Hem,'' iniziò Xeresia, sussurrando, appoggiando la mano sulla spalla
    di Hax ``scusate, ci sono problemi?'' I due si girarono per fissarla,
    con gli occhi che brillavano al buio per il passaggio di paradigma,
    ``C-che hanno i vostri occhi?'' chiese lei, preoccupata ``Non hai mai
    visto degli scienziati all'opera?'' chiese Hax ``Non lavorate in
    laboratori lontani dagli occhi degli altri? Non avete paura che vi
    denunci?'' ``E con chi hai parlato?'' incalzò Lissa, acida ``Con
    qualche setta iper-conservazionista?'' Hax decise di lasciar stare la
    conversazione e tornare a mettere a posto gli ultimi dettagli. Entro
    breve sarebbero arrivati a contatto con della gente poco simpatica. ``E
    poi a chi ci denunci, principessa? Alla polizia di stato che non riesce
    neanche a proteggere il tuo palazzo?''

    Mentre Lissa litigava con l'altra ragazza Hax tornò a sistemare
    l'algoritmo d'impulso. Tentò di trovare il sistema che avrebbe dovuto
    fare meno danni possibili, passando per accelerazioni elettromagnetiche
    lineari, esplosioni controllate ed effetti frusta. Alla fine decise che
    creare una bolla di aria compressa per poi rilasciarla in direzione
    opposta a dove doveva andare era l'idea meno pericolosa per la
    struttura e per se stesso.

    Hax sentì dei rumori farsi sempre più forti dalle scale. Probabilmente
    tutti quei discorsi avevano attirato attenzioni indesiderate. Molto più
    che probabilmente. Finì di collegare tutti gli algoritmi tra di loro e quindi,
    rivolgendosi a Lissa, le fece un cenno con la testa. ``Ok,
    principessa.'' disse prontamente la ragazza ``Immagino dovremo spostare
    questo discorso a dopo.''

    Lissa si riposizionò alla destra della rampa di scale, mentre Hax si
    schiacciò contro il muro sull'altro lato. Doveva solo aspettare che
    l'amica facesse una mossa poi sarebbe scattato fuori dalla sua
    copertura e si sarebbe lanciato a tutta velocità contro l'altro nemico.
    Se sempre ci fosse stato un altro, ovviamente. 
\cleardoublepage{}
