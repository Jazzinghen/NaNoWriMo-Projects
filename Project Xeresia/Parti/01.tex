\chapter{Tutte le storie iniziano con una macchina in panne...}

\section{Rollers - Piane di Harihan - Giugno 2635}
    Era una splendida giornata per viaggiare attraverso le Piane di
    Harihan, col suo cielo terso ed il sole che spaccava le pietre. Musica
    a tutto volume dal sintetizzatore installato nell'automotiva,
    finestrini tirati giù, vento tra i capelli, quello che ci voleva prima
    di uno dei soliti lavori assolutamente pallosi.

    Almeno, questo era quello che pensava Hax, il quale, ora, si trovava
    sotto il sole allo Zenith, in mezzo alla piana, con una cosa come mezzo
    migliaio di furlong da un qualunque centro abitato con il veicolo senza
    combustibile e quattro persone che avrebbero voluto ucciderlo. Hax era un
    \emph{tuttofare}: faceva parte di uno di quei gruppi di avventurieri che,
    per interesse, ideologie o altri motivi che a lui sfuggivano, andavano
    in giro a raccogliere contratti di lavoro. Questi contratti prevedevano
    il risolvere problemi di qualunque tipo, dal recupero di oggetti,
    all'eliminazione di pericoli per comunità segregate in lughi
    abbandanati dagli Dei. Il gruppo col quale andava in giro era,
    fondamentalmente, formato da un gruppo di amici, non era come quelli
    che si vedono in giro spesso, formato da gente che lo fa solo di
    professione. Chissà. Forse era proprio per quel motivo che erano quasi
    sempre senza soldi.

    Soldi o no, comunque, lui e gli altri dei \emph{Rollers} avevano un
    problema. Era già da cinque minuti che se ne stava là sotto il sole, il
    quale gli faceva chiudere ancora di più gli occhi che già di loro non
    stavano molto aperti. Un problema che ha iniziato ad avere durante gli
    anni dell'accademia. Causa dei problemi di sonno, della sua sensibilità
    alla luce solare datagli dalla colorazione chiara delle iridi oppure da
    quella dannata iperdilatazione delle pupille da drogato che ha sempre
    avuto. Si grattò i capelli, già disordinati di loro, sperando che
    questo avrebbe messo in moto delle parti di cervello addormentate,
    facendogli venire in mente come fare. O forse, magari, scatendando
    qualche reazione entropica che, per effetto farfalla, avrebbe fatto
    passare di là un autobotte piena di combustibile, proprio per loro.
    ``Stronzate.'' disse, tra sè e sè, tentando d'ignorare questo ronzio
    che girava in sottofondo da un po', ovvero da quando l'automotiva si
    era fermata in mezzo al nulla. Putroppo questo non bastò per non
    sentire più chiaramente cosa fosse questo rumore.

    ``\dots{}ottima idea davvero, Hax!'' fece la causa del ronzio,
    evidentemente alterata. Hax si girò, ora di discutere e di provare a
    sistemare in maniera costruttiva questo problema. ``Senti, Lissa, ma
    che cazzo devo dirti, scusa?'' rispose lui lievemente irritato,
    girandosi verso la ragazza appoggiata al tettuccio della veicolo in
    legno, bronzo ed acciaio. Lissa era la seconda del nucleo storico dei
    Rollers, con la quale aveva frequentato tutta l'accademia di
    algoritmica. Per quanto litigassero spesso i due erano particolarmente
    affiatati, una specie di fratelli di battaglia. Non è sempre stato
    così, ovviamente, visto che, all'inizio, Lissa lo odiava per come era
    riuscito ad accedere all'accademia. ``Per intanto, maledizione, mi
    spieghi da dove è venuta fuori questa storia di andare in auto fino a
    Shinthya per fare questo lavoro?'' lo incalzò lei ``Beh, tu e Jam
    continuate a scassare le palle che dovremmo farci un viaggio e l'unica
    volta che decidiamo di andare a fare un lavoro prendendocela comoda vai
    fuori di testa?'' rispose il ragazzo ``Ma è un'altra cosa, dannazione.
    Stiamo andando a lavorare, dovremmo levarci dalle scatole il contratto
    il prima possibile e POI prenderci una pausa.'' ``Ma va, su. Sei sempre
    la solita, dovresti prenderla con calma. Non abbiamo neppure una
    deadline per questo lavoro, perchè spaccarci il culo inutilmente?''

    La ragazza, sentito quello, era lì che lo guardava con gli occhi
    sbarrati. Hax è sempre stato così, fin da quando lo ha conosciuto
    all'accademia. Non fare ora quello che puoi fare domani. Oh, beh,
    avrebbero comunque dovuto risolvere quel problema. Per quanto
    litigavano, comunque, sapeva che non era completamente colpa sua. A
    quanto aveva capito il manometro che indicava la pressione all'interno
    della bombola contenente il combustibile si doveva essere rotto durante
    il viaggio, bloccandolo a mezza corsa. Era per quello che si erano
    trovati in mezzo al nulla senza un goccio di propellente. Hax era uno
    scansafatiche cronico, non uno sprovveduto, non avrebbe mai fatto un
    errore simile.
    
    ``Vabbeh, sentite, ragazzi, litigare non serve a nulla, seriamente.
    Cosa possiamo fare per uscire da questa situazione?'' disse Jam,
    aggiungendosi alla conversazione. Jam era una ragazza di colore, la
    quale veniva dall'arcipelago di Sanpan, un'eccellente matematica ed
    un'ancora più abile spadaccina. Aveva una ventina d'anni, la maggior
    parte dei quali passati a studiare le arti matematiche e quelle della
    spada. Normalmente non interveniva quando quei due litigavano, ma non
    aveva molta voglia di passare dell'altro tempo in quella scatola
    dannata che aveva bisogno di propellente per muoversi, anche perchè
    stava per morire di caldo. Aprì la portiera e scese, sentendo i
    muscoli, addolenziti dal viaggio, riprendersi dalla posizione tenuta
    per ore. Si stiracchiò e poi riprese ``Siamo in cinque scienziati e non
    riusciamo a far muovere questa carretta?'' dando un calcio ad una delle
    ruote posteriori. Un altro motivo per il quale non guadagnavano
    moltissimo.

    Detto quello scesero anche gli altri due del gruppo: Elythia e The
    Fixer. Elythia, una trentenne particolarmente attraente vestita di
    qipao viola con inserti oro, era la terza del nucleo dei Rollers, la
    quale aveva seguito Hax e Lissa fin dal primo momento. Per lei erano i
    suoi due fratelli minori che non aveva mai avuto. Elythia era una
    medica che aveva imparato tutte le basi dal padre, un dottore del
    mercato nero. The Fixer, invece, era un uomo brizzolato, sui
    trentacinque, alto e dal fisico scolpito, con tratti evidentemente
    nordici. Era diventato un esperto di chimica e metallurgica poi, dopo
    il fallimento dell'azienda per la quale lavorava, aveva passato un
    brutto periodo, fatto di vita sulle strade e metallo pesante. Si era
    unito ai Rollers dopo che lo avevano contattato come assistente per un
    contratto. Storia lunga, comunque. The Fixer non era il suo nome, ma
    nessuno sapeva come si chiamava. Però nessuno aveva mai chiesto, in
    quanto riparare era esattamente quello che aveva sempre fatto durante
    i loro viaggi.

    ``In effetti a dirla così sembra assurdo, Lady'' fece Elythia. Jam
    veniva anche chiamata Lady Jam dagli altri visto che, a quanto avevano
    capito loro, da dove veniva ogni guerriero o matematico aveva un titolo
    che corrispondeva al loro \emph{Sir} o giù di lì, perciò avevano deciso
    di affibiarle quel termine. ``\dots{} il problema è che,
    effettivamente, non possiamo alimentare questa cosa con le nostre
    capacità. Al massimo potremmo utilizzarla come base per la loro
    applicazione.'' continuò ``In effetti\dots{}'' aggiunse The Fixer
    ``\dots{}potremmo utilizzarla come \emph{contenitore} sul quale
    imprimere delle forze o cose simili. Sapete che non sono proprio il
    massimo in questo campo.''

    ``Uff.'' fece Jam sbuffando, mentre si massaggiava gli occhi ``Ok,
    allora, aspettiamo che passi di quì un camion pieno di combustibile.''
    finì, sarcasticamente. \emph{Il camion del destino!} pensò subito Hax,
    stupito, per poi, però,
    rimanere fulminato dalla frase che aveva pronunciato poco prima The
    Fixer. Si girò verso il gruppo con gli occhi che parevano brillare
    dicendo ``Ok, penso di avere un'idea.''

    \section{Xeresia - Metra - Giugno 2635}
    Era finalmente iniziata l'estate anche a Phaion. Xeresia era sul
    Balcone della sua stanza all'interno del palazzo reale della capitale
    di Phaion, Metra, ad ammirare quanto bella fosse la città, così
    piena di vita, pulsante. Non era mai stata una gran fanatica
    dell'estate, in effetti non l'aveva mai attesa, in quanto portava
    svariate conseguenze date dal fatto di essere principessa e
    sacerdotessa di Metrasys, dea del ciclo vitale, della famiglia, e madre
    di tutte le divinità, alla quale la città era dedicata. I mesi
    seguenti, per lei, erano sempre significati noiosi incontri di palazzo
    e lunghissime cerimonie.

    \'E stato però durante una di quei incontri di palazzo che ha
    incontrato Gabriel, un bardo che sarebbe rimasto per tutta l'estate.
    Due anni prima passò uno dei momenti più belli della sua vita, nel
    quale conobbe l'amore. Amore che, però, venne ostacolato
    dall'impossibilità, finita l'estate, di rimanere. Non è che questa
    impossibilità fosse data dal fatto che Gabriel volesse andarsene, ma
    perchè dopo le leggi della propria nazione non permettevano a visitatori
    esterni di permanere oltre ad una certa quantità di giornate
    consecutive.

    Per quanto questa legge potesse sembrare bigotta alle nazioni esterne
    questa aveva permesso di mantenere un certo controllo sul tasso di
    criminalità, non permettendo scambi commerciali non controllati. Non
    era un fatto di razzismo, in quanto chiunque avesse un contratto
    lavorativo continuativo poteva richiedere la cittadinanza. Quindi non
    si trattava di purezza della razza o cose del genere. Almeno, questo
    era quello che pensava lei. Non che Gabriel non avesse pensato di
    rimanere là. Il fatto è che non c'erano posti come bardo disponibili,
    quindi avrebbe dovuto smettere di viaggiare, facendo qualcosa che non
    amava ed ad entrambe sembrava una cosa stupida.

    Era per quello che stava aspettando che qualcosa cambiasse, magari
    quando lei sarebbe salita al potere avrebbe fatto qualcosa\dots{} Per
    ora avrebbe atteso l'estate sapendo che quello era il periodo nel quale
    poteva rivedere il suo amato.

    \section{Gabriel Bard - Dera - Giugno 2635}
    Si stava finalmente avvicinando l'estate. Era quella parte di anno dove
    un bardo brillava. Feste all'aperto, viaggi senza intemperie ma,
    soprattutto, era la stagione nella quale Gabriel poteva andare a Phaion
    e vedere la sua amata Xeresia.

    Quella storia del non poter stare per un periodo più lungo di una
    settimana era una stupidaggine. Una regola ovviamente razzista da parte
    del parlamento che governava la nazione, una monarchia costituzionale.
    Gabriel non era un grande esperto di politica internazionale, ma era
    venuto in contatto con un gran numero di governi grazie al suo lavoro e
    poteva capire che cosa stava dietro a determinate scelte di governo.
    Strano, però, visto che il re aveva sempre dimostrato una certa
    propensione alla compenetrazione culturale. Ecco perchè era ancora
    possibile visitare Phaion, quasi senza restrizioni, per tutta l'estate. 

    Comunque, come per gli anni passati, doveva prepararsi per partire alla
    volta della capitale, Metra, oltre a preparare storie e spettacoli per
    stupire ancora una volta la gente che avrebbe incontrato lungo il
    viaggio. Chissà se quasta volta avrebbe potuto prende l'automotiva per
    viaggiare\dots{}

    \section{Rollers - Shinthya - Giugno 2635}
    Ok, ora doveva proprio ammetterlo. Prendere l'auto era stata una
    cazzata di quelle astronomiche. Certo, a posteriori era sempre facile
    fare questi ragionamenti. E questo concetto bruciava ancora di più
    adesso che Hax si trovava a testa in giù contro un muro dentro un
    salotto che, prima di vedersi sfondato da una auto volante, doveva
    essere molto ben arredato. Nella sua mente ora frullavano una serie di
    domande, tra cui \emph{Dove sono gli altri?}, \emph{Come cazzo faremo
    a ripagare tutti questi danni?} o ancora \emph{Perchè il piano geniale
    non aveva funzionato?}

    \subsection{Poche ore prima\dots{}}
    ``Non posso credere che ce lo stiate proponendo sul serio.'' Elythia
    ascoltava sempre le proposte di Lissa ed Hax, la sua idea era che,
    fondamentalmente, non dicevano stupidaggini a priori. Sarebbe stato
    molto scortese. E spesso le loro idee li avevano salvati da situazioni
    ben peggiori. Altre volte rischiavano di far scoppiare guerre, perciò
    le scartava e proponeva una versione edulcorata delle stesse oppure
    vedeva assieme agli altri se era possibile risolvere la questione in
    qualche altro modo.

    Questo era uno dei casi nei quali evitare di seguire quell'idea si
    rivelava essere la scelta migliore. Si era chiesta, in effetti, come
    mai, dopo che Hax aveva avuto una delle sue solite illuminazioni, lui e
    Lissa si fossero messi a confabulare per una ventina di minuti
    scrivendo con dei gessetti sulla strada. E la risposta era\dots{}

    ``\dots{}creare una catapulta magnetica utilizzando le guardrails e le
    rotaie abbandonate della ferrovia quì a lato della strada. Non mi
    sembra un'idea così stupida. Non possiamo alimentare in maniera
    continuativa l'auto perchè ci ammazzerebbe, però possiamo creare
    qualche sistema che fornisca un impulso abbastanza potente per farci
    andare lungo la rotaia fino a Shinthya.'' rispose Lissa, sicura di sè,
    seguendo le rotaie abbandonate con le mani, come per mimare un qualcosa
    che ci scivola sopra. ``Senti, Lissa, lo so che spesso utilizziamo i
    vostri piani assurdi per tirarci fuori dalle peste ma questo non è un
    po' troppo pericoloso solo per andarcene da questa piana desolata?''
    incalzò Elythia, per poi girarsi verso Hax ``Dai, Hax, lo sai anche te
    che questa cosa comporta dei rischi troppo alti. Non sono una fisica ma
    immagino che se solo troviamo qualcosa come un sassolino o, Eclipse ce
    ne scampi, un pezzo di rotaia in meno rischiamo di andare fuori dal
    tracciato e di schiantarci contro qualcosa!'' Il ragazzo, però,
    stava fissando l'orizzonte, con gli occhi che gli brillavano come se
    avesse visto qualcosa che gli altri non potevano osservare. Qualcosa di
    epico in tutti i sensi. ``\'E troppo dannatamente epico per non farlo,
    Elythia. Cosa vuoi che accada?'' rispose, stringendo la mano destra,
    come per afferrare qualcosa.

    La dottoressa sapeva che la riposta da parte del ragazzo sarebbe stata
    quella, ma sperava in una sua improvvisa lucidità. Cosa che, appunto,
    non accadde. Perciò si rivolse a Jam ed a The Fixer ``Ok, sono partiti.
    Avete qualche idea voi?'' ``A parte tirare fuori tutte le nostre cose
    dal baule, mettercele in spalla ed iniziare a camminare da veri
    uomini?'' rispose prontamente The Fixer con la sua solita voce calma ed
    impassibile. \emph{Io sono una donna, anche fosse} pensò Elythia,
    cacciandogli un'occhiataccia ``Scusa.'' rispose lui, quasi imbarazzato,
    neanche le avesse letto la mente. ``Hem, comunque, secondo me potremmo
    calibrare i loro calcoli. Ho abbastanza competenze, penso.'' si
    intromise Jam ``Voglio dire, farlo senza una base abbastanza buona di
    modellazione matematica può essere pericoloso, senza niente togliere ai
    nostri programmatori\dots{}'' guardò Lissa ed Hax che, intanto, stavano
    ignorando questo discorso mentre smontavano con l'ausilio di due piedi
    di porco delle guardrails ``\dots{}però aggiungendo delle correzioni di
    tipo matematico alle iscrizioni che danno l'impulso possiamo dare una
    carica sia alle rotaie che all'auto con la quale gallegiare sopra le
    rotaie. Non è una cosa che può essere mantenuta per un lungo periodo,
    ma dovrebbe bastarci per arrivare evitando rotaie mancanti e sassolini,
    no?''

    Elythia si mise una mano sul volto in segno di resa ``Ok, ok.
    Facciamolo. Quanto ci vorrà, secondo voi?'' fece, rivolta ai due, che
    intanto avevano già creato una specie di gabbia di partenza. ``Dacci
    mezz'ora!'' fece Hax, mentre Lissa annuiva mentre smartellava sulle
    guarda rails in modo da fissarle a dei pali. Voleva dire due ore. Quei
    due saranno anche stati bravi nel loro campo ma erano dannatamente
    esagerati. O troppo o troppo poco. Andavano calibrati. Questo comunque
    non significava che non si fidava di loro. Se dicevano che era
    possibile farlo, lo avrebbero fatto. Non c'era stata una volta che
    avessero esagerato su quello, perciò non avrebbe dubitato di quella
    parte. E poi era un piano troppo assurdo perchè loro non ci dessero il
    massimo. Erano fatti così.

    Ci vollero quasi due ore (appunto) per completare tutta
    l'infrastruttura per il lancio e per inscrivere tutti i codici e le
    formule matematiche di calibrazione, quando poi arrivò Hax (lei non
    aveva lavorato. Non lo faceva quasi mai, non poteva permettersi di
    trovarsi inabile a curare, visto che era lei la medica del gruppo) e le
    disse ``Sorella, abbiamo tutto pronto, se vuoi prendere
    posizione\dots{}'' ogni tanto Hax la chiamava così, non era perchè era
    fricchettone, è solo che, come lei considerava lui e Lissa i suoi
    ``fratellini'', lui aveva preso a considerarla una sorella. Comunque,
    ancora poco convinta, rispose ``Ok, ok. Vediamo se riuscite ad
    ammazzarmi questa volta. Voi ed i vostri piani assurdi.'' aprì la
    portiera posteriore e si sedette dietro il sedile dell'autista.

    Una volta che tutti salirono in auto, con Hax al ``volante'', una serie
    di linee squadrate bianche e verdi acido, seguite da una serie di curve
    viola, partirono dalle varie scritte, poste sulle guardrails montate
    intorno al mezzo di trasporto, quasi incomprensibili ad Elythia. Erano
    gli effetti di due programmi ed una serie di funzioni che erano state
    iscritte da Hax, Lissa e Jam. Una serie di piccoli fulmini comparirono
    tra la macchina e le guardrails usate come ``rotaie magnetiche'' per
    l'auto e, in una frazione di secondo, si trovarono tutti impiantati sui
    sedili.

    Elythia riuscì con grande sforzo a non svenire per l'accelerazione, per
    poi adattarsi lentamente alla nuova velocità. Le sembrava di avere gli
    organi riordinati in ordine alfabetico. Si soffermò un secondo a
    pensare come sarebbe stato ad avere l'apendice nel cranio per poi
    smettere di pensare quando si ricordò che esisteva un altro ``organo''
    prima di appendice in lista. Mentre era persa via in quei ragionamenti
    sentì Jam che commentava, eccitata ``Stavao pensando, comunque, The
    Fixer, Hax e Lissa\dots{} Siete un sacco forti se riuscite a spostare
    un'auto spingendola. Vabbeh che ha le ruote, ma spostare tutto
    quel peso\dots{}'' Hax si girò per guardare Jam in volto ``In che
    senso, scusa? Normalmente una persona riesce a spingere un'auto da
    sola.'' ``Eh? Com'è possibile, scusate?'' continuò lei, guardandosi
    intorno spaesata ``Perchè? Un'auto pesante non peserà più di una
    tonnellata.'' rispose The Fixer ``EH?!?'' chiese a voce alta Jam,
    girandosi verso l'uomo ``Ma allora\dots{}''

    La macchina, ad un tratto, sembrò alzarsi da terra, per poi tornare
    sulle rotaie in poco tempo. Accadde una, due volte, l'intervallo tra lo
    stacco ed il ritorno a terra era sempre più lungo. ``Oh, cazzo.
    Potremmo avere un problema.'' fece Hax, in un misto di preoccupazione e
    perplessitudine ``Ma com'è possibile? Il programma era corretto.''
    ``T-Temo di aver sbagliato qualcosa io.'' rispose Jam, preoccupata
    ``Pensavo che le auto pesassero in media una decina di tonnellate.''
    ``SCUSA?!? Ma in che mondo vivi?''

    Lady Jam era una ragazza molto intelligente, aveva solo un problema.
    Era un po' fuori dal mondo. Non conosceva il funzionamento di molte
    cose. ``Ad un matematico non devono importare cose così tanto da\dots{}
    ingegneri.'' era solita pensare. Non lo faceva per scherno nei
    confronti degli ingegneri. Era proprio un'idea particolare che aveva di
    come funzionava il mondo. Le era data dal passato accademico che
    aveva. Troppo rapido e troppo concentrato solo sullo studio.

    Tutto a posto, in realtà, nella maggioranza delle situzioni, in quanto
    non le serviva avere quelle conoscenze per fare calcoli o tracciare
    formule. Il problema era quando doveva relazionarle a cose di tutti i
    giorni. Spesso avevano provato ad insegnarle dei concetti base, a farle
    imparare come misurare ad occhio le distanze o i pesi ma non ce la
    faceva. Era più forte di lei.

    ``Vabbeh, tutto a posto finchè\dots{}'' Hax non fece in tempo a finire
    la frase che ci fu uno scossone e l'auto si staccò da terra. La
    velocità era troppo alta per scendere in quel momento. Se fossero
    saltati giù in quel momento avrebbero rischiato di grattarsi sul
    terreno. Lissa aveva calcolato che, con la spinta elettromagnetica
    iniziale, sarebbero arrivati in un quarto d'ora. rotolare sul terreno
    alla velocità di duemila furlongs all'ora sarebbe significato che un
    qualunque ostacolo avrebbe fatto esplodere almeno la metà delle ossa
    presenti in un corpo umano.

    Ma che importava, ormai? Dopo neanche mezzo minuto si trovavano ad
    un'altezza tale che quello che li avrebbe uccisi sarebbe stato
    l'impatto col suolo, non con qualche albero o pietra posta ai lati di
    una strada. Hax era paralizzato. Soffriva di vertigini: aveva superato
    solo in parte la fobia di viaggiare sulle airships, ma volare a quella
    velocità in un'auto andava fuori dal suo concetto di airship. Riuscì
    a sentire in maniera ovattata Elythia che, incazzata, fece ``Ok, ma che
    cazzo devo fare con voi per non rischiare la morte per una qualsiasi
    puttanata?'' non imprecava mai, a meno di non essere DAVVERO irritata
    ``Però ormai ci siamo dentro. Come ne usciamo?'' ``Hem\dots{}'' fece
    Jam ``considerando l'energia iniziale\dots{}'' ``Senti, riesco a vedere
    la città da quì. Questo mi dice che non abbiamo molto tempo. Soluzione.
    ORA.'' ``Ok, ci sono.'' si intromise, sbrigativa, Lissa ``Non so
    moltissimo di aereodinamica, ma questa scatola dovrebbe rallentare un
    tot, no?'' ``Sì,'' rispose Jam ``dovrebbe arrivare a\dots{} hem. Quanto
    va un'auto?'' ``Ok, non importa. Quello che volevo dire è che dovremmo
    prepararci ad aprire le portiere. Rallentiamo la macchina con le
    portiere aperte\dots{}'' ``Se le riusciamo ad aprire vuol dire che
    questa carretta è abbastanza lenta, Lissa'' aggiunse Elythia
    ``\dots{}anncora meglio. In pratica, arriviamo sulla città e ci
    buttiamo su un tetto, un albero. Qualcosa.''

    \emph{E a me non ci pensano?} si chiese Hax, immobilizzato, con tutti i
    muscoli tesi che stringevano il più forte possibile il sedile sul quale
    si trovava, tenendolo incollato a quest'ultimo.

    In pochissimo tempo tutti furono saltati fuori dall'auto. ``Se ci
    perdiamo di vista dirigetevi verso la torre dell'orologio, poi vedremo
    come sbrigarcela.'' urlò Lissa, la prima a saltare, prima di buttarsi
    in una di quelle giganteschi laghi artificiali dai giardini pubblici.
    Hax, troppo intento a tenere sotto controllo la paura non sentì nulla.
    Riuscì a pensare una cosa, però, prima che l'auto si schiantasse
    sul quinto piano di una palazzina: ``Tanto le cinture proteggono contro
    gli incidenti, no?''

    Ovviamente, le cinture andavano messe.

    L'auto sfondò la parete della casa, catapultandolo fuori dal
    parabrezza, oltre una stanza da letto, dentro una porta, attraverso un
    soggiorno, su un divano. L'auto distrusse la camera da letto, sfondò la
    parete che divideva camera da letto e soggiorno e si fermò, capottata
    ed accartocciata, su uno di quegli splendidi tappeti orientali che
    tanto aggradavano la medio borghesia.

    ``Cazzo\dots{}'' mormorò Hax, con il corpo che urlava.

    Ok, ora doveva proprio ammetterlo. Prendere l'auto era stata una
    cazzata di quelle astronomiche. Certo, a posteriori era sempre facile
    fare questi ragionamenti. E questo concetto bruciava ancora di più
    adesso che Hax si trovava a testa in giù contro un muro dentro un
    salotto che, prima di vedersi sfondato da una auto volante, doveva
    essere molto ben arredato. Nella sua mente ora frullavano una serie di
    domande, tra cui \emph{Dove sono gli altri?}, \emph{Come cazzo faremo
    a ripagare tutti questi danni?} o ancora \emph{Perchè il piano geniale
    non aveva funzionato?}
    
    Prima di rispondere a tutte queste ottime domande quello che doveva
    fare era fuggire il prima possibile da una costruzione che rischiava di
    crollare. I rumori non erano proprio il massimo. Poi c'era la questione
    della polizia. Bravi quanto volevi, ma potevano rompere le palle. Tanto
    avrebbero dovuto comunque pagare i danni. Ma gli agenti erano
    particolarmente suscettibili quando trovavano qualcuno direttamente
    sulla scena degli incidenti.

    Ignorando i muscoli che facevano male vuoi per l'impatto, vuoi per aver
    passato cinque minuti ad essere tesi come la corda di un violino, corse
    verso l'auto, la quale non poteva esplodere, fortunatamente, e tirò
    fuori il suo zaino. Controllò rapidamente che tutto fosse a posto e,
    una volta tirato un sospiro di sollievo, se lo mise in spalla e corse
    verso la porta dell'appartamento. Era stata proprio una fortuna che non
    ci fosse nessuno al momento dell'impatto. Ora, però, non poteva
    rischiare che la gente morisse nel crollo della casa, perciò cercò un
    interruttore per l'allarme antincendio e lo attivò. Ci volle un pochino
    perchè il sistema si accorgesse della sua richiesta. Evidentemente veniva
    ancora utilizzato un sistema a pressione idraulica invece che
    interamente elettrico. Funzionava un po' come i freni delle auto.
    \emph{'Fanculo} Risolto anche questo problema, tra il rumore assordante
    della campanella antincendio, le sirene della polizia e le urla degli
    abitanti della palazzina,
    rubò una corda dall'appartamento che aveva sfondato, legò ad un capo un
    candelabro e corse ad una finestra.

    Lanciò il rudimentale rampino vero il tetto di una casa vicina e, dopo
    aver testato rapidamente che reggesse, si buttò fuori dalla finestra.
   
    \section{Rollers senza Hax - Shynthia - Giugno 2635}
    Elythia era caduta su un tetto abbastanza alto quindi riuscì a non
    farsi nulla, aiutata anche dalle sue competenze nelle arti marziali.
    Era infuriata per quello che era successo, anche se era molto più
    preoccupata di come stessero gli altri. Essendo saltata subito prima di
    Hax si era assicurata che tutti fossero atterrati in zone poco
    pericolose.

    Riuscì a trovare una scaletta antincendo che portava sulla strada,
    quindi si diresse verso la torre dell'orologio, come Lissa aveva detto
    prima di saltare. Non era molto preoccupata per lo stato di Lissa o di
    The Fixer. Erano coriacei, quei due. Era molto più in pensiero per le
    condizioni di Jam e di Hax. La prima era comunque una ragazzina, anche
    se allenata al combattimento con la spada, l'altro\dots{} Vabbeh, oltre
    a soffrire di vertigini, che avrebbero potuto fargli fare qualche
    stupidaggine, se staccato dal gruppo rischiava sempre di ficcarsi in
    guai assurdi. Una volta aveva quasi fatto partire una rivolta popolare,
    indignato dalla condizione nella quali vivenano i lavoratori in quella
    nazione.
    Tremò al solo pensiero di tutte le ore che avrebbe dovuto passare in
    municipio a spiegare la situazione se non l'avesse trovato con gli
    altri al punto d'incontro.

    Camminò tra le strade piene di gente della città, ignorando gli scorci,
    per quanto belli: lo stato degli altri era molto più importante. Arrivò
    dopo un quarto d'ora di strade sconosciute, incroci presi ad intuito,
    bistrot ed altre botteghe piene di specialità locali che avrebbe
    assaggiato molto volentieri più tardi.

    Una volta lì c'erano Lissa, fradicia, seduta su una panchina al sole,
    svaccata, che si stava riprendendo e Lady Jam, che stava togliendosi
    foglie e rametti dalla lunga chioma bianca, in piedi vicino all'altra
    ragazza. Niente Fixer.

    \dots{}

    NIENTE HAX! Corse verso le altre due. ``Ci siete solo voi?'' fece
    Elythia, concitata ``Ah, Elythia! Ciao.'' rispose Lissa,
    per poi ributtarla indietro ``Sì, stiamo aspettando gli altri.''
    ``\dots{}e ci togliamo questi dannati ramoscelli dai capelli.''
    aggiunse irritata, se mai potesse veramente esserlo, Jam, prendendo un
    pezzetto di legno dai capelli e buttandolo a terra. ``Tutto a posto voi
    due?'' chiese Elythia, rendendosi conto che anche loro avevano bisogno
    di attenzioni ``Lissa? Jam?'' ``Sìsì. Ho fatto salti ben peggiori. Però
    è stato dannatamente epico, devo ammetterlo.'' rispose, senza neppure
    muoversi, Lissa ``Ha ha.'' rise l'altra ragazza ``Sono stata fortunata
    che dietro le fronde di quell'albero ci fossero dei cespugli belli
    alti. A parte alcuni graffi al mio vesito.'' guardò il suo abito, che
    era formato dalla parte sopra di un kimono, con colori sgargianti ed il
    disegno di una volpe, e dei pantaloncini corti, più comodi per il
    combattimento. La parte sopra recava uno strappo alla manica sinistra.
    ``Mi ci vorrà un po' per metterlo a posto.'' finì, sbuffando, per poi
    rimettersi a togliere ciò che le si era impigliato nei capelli durante
    la caduta.

    Rimasero per un po' lì a discutere sul da farsi quando, ad un tratto,
    comparve The Fixer da un vicolo. Elythia fece per salutarlo quando
    rimase interdetta da un paio di dettagli. Per prima cosa sembrava non
    fosse neppure caduto da un'auto in volo. Era impeccabile. Aveva in mano
    una tazza con del caffè dentro e, sotto il braccio destro, aveva un
    paio di giornali locali. Si avvicinò sorseggiando il caffè, superò
    Elythia che non sapeva neppure da dove cominciare e si sedette vicino a
    Lissa, la quale lo guardava anche lei con gli occhi sbarrati. Ancora
    non riuscivano ad abituarsi a quelle scene.

    ``Giornale?'' fece, The Fixer, porgendo le riviste a Lissa ``N-No,
    grazie, Fixer. Non ho molta voglia di leggere, veramente.'' ``Intendo
    per asciugarti.'' ``Eh?'' Lissa, senza fare altre domande, prese un
    giornale ed iniziò ad utilizzarlo come asciugamano per braccia e testa.

    ``Ok. Visto che siamo tutti quì posso affermare con sicurezza che
    abbiamo un problema.'' fece Elythia mentre, dando le spalle agli altri
    tre, fissava una scia di fumo che s'innalzava nella parte orientale
    della città. Doveva essere dove era impattata l'auto. ``Hax non c'è.''
    ``Beh, dai. Non potrà fare casino. Quì non mi sembra che la situazione
    sia tale da spingerlo a creare problemi.'' rispose Lissa, tentando di
    proteggere l'amico. Sapeva anche lei che Hax rischiava di esagerare, ma
    era scorretto nei suoi confronti pensare che causasse problemi apposta
    oppure ogni volta che era staccato dalla squadra. Era abbastanza
    intelligente. La maggior parte delle volte che si sono trovati davanti
    ad un qualche tipo d'incidente causato dal ragazzo era a causa del suo
    senso molto forte di giustizia. Tutto suo, in realtà, però se non
    c'erano le basi non sarebbe dovuto capitare nulla di problematico.
    ``Sì,'' iniziò Elythia ``ma ti ricordi perchè siamo venuti quì,
    giusto?''

    Ecco, adesso sì che era preoccupata anche Lissa.

    \section{Seysill - Metra - Giugno 2635}
    Fare il parlamentare in una nazione come Phaion era come rubare le
    caramelle ad un bambino. Almeno, questo era quello che pensava Seysill
    Mann, un uomo sulla quarantina, sempre elegante, brizzolato, con i
    gusti raffinati.

    Lui era uno di quei cattivi gentleman dei quali leggi nei libri o senti
    narrare dai bardi. Gli piaceva autodescriversi così, ogni tanto, quando
    pensava al suo lavoro. Ma tornando al discorso sull'eticità del suo
    lavoro. Non è che tutti quelli che lavoravano con lui fossero dei
    corrotti bastardi, figuriamoci! In uno stato onesto come questo? E poi,
    neppure lui era corrotto. Diciamo solo che aveva intravisto un'altra
    realtà. Qualcosa di più serio, di più in linea con il funzionamento del
    mondo.

    Se nel mondo i parlamentari delle nazioni democratiche erano i soldati,
    lui era un mercenario. Ecco, un mercenario della democrazia. Veniva
    pagato da della ``gente molto gentile'' per esportare le loro idee
    all'interno dei governi ``meno illuminati''. Tutte puttanate,
    veramente. Ma chi era lui per criticare? Non era mica un politico. Lui
    era un professionista che faceva il proprio lavoro.

    E, da vero gentleman, se ne stava nel suo studio, seduto sulla sua
    poltroncina, a leggere ed a sorseggiare del tè quando, all'improvviso.
    Qualcuno bussò alla porta. \emph{Dannazione} pensò, sbuffando ``Sì?''
    chiese, verso la porta ``Signor Mann?'' rispose una voce da dietro la
    porta. Messi. Possibile che vengano a disturrbarlo sempre quando si
    gode la sua pausa? ``Entri pure.'' La porta si aprì ed entrò un
    ragazzo, uno di quei segretari del palazzo, con in mano una lettera
    ``Signor Mann, signore, ho una lettera per lei da parte del Duca
    Philip, mi è stato detto che è importante.''

    Oddio, Philip. Quel tizio era di un viscidume quasi insopportabile.
    L'aveva incontrato solo alcune volte ma ogni volta ha avuto
    l'irrefrenabile necessità di lavarsi le mani dopo avergliele strette.
    Il \emph{Duca} Philip III di Shien era il suo datore corrente. Lo aveva
    contattato per far approvare una legge di stampo alquanto\dots{}
    razzista, avrebbe detto una persona normale. Ma, ovviamente, dietro c'è
    qualcos'altro. Era stupido pensare che una persona l'avrebbe pagato
    quella cifra solo per proteggere la purezza della propria razza quando,
    comunque, uno da fuori poteva stabilirsi se aveva un contratto
    lavorativo di una certa lunghezza.

    ``Sì, grazie.'' fece, sbrigativo, prendendo di mano la busta dal
    ragazzo. ``Bene, puoi andare.'' ``Sì, signore.'' Una volta che il
    ragazzo fu fuori dalla porta Seysill aprì la busta con un taglia carte.

    Lesse in fretta la lettera senza rimanere troppo stupito dal contenuto,
    che l'avvisava che l'indomani si sarebbe votato per la legge che era
    stata proposta e che, grazie agli sforzi del ``mercenario'' più dei due
    terzi del parlamento avrebbe votato a favore. Si faceva anche menzione
    al pagamento ed a quanto fosse soddisfatto il Duca dei suoi sforzi.

    Mann, soddisfatto, riprese posto sulla sua poltrona, riprese il libro
    che aveva lasciato a metà e prese un sorso del suo tè, per poi dire,
    tra sè e sè ``Ovvio, sciocco. Chi credevi di aver ingaggiato?''

    \section{Hax - Shynthia - Giugno 2635}
    La corda ed il ``rampino'' avevano retto, fortunatamente, evitando ad
    Hax una caduta sulla strada o, ancora peggio, sui veicoli della
    polizia. Certo, non poteva dire la stessa cosa per la finestra
    dell'appartamento nel quale si era fiondato. Dopo aver attutito il
    volo con una capriola a terra si alzò rapidamente, per dare un'occhiata
    in giro. C'erano due persone anziane che lo stavano fissando,
    esterrefatte. Non capita spesso che un ragazzo bello come lui, in
    effetti\dots{} ti sfondi finestra e tavolino dopo essere volato fuori
    da una casa che, presumibilmente, stava prendendo fuoco.

    ``Hem.'' iniziò Hax, con i due signori che sembravano non saper che
    dire ``So che può sembrare pretenzioso ma, in realtà\dots{}'' il
    ragazzo si mise a pensare il più forte possibile a cosa avrebbe dovuto
    dire ``\dots{}sono un agente dei servizi segreti del vostro paese! Sono
    sulle tracce di pericolosissimi orchi affiliati ad una banda criminale
    segreta che opera nei bassifondi della nostra bellissima Shynthia.
    Avete visto qualche orco sospetto nell'ultimo periodo?'' l'uomo annuì
    ``Bene, allora. Buona giorna\dots{} Scusi?!?'' Hax si accorse solo dopo
    di aver urlato. ``Sì, signore. Ho notato uno di quei tizi con la pelle
    grigia, effettivamente ha provato a passare inosservato, ma l'ho visto
    andare verso quella direzione'' rispose l'anziano, indicando verso
    nord. ``E\dots{}Eh. G-Grazie, cittadino. Salverete la vostra città
    grazie al vostro apporto. Ora devo andare ad inseguire il criminale.
    Hem. Mi raccomando. NON dite a nessuno che sono passato, neppure alla
    polizia. \'E una missione segreta.'' fece, Hax, dirigendosi alla porta
    e quindi, mentre stava chiudendo la porta ``Ah, scusate per la
    finestra.''

    \emph{Che scusa del cazzo era?} si chiese. Per quanto fortunato fosse
    stato, se di fortuna si poteva parlare, non poteva fare ancora di
    quegli errori. Si era abituato troppo bene con Elythia nel gruppo.
    Mentre fuggiva dalla costruzione si fece una certa quantità di domande,
    in effeti, tra cui come fare per avere delle uscite migliori a cosa
    diavolo ci facessero degli orchi in incognito in città. Per quanto si
    sforzasse non riuscì a ricevere delle risposte.

    Era mentre percorreva il vicolo che, a quanto pareva, aveva imboccato
    l'orco, si fece la domanda più importante di tutte

    \emph{Quale diavolo è il lavoro per il quale siamo quì?}

    \section{Rollers senza Hax - Shynthia - Giugno 2635}
    ``Beh, non siamo venuti quì per mettere a posto il calcolatore centrale
    dell'amministrazione comunale per la quale c'era bisogno di una
    riparazione hardware specializzata, no?'' chiese Jam, ingenuamente
    ``Esatto.'' rispose, preoccupata Elythia, con i sudorini freddi ``E
    allora? Non penso che se Hax arrivi al municipio e provi a metterlo a
    posto ci siano problemi. \'E quello su cui ha sempre lavorato, in
    pratica.'' ``Non capisci, Lady'' fece Lissa ``il problema non è se
    arriva al municipio. Il problema è se NON ci arriva.'' ``Eh?'' fece
    Jam, non capendo. ``Jam, non abbiamo detto ad Hax qual'è il lavoro per
    il quale siamo stati ingaggiati.'' rispose Elythia ``Non ne ho avuto il
    tempo, inoltre pensavo che, visto che non era nulla di assurdo, potevo
    spiegarglielo dopo. Il problema è che ora è in una città che non
    abbiamo mai visitato prima, sapendo di essere sotto contratto ma non
    sapendo quale sia il lavoro. Potrebbe prendere un qualsiasi segnale
    come un indizio.'' ``Oh.''

    In quel momento, in lontananza, il palazzo sul quale si era schiantata
    l'auto dei Rollers, crollò, sollevando un'enorme nube di polvere.

    \section{Gabriel - Confini di Phaion - Giugno 2635}
    Finalmente era arrivato ai confini di Phaion. Avrebbe soltanto dovuto
    sbrigare un paio di formalità burocratiche e poi avrebbe potuto
    continuare con il viaggio per raggiungere la sua amata.

    ``Ah, Xeresia, stella che guida i miei viaggi, ancora pochi giorni e
    sarò da te\dots{}'' si mise a recitare, poi si fermò ``\dots{}hum, che
    ci metto?'' aspettare il responso era, al solito, una cosa lunga e
    noiosa e comporre era un buon metodo per mantenere il cervello allenato
    e la creatività fertile.

    ``Signor Gabriel?'' fece una signorina, anche abbastanza attraente,
    vestita da segretaria, chinandosi sul bardo seduto su una delle sedie
    d'attesa ``HUH? Sì?'' rispose, sobbalzando sulla sedia, il ragazzo,
    sbalzato fuori dai suoi pensieri ``Devo informarla che, purtroppo, non
    possiamo farla passare.'' ``Come scusi?'' ``I suoi giorni di
    permanenza sono finiti, secondo i nostri registri, il 5 marzo di
    quest'anno.''

    Gabriel si sentì crollare il mondo addosso. Cosa voleva dire? ``Scusi,
    non capisco. Cosa vuol dire \emph{I suoi giorni a disposizione sono
    finiti}?'' ``Vuol dire che non possiamo farla entrare in base alle
    regolazioni regie.'' ``Ma, scusi. \'E luglio. Inizia la fiera estiva.''
    ``Purtroppo una legge approvata qualche giorno fa ha rimosso
    coompletamente la fiera estiva.'' ``A\dots{} Ah.'' rispose, abbattuto,
    il giovane. Ringraziò la signorina, recuperò le carte e quindi si
    trascinò fuori dall'ufficio di confine.

    Cosa significava che avevano rimosso la fiera? Come avrebbe fatto?
    Questo significava che avrebbe potuto vederla solo per una settimana
    all'anno! Doveva trovare un altro modo. Ma superare i controlli del
    confine era impossibile. C'erano moltissime guardie e lui non era nè un
    combattente nè uno che riusciva a sgattaiolare oltre i posti di
    controllo.

    Alla fine gli venne in mente un metodo. Gli avrebbe richiesto del
    denaro, ma era l'unica cosa\dots{}

    \section{Rollers senza Hax - Shynthia - Giugno 2635}
    Erano passati quattro giorni da quando erano arrivati a Shynthia e non
    avevano ricevuto messaggi da Hax. Ora Elythia si stava preoccupando per
    il ragazzo. Non le importava se avrebbe fatto casini o se avrebbe
    dovuto passare anche giornate intere alla sede della polizia a tirarlo
    fuori di prigione.

    Avevano già riparato il calcolatore ed erano addirittura riusciti a
    convincere l'amministrazione che non era stata colpa loro della
    macchina impiantata dentro un condominio, il quale era crollato,
    avevano detto, per un incendio. Potevano andarsene, insomma. Ma non
    senza Hax.

    Il fatto che era\dots{} preoccupante e rassicurante allo stesso
    momento era che nessuno lo aveva visto. Nè la polizia, nè i pompieri
    nè tantomeno, e questa è la cosa più importante, negli ospedali.
    Elythia era seduta ad uno dei Bistrot, che aveva ignorato il primo
    giorno, assieme agli altri. ``Su, Elythia. Lo so cosa stai pensando''
    disse Lissa, mentre sorseggiava del caffè ``però se non troviamo Hax
    negli ospedali vuol dire che è ancora vivo. Non sarà il massimo a
    picchiare ma sa sopravvivere.'' ``E se è ferito?'' ``Sarebbe andato in
    un ospedale, non è uno sprovveduto. Quante volte io e lui ti abbiamo
    fatto preoccupare così e poi non è successo nulla?'' ``E se l'avessero
    imprigionato?'' ``Ma figuarti. In una città del genere? Haha!'' Lissa
    scoppiò in una fragorosa risata ``Ma lo sai che sei davvero un caso
    disperato di tsundere?'' ``Scusa?'' ``No, lascia stare.''

    The Fixer, che era lì ad ascoltare silenziosamente il discorso, abbassò
    la tazza, fissò serio le ragazze al tavolo con lui e poi, impassibile,
    fece ``Potrebbe essere diventato un'entità di pura scienza. Come in
    quella storia della donna che si è trasformata in uno spirito che
    viveva nei calcolatori.'' Elythia e Lissa lo guardarono esterrefatte,
    mentre Jam andò in panico ``COSA?!? Può succedere davvero? No! Hax! Non
    diventare uno spirito!'' urlò, per poi continuare, rimuginando ``Però,
    forse, potremmo ancora prendere un calcolatore, farcelo entrare e poi
    portarcelo in giro.'' alzò lo sguardo per fissare gli altri ``Andiamo a
    comprare un calcolatore.''

    ``Scherzavo.'' rispose prontamente The Fixer ``Sul serio?'' fece Jam,
    quasi in lacrime ``Sì.'' le accarezzò la testa e poi tornò a leggere il
    giornale. Elythia notò una delle foto in prima pagina, la quale
    ritraeva un uomo che veniva portato via in barella. Il testo
    dell'articolo relativo citava:

    \emph{Altri venti feriti per le strade}
    ``Per il terzo giorno consecutivo la notte a Shynthia si è cosparsa di
    feriti per le strade. Tutti hanno almeno un osso rotto o sono caduti
    (stati buttati?) da un palazzo. Molti di questi non avevano documenti
    oppure non hanno rilasciato informazioni che permettessero il loro
    riconoscimento. \[..\] Gli investigatori hanno paura che questi casi
    siano correlati con le scomparse degli ultimi mesi. I dettagli a pagina
    25.''

    \emph{Stai scherzando?} si chiese Elythia. ``Fixer. Hai letto
    l'articolo a pagina venticinque?'' ``Sì, perchè?'' ``Di che parla?''
    ``Mah, nulla. Scomparse, mercato nero, traffico di schiavi, le solite
    cose.'' \emph{Le solite cose.} appuntò Elythia ``Non è che l'hanno
    preso?'' ``Hum.'' ``No, niente. Forse sono un po' troppo paranoica.
    Però questi feriti. Non è che rischia per questa cosa?'' ``Beh, non ti
    preoccupare. Anche se provassero a fargli del male rischierebbero
    duro.'' minimizzò Lissa, prendendo un altro sorso di caffè.

    \section{Hax - Shynthia - Giugno 2635}
    Era il quarto giorno che stava girando per la città, utilizzando ripari
    improvvisati per dormire. Non poteva permettersi il lusso di affittare
    una camera da letto in una qualche locanda. Avrebbero potuto trovarlo.

    Dopo essere fuggito da dove si era schiantato con l'auto Hax aveva
    seguito una traccia. Questa traccia era un orco piuttosto sospetto che
    gli avevano indicato due vecchietti. Dopo una giornata di
    investigazioni era venuto a scoprire di un giro di mercanti di schiavi,
    rapitori, contrabbandatori di organi. Senza neanche avere il tempo di
    decidere si era trovato in una lotta senza quartiere con una banda che
    operava da più di sei mesi a Shynthia.

    Non gli ci volle molto per capire che quella era la missione per la
    quale i Rollers erano stati chiamati là. Era impossibile che non lo
    fosse. Non importava se Elythia non glielo avesse detto. Il problema
    era grave. A quanto aveva capito nell'ultimo periodo c'erano state una
    serie di sparizioni tra la popolazione locale. Da un giorno all'altro
    delle persone scomparivano senza lasciare traccia. Non c'erano
    connessioni tra queste persone. La polizia, a leggere dai giornali che
    era riuscito a consultare alla biblioteca cittadina, stava provando a
    fare qualcosa, intensificando i controlli la sera.
    
    Ma questo non
    bastava. Quello che avrebbero dovuto fare era quello che stava facendo
    lui. Andare in giro, prendere la gente sospetta ed interrogarla. Se non
    parlavano\dots{} Convincerli con le cattive. Non gli piaceva farlo, ma
    c'era un'altra cosa di cui tenere conto: aveva perso traccia degli
    altri. Non c'erano stati contatti. Purtroppo non si ricordava niente
    del volo. La paura aveva bloccato qualsiasi processo che non fosse
    quello di pensare ``muoiomuoiomuoio'' ed avanti così. Fatto sta che non
    li aveva trovati negli ospedali.
    Fortunatamente il crollo della casa nel quale si era andato a
    schiantare avrebbe nascosto la sua ricerca. Quello di cui aveva paura
    era che lo collegassero agli altri. Aveva paura che li avessero presi,
    visto i casi di queste scomparse, e che li utilizzassero come mezzo di
    ricatto. Avrebbero potuto toccare tutto, ma non gli altri degl gruppo.
    Avrebbe continuato a neutralizzare un criminale alla volta, fino ad
    arrivare alla loro base, per liberare gli altri, se fossero stati
    presi, oppure per eradicarli dal territorio.

    C'era un problema. Lui non era un combattente addestrato. Se la era
    sempre cavata in tutte le situazioni, ma non aveva mai imparato arti
    marziali o altro. Durante gli anni prima dell'accademia aveva dovuto
    imparare a vivere nel mondo. Andare in giro per la città la notte a
    prendere un tizio alla volta, fargli delle domande, rompergli un arto o
    buttarlo giù dal terzo piano di un palazzo era una cosa. Picchiare una
    certa quantità di persone contemporaneamente dentro un sotterraneo
    (perchè così se l'era immaginato ``parlando'' con quelle gentili
    persone) era un altro affare.

    Però doveva farlo. Finalmente aveva capito dove si trovava la loro
    base. Quella sera avrebbe agito. Ogni giornata passata in più diminuiva le
    sue possibilità di finire il lavoro. E di rivedere gli altri. Beh,
    Lissa se la sarebbe presa, ma quella sera avrebbe utilizzato gli
    algoritmi che conosceva.

    Debug o meno.

    \section{Rollers senza Hax - Shynthia - Giugno 2635}
    Era calata la sera, la città era illuminata dai lampioni, dalle luci
    dei negozi e da quelle dei ristoranti. I Rollers stavano gironzolando
    sperando che qualcuno potesse dir loro se avesse visto Hax. Un sacco di
    gente che non lo aveva visto, qualcun'altro aveva scambiato la sua
    descrizione per un amico, altri, infine, non avevano capito
    minimamente.

    Come quella coppia di anziani che ha risposto ``Non possiamo
    rispondere, è in gioco la sicurezza nazionale.'' Lissa passò un po' di
    tempo a pensare a quella frase, ma poi lasciò stare, non ne avrebbe
    cavato un ragno dal buco. Il discorso tra Elythia e The Fixer si era
    preso uno spazio nella sua testa più grande di quanto lei non volesse
    ammettere. E se veramente Hax fosse stato preso da quei criminali? \'E
    vero che lui, per quanto in maniera improvvisata, sapeva far del male a
    chi lo minacciava, ma contro tanta gente sarebbe sicuramente stato
    sconfitto.

    A meno che\dots{} ``Oddio.'' mormorò, fermandosi in mezzo alla strada.

    \section{Hax - Shyntia - Giugno 2635}
    ``Ha.'' fece Hax, in segno di scherno al gruppo di gente che aveva di
    fronte. Una serie di tizi poco raccomandabili avevano le ``armi''
    spianate. Era riuscito ad infiltrarsi nella base, una specie di
    complesso sotterraneo profondo una decina di piani.

    Per un po' era riuscito a neutralizzare criminali poi, però, visto che
    lui non era Lissa, lo avevano beccato. Adesso si trovava in uno
    stanzone che fungeva da magazzino con quelli che sembravano essere gli
    ultimi\dots{} dieci? Tizi.

    Facciamo nove, visto che uno era steso davanti a lui, fumante. Era
    riuscito a fulminarlo in maniera non letale utilizzando uno degli
    algoritmi che avevano utilizzato per la catapulta elettromagnetica,
    fatto passare attraverso un tubo di ferro che aveva trovato là.

    Fare il programmatore significava questo: vedere oltre la realtà. Non
    sapeva come funzionasse per matematici, medici, chimici, ecc. Per lui,
    Lissa e, probabilmente, per gli altri programmatori la cosa era
    molto\dots{} sinestetica. Quando si voleva utilizzare un algoritmo per
    modellare la realtà si cambiava temporaneamente paradigma. Alla realtà
    si sovrapponevano una serie d'immagini stilizzate, di scritte, linee
    squadrate, cerchi concentrici. Ognuno con i suoi significati specifici,
    indicati dalla posizione sopra i singoli oggetti, dal colore, dalla
    quantità di luce che emanava, dall'opaticità. Così si potevano vedere i
    suoni, sentire i sapori, toccare i colori.

    Utilizzare un algoritmo significava modificare alcuni di questi valori
    tramite l'utilizzo di alcune formule specifiche, rappresentate a loro
    volta dall'unione di cerchi, linee squadrate e tantissime scritte,
    chiamate ``codice''. Almeno, questo era
    quello che si faceva di solito. Lui era un Retro-Engineer, significava
    che prendeva gli algoritmi o i programmi degli altri e li smontava.
    Quando lo faceva gli sembrava di creare degli esplosi dei programmi,
    con ogni cosa ancora al suo posto, ma su piani differenti. Aveva
    imparato così a programmare. Una cosa utilissima, perchè ti permette di
    essere dannatamente flessibile. Però era anche a causa di questo che
    non era mai stato così bravo come Lissa. \'E vero, lui sapeva fare un
    po' di tutto perchè aveva visto come funzionava più o meno tutto, ma
    lei era un genio quando si trattava di controllo delle energie
    elementali e della luce.

    E, ovviamente, era anche un genio nel Debug. Il Debug era una pratica
    arcana che permetteva ad un programmatore abbastanza bravo di
    correggere al volo l'algoritmo di un altro, evitando così possibili
    casini. Sbagliare un algoritmo poteva significare che non avrebbe
    funzionato o, nei casi peggiori, che si sarebbe comportato in maniera
    quasi casuale, andando a toccare parti di realtà non considerati da chi
    l'aveva lanciato. Uno dei problemi di Hax era proprio che, a causa
    del suo passato accademico, faceva spesso errori di scrittura del
    codice. Spesso erano sviste, quindi non avevano ripercussioni troppo
    dure, solo che rischiava di creare più danni di quanto non prevedesse
    (inserendo un numero più grande), con energie che non aveva considerato
    (modificando un riferimento senza volere) oppure alla cosa sbagliata
    (invertendo puntatori o cose così).

    Utilizzare del codice sulla realtà, comunque, non era gratis. Dopo aver
    lanciato un algoritmo, normalmente, i muscoli facevano male, ci si
    sentiva spossati, si cadeva più facilmente per terra, cose così. Gli
    effetti erano proporzionali alla magnitudine del cambiamento che si
    voleva imprimere alla realtà, in quanto un algoritmo veniva fatto
    girare utilizzando il corpo del programmatore come ``hardware''. I
    programmi erano una cosa differente. Ci si voleva molto per crearli,
    normalmente anche in più di una persona, e partivano da un oggetto che
    non fosse il programmatore, evitando questo affaticamento. Non era una
    cosa alla ``mana'' dei libri. Chi non praticava quell'arte non capiva.
    Non c'era niente che ti consumasse il sangue, era più alla stanchezza simile a correre
    per duecento metri. Solo che ti si accumulava in un secondo mentre
    stavi fermo

    Era per quello che aveva leggermente il fiatone. Fortunatamente quello
    che aveva fatto poco prima al malcapitato aveva funzionato al primo
    colpo senza effetti indesiderati. ``Allora?'' fece, puntando il tubo
    alle altre nove persone ``Chi vuole finire come il suo amico? Ne ho per
    tutti.'' Quelli lo guardarono leggermente impauriti e poi, lasciando a
    terra le loro armi, si girarono e se ne andarono.

    Non erano spaventati, sembravano più scazzati. In effetti uno di loro
    mormorò ``Non vengo pagato abbastanza per questo.'' un altro ``Un tizio
    che picchia ok, ma uno che fulmina la gente?'' In pochi minuti non li
    sentì più.

    ``Che diavolo?'' si chiese Hax, mentre stava tentando di capire se
    avesse dovuto inseguirli oppure lasciarli andare per controllare se,
    veramente, avessero preso gli altri. Optò per la seconda. I suoi amici
    erano più importanti. Attraversò tutti i piani, che contenevano stanze
    da letto, sale ricreative ed altri luoghi per una base, fino ad
    arrivare all'ultimo. Aprì la porta che dava sulle scale ed entrò in un
    salone che conteneva una serie di capsule metalliche inclinate di
    quarantacinque gradi rispetto al pavimento, con dei vetri sulla parte
    superiore. Queste erano collegate ad una serie di tubi e cavi che
    entravano in pavimento, soffitto e pareti. Da dentro le capsule usciva una luce blu che illuminava
    tutta la stanza. Hax appoggiò a terra il suo zaino e quindi si avvicinò ad una delle capsule, per vedere che
    cosa contenessero, anche se aveva paura di averlo già capito.

    Si chinò sulla prima e, passando la mano sul vetro, rimosse la condensa
    che si era formata. Purtroppo aveva avuto ragione. Dentro quella
    capsula c'era una ragazza nuda, che sembrava stesse dormendo. Dai
    capelli sembrava che fosse stata immersa in un qualche liquido. Azzardò
    che quella fosse una capsula criogenica. Gliene aveva parlato Elythia,
    tempo addietro. Sembra che vengano utilizzate per una gran quantità di
    cose: dalla prigionia alla terapia.

    Quest'ultima cosa gli fece venire in mente che doveva andare a vedere
    se c'erano gli altri per poi trovare un modo per tirar fuori tutta
    questa gente dai tubi.

    Era alla quinta capsula quando un rumore lo fece girare di scatto.
    Riuscì, appena in tempo, ad alzare il tubo che aveva ancora in mano per
    parare un colpo di spada sferrato da una figura che non riuscì subito a
    riconoscere che si trovò per terra.

    Per terra, schiacciato da una ragazza dannatamente attraente. Occhi
    verde smeraldo, lunghi capelli rosso rubino legati in una coda, corpo
    con tutte le curve al loro posto, messe in risalto da un armatura
    attillata in pelle, con delle protezioni aggiuntive sui seni, sulle
    ginocchia e sui gomiti. Aveva già visto questa perfetta
    alchimia\dots{}

    ``Hehe'' ridacchiò Hax, mentre la ragazza continuava a spingere sfruttando
    il suo peso ``Celty! Che diavolo?'' Celty era una ragazza splendida.
    L'aveva conosciuta all'accademia, non faceva algoritmica, studiava
    storia. Era molto simpatica e, soprattutto, era la ragazza di Lissa.
    Almeno, lo era stato fino ad un anno e mezzo prima, quando aveva
    incominciato a comportarsi in maniera strana fino a scomparire. Lissa
    aveva capito di esser stata lasciata, mentre Hax non era convinto. In
    quel periodo era scomparso dal gruppo per una certa quantità di tempo
    per controllare Celty. Ogni volta che cambiava paradigma per manipolare
    il codice vedeva che c'era qualcosa che non andava ``intorno'' alla
    ragazza, ma non capiva cosa. Quando poi perse le sue tracce tornò dal
    gruppo.

    Lissa aveva passato un momento terribile, nel quale si chiedeva se
    avesse fatto qualcosa di sbagliato, se l'avesse ferita in qualche
    maniera irriparabile.

    Le solite pare, insomma. Anche se Hax non era particolarmente d'accordo
    col comportamento della sua amica lei saembrava soffrire
    particolarmente perciò non aveva proprio ``perdonato'' Celty per quello
    che aveva fatto. Certo, sempre SE lo avesse fatto di sua libera
    iniziativa.

    Comunque, eccolo là, dentro un covo di malfattori che rapivano gente
    per metterla in tubi criogenici, sdraiato a terra con sopra di lui una
    stupenda ragazza. Occupata. Che voleva ucciderlo. Tutto questo gli
    ritornò in mente, dopo aver fatto quel tuffo nel passato, come
    un'esplosione all'interno della sua testa.

    Hax strinse i denti, impuntò il piede destro sull'addome della ragazza
    e, con le forze che gli rimanevano, la spise via facendo forza su
    braccia e gambe. Celty finì cinque metri più in là, rotolando. Il
    ragazzo saltò in piedi, finendo con le gambe leggermente divaricate,
    per fare in modo che rimanere in piedi risultasse più comodo. Aveva il
    fiatone. L'algoritmo utilizzato prima, lo spavento, l'aver spinto via
    la ragazza lo avevano stancato non poco. Purtroppo utilizzare un
    algoritmo che hai visto solo poche ore prima richiedeva molta più
    energia di quanto non avrebbe dovuto, vista la mancanza di
    ottimizzazioni per l'uso diretto.

    ``Allora?'' fece, ansimando, guardando ancora verso terra con la mano
    sinistra appoggiata sulle ginocchia, cercando di
    recuperare le forze. Poi, alzando la testa, continuò, facendo un cenno
    verso la ragazza ``Che vogliamo fare, Celty? Una battaglia all'ultimo
    sangue tra due tizi che non sanno neppure combattere?'' ``No, Hax.
    Vattene, lo sai che non voglio farti nulla.'' ``Lissa è in pensiero,
    sai? Potrei farti svenire e portati da lei, così fate due
    chiacchiere.'' ``Me ne sono andata, basta. Non posso tornare.''
    ``Perfavore non tirarmi fuori queste stronzate. Non è da te non parlare
    dei tuoi problemi con Lissa.'' ``\'E diverso. Senti, se non la vuoi
    capire dovrò fermarti con le cattive.'' ``Provaci.''

    Aveva provato a fare il figo ed ora doveva subirne le conseguenze.
    Strinse il tubo e si rimise in posizione eretta. Questa battaglia era
    impari. Era vero che nessuno dei due era propriamente allenato nel
    combattimento. Lui perchè non aveva mai avuto voglia di studiarlo,
    l'altra perchè avrebbe avuto solo un anno e mezzo per imparare
    qualcosa. Il problema era che lei stava brandendo una spada. Appuntita
    ed affilatissima spada.

    Celty caricò con la spada al lato, pronta a sferrare un colpo appena in
    raggio. Quando fu abbastanza vicina impuntò il piede sinistro e fece
    partire un fendente col piatto da sinistra verso il fianco del ragazzo.
    Hax mise in mezzo il tubo, permendicolare al terreno, per fermare il
    colpo. Sentì arrivare la vibrazione fino alla testa. La ragazza,
    prontamente, ruotò l'arma e lo colpì nel plesso solare con l'elsa
    dell'arma, facendolo arretrare.

    Hax vedette tutto sfuocato per un istante, per poi vedere una serie di
    lucine d'ovunque, le quali stavano scomparendo un po' alla volta. Si
    sentì come se gli avessero fatto uscire tutta l'aria dai polmoni.
    \emph{Oh, cazzo. Un altro colpo così e non mi rialzo più} pensò,
    continuando a tenere lo sguardo sulla ragazza, che si stava avvicinando
    camminando \emph{Non sta provando ad uccidermi, almeno, altrimenti ci
    sarebbe già riuscita.}

    Mancava pochissimo che Celty fosse a portata di un altro colpo perciò
    Hax decise di utilizzare l'ultima risorsa disponibile. Avrebbe caricato
    elettricamente la sua arma ed avrebbe bloccato il colpo, sperando di
    far svenire l'altra.

    Saltò indietro e fece ``Celty, ascolta.'' iniziò a vedere il ``codice''
    del mondo ``Ho notato che non vuoi uccidermi. Che è successo?''
    estrasse dalla sua mente l'algoritmo per cambiare la carica di un
    oggetto. Di fronte a lui comparvero una serie di cerchi, uno sopra
    l'altro, separati da scritte e linee di vari colori. ``Hax, te l'ho già
    provato a dire.'' rispose la ragazza ``Me ne sono dovuta andare e non
    posso tornare altrimenti vi metterei in pericolo. Quindi non voglio
    neppure fare del male a te'' le parole della
    ragazza modificarono dei valori intorno a lei. Intorno alla sua gola le
    ``aureole'' divennero spettri circolari, cambiando colore in base al
    volume. Ottimo. Hax avrebbe utilizzato quei dati per indirizzare
    l'algoritmo verso di lei. Vide ancora quei
    strani pezzi di codice intorno alla testa della ragazza, come quando se
    n'era andata. Strani, ma troppo piccoli e, soprattutto, troppo poco
    inerenti a quello che avrebbe fatto. ``No,'' continuò il ragazzo
    ``questo non me l'avevi detto.'' Importava poco. Ora doveva completare
    l'algoritmo modificandolo quì e lì per essere sicuro di colpirla e di
    non ucciderla per la scarica troppo alta. ``Comunque, per quanto
    potresti metterci in pericolo, non pensi che lavorare per della gente
    che commercia esseri umani sia un pochino\dots{} Poco etico?'' aveva
    quasi finito di aggiungere gli ultimi dettagli. Sperava solo di averlo
    fatto correttamente. ``Non lavoro per loro. Faccio la corriera.''
    ``Scusa?'' finito. Compresse i dischi e le scritte e le mise sopra il
    tubo, il quale s'illuminò di azzurro per un istante. ``Hai fatto
    qualcosa, sì? Dannazione. Avrei dovuto capirlo quando ho visto quei
    riflessi nei tuoi occhi.''

    Appena finì di dire quella frase Celty si lanciò sul ragazzo, caricando
    un colpo alla testa. Hax sollevò rapidamente il tubo per bloccare il
    colpo. Appena le due armi si toccarono l'algoritmo venne attivato,
    rilasciando una serie di linee sul tubo e facendo comparire dei cerchi
    rossi lungo di esso.

    \emph{Aspetta. Rossi?}

    \section{Rollers senza Hax - Synthia - Giugno 2635}
    ``Elythia. \'E vero che ho detto che Hax se la saprebbe cavare, ma non
    è che se si sentisse in pericolo utilizzerebbe degli algoritmi?'' fece
    Lissa, ferma in mezzo alla strada ``Beh, gliel'abbiamo detto più di una
    volta di non utilizzarli a meno che non abbia tempo per testarli.''
    rispose Elythia, girandosi ``Me se fosse in pericolo non ne avrebbe
    tanto. Potrebbe\dots{}''

    Lissa non fece in tempo a finire la frase che, in lontananza, si sentì
    un esplosione. Un paio di centinaia di metri più a nord una colonna di fumo si
    ergeva dai tetti della città. In giro per strada la gente si stava
    fermando per strada per vedere che cosa fosse successo.

    ``Oh, cazzo.'' fece Lissa, per poi scattare in quella direzione
    ``Lissa!'' urlò Jam per poi seguirla. ``Dannazione. Fixer.'' fece
    Elythia, iniziando a muoversi a passo spedito dietro alle altre due.

    Fixer prese un'altra strada. Aveva bisogno di altro.

    \section{Hax - Synthia - Giugno 2635}
    Hax si trovò sdraiato a terra. Aprì gli occhi, per vedere tutto
    sfuocato. Non sentiva quasi nulla, a parte un fortissimo mal
    d'orecchie ed un fischio. Quando incominciò a mettere a fuoco la scena
    vide Celty che stava andandosene zoppicante mentre in tutta la stanza
    qualunque cosa avesse potuto prendere fuoco stava bruciando. Doveva trovare le energie
    per alzarsi. Era come essersi appena alzati dal letto, solo con tutto
    il corpo che urlava ``Ma che cazzo hai fatto?''

    L'algoritmo aveva un bug. \emph{Cazzo} Probabilmente aveva funzionato
    su quel tizio prima solo per una botta di culo. Doveva aver sbagliato
    ad inserire i riferimenti a che elemento doveva venir modificato,
    oppure quei frammenti di codice che giravano intorno a Celty avevano in
    qualche modo interferito con la sua esecuzione.

    Chi se ne fregava. Troppo tardi, troppo pericolosa questa stanza per
    stare quì a pensare. Raccolse le energie, si girò a faccia a terra ed
    impuntò la mano destra sul pavimento. Spingendo con le ultime energie
    rimaste si rimise in piedi. Iniziava anche a ritornare l'udito, che gli
    permise di sentire lo scoppiettio delle ultime fiamme.

    ``Cazzo!'' urlò ``Lo zaino.'' doveva trovarlo. Dentro c'era tutta la
    sua vita. Dopo un paio di secondi lo vide. Fortunatamente era fatto in
    materiale anti proiettile. Se lo mise in spalla per poi andare a vedere
    come stavano le capsule. Incredibilmente non c'erano problemi e
    l'esplosione sembrava non aver fatto danni alla struttura. Sarebbe
    comunque dovuto andarsene il più in fretta possibile.

    Sempre la polizia. Se l'avessero beccato l'avrebbero portato alla
    centrale e poi messo in gattabuia per almeno la serata, a meno che non
    passasse Elythia a parlare con loro.
    Improvvisamente, però, anche se il suo lavoro, probabilmente, non
    sarebbe stato riconosciuto, era sollevato. Non aveva trovato gli altri
    lì sotto, quindi voleva dire che non li avevano presi. Iniziò a
    zoppicare verso le scale e le percorse. Arrivò alle porte divelte per
    notare che fuori c'erano già degli agenti.

    ``Uff.'' fece ``Beh, ragazzi, spero che siate ancora in città,
    altrimenti sono cazzi.'' e si avviò verso l'uscita con le mani alzate.

    \section{Rollers - Shynthia - Giugno 2635}
    Arrivarono sulla scena con una casa che pareva essere esplosa dal
    basso. Non era crollata, ma aveva una colonna di fumo che saliva dalle
    cantine e tutti i vetri dei piani superiori che erano saltati
    dall'interno.

    ``Che è successo?'' chiese Lissa ad un passante ``Non lo so, ad un
    tratto si è sentito un botto e la casa è come esplosa. Spero non si sia
    ferito nessuno.'' rispose questo. Dopo poco arrivò la polizia che creò
    un cordone intorno al palazzo. Elythia provò a spiegare che, forse, là
    dentro c'era un loro amico che poteva essersi fatto male, ma loro
    risposero che non potevano entrare finchè non arrivavano i pompieri.
    
    Dopo poco, però, due poliziotti scattarono in avanti con la loro
    pistola in mano, entrarono nel palazzo e ne uscirono con Hax, il quale
    sembrava rovinato. Aveva tutti i vestiti rovinati dall'esplosione, la
    faccia sporca di fuligine, gli occhi arrossati dal fumo e zoppicava.

    ``No, dai, seriamente. Vi ho detto che\dots{}'' stava dicendo, per poi
    venir spinto in avanti ``Hei! Ma che cazzo avete? Vi ho detto che giù
    di sotto ci saranno una quindicina di persone da salvare.'' continuò,
    liberandosi dalla presa degli agenti. ``Ok, ho capito, vado dal vostro
    capo.'' fece, alzando le mani al petto, per poi continuare dritto.
    \emph{Stronzi} pensò.

    Jam sentì un tuffo al cuore vedendo Hax. Stava facendo il duro con la
    polizia. Stava bene. Diede uno strattone ad Elythia, la quale stava
    ancora litigando con l'agente, la quale si fermò e fece ``Eh?''
    ``Hax!'' rispose Jam, indicando il ragazzo che stava camminando verso
    l'agente più alto il comando al momento. ``Oh, per Luna!'' fece
    Elythia, evidentemente sollevata ``Sta bene!'' poi lo sentì dire
    qualcosa del tipo ``Ma chi cazzo credete che io sia?!?''

    Si sentì cadere le braccia pensando a tutto quello che avrebbe dovuto
    fare per tirare fuori dalle peste il suo ``fratellino'', ma almeno
    sapeva che stava bene.

    Dopo un'ora di spiegazioni le permisero di incontrarlo. ``Elythia!''
    fece, allegro ``Che bello vederti! Stai bene?'' ``Sì, grazie, te?''
    ``Eh. Dici, interrogatorio e contusioni da esplosione a parte? Bene.
    Comunque. Gli altri sono con te?'' ``Sì. A proposito, The Fixer mi ha
    detto di darti questo.'' e passò una bottiglia di Porto ad Hax, oltre
    che a delle salviette rinfrescanti che doveva aver rubato in qualche
    ristorante ``Grande! Sono così contento che stiate tutti bene. Avevo
    paura vi fosse successo qualcosa.'' ``Anche noi, scemo.'' rispose lei,
    con le lacrime agli occhi ``Eh. Scusate. Ma ho fatto il lavoro,
    visto?'' ``Eh?'' ``Beh, sì, quello della banda criminale, no?'' ``Hax,
    noi siamo venuti quì per un elaboratore rotto.'' ``Come?'' ``Vabbeh,
    paladino della giustizia. Sono contenta tu non ti sia fatto troppo
    male. Ora ti tiro fuori di quì e ne parliamo.'' ``O\dots{} Ok.''

    Hax passò tutta la sera a tentare di capire come mai non fosse stato
    quello il loro lavoro, per poi lasciar stare. Aveva del porto a fargli
    compagnia. The Fixer aveva sempre la cosa giusta al momento giusto.

\cleardoublepage{}
