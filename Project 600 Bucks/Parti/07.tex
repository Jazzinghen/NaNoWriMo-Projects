\chapter{Let's rock, baby!}

  \section*{Valentine}
    
    Valentine si alzò tardi. Era stata su quasi tutta la notte per considerare le varie possibilità. Ed il piano di
    quella giornata era quello con le più alte possibilità di portare a casa capre e cavoli. Aveva mandato Catherine a
    controllare i dintorni Bazaar, mentre Sofia doveva intercettare le comunicazioni delle telecamenre alle entrate del
    mercato per controllare quando fossero entrati i tizi dei Silicon Knights con la ragazza. A quel punto Sofia doveva
    andare dentro e dumpare il cervello di Zoè dentro quello esterno. Tutto stava a vedere dove avrebbero mandato i vari
    Frames. E, soprattutto, se \emph{lui} fosse veramente lì o meno.

    Intanto aveva mandato il Pivello alla base in stand by, così che potesse muoversi appena ce ne fosse stato bisogno.
    Lei doveva aspettare. Lo scambio sarebbe avvenuto alle 20:00 al Bazaar.

    Il Pivello si era mosso per andare alla base verso le 17:00. Solo che, una volta arrivato aveva trovato più di metà
    della spedizione della Chevalier che girava attorno all'edificio, il che significava anche che c'era un frame in
    quella zona. Il Pivello l'aveva chiamata per quello. Era definitivamente un'emergenza. ``Ok, senti, Pivello. Come
    vedi la situazione?'' chiese lei, attraverso la cornetta ``Mh, signorina. Il numero di guardie è alto, però se
    riesco ad arrivare all'ATHENA dovrei riuscire a sbaragliarle senza problema. Potrebbe essere complicato con il
    Frame, ma grazie al Fr...'' sembrava non ricordarsi il nome ``Fragach. Se non ti ricordi il nome dì
    \emph{pistola}.'' ``Sì, lo farò, anche se preferirei rispettare le vostre scelte. Comunque, siccome sono addestrato
    al combattimento ravvicinato mentre la maggior parte dei piloti fa solo riferimento al combattimento a media e lunga
    gittata, dovrebbe essere possibile se sfrutto l'effetto sorpresa.'' ``Ovvero?'' ``Dovrei uscire direttamente dalla
    base. Questo, però...'' ``Significherebbe demolirla in gran parte, giusto?'' ``Sì, mi dispiace. Se vuole posso
    evitare.'' Non c'era da chiederlo ``\emph{Pivello}, a me pare una buona idea. Vai con quello.'' ``Roger. Arsenicos,
    chiudo.'' e mise giù.

    Chissà. Avrebbe dovuto infiltrarsi nella base senza farsi scoprire e poi, dopo aver attivato l'ATHENA, saltare fuori
    nel minor tempo possibile. Vediamo se il ragazzo aveva abbastanza palle per farlo. Poi arrivò un'altra chiamata.

    Sarebbe stato un cazzo di viaggio.

  \section*{Arsenicos}

    Mise giù. Era ora di entrare in azione. Tentò di calcolare il tragitto migliore per entrare nella base. Quasi
    sicuramente il meglio era passare attraverso la discarica, poi su per le scale in grata non facendo rumore e poi
    scivolare verso l'Armoured Frame. Una volta dentro, poi, sarebbe stato un gioco da ragazzi. Ovviamente non avrebbe
    avuto tempo per risparmiare nemici. Se erano lì voleva dire che volevano o rubar loro qualcosa, tendere loro una
    trappola oppure piazzarne in giro proprio come temeva ieri la signora Mac an Bahird.

    Iniziò a muoversi nei vicoli, stando attento a non farsi troppo vedere nelle strade principali. L'Armoured Frame che
    era là assieme al gruppo della Chevalier avrebbe potuto rilevarlo. Fece il giro da lontano, tentando di passare
    dalla parte con meno guardie possibili. Gli ci vollero una decina di minuti però, finalmente, era nel punto più
    vicino possibile alla barriera che divideva la discarica dalla strada. Sarebbe dovuto salire sulla costruzione a
    lato e saltare oltre la grata. Si sarebbe fatto un male impossibile. Ma questo ed altro per la giustizia!

    Iniziò a salire le scale antincendio di uno dei palazzi quando, a metà, ricevette una chiamata dalla signorina
    Arthur. Immediatamente rispose ``Arsenicos online.'' ``Pronto? Ascolta. Ci ho ripensato.'' disse lei, dall'altra
    parte della comunicazione ``Eh?'' ``Farti recuperare il Frame è troppo pericoloso. Ho bisogno che tu sia tutto d'un
    pezzo per lo scambio. Vieni a recuperare dell'altro equipaggiamento.'' ``Mh.'' poteva capire che fosse pericoloso.
    Lo era, tantissimo, ma i piani pericolosi erano il marchio di fabbrica della signorina Arthur. Evidentemente aveva
    bisogno di certezze. ``Ok, vedrò di tornare il più in fretta possibile.'' ripose lui ``Ottimo. Vedi di non morire là
    fuori, Arsenicos.'' ``Roger. Arsenicos, chiudo.''

    Iniziò a scendere le scale silenziosamente. D'altronde la signorina Arthur l'aveva chiesto espressamente...

  \section*{Valentine}

    Valentine aveva ricevuto una comunicazione da parte di Catherine. A quando pareva c'era stato un assemblamento di
    Frames intorno al Bazaar. Sarebbe stata dura uscire da lì, ma era abbastanza confidente nel fatto che avessero
    abbastanza abilità ed equipaggiamento per uscirne vincitori. Un'altra informazione che le aveva dato era che uno dei
    due Frames dei Silicon Knights, quello con più graffi da usura alla corazza esterna, aveva segni indisputabili che
    si trattasse del veicolo di Nicolas.

    Appena sentì quel nome Valentine ebbe un momento ddi sconforto. Era bravo. Era dannatamente bravo. Il Pivello non ce
    l'avrebbe fatta a neutralizzarlo. Questo significava che Catherine avrebbe dovuto porre l'attenzione a lui,
    sprecando colpi del Tekko, probabilmente, visto che lui era uno che preferiva le tecnologie studiate, perciò avrebbe
    montato delle contromisure attive antitarget invece che gli scudi elettromagnietici ancora in via di
    sperimentazione.

    Poi, però, guardò alla valigia. Sul suo volto comparve un sorriso quasi maligno. Ora di farla pagare allo stronzo.
    Avrebbe capito con chi cazzo aveva a che fare. Gli avrebbe cancellato dalla faccia quel suo sorriso da coglione
    \emph{so tutto io}.

    Scattò in piedi, prese la valigia e si diresse verso l'uscita. Doveva trovare un posto lontano ed alto.

  \section*{Sofia}

    Sofia era seduta sul tetto di una costruzione, ad occhi chiusi e gambe incrociate, collegata al suo portatile. Tentare di tener
    d'occhio una ventina di telecamere con gli occhi aperti era un casino, anche perchè realtà ed aggiunte del sistema
    integrato si mescolavano. Verso le quattro notò un gruppo entrare da uno degli archi del Bazaar, scortando Zoè.
    ``Ora di andare.'' disse a sè stessa, alzandosi in piedi. Il piano era quello di correre dentro nel Bazaar,
    raggiungere Zoè, fare il dump del cervello e scappare senza farsi vedere. O, se veniva vista, eliminare le guardie
    che l'avevano vista. Non le piaceva molto come cosa, ma non era quello il momento di mettersi a fare discorsi
    morali. L'unica cosa era che, se veniva scoperta e doveva veramente eliminare della guardie, queesti tizi avrebbero
    potuto vendicarsi su Zoè.

    Che, però, era l'unico modo che avevano per recuperare la fantomatica \emph{tecnologia}. Boh, non è che ci capisse
    molto dal piano. Il modo migliore era non pensarci proprio ed agire. Mise in spalla lo zaino con gli strumenti,
    puntò l'Hook e partì.

    Volò attraverso vie, sopra tetti e strade. Non ci mise molto ad arrivare all'interno del Bazaar. Una volta lì iniziò
    a chiedere in giro. I mercanti non hanno problemi a dirtelo se hanno visto qualcuno di sospetto. Devono tenere
    d'occhio la gente che potrebbe fare danni. Le riferirono che li avevano visti passare per la strada dei negozi
    d'elettronica ed imboccare quella degli Otaku. Le dissero, inoltre, che sembrava si fossero divisi. Due seguivano la
    ragazza, mentre in giro c'erano tutti gli altri. Seguì le tracce finchè non vide che il gruppo che scortava la
    ragazza stava salendo una scala per andare sui piani intermedi per poterdi nascondere meglio.

    Era impossibile avvicinare Zoè senza neutralizzare le guardie, perciò pensò che, se non la vedevano, poteva
    attaccarli senza ucciderli, fare il suo lavoro e poi andarsene. Li seguì su per le scale. Questi arrivarono fino al
    penultimo piano intermedio e poi si fermarono davanti ad un negozio a discutere dove andare dopo di quel punto, visto
    che lì c'erano solo negozi.

    Senza aspettare oltre Sofia impugnò il piede di porco con la sinistra, il quale era attaccato al lato dello zaino da
    due straps, scattò dietro le due guardie e colpì una delle due al braccio sinistro. Durante il tragitto alcune patch
    esagonali sul braccio sinistro scoperto di Sofia si aprirono, facendo uscire dei micro thrusters, i quali spinsero
    il braccio in direzione del colpo. All'impatto il braccio cambiò leggermente geometria per scaricare tutta la forza
    accumulata durante il colpo direttamente all'impatto. Le avevano fornito il braccio Cybernetico, ma nessuno le
    aveva detto che non poteva installare dei potenziameti cinematici per combattere in corpo a corpo. Una ragazza
    doveva difendersi.

    L'uomo colpito volò via, in direzione del colpo, andando a schiantarsi contro una paratia, per poi svenire a terra.
    Sperava fosse svenuto. Subito provò a colpire anche l'altra guardia, la quale, però, fermò il colpo con il braccio
    destro. ``Oh, cazzo.'' commentò Sofia, mentre guardava la geometria del braccio di questo rimettersi a posto dopo
    aver parato il colpo ``Ha... HAHA.'' tentò l'approccio sociale. Sembrava non funzionare. Notò, allora, che quel
    tizio aveva un cervello Cybernetico. Lo notò dai connettori extra che si era fatto installare sulla tempia per poter
    connettere più rapidamente gli smartpacks. Guerrafondaio del cazzo.

    Il tizio, tenendo saldamente il piede di porco della ragazza, caricò un colpo con la sinistra. Sofia dovette agire
    in fretta. Raggiunse con la mano destra uno dei suoi connettori retraibili sul collo e lo estrasse, tracciando in
    aria un arco con il cavo. In rotazione schivò il colpo dell'uomo accucciandosi e poi, prima che potesse rimettersi
    in posizione di difesa, saltò verso l'alto, aiutandosi con tutto il corpo. A mezz'aria completò la rotazione,
    collegando il suo connettore ad uno di quelli sul collo dell'uomo.

    Non voleva connettersi senza portatile ad uno così, ma non aveva tempo.

    Inoltre. Era lui che doveva preoccuparsi.

  \section*{???}

    ``Prego selezionare la rappresentazione virtuale per bloccare il tentativo d'invasione.''

    ``Avete selezionato Karsperski-NOD Titan 12-Pro. Si è sicuri della scelta?''

    ``Inizializzazione del motore.''

    ``0============100''

    ``Pronti.''

  \section*{Sofia - Regina centenaria delle streghe}

    Era da un po' che non provava un'invasione di sistemi attraverso la realtà virtuale. Lì la cosa era completamente
    differente. Aveva bisogno di piegare la volontà dell'utente, non di bypassare un codice. Era più difficile. E più
    facile. Normalmente gente come il tizio che aveva attaccato usava software di aiuto standard, per non parlare del
    fatto che avrebbe scelto una rappresentazione virtuale di quelle gigantescbe, che dovrebbero far paura per la
    dimensione. Quello che i tizi come quello lì non sapevano era che potevano spaventarci qualche Noob minchia, non una
    come lei.

    Non a caso lei scelse di utilizzare una rappresentazione come Bayonetta, però coi capelli corti viola ed una tenuta
    un po' più High Tech come aspetto. Però il resto era uguale, katana ed artiglioni elettrici ai piedi.

    Eccolo lì, infatti, un avatar che spaziava su quasi tutto il campo visivo a quella distanza, il che significava
    dalle 100 alle 200 volte più grande di lei. Noob.

    Il titano provò a colpirla con un pungo, che era la rappresentazione di un attacco alla sicurezza. Una puttanata,
    visto che era gigantesco, quindi poteva fare tanti danni, ma chiamato come non mai. Lei schivò il colpo, ovvero
    riuscì ad intrappolarlo in qualche socket finta, ed utilizzare l'apertura per iniettare un virus stupido quanto
    efficacie. Era un virus che rallentava completamente il sistema. Non cancellava nulla, però permetteva a chi
    l'avesse lanciato di eseguire un sacco di altri programmi ai quali l'avversario non sarebbe riuscito a reagire.
    Certo. Ormai i sistemi hanno un metodo automatico per bloccare questi attacchi. Ma devono accorgersene e quindi
    disattivarli, il che prende tempo a cause proprio del virus. Questo si riassume in... Witch Time. O Bullet Time,
    dipende dall'ambientazione di sui si parlava. LOL.

    Lanciò il suo attacco, al grido di ``Don't touch me!'' Witch Time. Avrebbe dovuto fare qualcosa per far diventare
    tutto viola, ma non era quello il momento. Fece partire una serie di attacchi imparabili con la sua katana, la quale
    era la rappresentazione di attacchi di Overflow molto diretti ma distruttivi. Non era facile farli entrare, ma una
    volta che colpivano poteva far girare codice arbitrario nel sistema avversario. Tipo eliminazione di file,
    disattivazione di servizi, distorsione delle percezioni. Adorava quest'ultimo attacco. Era il massimo per stordire
    gli avversari. Non voleva \emph{formattarlo}, voleva metterlo fuori combattimento per poi eliminare i ricordi che la
    riguardavano. Prima, però, avrebbe dovuto demolire il sistema di sicurezza rappresentato dal titano. Avrebbe potuto
    attaccare direttamente il proprietario del cervello, ormai senza difese.

    Schivò un paio di colpi diretti, poi il nemico lanciò una serie di attacchi a ricerca rappresentati come dei
    missili. Ne schivò uno e, dopo aver iniettato il solito sistema per rallentare il tempo corse su per il braccio
    del nemico. Il tempo durò abbastanza per permetterle di scattare (ovviamente la velocità della corsa non era proprio
    proporzionata in base agli avatar) e raggiungere la spalla. Quì lanciò un attacco abbastanza mirato da permetterle
    di staccare il braccio. Ed intendeva disattivare una subroutine d'attacco. Il titano urlò. Non sentiva dolore, solo
    la guardia sarà stata incazzata a bestia. Che pippa. Saltò all'indietro, finendo sul terreno. Dietro di lei gli
    attacchi a ricerca andarono a schiantarsi sul terreno. Degli attacchi del genere, se pensano di essere arrivati a
    destinazione, poi non fanno più controlli, perciò falliscono miseramente.

    Il combattimento andò avanti così, finchè nomn staccò anche l'altro braccio del nemico. Invece che saltare a terra
    passò sulle spalle del nemico. Aveva una modifica che le permetteva di camminare sulle pareti verticali. Ora. Dove
    poteva stare uno come quello all'interno del sistema? Ovvero. Quale era la componente del sistema di protezione che
    andava staccata per disattivare l'intero titano? Era un po' come tentare di capire dove stessero le batterie
    all'interno di qualche gadget o di dive fosse il cavo dell'elettricità di qualche elettrodomestico. Stacchi quello,
    non funziona più nulla. Ovviamente la rappresentazione umana urlava \emph{Tagliategli la testa}, ma in quel mondo
    seguire dei consigli che darebbe la Regina di Cuori non serviva ad una beneamarita fava. Proprio perchè la mente
    umana è portata a pensare che il punto più importante sia la testa non si mettevano lì le socket per la connessione
    al sistema operativo. Qualcuno le metteva nei talloni, altri nel cuore (anche lì c'era da discutere se fosse una
    scelta tattica, in realtà), altri in qualche punto che non si colpisce mai.

    Però. Era anche vero che quello aveva scelto una rappresentazione abbastanza macha. Probabilmente avrà pensato che
    il punto più alto era anche il migliore. E poi che le costava tagliare la testa a questo coso? Ormai era là. Partì
    all'attacco. Tagliò a metà uno dei missili, facendo in modo che una delle funzioni restituissero il famosissimo
    valore \emph{1.0/0.0}, e poi si avvicinò al collo. Il quale, però, era cinto da un collare ricolmo di cannoni laser.
    Pessimo, PESSIMO gusto. Se voleva andare sulla testa per vedere di fare dei danni seri doveva passare quel sistema.
    
    I cannoni si girarono verso di lei, pronti a fare fuoco. Poteva schivare i colpi, soprattutto con quei tempi di
    carica, però doveva disattivarne un botto in un colpo, non poteva perdere tutto tempo. Iniziò a pensare se, magari,
    non fosse possibile far girare uno specifico comando all'interno del sistema di protezione. Qualcosa come
    \emph{killall -KILL lazah} o cose così. Schivò due o tre colpi tentando di recuperare l'handle delle postazioni
    laser. Verso il quarto riuscì a farselo restituire. Usandolo caricò un colpo e, dopo aver schivato una quinta
    laserata, lo lasciò andare. Un arco di luce partì dalla punta della katana, mentre volava parallelamente al terreno,
    colpendo tutte le torrette. Queste esplosero in una nube di Voxels di svariati colori. Dati non eliminati dalla
    cache. Corse attraverso i cubetti colorati, i quali scomparivano mentre il sistema eliminava i dati inutili dalla
    RAM, lungo il collo, arrivando agli occhi dell'avatar della guardia.

    Con la katana tagliò uno degli occhi e poi, senza pensarci un secondo, si tuffò all'interno. Attraversò le mesh
    trasparenti, visto che il lato posteriore (ovvero quello che non poteva venir visto perchè all'interno) non veniva
    renderizzato per il Back-Culling. Roba tecnica. Arrivò al centro. Lì la scena era differente. Non più un
    combattimento tra due personaggi usciti da un film fantasy o da un videogame. All'interno era presente un nucleo
    cubico formato da tanti altri piccoli cubi. Luci psichedeliche. Pensava di fermarla confondendola? Aveva giocato a
    REZ. Con il Trance Vibrator, cazzo. Quello non l'avrebbe fermata neanche se pagava.

    Lo spazio di distorse, come se all'interno della testa ci fosse stato più spazio che in tutta l'\emph{arena}. Il
    cubo si fece sempre più distante. Era il kernel del sistema di protezione. Spaccava quello e poi poteva ravanare
    quanto voleva all'interno del cervello del tizio. Si fece comparire dei thrusters dietro la schiena e, all'urlo di
    ``Ok, coso. Ora di farla finita.'' si catapultò sul cubo, tagliandolo a metà.

  \section*{Sofia}

    Non era passato neanche un minuto da quando si era connessa. Il cavo si riavvolse all'interno della sua nicchia nel
    collo, mentre la guardia si accasciò a terra. Aveva eliminato tutti i suoi ricordi, impiantando delle altre
    informazioni, in modo che, una volta sveglio, non sospettasse. Aveva anche dato una giustificazione per il collega,
    a terra.

    ``Fiu.'' disse la ragazza, facendo a finta di asciugarsi il sudore dalla fronte. Zoè la stava guardando,
    esterrefatta. ``Che hai fatto?'' le chiese, guardando l'uomo a terra ``Oh, niente, ho trappolato con la sua testa.
    Cose sgaggie.'' ``Mh.'' annuì lei, per poi tornare a guardarla ``Sei venuta a salvarmi? Ti manda Arsenicos?'' ``Sono
    venuta a fare una cosa per salvarti, sì. Mi manda lui ma, soprattutto, Valentine.'' ``La nevrotica?'' ``Hem... Sì,
    lei.'' ``Allora andiamo?'' chiese lei, senza troppa inflessione ``Mh, no.'' ``Come no? Non dovevi salvarmi?'' ``Con
    tutte le guardie che ci sono in giro non ce la farei mai. Valentine ha un piano per salvarti, però prima è
    necessario che io faccia il dump del tuo cervello dentro di...'' Sofia fece scivolare via la spallina sinistra dello
    zaino e lo portò davanti. Aprì lo scompartimento centrale tirando la zip. ``...questo.'' e mostrò il cervello dentro
    la scatola di trasporto autoalimentata.

    ``Eh? E poi?'' chiese Zoè, che pareva solo un po' preoccupata ``Ah, non lo so. Secondo me non serve, però è una
    precauzione che vuole prendere Valentine. Devo fare il dump ed aggiungere un demone che continua ad uploadare ciò
    che registra il tuo cervello in uno spazio nell'Internet. Grazie al demone, in caso qualcosa andasse storto,
    possiamo far passare la tua coscienza in questo cervello.'' e battè la mano sul componente che aveva nello zaino ``E
    poi fare il merge con le informazioni. Ma sono sicura che non servirà.'' ``Capito. Allora facciamolo. Visto che
    Arsenicos sembra fidarsi di voi io mi fiderò di voi.''

    ``Bella scelta.''

  \section*{Valentine}
    
    Valentine montò in macchina. Accarezzò il volante. Accese il motore e, una volta caricata la mappa interattiva della
    città nel computer di bordo, iniziò a cercare un posto utile. Aveva bisogno di una posizione ad ovest. La giornata
    stava per finire ed aveva bisogno di un posto con il sole alle spalle. Una delle prime regole per gli snipers. Trovò
    una torre che faceva al caso suo, lontana due chilometri dal punto dove si trovava il Bazaar. Da lì avrebbe potuto
    sparare indisturbata ai Frames. Soprattutto a quello di Nicolas.

    Iniziò a guidare. Fece un paio di curve poi notò che le macchine che erano dietro di lei non erano lì per caso, la
    stavano pedinando. Dovevano aver seguito qualche comunicazione. Quelle chiamate con il Pivello e Catherine erano
    state un po' rischiose, in effetti.

    guardò l'orologio. Aveva un po' di tempo. Questo significava che si sarebbe divertita un po' con il gruppo di idioti
    che stava provando a seguirla.

    Iniziò a prendere bivi a caso, tagliò rossi, schivò il traffico in contromano, entrò per sensi unici. Alcuni avevano
    desistito, mentre altre quattro macchine, quelle guidate dai più tenaci, le erano ancora dietro. A quel punto decise
    che era ora di scrollarseli definitivamente di dosso. Entrò nell'autostrada il prima possibile.

    Poteva sembrare una scelta idiota, anche perchè come fai a pedinare della gente su uno stradone a quattro corsie
    durante il momento della giornata nella quale la strada non è ancora piena di lavoratori che tornano a casa?
    Valentine sorrise pensando a quel tipo di ragionamenti. Sorrise soprattutto pensando ai ragionamenti che stavano
    facendo quelli che la pedinavano.

    Fece scivolare la mano destra sul volante, fino a raggiungere con il pollice uno dei quattro ponti che collegano la
    ruota del volante alla parte centrale. Lì era presente un copri tasto rosso. Era una delle pochissime cose che
    stonavano con la macchina, ma doveva essere sicura di ricordarsi cosa ci fosse lì dentro. Con uno scatto del pollice
    sollevò il coperchio, scoprendo un pulsante rettangolare. Ci fece scivolare sopra il pollice e poi disse, con uno
    brillio di lucida follia negli occhi ``Ok, stronzi, vediamo come reagite a questo.'' e premette con tutta la forza
    che aveva il tasto, come se servisse ad avere un risultato migliore.

    Il bagagliaio della macchina si aprì leggermente, solo per lasciare spazio alla macchina per cambiare geometria, la
    quale fece uscire due thrusters. Questi iniziarono a girare e, in men che non si dica, iniziarono a spingere la
    macchina. Valentine si sentì immediatamente schiacciata contro il sedile, le sembrava che lo spazio si stesse
    schiacciando di fronte a se. Altri micro-thrusters comparirono sui lati della macchina, per permetterle di
    controllare meglio la traiettoria.
    
    Le macchine provarono a starle dietro, spingendo al massimo, ma dovettero abbandonare dopo poco tempo, quando lei
    riuscì a schivare all'ultimo momento una macchina che stava superando un camion. Valentine schizzò lungo
    l'autostrada per una decina di kilometri, fino a che non arrivò nelle prossimità della torre che le serviva. A quel
    punto disattivò i thrusters, i quali rallentarono la macchina a velocità accettabili e poi si rimisero all'interno
    del bagagliaio. Valentine era soddisfatta. Se c'erano due cose che la eccitavano più del sesso erano guidare quella
    macchina ed i fucili di precisione. La storia dei fucili di precisione era un po' particolare. Diciamo che le
    permettevano di vincere la guerra senza scendere in battaglia. Certo, probabilmente quello non era il modo predicato
    da Sun Tzu, ma funzionava davvero bene.

    Uscì dall'autostrada e si diresse verso il palazzo. Ci mise un po' per trovare un parcheggio gratuito, poi recuperò
    il fucile, chiuse la macchina e s'incamminò.

  \section*{Arsenicos}

    Arsenicos stava correndo sul tetto di un palazzo. Arrivato a cornicione spiccò il salto più lungo che potè. A metà
    del volo, subito dopo che la forza di gravità riuscì a contrastare la sua spinta, mugugnò una sola parola
    ``MAVAFFANCULO.'' Volò dall'altra parte della grata che dava sulla discarica della base e finì dentro un cumulo di
    ciarpame. Alterato come non mai sbucò dai rifiuti.

    Come cazzo si erano permessi? Per quanto a lui sarebbe piaciuto, per quanto lo stava aspettando la signorina Arthur
    non avrebbe iniziato a chiamarlo per nome durante quella missione neanche a pregarla. Qualcuno doveva aver
    intercettato le comunicazionii e si doveva essere finto da signorina Arthur. Si mosse di gran passo verso
    le scale in metallo. Una volta arrivato lì iniziò a salirle con passo felpato per non fare rumore. Arrivò alla
    porta. Guardò attraverso 
