\chapter{You can't live on love alone}

  \section*{Valentine}

    ``Guardate quà, porca puttana.'' Valentine era incazzata come non mai. Lei, il Pivello e Sofia erano nella base. Sul
    monitor da 42 pollici erano presenti varie tabelle dati ed un grafico a torta. ``Due caricatori di proiettili
    APFSDS-EX (Armour Piercing Fin Stabilysed Discarding Sabot - Explosive) 55.6x450mm, costo finale 15000 Creds, una
    granata HEAT (High Explosive Anti Tank) da 400mm, 680\$'' continuò lei, indicando delle linee nelle tabelle ``Sono
    spese standard. Ho sparato quei colpi per proteggerla.'' tentò di difendersi il pilota del Frame.
    
    ``Spese standard?'' ripose lei, infuriata ``Hai scaricato tutti i proiettili del fucile d'assalto per coprirmi da
    dei soldati a piedi. Ne bastano venti per una cosa simile! Ma mettiamo anche che non sia un problema. Vogliamo
    parlare della quantità di danni al Frame? Non so quanto ci vorrà per rimettere tutto a posto. In pratica con questa
    missione siamo arrivati a stento a guadagnarci un'altra settimana di autonomia.''

    Durante quella discussione Sofia era stata in silenzio ad ascoltare il discorso, divertita. Alla fine non riuscì a
    trattenere una risata. Valentine rivolse lo sguardo verso di lei e le fece ``Non c'è molto da ridere, Sofia.'' e
    quindi aprì un documento che le era arrivato via mail ``Questo è il costo di riparazione del serbatoio idraulico da
    parte della Ville-Lumière Holdings.'' ``Ma, quello serviva per disattivare le tute termo-ottiche dei Noobs. Com'è
    che ci hanno fatto pagare anche quello?'' ``La nostra assicurazione copre soltanto la distruzione di sulo pubblico
    quando è strettamente necessario alla missione, infatti non ci hanno fatto pagare gran parte dei buchi nei palazzi e
    vetri distrutti nel raggio di un campo da calcio.'' e guardò male il Pivello ``Hanno decretato che il danno al
    serbatoio poteva essere evitato.'' ``E come?'' chiese, genuinamente, Sofia. Davvero non riusciva a capire che
    avessero in testa quelli delle aziende d'assicurazione.

    ``Con delle granate EMP, ad esempio.'' rispose Valentine, secca ``Ma.'' iniziò Sofia, stupida ``Ma sei matta? Cosa
    vuoi che faccia? Che rischi di friggermi il cervello perchè dobbiamo usare delle granate che distruggono la
    componentistica elettronica?'' ``Pensavo che la tua componentistica fosse isolata contro le scariche
    elettromagnetiche.'' ``Sì, certo, ma c'è sempre una soglia di sopportazione in queste cose, Vale. Una scarica troppo
    potente può comunque attraversare il rivestimento.'' ``Beh, ormai dovresti esserti allenata con quelle granate per
    sapere quale distanza puoi sopportare, la killing zone e tutti gli altri dati rilevanti.'' ``Questo discorso non ha
    senso. È come se stessi chiedendo ad una persona con problemi al sistema immunitario di usare granate radioattive!''

  \section*{Catherine}

    Catherine, la sera della missione, era rimasta alla base finchè, ad un certo punto non aveva ricevuto una chiamata
    da parte del Boss di andare a recuperare ATHENA ed un altro Frame. Al tempo non si era capacitata di come fosse
    possibile che l'Armoured Frame del team fosse stato danneggiato così tanto durante una missione di scorta. Capiva
    fosse stato un capo di una Corp importante o un ricercatore nel campo degli Armoured Frames, della nanotecnologia o
    dello sviluppo di sistemi di criptazione.

    Durante le due guerre aveva dovuto tenere al sicuro i ricercatori Europei da spie che volevano eliminarli o
    portarseli a casa per estrarre più conoscenze possibili prima di metterli a lavorare ai loro progetti, se
    collaboravano, o trovare loro un posto in una bara di legno sperduta in qualche punto della Panamerica o della
    Gloriosa Autarchia Capitalista Asiatica. Era necessario avere più ricercatori possibili al tempo. Soprattutto di
    quei tre tipi, visto che erano le tecnologie Bleeding Edge. Gli Armoured Frames, ovviamente, erano l'arma
    definitiva. Qualche pazzo, alla fine della Terza Guerra Mondiale, aveva deciso di montare dei lanciamissili
    termonucleari di Media Gittata su un Armoured Frame progettato apposta per lanciare missili da qualunque parte del
    globo, bypassando così definitivamente lo scudo spaziale. ``Un tributo'' lo chiamò. Incredibile.

    I ricercatori di nanotecnologie furono ignorati per tutta la terza Guerra poi, quando il mondò notò che l'Europa e
    la Russia riuscirono a recuperarsi dai danni causati dagli attacchi nucleari che fecero finire la Guerra, danni che
    obbligarono le altre nazioni a spostare le proprie città ed a rinchiuderle in gusci per proteggerle dalle
    radiazioni, capirono. L'Europa aveva formato un team di ricerca nanotecnologica in tandem con la Russia. Avevano
    capito che la Guerra sarebbe andata a finire come la Seconda, ma più su vasta scala. Fu così che decidero di
    sviluppare delle nanomacchine che venivano disperse nell'ambiente come sabbia le quali potevano eliminare qualunque
    traccia di polvere radioattiva nel raggio di centinaia di chilometri. Riuscivano anche a penetrare nei polmoni degli
    esseri umani e ripulirli dalle radiazioni, togliendo il problema delle malattie e delle mutazioni. Ovviamente prima
    venivano disperse meglio era. In alcune parti d'Europa si riuscì a disperderle in anticipo rispetto agli attacchi,
    non permettendo di espandersi. Fu così che, assieme all'evacuazione preventiva delle città, si riuscì a prevenire la
    creazione di una nuclear Wasteland in Europa. Molte delle capitali furono distrutte quasi completamente ma almeno la
    gente era viva. Tra l'altro l'Europa, preoccupata per la condizione della Foresta Amazzonica, spinse sulle
    tecnologie ambientali trasformando l'Europa, le terre intorno al Mediterraneo e, dove possibile, la Russia nel nuovo
    polmone verde del mondo. La Quarta guerra mondiale fu portata avanti senza armi nucleari, tutte smantellate durante
    una fase di transizione e, appunto, scaturì perchè Europa e Russia non volevano fornire le tecnologie necessarie ad
    Asia ed America per ripulire i loro continenti. Avevano paura che, una volta posti allo stesso livello, sarebbero
    stati conquistati in pochissimo tempo. La Quarta non durò molto, visto quuanto erano indeboliti i due imperi e visti
    i dissapori tra di loro.

    Per quanto riguarda la tecnologia di criptazione la storia era facile. Con l'avvento dei nuovi processori i vecchi
    sistemi di sicurezza erano utili come un foglio di carta contro un proiettile da 9mm. E in guerra chi controlla le
    informazioni controlla la battaglia.

    Catherine era passata attraverso tutte e due le ultime Guerre. Le cicatrici sul suo corpo erano delle testimonianze
    indelebili. Aveva dovuto farsi curare per intossicazione da radiazioni. Ne era uscita con una pacca sulla spalla e
    basta. Niente pensione, niente onorificenze, niente premi. Lei era una delle migliori, se non la migliore meccanica
    del Reparto Meccanizzato Morrigan, il gurppo di Armoured Frames del Regno Franco-Britannico, nato dopo la crisi
    economica d'inizio secolo. Erano le SAS degli Armoured Frames. Lei non era portata per pilotare gli Armoured Frames
    ma più volte c'è bisogno che i meccanici scendano in battaglia con i piloti durante missioni sotto copertura, perciò
    aveva ricevuto un addestramento che non aveva niente da invidiare alle vere SAS. Era diventata esperta nell'utilizzo
    di fucili da combattimento oppure di armi Anti-Tank/Anti-Frame. Dopo la fine delle guerre e lo smantellamento degli
    eserciti per la creazione di gruppi privati al soldo delle Corp Catherine si era ritrovata senza lavoro.

    Una settimana dopo il suo congendo, però, qualcuno suonò alla porta di casa sua. Era Valentine Arthur. La conosceva
    dai giorni della Quarta Guerra. Lei faceva parte del dipartimento d'intelligence della marina del Regno. Aveva
    partecipato solo all'ultima perchè era troppo giovane per la Terza. Al tempo Valentine aveva ancora gli occhi di
    qualcuno che stava combattendo per ciò che era giusto e non per i soldi. Era per quello che aveva deciso di creare
    un gruppo di agenti esterni per Corps che non accettasse i lavori più sporchi. In questi anni il Boss (Catherine
    chiamava così Valentine perchè, non importava che lei fosse più vecchia, la ragazza aveva più il senso degli affari
    ed era quella che faceva galleggiare la baracca) provava in tutti i modi a mantenere quello sguardo, ma quando
    arriva la fine del mese i conti vanno fatti quadrare. Non che avesse iniziato ad accettari lavori sporchi, ma aveva
    iniziato a lamentarsi sempre di più degli sprechi ed aveva perso un po' di vista il combattere per la giustizia.

    Catherine si trovava nella sezione principale della base, la quale era un Magazzino ai limiti del centro di Parigi
    (ricostruita dopo i danni della Teza Guerra. Almeno gli Europei hanno avuto il buon senso di non cambiare nome, tipo
    New New York City o, God-Save-Us, Neo Tokyo), grande abbastanza per contenere ATHENA e tutta la strumentazione per
    lavorarci, ed aveva un soppalco da due piani contenente gli uffici, una cucina, un bagno con doccia e la stanza per
    lei e Sofia. Stava finendo di smontare i pezzi del Frame che Arsenicos aveva neutralizzato durante l'ultima
    missione. Doveva dire che, per quando avesse distrutto parte del proprio veicolo, aveva fatto un buon lavoro. La
    componentistica dell'Armoured Frame d'assalto del nemico era tutta a livello militare.

    Però adesso doveva prendersi una pausa. Era stata là gran parte della giornata. Tra l'altro oggi avrebbero dovuto
    fare il debrief completo per la missione. Sarebbe stato un massacro. Era ovvio che Valentine se la sarebbe presa con
    i ragazzi. E comunque lei aveva bisogno di un drink. Lavorare tutto il giorno con delle macchine non è il massimo.
    Lei non era una di quelle persone che sta nella propria officina e trattava le macchine come amici e ci parlava.
    Quelli erano matti. Lei era un tecnico. Una macchina non va? Ottimo, la mettiamo a posto. La cosa assurda era che,
    da quando giravano i sistemi di cura a nanomacchine e la componentistica cybernetica, lei era capace anche di curare
    la gente. Ovviamente non riusciva a fare un'operazione a cuore aperto, ma se si trattava di controllare un'iniezione
    di macchine proteiche o di fare controlli diagnostici e riparazioni di protesi cybernetiche non c'erano problemi.
    Alla fine era come mettere mano ad un Armoured Frame, solo dieci volte più piccolo. O di più, se si trattava di
    Frames non urbani.

    Passò le mani dentro nel sistema di pulizia montato su una delle colonne che sorreggevano la base e quindi si
    diresse su per le scale che portavano agli uffici. Sentì gli urli del Boss fino da metà della scalinata. Sembrava
    che stesse litigando con Sofia a proposito dell'uso di granate ad impulsi elettromagnetici. Aprì la porta e sentì
    BJ finire una frase con ``...granate radioattive!'' Valentine si girò a guardare chi fosse entrato dalla porta e,
    appena riconobbe Catherine, fece ``Oh, Catherine.'' ``Sono venuta per il debriefing.'' rispose lei, dirigendosi
    verso la cucina. Arrivò alla credenza con i bicchieri e prese uno dei suoi Old Fashioned tumblers. ``Allora, di che
    stiamo parlando?'' chiese, andando verso il frigorifero ``Delle spese inutili che questi due hanno causato
    nell'ultima missione.'' rispose Valentine, irritata. Catherine versò dei cubetti di ghiaccio nel bicchiere usando il
    distributore integrato nel frigo. ``Boss,'' si girò per andare all'armadietto degli alcoolici ``sei troppo dura con
    la ragazza ed il Pivello.'' dopo aver scorso con lo sguardo tutte le sue bottiglie decise per andare con il
    Lagavulin 21. Ne versò abbastanza per riempire a metà il bicchiere e poi lo ripose. ``Non sono troppo dura. Se ci
    fosse ancora Nicolas.''

    ``Se.'' fece Catherine, buttandosi sulla poltrona libera intorno al tavolo per il briefing ``Non ti avesse lasciata
    e poi non se ne fosse andato dal gruppo attratto dai soldi offertigli da un gruppo di mercenari pronti a tutto...
    Sì, forse non avremmo tutte queste spese.'' Valentine sembrava non saper cosa rispondere. ``Boss'' continò il
    meccanico, dopo aver preso un sorso di Scotch ``Non puoi chiedere ad un pilota appena uscito dall'accademia ed una
    hacker che non ha nessun addestramento militare di comportarsi durante una situazione come quella come se fossero
    dei veterani. Abbiamo portato a termine la missione senza nessun ferito, no? Penso che questo basti. Inoltre siamo
    riusciti a consegnare un tizio che stava provando a vendere informazioni sui sistemi di pulizia delle polveri
    radioattive a delle spie Americane.'' ``Sì, ma ci abbiamo quasi perso, Catherine.'' ``Beh, oddio. Guardiamo come
    stanno le cose. Il Pivello è riuscito a neutralizzare un Frame senza distruggerlo, il che significa nuovi pezzi
    gratis, inoltre tu te ne sei tornata alla base con quattro tute Termo-ottiche e svariate armi nuove, per non parlare
    delle munizioni.''

    Arsenicos aveva la faccia di chi era stato appena coperto di complimenti, mentre Sofia se la rideva sotto i baffi.
    Catherine, quindi, puntò l'indice destro, mentre con le atre dita teneva il bicchiere dall'alto, nella loro
    direzione. ``Non pensiate che non abbia nulla da dirvi.'' fece lei, seria ``Pivello. Che diavolo era quella
    tattica?'' ``Quale tattica, signora Mac an Bahird?'' chiese lui, stupito ``Ecco, appunto. Ho guardato i video feed
    di ATHENA del combattimento. Che intenzione avevi saltando sul tetto della casa mentre un Frame d'assalto stava
    arrivando con i Boosters al massimo?'' ``Farlo deviare come ho fatto?'' ``E se non fossi riuscito a colpire e
    distruggere il thruster laterale?'' ``Mi avrebbe colpito, immagino.'' ``Esatto. Quella tattica non aveva nessun
    senso. Dovevi aspettare che passasse. E vogliamo parlare di quando l'ha caricato a testa bassa?'' sembrava che
    Arsenicos non ne avesse tanta voglia. Sapeva di aver fatto una scelta sbagliata. ``Non lo sto dicendo perchè non mi
    piace il metodo, capisci? Se è un braccio distrutto lo possiamo riparare, ma se quel pistone da demolizione ti
    avesse colpito avrei dovuto raccogliere te col cucchiaino, invece che il braccio. Probabilmente potevi raggiungere
    lo stesso risultato sparando con Gae Bolg dalla posizione dove ti trovavi.''

    ``E, Sofia?'' Sofia saltò sul posto posizionandosi con la schieda dritta sul divano dove si trovava ``Nya?'' si fece
    sfuggire ``Devo ammettere che capisco le tue paure, ma il Boss ha ragione. Non è che non usiamo le granate perchè
    possono ucciderci, ma ogni tanto servono. È per quello che ci alleniamo ad usare armi esplosive, così che possiamo
    danneggiare il nemico senza farci del male. Hai rischiato che l'acqua non disattivasse tutte le tute. Pensi che
    saresti quì senza danni se fosse successo?''

    Catherine non sopportava fare quei discorsi. Non era il suo compito. Era quello del Boss, ma Valentine stava
    passando un brutto momento a causa di Nicolas. Però non poteva rischiare di mandare tutto all'aria. Il loro business
    era qualcosa di molto vicino all'esercito, alla fine, magari senza tutte le storie di salutarsi, di tagliarsi i
    capelli e d'indossare delle tenute ridicole durante le parate. Ma le parti di addestramento ed il rischiare la vita
    ogni giorno c'erano. Oh, aspetta. Allora, forse, doveva dire che era più simile a lavorare per le vecchie agenzie
    d'intelligence nazionali.

    ``Mh.'' mugugnò quindi il meccanico, per poi bere un altro sorso di Scotch ``Ok, Boss, scusa. Non dovrei
    intromettermi così. È solo che una volta non erano i soldi la parte importante di questo lavoro. Devo ammettere che
    mi fa un po' male vederti così.'' ``Catherine. Ne parliamo davanti ad un bicchiere dopo, ok?'' ``Ok.'' rispose lei,
    alzando le mani, per poi apposggiarsi sullo schienale della propria poltrona.

  \section*{Valentine}

    Il discorso di Catherine l'aveva riportata alla realtà. A guardare bene la missione era andata a buon fine, magari
    non come Mr. Fabulous avrebbe voluto, ma sicuramente era stata impartita la giustizia. Inoltre erano riusciti a
    recuperare dei gingilli piuttosto interessanti. Probabilmente voleva solo dimostrare a Nicolas che riuscivano a
    lavorare come, se non meglio, quando lui era con loro. Doveva smetterla di pensarla così, ma veniva in automatico.
    Era più forte di lei.

    ``Va bene, ammetto che sono stata esagerata con voi due. Il problema è che la prossima volta potremmo non essere
    così fortunati. State più attenti e vedete di rinforzare le tattiche, ok?'' fece lei, leggermente riluttante
    all'inizio, rivolgendosi ai due. Il volto di Arsenicos si illuminò, mentre la ragazza si rilassò sulla sedia. ``Sì
    signora, signorina Arthur.'' rispose, prontamente, il pilota ``All right.'' aggiunse Sofia, risollevata.

    ``Bene, visto che avete capito, adesso vorrei parlarvi dell prossimo lavoro.'' fece Valentine, aprendo un nuovo file
    sul suo PC. Sul proiettore comparirono una serie di foto, mappe, planimetrie e strutture di sistemi di sicurezza.
    ``So che abbiamo portato a termine la missione per Mr. Fabulous da poco tempo, ma ci è arrivata una richiesta per un
    salvataggio. La paga è buona e le motivazioni sembrano solide.''

    Le missioni di salvataggio erano quelle più interessanti per il gruppo. Normalmente significava salvare qualche
    persona dalle mani di rapitori, oppure da delle Corp rivali. Questo significava che non c'era il rischio di
    backfire, com'era successo con Mr. Fabulous, o di aiutare la parte sbagliata. Non che accettassero una missione
    senza prima aver fatto del backgound check ma, normalmente, era meno pericoloso accettare missioni del genere che
    non, diciamo, quelle di Recupero Materiale. Il termine \emph{Recupero Materiale} era un modo alternativo per
    chiamare i furti che, ogni tanto, il gruppo portava a termine per recuperare materiale troppo pericoloso per venir
    lasciato in mano a chi lo possedeva, per poi consegnarlo a centri di ricerca.

    ``Allora,'' iniziò Valentine, portando in primo piano la foto di una ragazzina

  \section*{Sofia}

    Ninja.

    Ecco cosa sarebbe stato figo in quel momento. Proprio come quando giochi a Ninja Gaiden. Sofia, dopo la ramanzina
    che aveva ricevuto, si era persa al solito nei suoi pensieri. Così s'immaginò un attacco da parte di ninja durante
    il briefing. Tracciò nella sua mente la traiettoria che le scheggie di vetro generate dallo sfondamento delle
    finestre avrebbero povuto prendere. S'immaginò mentre le schivava e saltava dietro il divano per proteggersi dagli
    attacchi, mentre gli altri del gruppo estraevano delle armi. Catherine un fucile da combattimento Jackhammer,
    Valentine la sua pistola uscita da Metal Gear Solid e il Pivello...

    Mh, il Pivello probabilmente si sarebbe messo ad urlare qualcosa come <<GIUSTIZIA!>> ed avrebbe lanciato il
    tavolino. Sìsì. Quella era il modo più coreografico.

    ``...in pratica il piano è quello di disabilitare il sistema di sicurezza per accedere ai piani superiori. Per fare
    questo dovremmo far infiltrare Sofia...'' sentì in lontananza Vale parlare. Sofia fece un cenno con la testa, senza
    distogliere l'attenzione dalla coreografia della battaglia con i ninja.

    Una volta dietro il divano Sofia avrebbe raggiunto il piede di porco che aveva attaccato sotto il mobile (messo lì
    proprio in caso di situazioni simili) e sarebbe strisciata lungo di esso per cambiare posizione. Intanto dall'altra
    parte Catherine ne stava ammazzando a fiotti, lanciandonepure uno giù dalla finestra con un calcio. Vale avrebbe
    usato quelle sue tecniche strane per metterne in leva uno ed usarlo come scudo, uccidendone altri due con dei colpi
    piazzati in fronte.

    OH WAIT. Ma i ninja sanno parare i proiettili con i loro tanto. Mh. Ah, ok. Allora facciamo così. Hai presente come
    Vale porti sempre con se tre caricatori sulla fondina che ha sulla coscia sinistra? Probabilmente uno dei tre
    contiene proiettili APEX (Armour Piercing - Explosive), solo che non li usa mai perchè sono costosi. Eietterebbe il caricatore nella sua arma
    mentre schiva degli attacchi con una piroetta a scendere e poi, una volta arrivata in ginocchio, estrarrebbe il
    caricatore APEX. Però le starebbe per arrivar un colpo dall'alto da parte di un ninja che non aspetta che le
    coreografie siano complete. Perciò Valentine farebbe una capriola all'indietro, schivando il fendente, non lasciando il caricatore e
    tirando un calcio in faccia al ninja. Il ninja volerebbe indietro con tutte quelle movenze da cattivo dei Super
    Sentai, senza però perdere la guarda.

    Ci sarebbe un effetto di rallentamento del tempo, durante il quale Valentine si rialza in piedi (probabilmente
    telecamera rotante alla Michael Bay, famoso regista di film d'azione d'inizio secolo) e punta l'arma verso il ninja
    in una posa figa. Tipo gambe leggermente aperte e braccio esteso. Il ninja si metterebbe in posizione difensiva,
    sapendo di poter parare il colpo. Valentine premerebbe il grilletto, facendo partire uno dei proiettili perforanti.
    Il ninja sposterebbe la spada per tagliare il proiettile ma questo, a contatto con la lama, salterebbe in aria,
    lasciando il nemico distratto abbastanza per venir colpito da un secondo colpo sparato dall'arma della donna.

    ``Figo.'' sussurrò Sofia, ancora immersa nei suoi pensieri.

  \section*{Arsenicos}

    La Signora Mac an Bahird aveva fatto dei complimenti ad Arsenicos. Almeno, questo era quello che aveva capito lui.
    Era da un bel po' che non riceveva dei commenti positivi dai membri del team. Era anche vero che non aveva fatto
    nelle altre missioni. Comunque adesso era il momento di seguire il briefing. I lavori di salvataggio sono quelli più
    difficili. Bisogna proteggere qualcuno che non sa combattere minimamente. Dopo averlo salavato da della gente che,
    probabilmente, potrebbe ucciderlo se il team fa un errore.
    
    ``Allora,'' inizio la signorina Arthur, portando in primo piano la foto di una ragazzina. A lato erano presenti i
    dati biometrici, oltre a dati più <<classici>> come altezza, peso, misure? Chi aveva compilato la scheda? Sofia?
    C'era comunque qualcosa di strano in quei dati. Per prima cosa non riuscì a vedere nessuna sequenza di DNA, anche se
    non se ne preoccupò particolarmente, visto che normalmente solo i DNA dei criminali venivano inseriti nei database
    per rintracciabilità futura. L'altra cosa che lo colpì fuorono gli occhi della ragazza. Erano perfetti ed avevano un
    pattern perfettamente simmetrico.
    
    ``in questa foto potete vedere Zoè Chevalier.'' continuò Valentine ``Lei è la figlia del CEO della Chevalier
    Technologies Unlimited, un'azienda di produzione di componentistica Cybernetica con base a Lione. A quanto mi è
    stato detto la ragazza è stata rapita meno di due giorni fa da parte di agenti della Missing Link a causa di una
    faida tra le due aziende. Non me l'hanno detto, ma potrebbe essere che le due aziende siano in gara per il rilascio
    di nuova componentistica e che quindi la Missing Link abbia deciso di rallentare la messa sul mercato del prodotto
    della Chevalier Ultd. utilizzando la ragazza come deterrente. Abbastanza stronzi.'' sì, di fatto ``Ma che ci volete
    fare? Per questi squali il Business è guerra, quindi penso che siamo fortunati se la troviamo senza un graffio, in
    realtà.'' 

    Valentine minimizzò la foto della ragazza e portò in primo piano una renderizzazione 3D di uno dei palazzi della
    Missing Link. Nella renderizzazione si potevano vedere i vari piani con le mura trasparenti ed un punto rosso
    pulsante, in uno degli ultimi piani, oltre ad una serie di linee, che potevano indicare delle vie di attacco o di
    fuga. Arsenicos non conosceva ancora bene i metodi per il briefing d'assalto del gruppo. ``Questa è una delle basi
    di ricerca della Missing Link a Londra. Secondo le informazioni che la Chevalier è riuscita a recuperare gli agenti
    della compagnia avversaria hanno portato la ragazza al 32° piano della struttura dentro una stanza di sicurezza. Non
    so perchè abbiano optato per un centro di ricerca, in realtà. Probabilmente perchè pensavano di confondere quelli
    della Chevalier.''

    ``Questa non è una cattiva notizia perchè significa che le difese saranno meno pesanti di una base operativa, in
    realtà. Fatto sta, comunque, che dovremo disabilitare i sistemi di sicurezza, perchè non abbiamo abbstanza personale
    per andare a distruggere tutte le torrette e le telecamere che sono presenti nel palazzo, per non parlare delle
    guardie che avranno aggiunto per proteggere la ragazza.'' Valentine guardò Sofia ``Abbiamo una persona in grado di
    disabilitare un sistema del genere in poco tempo, il problema è che, come ogni sistema di sicurezza è separato dalla
    linea esterna. In pratica il piano è quello di disabilitare il sistema di sicurezza per accedere ai piani superiori.
    Per fare questo dovremmo far infiltrare Sofia'' sofia sembrò annuire a questo piano ``in un modo o nell'altro e poi,
    una volta tolte di torno telecamere e chiusure elettroniche delle porte, andare a recuperare la ragazza mentre
    entriamo vestiti da addetti alle pulizie.''

    Arsenicos alzò la mano ``Sì, Pivello?'' ``Veramente mi chiamerei Arsenicos.'' ``Che c'entra?'' ``Ah, no. La domanda
    era un'altra: avremo bisogno dell'Armoured Frame per questa missione?'' ``No, non penso proprio. Per due motivi. Il
    primo è che dobbiamo infiltrarci in una costruzione ed usare un Frame non è proprio la cosa più silenzosa di questo
    mondo. Il secondo è che non penso sarà riparato per domani. Giusto, Catherine?'' ``Impossibile.'' rispose il
    meccanico, prontamente ``Anche attaccandoci direttamente le braccia del Frame d'assalto che abbiamo razziato non
    riusciremmo a calibrarlo in tempo.'' ``Eccoti la risposta. Altre domande prima che continui?'' ``Mh, no. Non
    penso...''

    Arsenicos si fermò a pensare. In realtà... ``In realtà,'' fece ``un'altra domanda ce l'avrei.'' ``Ah, sì? E quale
    sarebbe?'' ``Com'è che la Chevalier Technologies Ulimited, che è un'azienda di componentistica Cybernetica al di
    sotto solo di quelle italiane, ha bisogno di richiedere i servigi della Troubleshooters Sans Frontieres? Voglio
    dire, non per sminuire il nostro lavoro, ma sono sicuro che potrebbero permettersi delle agenzie di più alto
    calibro. Anzi, dovrebbero averne una ufficiale, addirittura.'' Valentine sembrò stupita dalla domanda, probabilmente
    era rimasta colpita da quanto era arguta. Era sicuramente così. ``Domanda interessante. In effetti ho fatto la
    stessa domanda a loro. A quanto pare non vogliono che la notizia si sparga. Utilizzare un'agenzia ufficiale è una
    cosa che non passa inosservata. Immagino che al <<paparino>> Chevalier importi del valore delle azioni della propria
    azienda quasi quanto la propria figlia, se non di più. Ed immagino che dimostrare che il signor Chevalier non riesca
    a proteggere la figlia potrebbe dare un colpo alla credibilità del gruppo. Se tu fossi un finanziatore pagheresti
    un'azienda che rischia di vedersi rubare i progetti da sotto il naso?'' ``No, ok, ma non so se è la stessa cosa.''
    ``Inoltre c'è sempre l'altra scusa. Se veniamo beccati possono dire che loro non c'entravano nulla e che non avevano
    mandato nessuno. Che noi eravamo solo dei ladri.'' ``Aspetti. Mi sta dicendo che ha accettato questo lavoro anche
    sapendo che potevamo diventare dei capri espiatori?''

    Arsenicos era particolarmente stupito dalla temerarietà della signorina Arthur ``Sì, perchè noi siamo abbastanza
    bravi per non farci beccare. Pivello, fatti crescere le palle, maledizione.'' ``Sì signora, Signorina Arthur! Non ho
    altre domande, per adesso.'' ``Ok, ottimo. Allora, continuiamo con il piano.''

    ``Prenderemo l'aereo Paris-London domani, arriveremo a Londra alle 12:45. Dopo aver messo le nostre cose in albergo
    andremo a compiere un primo sopralluogo. La sera stessa dovremo entrare in azione. Non possiamo perdere molto tempo.
    Inoltre stasera dovremo andare al Bazaar.''

    Il Bazaar era uno dei pochi mercati liberi rimasti sulla Terra. Quello di Parigi era uno dei più grandi, secondo
    solo a quello di Palermo, quello di Nuova Dheli e quello di Khan el-Khalili, che aveva perso la sua connotazione
    turistica dopo la ricostruzione alla fine della Terza Guerra Mondiale. I mercati liberi sono dei mercati che non
    sono controllati minimamente dalle Corps. Hanno delle leggi economiche completamente differenti, un sistema
    giuridico separato e, addirittura, degli eserciti personali. È per quello che nessuno si permetteva di disturbarli,
    anche perchè il flusso non controllato d'informazioni nel Bazaar faceva comodo a tutti, in realtà. Chi dava
    veramente soldi alle Corps, tra l'altro, non sarebbe mai andato a comprare là.

    Una delle leggi fondamentali del Bazaar era quella che non si potevano usare armi di nessun tipo all'interno dei
    suoi territori. Questo manteneva i conflitti all'esterno. E se qualcuno si azzardava ad andare contro quella legge
    le guardie in borghese sparse per tutto il suolo del Bazaar avrebbero fatto in modo di far scomparire la fonte di
    disturbo. Per sempre. Tra l'altro si dice che le <<guardie>> del Bazaar, in realtà, siano TUTTI coloro che vendono e
    che quindi non si può neppure provare a fare qualcosa di nascosto, visto che non c'è punto che non abbia almeno
    qualche mercante.

    Il team andava a comprare lì perchè le armi che avevano erano state altamente customizzate ed il loro creatore
    viveva all'interno di quella comunità. Inoltre non essere tracciabili era una cosa ottima per il gruppo, visto che
    spesso venivano contattati proprio per la loro anonimia.

    ``Quindi'' concluse Valentine ``Pivello, Sofia, studiatevi le planimetrie, i sistemi di sicurezza e le vie di fuga.
    Dovremmo entrare e non fare errori se vogliamo uscirne senza troppi casini.'' e lasciò il PC acceso in modo che
    potessero recuperarsi i dati per portarli sui propri laptops. ``Io e Catherine andremo al Bazaar.''

  \section*{Sofia}
    
    Sofia si lasciò sfuggire un risolino pensando alla scena d'azione con i ninja e poi sobbalzò sulla sedia. Non
    avevano mica capito che non stava seguendo, vero? Si guardò intorno solo per scoprire che era da sola sulla
    poltrona. Da sola mentre il Pivello stava leggendo delle cose sul suo Laptop. ``Hem. È... È interessante?'' chiese
    lei, sperando di non far capire che non aveva seguito una sola parola del briefing, in pratica. ``Mh?'' fece il
    ragazzo, alzando la testa dal portatile ``Oh, ah, sì Sofia. Certo potrebbe essere un po' dura per la tua parte.''
    ``Ah, capito.''

    Aspetta. Dura per la sua parte in che senso? Avrebbe dovuto saltare da un palazzo ad un altro sfondando i vetri e
    rischiando di trovarsi in mezzo a dei cyborg da combattimento? Avrebbe dovuto andare a letto con un vecchio porco
    solo per rubargli i dati del DNA? Avrebbe dovuto uploadare il suo conscio nell'Internet per poi scoprire che
    esisteva già un altro essere senziente? Combattere su un mecha mezzo organico contro degli esseri sovrannaturali
    attratti da un ragazzino giapponese? Andare nello spazio con un'intelligenza artificiale attaccata alla sua armatura
    per distruggere un'arma di annichilazione universale? Sarebbe stata messa in una bara per cinquecento anni solo per
    risvegliarsi senza ricordi, combattere contro Valentine in una guerra per il recupero di un gioiello. CHE ERA LEI?!?
    Aiutando la se stessa del passato, portata nel futuro da suo padre per risvegliare una divinità...

    ``Avrò bisogno di quattro pistole demoniache, Pivello!'' fece lei, serissima, al pilota ``E fai in modo che calzino
    in mio 40, ok?''

    Arsenicos alzò lo sguardo, con gli occhi fuori dalle orbite. Aveva la bocca aperta dallo stupore. ``Ah...'' tentò di
    dire. Pareva non sapere che cosa dire, se prendere la storia sul serio o meno. Lei era una hacker. Potevano essere
    dei termini specifici ``...in--in che senso?'' ``Hai capito benissimo.'' ``Ok.'' e si mise a lavorare sul portatile.

    Sofia ancora non ci poteva credere bene. Forse la storia di venir messa in una bara non era proprio quella più
    corretta. Decise così che avrebbe controllato sul PC di Vale. Probabilmente c'erano dei dati. Non le ci volle molto
    a scoprire che il piano, in realtà, prevedeva che lei s'infiltrasse all'interno del centro di ricerca, arrivasse
    fino al seminterrato 4, accedesse alla sala server e lì modificasse i permessi d'accesso alla struttura, facendoli
    rispondere ai campi bioelettrici generati dai corpi di Valentine e del Pivello. Cose fighe, insomma. Non sarà stata
    difficile come quello che credeva lei, ma comunque era tosta. Infiltrarsi lì dentro significava dover utilizzare i
    sistemi d'areazione oppure impersonarsi in qualche inserviente o cose del genere. Avrebbe preferito farlo
    dall'esterno, ma i sistemi di sicurezza sono isoltai dal mondo esterno, normalmente. E quando hanno connsessioni
    sono connessioni private One-Way-Only verso le sedi più vicine dei gruppi militari associati alla Corp, criptate con
    chiavi da 1024qbits o forse anche di più. Il discorso si stava facendo complicato e, spesso, anche per lei arrivava
    ad un punto dove non riusciva più a capire di che stesse parlando. Era comunque assodato che sarebbe dovuta arrivare
    fisicamente al centro di controllo per fare quello che Valentine le aveva chiesto.

    Due palle. E così niente Bayonetta neanche quella sera.

  \section*{Arsenicos}

    Non aveva capito. Minimamente.

    Che significava <<armi demoniache>>? Nel senso. Sarà stato di sicuro un termine tecnico da hacker. E Sofia sembrava
    seria, dannatamente seria. Voleva dire che ne aveva bisogno. Iniziò a guardare in giro.

    Fare una ricerca sull'Internet cercando <<armi demoniache>> faceva venire fuori un casino di cose strane. Ed erano
    tutte armi in corpo a corpo. Spadoni, asce a due mani, alabarde... Non sembravano le cose che si sarebbe portata
    dietro una come la ragazza. Ad un certo punto arrivò ad una pagina che parlava di demoni giapponesi. Ma, anche lì,
    le armi non sembravano proprio quelle giuste. Erano degli tetsubo: delle mazze da baseball in ferro gigantesche con
    gli spuntoni.

    Ok, però i demoni giapponesi avevano un altro nome. Oni. Guardò Sofia, mentre stava guardando un episodio di una
    serie animata giapponese. Probabilmente, visto che era una fan, voleva che cercasse il termine in giapponese. Provò
    con <<Oni Weapons>>. Avrebbe cercato il nome in giapponese se non avesse trovato nulla. Discussioni sulle armi dei
    demoni del folklore nipponico, una pagina su un vecchio videogame chiamato <<Halo>>. Ah, conosceva il gioco. Aveva
    provato l'ultima versione a casa di un amico. Però, comunque, non sembrava corretto. Poi, alla fine, arrivò sulla
    pagina dell'illuminazione.

    Era una pagina che parlava di un gioco in terza persona chiamato <<ONI>>. Aveva una serie di armi. Alcune
    impossibili, altre erano effettivamente realistiche. E il gioco sembrava abbastanza oscuro da venir apprezzato dalla
    Hipster dei videogames. Decise di scorrere meglio la lista delle armi. Forse era quello che intendeva. C'erano
    pistole, armi al plasma... Un fucile di precisione ad accelerazione magnetica che sparava proiettili al mercurio.

    Aspetta un attimo. Questa sembra una di quelle cose fatte a posta per una missione ad alto profilo. Iniziò a
    controllare in giro per la rete e per i siti ai quali aveva accesso grazie al suo passato all'accademia. C'erano
    notizie di una nuova tecnologia per le armi, in effetti. Erano dei fucili ad alta precisione con computer integrato
    per il calcolo della traiettoria, Satellite Uplink, mirino multi modalità 2.5-20x, sistema di fuoco a controllo
    remoto e stabilizzatore inerziale. Erano costruiti per sparare
    i colpi usando delle rotaie ad accelerazione magnetica, a differenza delle armi che usano la polvere da sparo o
    altri esplosivi solidi. I proiettili variavano da semplici paletti in acciaio a, effettivamente, colpi in mercurio
    raffreddati con serbatoi separati di azoto liquido. I proiettili venivano espulsi dalla canna a Mach 23, dando loro
    la capacità di perforare più o meno qualsiasi cosa.

    Tecnologia del futuro. Sul serio. Quell'arma era un mostro.

    A quanto pare la Carl Walther GmbH aveva prodotto centocinquanta unità di quest'arma, in quanto molto costosa e,
    soprattutto, ricolma di tecnologie troppo all'avanguardia per un uso standard. La storia del WA2100 ricordava
    moltissimo quella del suo predecessore, il WA2000, fucile di precisione così dannatamente ben costruito che poteva
    venir definito lo Stradivarius degli Sniper, anche quello costruito in circa 180 unità e poi abbandonato. Sembrava
    abbastanza oscuro, tecnologico e dannatamente letale da poter essere quello che aveva richiesto la ragazza.

    Ora, però, lo doveva recuperare. Quella era una di quelle armi che riescia trovare dai collezionisti, dai matti
    oppure...

    Arsenicos si mise a scrivere una mail ad uno dei suoi mentori.

    Dopo un'ora fece, rivolgendosi alla ragazza ``Ok! Allora domani verso sera dovremmo avere la cosa che chiedevi.''
    ``Ah?'' fece Sofia, distrattamente ``Ah! Ok, grazie.''

    Un peccato che ne avrebbe recuperata una sola e non quattro. Ma probabilmente avrebbe fatto abbastanza danni già
    così.

  \section*{Sofia}

    Di che stava parlando il Pivello?
