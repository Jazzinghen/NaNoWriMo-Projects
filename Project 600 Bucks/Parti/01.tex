\chapter{Mai speso così tanto per una missione}

  \section*{Valentine}
    
    Il lavoro era una cazzata, giusto?

    Valentine voleva ricontrollare bene i dettagli di quella faccenda.
    Quella era una missione di protezione come tante altre. Anzi.
    Probabilmente era ben più facile di molte altre missioni che lei, con
    il suo gruppo, avesse mai portato a termine. Fondamentalmente si
    trattava di scortare Mr. Fabulous, il CEO della \emph{Real Stuff},
    un'azienda di advertising,
    così si chiamavano le aziende che producevano lo spam che la gente si
    trovava nelle caselle di posta, nei cellulari, nei tablet e in
    qualunque altra cosa elettronica che uno potrebbe trovarsi in tasca.

    Mr. Fabulous. Lei non si ricordava molto bene come fosse il mondo
    quando ancora la gente non poteva cambiarsi il proprio cognome senza
    controlli (ora basta pagare una somma che la povera gente non può
    permettersi), era
    troppo piccola. Sicuramente non ti trovavi certi acquirenti che, per
    aumentare la dimensione per proprio E-Penis, andavano in giro con dei
    termini pacchiani invece che quelli lasciatigli dai propri genitori.
    E così ti trovavi a lavorare per Mr. Cool, il Signor Corleone oppure,
    com'è capitato una volta, Mr. Hammerhead.

    ``E Hammer non sono le mani.'' Doveva essere un meme di più di
    cinquant'anni prima. Lei non l'aveva capito, ma Sofia sì, quindi era
    qualcosa da nerd di sicuro.

    Valentine fuggì rapidamente dai suo pensieri, richiamata da uno stimolo
    esterno. Era il suo datore di lavoro temporaneo, che stava chiedendole
    qualcosa. ``È sicura che sia tutto a posto, signorina Arthur?''
    Guardò Fabulous, Mr. Fabulous Maximilian (O Fabulous Max per gli amici
    e per gli acquirenti), da dietro i suoi occhiali rettangolari con la
    montatura in alluminio nero, e sfoggiò uno di quei sorrisi di facciata
    che tanto piacevano agli uomini di potere. ``Non si preoccupi, Mr.
    Fabulous, vedrà che il viaggio andrà liscio come l'olio.'' ``Spero
    proprio.'' fece, arrogante, l'uomo ``Con tutto quello che vi sto
    pagando.''

    Valentine non fece nenanche una piega sentendo l'ultimo commento. Il
    bastardo. Era seduto assieme a lei nella sua Infinity Limo 2.0, mentre
    aveva un autista al volante. La gente con i soldi adesso aveva i
    sistemi di guida automatici. Come poteva uno che invadeva la privacy
    delle persone con le sue informazioni costruite su misura solo per
    vendere delle cianfrusaglie che non interessano a nessuno, permettersi
    di pagare un autista?

    Comunque non era quello di cui doveva preoccuparsi. Il lavoro del suo
    team era di scortare Mr. Fabulous per tutta la serata, mentre lui se ne
    andava ad una di quelle serate di gala di benificenza. Non vedeva l'ora
    che fosse finita. Odiava indossare i vestiti da sera: o era impossibile
    muoversi in maniera comoda oppure attiravi su di te tutti gli sguardi
    degli squali della finanza in cerca di un'avventura per fuggire dalla
    noia coniugale. Fatto sta che aveva scelto la versione scomoda, odiava
    i vecchi bavosi.

    Era seria, comunque, quando diceva che non c'era da preoccuparsi. A chi
    vuoi che gliene freghi che un esperto della produzione di
    \emph{advertising}? Al massimo qualche poraccio che ha deciso di usare
    un client di posta invece che il servizio da browser e che, quindi,
    deve aspettare il caricamento di tutte le immagini prima di poter
    eliminare le mail inutili. Che normalmente sono il $90\%$.

    Era così stamaledettamente sicura di quella cosa che aveva messo in
    standby metà del team e si era portata dietro solo ciò che faceva scena
    del suo gruppo: lei e l'Armoured Frame del team, il quale stava
    seguendo la macchina vicino. Ah, sì, c'era dentro anche il Pivello, ma
    non interessava a nessuno.

    La Limo era quasi arrivata a destinazione quando un esplosione un
    sibilo ed un esplosione la riportarono alla realtà. Sì, c'era anche
    quella parte della macchina capppottata e di Mr. Fabulous che le era
    stato proiettato addosso. Valentine mugugnò massaggiandosi la testa,
    dopo aver fatto rotolare via l'uomo ed essersi raddrizzata. Si trovava
    sul tettuccio della macchina, mentre l'autista era ancora legato a
    testa in giù. ``Ma cazzo!'' urlò, sbattendo il pugno sul finestrino
    antiproiettile. Mr. Fabulous sembrava svenuto. Un problema in meno.
    Odiava avere palle al piede in quelle situazioni. O mani morte, a
    giudicare dal personaggio. L'autista... Era ricoperto di scheggie
    metalliche su tutto il corpo e stava sanguinando copiosamente.
    Probabilmente era già morto.

    Ma non era quello il momento. Doveva uscire di lì prima che
    l'automobile saltasse in aria. E doveva farlo portandosi dietro
    centoventi chili di uomo. E prima ancora di quello...

    Appoggiò indice e medio della mano destre, unite, subito dietro
    l'orecchio destro. Era il modo per attivare l'auricolare del cellualre,
    stimolava direttamente il timpano, facendo in modo che nessuno potesse
    udire le conversazioni, a meno che non ascoltassero la persona che
    parla. Sofia diceva che era una citazione ad un vecchio gioco. Un
    classico. Storie di gente che mangiano serpenti. Abbastanza macabro, se
    chiedete a Valentine. ``Pivello'' disse lei, aspettando che il
    cellulare, una volta riconosciuto il comando, componesse il numero del
    pilota dell'Armoured Frame.

    ``Arsenicos online.'' rispose qualcuno dall'altra parte della linea
    ``Ascoltami Pivello.'' fece lei, tempestivamente, non aveva tempo da
    perdere ``Veramente mi chiamerei Arsenicos.'' ``Non me ne frega nulla,
    ok? Senti, non...'' ``Sì, lo so cosa vuole dire, Signorina Arthur.
    \emph{Non morire}, giusto? Non si preoccupi. Vado e torno!'' ``No,
    asp...'' Valentine non riuscì a finire la frase che il Pivello aveva
    già messo giù.

    ``Non sprecare munizioni.'' mugugnò, ma non c'era nessuno ad
    ascoltarla, neanche Mr. Fabulous

  \section*{Sofia}

    ``Let's Rock, Baby!'' fece Sofia, schioccando le dita di entrambe le
    mani e mettendosi in una posizione d'effetto, con entrambe gli indici
    che puntavano al nulla.

    Era in vacanza. Niente missioni quella sera. Valentine le aveva detto
    che potevano riposarsi, visto che il lavoro era semplice. E Valentine
    aveva naso per quelle cose.

    Sofia era riuscita a mettere le mani su un vecchio gioco, uno dei
    classici dell'inizio secolo e, finalmente, sarebbe riuscita a
    giocarci. La gente le diceva che era una hipster, ma non era vero. Non
    era mica colpa sua se i videogiochi erano più belli una volta. Da
    quando la Apple aveva acquisito la Nintendo, creando la Ringtendo, una delle
    corporazioni più \emph{Design-prima-della-funzionalità} driven (tanto
    che ormai non si sforzavano più neanche di nasconderlo: <<Comprate le
    nostre console, sono più belle delle altre, chi se ne frega del
    supporto full specturm? Think equal.>>), non si erano più visti giochi
    che non fossero da Casual Gamers.

    Casual Gamers. Ecco un altro termine per il quale era stata etichettata
    come Hipster. Ormai qualsiasi Noob pensava di potersi chiamare Hardcore
    Gamer solo perchè aveva preso tutte le stelline ad \emph{Angry Birds
    Galaxy 2} (Voto della critica pagata dalla Ringtendo: 15/10, perchè
    avevano finito i decimali per il 14). Non avevano capito nulla. Che
    fina avevano fatto gli Shooter Hell? Che fine avevano fatto gli RPG
    a la ``Dark Souls''? Che fine aveva fatto Kojima?

    Ah, no. Sapeva che fine avesse fatto il regista. Poco dopo la creazione
    della Ringtendo (Che, tra l'altro, è un portmanteau di \emph{Ringo},
    che è la traduzione in giapponese di Apple, e \emph{Nintendo}) Kojima
    ha richiesto che il suo cervello cybernetico venisse messo in standby
    ed alimentato finchè il mondo non fosse stato pronto per dei giochi
    seri. Sofia sperava veramente di essere lì in quel giorno. E allora gli
    ``Harcore'' Gamers avrebbero capito.

    Tra l'altro si sentiva anche abbastanza fiera. Lei era stata una delle
    prime ad utilizzare un cervello interamente cybernetico. Non che
    l'avesse veramente richiesto, in realtà. Da piccola aveva rischiato di
    morire a causa di un incidente e la \emph{Pisa Nanotechnologies
    Incorporated}, una Corp nata dall'unione della Scuola Superiore
    Sant'Anna, la Normale di Pisa e dai rimasugli pacifici del CERN, aveva
    bisogno di cavie per una loro nuova tecnologia. Una tecnologia
    rivoluzionaria. Erano finalmente riusciti a creare un cervello
    artificiale che potesse contenere tutte le informazioni di quello vero
    e, soprattutto, che fosse in grado di supportarne tutte le funzioni. Da
    quelle base come la vista a quelle più complesse come provare emozioni,
    avere intuizioni o \textbf{GASP} l'anima. Fatto sta che non potevano
    venderlo al pubblico senza testarlo e quale maniera migliore di andare
    a raccattare la gente che stava per morire che non aveva nulla da
    perdere?

    E voilà. Eccola con un cervello cybernetico formato da una quantità
    impressionante di microprocessori quantistici in parallelo ed un
    contratto con la Pisa Nanotechnologies che le garantisce upgrades
    ogni volta che esce una versione nuova ed ottimizzata sia di hardware
    che software. Ovviamente lei sta usando una versione OpenSource del
    sistema di controllo: ADA. Non le va di avere un qualche comando impiantato
    in una regione ad accesso limitato del cervello. Aveva dovuto lottare
    per un po' con la Corp, ma alla fine era riuscita a spuntarla spingendo
    sul piano spirituale del quale a lei fregava molto poco ma che toccava
    da vicino molta della gente a casa.

    Certo, nel primo periodo era stata un po' dura (per non parlare del
    rischio di non svegliarsi mai più finita l'operazione): oltre al
    cervello le erano stati sostituiti gli occhi ed il braccio sinistro,
    danneggiati durante l'incidente. Non solo non riusciva a controllare il
    braccio artificiale, distruggendo porte e mobili senza volerlo, ma nei
    primi stadi del sistema operativo il mondo le sembrava velocissimo.
    Mancavano ottimizzazioni. La connessione diretta ai computer, però, le
    permettevano di fare quello che faceva prima, ma molto, MOLTO più
    velocemente. Fu così che intervallò la sua vita al rallentatore con mad
    skillz al pc. Riusciva a scrivere programmi alla velocità del pensiero.
    Prima ancora che i Noobs potessero allungare le loro ditina da finti
    esperti alla tastiera lei aveva già compilato e debuggato un Code
    Breaker per AES-256. Non che fosse difficile con i processori
    quantistici farlo, ma sono comunque una bella quantità di righe di
    codice e, soprattutto, conoscere della teoria che ci sta dietro.

    Quella vita l'aveva portata non solo ad allenarsi nell'uso di protesi
    cybernetiche, ma anche a costruirsi una base tecnica solidissima. Non
    era un genio, era solo molto flessibile, imparava in fretta e,
    soprattutto, aveva l'hardware a supporto. Capì, comunque, una volta
    arrivata in università, che non era un vero genio. Quelli che erano dei
    geni li vedevi: prendevano il massimo dei voti quasi sempre e non
    dovevano sforzarsi. Lei aveva fatto fatica. Molta. Soprattutto perchè
    non aveva voglia di lavorare. Ora stava studiando per guadagnarsi il
    Master Degree, ma il modo di affrontare le cose era cambiato.

    Però adesso. ADESSO. Aveva recuperato le abilità motorie e di
    percezione della realtà di una persona normale, anzi, probabilmente lei
    aveva i riflessi migliori della maggior parte delle persone. E con quei
    riflessi avrebbe giocato a Bayonetta quella sera. Prese in mano la
    scatola in plastica verde trasparente del gioco. Sulla copertina,
    protetta da uno strato di plastica trasparente, oltre ai marchi della
    XBOX360, era raffigurata la protagonista, una ragazza con una bodysuit
    nera attillata e delle pistole in mano, in procinto di tirare un calcio
    caricato a chi stava ammirando la cover. Al piede aveva attaccata
    un'altra pistola. Una volta sapevano come fare i giochi. Sofia baciò la
    scatola ed iniziò a canticchiare ``Bayonetta, Bayonetta...'' mentre
    estraeva il disco per inserirlo nella console attaccata al monitor in
    Real Definition a pixel full Spectrum installato nella base del team.

    Inserito il disco la ragazza accese la console e corse al divano,
    pronta per godersi la serata di gioco. Una volta seduta sentì un trillo
    e le comparve una scritta a metà del suo campo visivo: CALL. La scritta
    era nera, circondata da un rettangolo rosso semitrasparente dagli
    angoli arrotondati. Sotto questa scritta si leggeva ``Caller: Vale''
    (Letto val, mi raccomando). Il tutto compariva con un effetto fade in e
    scompariva con l'effetto contrario. Si fermò per ammirare l'effetto.
    Unito al trillo formava una citazione perfetta. L'aveva programmato
    lei, così poteva dire di andare in giro sempre con un pezzetto di Metal
    Gear Solid con lei.

    Ma la gioia finì presto. Se Valentine la stava chiamando significava
    che avrebbe dovuto fare qualcosa. Odiava fare qualcosa quando c'era da
    giocare. Posizionò le dita dietro l'orecchio e disse ``Vale! Hei.''
    ``Sofia, non ho molto tempo.'' si sentì dire dall'altro lato della
    cornetta. Non era buono. Quando la chiamava Sofia e non BJ c'erano problemi. In sottofondo poteva sentire rumore d'incendi e passi di
    Armoured Frames. ``Ascolta, siamo stati attaccati da una forza
    sconosciuta. Ho bisogno d'informazioni. Soprattutto perchè diavolo ci
    hanno attaccato. Se questo stronzo è un criminale lo porto dai
    vigilantes.'' esplosione in sottofondo, pausa ``Per non parlare del
    Pivello. Non so quando ancora ci vorrà prima che parta con le sue
    solite puttanate.'' ``Ho capito. Mi ci metto subito.'' rispose,
    prontamente, la Hacker ``Ottimo. Ora vado. E la prossima volta che
    metto un vestito del genere ti autorizzo a darmi della rincoglionita.''
    ``Sicuro.''

    Sofia chiuse la comunicazione. Guardò il logo di Bayonetta sullo
    schermo, con dietro il simbolo della luna che ruotava e la scritta
    ``Press Start to Begin'' che la stavano invitando e disse loro,
    sconsolata ``Devo andare.'' e quindi spense la console e la TV.

    Da lì non avrebbe potuto fare molto. Se Vale non aveva sentito arrivare
    gli assalitori significava che giravano in Autistic Mode. Nessun
    firewall è più potente di spegnere le comunicazioni radio. Doveva
    andare lì fisicamente. Raccolse il suo portatile rinforzato con
    materiali militari ed il suo bracciale con rampino e corse su per le
    scale che portavano al tetto della base. Fece partire la riproduzione
    della sua scaletta musicale da running e si mise a correre verso il
    punto indicato dalla posizione GPS del cellulare di Valentine. Sarebbe
    stato un viaggio movimentato.

  \section*{Arsenicos}

    Arsenicos era nella cabina del suo Armoured Frame, con la <<u>> (non vorremo mica essere come quegli stronzi di
    Panamericani?), e stava tenendo sott'occhio i sensori per cercare tracce di minacce. Non che fosse diffile, visto
    che tutti i monitor necessari per la navgazione ed il combattimento erano posti nel campo visivo del pilota, senza
    che questo dovesse per forza girare la testa.

    Normalmente non era così. Durante i suoi allenamenti all'accademia aveva sempre usato delle unità piuttosto
    spaziose. Non solo perchè potessero contenere un istruttore oltre al pilota, ma anche perchè i camminatori della sua scuola erano di
    livello militare, quindi erano massicci ed imponenti.

    Era strano, quindi, trovarsi ad operare usando un Armoured Frame così piccolo. Era stato evidentemente progettato
    per il combattimento urbano. Sarà stato alto non più di una decina di metri, in piastre assorbenti nere con varie linee
    energetiche illuminate di azzurro. Servivano per mimetizzarsi nell'ambiente. Era strano vedere una macchia nera su
    un palazzo, anche a lunga distanza, perciò le avevano aggiunte per spezzare la figura. C'era anche la possibilità di
    coprire le linee con delle piastre nere, nel caso si combattesse durante un blackout o in un luogo senza
    illuminazione (certo, senza contare i sistemi visivi alternativi). Non che potesse passare per i vicoli, ma almeno poteva andare in giro per le strade a
    due corsie senza dover distruggere le facciate delle costruzioni ai lati. Era un veicolo abbastanza particolare, in
    realtà. Alcune componenti le aveva viste nelle liste delle case di produzione che ``sponsorizzavano'' la sua
    accademia, come la piastra frontale della FaLKnR oppure il pugnale (Anche se un coltello tattico da due metri in
    Acciaio nero con una lama da un metro rinforzata in Aggregated Diamond Nanorods aggangiati ad un sistema a
    microvibrazioni fa comunque il suo effetto) della Valhalla Blacksmiths.

    Altre, invece, come il fucile d'assalto con lanciagranate (un FN2000 ingigantito, in pratica) o la sidearm (FN-P90,
    il creaore doveva essere un fan del Belgio. O delle armi strane) erano customizzati. Dannazione. Anche cose
    sensibili come il processore centrale, il generatore ed il sistema di Thrusting principale erano customizzate. Non
    sapeva chi avesse lavorato dietro ad \emph{ATHENA}, questo era il nome dell'Armoured Frame, ma sicuramente non era
    uno alle prime armi.

    E a proposito di prime armi. Non riusciva a capire come mai Valentine, il capo del gruppo che l'aveva assunto,
    continuasse a chiamarlo Pivello. Non sembrava il tipo di donna che prendeva in giro gli altri. Quindi,
    probabilmente, non riusciva a ricordarsi il suo nome. Non era un problema. Gliel'avrebbe ricordato. Anzi! Di sicuro
    non era per quello, diciamocelo. Ha chiamato lui per questa missione, mica le altre due.

    Tutto doveva andare bene, doveva dimostrare alla sua datrice di lavoro che era uno di cui fidarsi.

    Stava seguendo la limousine del VIP da vicino, non voleva che qualche veicolo si intromettesse tra di loro. Una
    delle tecniche d'attacco più banali, certo, ma non puoi mai sapere che tipo di trucchi hanno nelle maniche. Aveva
    dovuto bruciare dei rossi, ogni tanto, ma non avrebbe preso multe. Gli Armoured Frames non hanno targhe da
    scannerizzare.
    
    Certo. Non che quella cosa lo facesse stare meglio. Lui era salito sugli Armoured Frames per la giustizia. Si era
    fatto sette anni di accademia per quello. Non aveva scelto quella carriera per l'adrenalina o per uccidere i
    cattivi. L'aveva scelto perchè in quella maniera poteva fare rispettare le regole. È anche per quello che, ora, non
    lavorava per qualche gruppo mercenario Europeo che si vendeva alla Corp che offriva di più, ma invece lavorava per
    i Troubleshooters Sans Frontieres. Il gruppo di Valentine Arthur sarà stato piccolo, avranno avuto problemi
    finanziari, ma almeno era coerente con le idee di Arsenicos e non accettava Wet Works.

    Ad un tratto un cicalino attirò la sua attenzione. Il sistema di controllo di ATHENA contornò di rosso una minaccia
    sullo schermo principale. Era una Rocked Propelled Grenade sparata da un MATADOR Mk.VI. Non risucì a reagire
    abbastanza in fretta. Quelle cose vanno ad una velocità impossibile. L'unica fortuna, per quelli dentro la
    limousine, era che il sistema è un Fire and Forget, quindi non ha nessun sistema di puntamento, e chi l'aveva
    lanciato non aveva moltissima mira oppure molta fretta. La granata colpì la strada subito davanti alla macchina,
    sbalzandola verso sinistra.

    Arsenicos cambiò modalità di propulsione, facendo in modo che le slitte con ruote, sospinte dai secondary thrusters,
    cambiassero configurazione e diventassero degli appoggi con tre punti di contatto col suolo, facendoli sembrare
    degli zoccoli stilizzati. In quella maniera avrebbe avuto più controllo. Imbracciò Excalibur, il fucile d'assalto
    (pessimo gusto), e scattò di lato, andando a coprirsi dietro una delle costruzioni a lato della strada. Doveva
    portare via Valentine ed il VIP.

    Ma come avrebbe fatto? Gli Armoured Frames non sono fatti per aver tanto controllo a livello di pressione delle
    dita. Quel tanto che basta per non rendere le armi degli ammassi di ferro senza forma, ma non poteva andare lì e
    prendere il capo per la collottola sperando di non trasformarla in un ammasso di carne macinata.

    Poco dopo, però, arrivò una chiamata da parte di Valentine. ``Arsenicos online.'' disse lui, prontamente, dopo aver
    attivato la connessione ``Ascoltami Pivello.'' fece la voce dall'altra parte della linea, frettolosamente.
    ``Veramente mi chiamerei Arsenicos.'' rispose lui, istintivamente ``Non me ne frega nulla, ok? Senti, non...''
    Valentine era viva, quindi significava che doveva uscire dalla macchina. Il problema era che si trovava sotto il
    tiro di chi aveva compiuto l'imboscata, perciò non poteva uscire finchè la situazione era questa. L'aveva chiamato
    per chiedergli una mano. Ma non c'era tempo.
    
    ``Sì, lo so cosa vuole dire, Signorina Arthur. \emph{Non morire}, giusto? Non si preoccupi. Vado e torno!'' fece
    lui, tirando fuori una frase figa sperando che questo facesse colpo, e poi chiuse la comunicazione. Magari poi si
    sarebbe ricordata il suo nome. La tattica standard in questi casi è quella di fornire fuoco di copertura per fare in
    modo che gli alleati si ritirino. E così sia.

    Facendo perno sul piede destro di ATHENA uscì dalla copertura, girandosi, fucile spianato. Arrivò accucciato, passò
    il dito dal grilletto del fucile a quello del lanciagranate. Rapidamente mirò al punto più probabile (calcolato
    durante la rotazione dal computer di bordo) della costruzione da dove era partito il missile e premette fino a fine
    corsa il grilletto. La granata HEAT da 400mm volò fino alla costruzione. Appena impattò sul muro saltò in aria,
    infrangendo tutti i vetri in un raggio di 100 metri con l'esplosione e demolendo parte della facciata. Non poteva
    essere sicuro, però, che non si fossero spostati da quel punto, perciò iniziò a sparare raffiche controllate in
    quella direzione. Ogni volta che premeva il grilletto tre bossoli di proiettile grandi come un comodino venivano
    esplusi dalla camera d'esplosione dell'arma, lasciando dei fori in terra, o nelle costruzioni vicine.

    Continuò così per qualche decina di secondi poi il sistema di controllo lo avvisò di un altro pericolo, questa volta
    dall'alto. Arsenicos fece in tempo solo a girarsi che un colpo di MATADOR lo prese sul braccio destro, danneggiando
    quasi completamente l'armatura. Fortunatamente ATHENA era abbastanza resistente per resistere ad un colpo simile,
    perciò prese la mira e sparò una raffica di tre colpi nella direzione del missile, distruggendo il muro esterno di
    uno o due appartamenti della casa. Sarebbe stata una battaglia lunga, se fosse continuata così. I nemici erano in
    Autistic Mode, per non parlare del fatto che potevano avere qualche sistema di camuffamento termo-ottico per
    proteggerli. ATHENA era stato creato per il combattimento contro veicolo, non anti guerriglia.

  \section*{Sofia}

    Poteva sembrare stupida l'idea di andare a piedi verso il luogo della missione. Ma non era così. Il traffico a
    Parigi era impossibile. Inoltre lei correva sui tetti. La strada era per Noobs. Si era allenata per anni, durante le
    superiori e l'università. La città dall'alto era uno spettacolo magnifico e non tutti erano capaci di apprezzarlo.
    
    Ma, ovviamente, saper saltare lontano ogni tanto non basta. Quando la distanza tra due costruzioni è più grande del
    record del salto in lungo (meno, in realtà, perchè lei non era una campionessa olimpionica), allora avevi bisogno di
    una mano. Quella mano si chiamava, semplicemente, Hook. Hook era un sistema d'aggancio elettromagnetico che
    sfruttava certe proprietà dei materiali superconduttivi che lei non aveva capito molto bene. Fondamentalmente quello
    che faceva l'Hook era intrappolare il campo elettromagnetico degli oggetti per agganciarvici. Non andava benissimo
    in fisica. Era fortunata che si studiava solo come base, altrimenti non avrebbe mai potuto prendere il Master.

    Grazie alle sue abilità di runner da Parkour e all'Hook, il sistema GPS a Mixed Reality che le indicava i waypoint
    era per scontato,  Sofia riusciva ad andare più o meno dappertutto alla velocità di un Taxi. Gratis.

    Dopo una decina di minuti riuscì ad arrivare al punto prefissato. Era arrivata sul tetto di una delle case a lato
    della strada, il quale non era messo benissimo. C'erano svariati buchi per terra, che proseguivano fino al lato
    della costruzione. Se guardava attraverso riusciva a vedere il fianco di ATHENA, l'Armoured Frame del gruppo,
    accucciato dietro ad una limousine rovesciata. Probabilmente il Pivello l'aveva posizionata in quel modo per
    proteggersi. Ma dove diavolo erano Vale ed il VIP?

    Sofia chiamò il pilota dell'Armoured Frame. ``Arsenicos Online.'' ``Come va?'' ``Ah, signorina Zen. Non benissimo.
    Siamo stati attaccati da un gruppo di nemici che...'' ``Fammi indovinare. Modalità Autistica o cose così?''
    ``Probabilmente anche tute Termo-ottiche.'' Perfetto, sul serio. Però, se utilizzavano delle tute a mimetizzazione
    attiva dovevano generare qualche distorsione nel campo visivo. Forse avrebbe potuto vederli. O forse...

    ``Coso.'' ``Veramente, io sarei Arsenicos.'' ``Mh. Abbiamo bisogno di qualcosa che disturbi i sistemi termo-ottici.
    Basta anche poco tempo, poi posso disabilitarli permanentemente.'' ``Ho creato delle nuvole di polvere, prima, ma
    non sono servite a moltissimo.'' ``Mh. Non va bene. Dentro nella polvere possono ancora nascondersi.'' Sofia si
    guardò in giro per vedere se riusciva a trovare qualcosa di utile. Tentò di applicare alla sua visione una
    rappresentazione 3D della rete di distribuzione dell'acuqa. La pioggia faceva un casino incredibile col sistema di
    mimetizzazione. Troppi update in fretta e la tuta si auto-disattiva per resettarsi, pensando ci siano problemi.
    Purtroppo dura finchè c'è pioggia.

    Nelle vicinanze c'erano delle tubature, ma significava dover demolire la strada per poter colpirle. Non poteva
    permettersi dei danni simili. In realtà avrebbe potuto permetterseli, in realtà, ma la ramanzina che avrebbe dovuto
    subire dopo la missione da parte di Vale non sarebbe valsa la pena. Fortunatamene l'azienda pubblica aveva deciso
    d'installare un serbatoio per la distribuione d'acuqa in caso di siccità prorprio dul tetto della costruzione di
    fronte a quella dove si trovava in quel momento.

    Aveva bisogno che venisse demolita.

    Sofia guardò in direzione di ATHENA. L'Armoured Farme era ridotto male: aveva parecchie piastre d'armatura
    danneggiuate da colpi di lanciamissile e, a giudicare dal numero di bossoli impiantati per terra, stava per finire
    le munzioni. Era il momento di darci dentro e <<Shoot to Thrill>> era la canzone giusta per il momento, perciò
    richiese al riproduttore di saltare a quel punto della scaletta.

    BJ chiamò il Pivello ``Hei.'' ``Mi dica Zen,'' ``Ascolta, ho bisogno che tu faccia saltare quel serbatoio d'acqua in
    cima al palazzo alla tua sinistra.'' ATHENA girò la testa in quella direzione e poi ritornò nella posizione normale
    ``Ok. Non c'è problema. Ma a che le serve?'' ``Voglio giocare uno scherzetto alle tute mimetiche di questi Newbs.''
    ``Roger.'' ATHENA si alzò e puntò l'arma in direzione dei piloni di supporto del serbatoio ``NO!'' urlò Sofia.
    L'Armoured Frame ritocnò in copertura dietro alla Limo.

    ``Ma non mi ha detto di...'' BJ non permise al pilota di finire la frase  ``Ti ho detto di farlo saltare.'' ``Era
    quello che stavo facendo.'' ``Con il fucile?'' ``Perchè?'' ``Hai una macchina distrutta davanti! Usa quella! È
    vuota, no?'' ``Beh, c'è un cadavere dentro.'' Sofia si mise la mano in faccia, non poteva crederci ``Ma che ti frega, Newb? Lancia la
    Limo contro quello stramaledetto serbatoio.'' ``Roger.''

    L'Armoured Frame si alzò in piedi, appoggiando Excalibur a terra, e poi prese con entrambe le mani la Limo verso il
    fondo. Si sentì il rumore dei pistoni idraulici e dei nanotubi in grafene che venivano sforzati per sollevare il
    veicolo. Dopo poco l'Armoured Frame aveva sollevato la parte frontale della macchina e quindi, per darsi la spinta,
    spostò in dietro il piede destro. Con una rotazione dei motori magnetici che formano spina dorsale del Mecha ATHENA
    inizò una rotazione di 360° con l'intero corpo. E così, come se fosse stato un atleta che partecipa ad una gara di
    lancio del martello, l'Armoured Frame lanciò la macchina verso il serbatoio.

  \section*{Valentine}

    Valentine era riuscita a trascinare a forza il peso ``morto'' (sperava di no, altrimenti addio soldi) di Mr.
    Fabulous all'interno del veicolo che si trovava direttamente a lato della poszione dell'incidente. Non era stato
    facile. Non solo era dovuta uscire da una macchina che rischiava di esplodere, ma aveva dovuto schivare i bossoli
    che venivano espulsi dal fucile d'assalto dell'Armoured Frame. Non sapeva cosa fosse preso al Pivello, ma uno di
    quei bossoli è abbastanza per uccidere una persona. O distruggere macchine, se è per quello.

    Lì doveva essere al sicuro, per il momento. Però se fosse entrata in una delle costruzioni forse avrebbe avuto
    ancora più possibilità di sopravvivere allo scontro. Sollevò la gonna del vestito per raggiungere la fondina che
    aveva legato alla coscia destra, subito sopra le calze autoreggenti attaccate alla giarrettiera. Quanto diavolo era
    scomodo sollevarla? Mai più, la prossima volta si sarebbe messa un vestito tattico mascherato.

    Estrasse la H\&K SOCOM Mk.30 con silenziatore e smart pack e buttò a terra la parte di gonna che stava tenendo
    sollevata. Attivò il mirino a laser verde e collegò il cavo che usciva dallo smart pack agli occhiali. A mezz'aria
    comparirono delle scritte e dei simboli luminosi, generati dai Full Spectrum Pixels di cui erano ricoperti i suoi
    occhiali, li quali indicavano la quantità di proiettili rimanenti, lo stato della batteria dello smart pack e,
    soprattutto, dove stava puntando l'arma. Poteva sembrare un gadged per ragazzini questa cosa, ma in realtà non le
    richiedeva il tempo necessario per spostare l'arma in posizione di mira per sparare con precisione. Non aveva quelle
    cose quando lavorava per la \emph{Sword of the Queen}, una Corp di protezione nazionale Britannica.

    Si guardò intorno per vedere se ci fossero stati nemici e poi si avvicinò ad una porta che dava in una delle
    costruzioni ai lati del vicolo. Improvvisamente sentì rumore di sforzi da trazione provenire dall'Armoured Frame e,
    dopo pochi istanti, il rumore di metallo contro metallo. ``Ma che ca...'' tentò di dire ma fu interrotta da uno
    scroscio d'acqua che proveniva dal tetto della casa nella quale stava provando ad entrare. In pochissimo si ritrovò
    col vestito mizzo, cosa che avrebbe significato muoversi molto più lentamente. Cosa aveva in testa quell'idiota?

    Quando sentì il rumore che fanno le tute ottiche quando si resettano capì. Si girò di scatto giusto in tempo per
    vedere un uomo vestito con una tenuta tattica correrle addosso, pugnale da combattimento in mano. Riuscì a stento a scartare
    di lato e far continuare il movimento dell'uomo oltre di lei, il quale si stava già girando per tirarle una
    pugnalata al torso. Rapidamente si voltò e, bloccando con l'avambraccio sinistro quello destro dell'avversario,
    glielo bloccò ponendo la mano ad uncino. Fu un fatto di alcuni battiti di ciglia. Prima che l'avversario riuscisse a
    coprirsi con la mano destra Valentine portò un affondo alla trachea dell'avversario con la canna della pistola
    (aiutata anche dall'allungo aggiuntivo fornitole dal silenziatore), rompendogliela all'istante. L'uomo si accasciò a
    terra, gorgogliando.

    Valentine non si faceva problemi ad eliminare nemici che provavano ad ucciderla. Non era come quegli eroi di cui si
    legge nei fumetti. Quelli andavano alla gola, non vedeva perchè non avrebbe dovo fare altrimenti. Diede un'occhiata
    alla tuta che si riattivò dopo aver completato il reboot: sembrava un ottimo gadget. Trascinò il cadavere dell'uomo
    verso la porta, la aprì usando due colpi di pistola e quindi, dopo aver portato dentro Mr. Fabulous ancora nel
    mondo dei sogni, ce lo trascinò dentro.

    Lì spogliò il cadavere della tuta tattica e di tutto il resto dell'equipaggiamento e, dopo essersi spogliata, lo
    indossò a sua volta. Fortunatamente era riuscita a non rovinarlo. Non trovavi una tuta termo-ottica di livello
    militare tutti i giorni. Non gratis, almeno. La tuta era auto regolante, quindi divenne della sua misura appena
    finì d'indossarla. Legò la fondina termo-ottica alla coscia e recuperò il pugnale dell'uomo. Decise di chiudere Mr.
    Fabulous dentro uno degli armadietti per le scope che si trovavano in quel piano. E di mettere il cadavere in quello
    accanto. Ci sarebbero stati meno casini così. Attivò la mimetizzazione e si mise a salire la costruzione per cercare
    risposte.

  \section*{Sofia}

    Sofia seguì la Limo con lo sguardo finchè questa non andò a sfondare il serbatoio d'acqua. Si ritrovò ad urlare
    ``BOOOM!'' mentre agitava il pugno in aria. Intanto, in giro per i tetti si potevano vedere gli effetti delle gocce
    d'acqua su dei sistemi termo-ottici. Un casino di reboot forzati. Grandioso. Una dei più sfortunati si trovava sul
    tetto esattamente di fronte al suo. Sfortunata perchè, prima di capire cosa gli stesse succedendo, sarebbe stato
    colpito da una ragazza di 54 Kg catapultata da un artiglio elettromagnetico.

    Ed, in effetti, è proprio quello che BJ fece. Puntò l'hook in direzione del pavimento del tetto e si fece trascinare
    in quella direzione, piede destro in avanti, pronto a tirare un calcio in faccia alla donna. Cosa che avvenne pochi
    secondi dopo. Le atterrò in volto, facendola schizzare indietro. Non che lei atterrò meglio. Quel piano era geniale,
    ma aveva il problema dell'atterraggio. Le placche rinforzate della sua tenuta la stavano proteggendo, quindi non
    c'era di che preoccuparsi.

    Appena si fermò andò di corsa verso la donna per stenderla del tutto, ma evidentemente non serviva. Il colpo l'aveva
    già fatta svenire.

    Sofia si guardò in giro per scoprire con cosa comunicavano. Di sicuro non usavano radio, li avrebbe intercettati.
    Poi notò un portatile rinforzato a terra. Non rinforzato come il suo. Non tutti possono permettersi quel livello di
    fiiganza. Corse verso il portatile, il quale era collegato via cavo ad uno degli appartamenti. Ovviamente non
    serviva comunicare senza fili se avevi una connessione ottica, no? Di sicuro non ti beccano molto facilmente in
    quelle situazioni. L'avevano pensata bene, i maledetti. Estrasse dal suo zaino il laptop e lo fece Boootare, mentre
    collegava i due via cavo. Una volta caricato il sistema operativo si collegò ad esso attraverso i connettori che
    aveva montati sul retro del collo. Era un classico. Ce li avevano così in Ghost In The Shell, in Shadowrun, in
    Matrix. Aspetta. In Matrix erano lungo tutta la spina dorsale. Ma quella era più una realtà alternativa. Oh, vabbeh,
    chi se ne fregava. Se li era fatti fare con una forma esagonale molto figa, per sembrare di più in personaggio di
    Ghost In The Shell. Adorava quella serie. Adorava anche il Maggiore. Peccato che lei non fosse così brava.

    Però un PC del genere riusciva ad hackarlo, soprattutto se gli era collegato direttamente. Anzi, era una cazzata.
    Come quando leggi la posta di qualcuno che ha lasciato il PC sbloccato. Non c'erano documenti da leggere, ma la rete
    che utilizzavano per comunicare tra di loro sfruttava una codifica militare a 1024qbits. Una cosa che non cracki
    dall'esterno. Da quando avevano introdotto i processori quantistici i sistemi di codifica si sono dovuti fare più
    potenti e sfruttare a loro volta le potenzialità intrinseche degli algoritmi quantistici. Quando un sistema può
    testare tutte le alternative possibili in un colpo (Avete presente la storia del gatto vivo e morto al tempo stesso?
    Applicatelo ai bit. 1 e 0 nello stesso momento) allora è ora di avere dei codici che riescano a resistere a questi
    attacchi.

    Quindi questi tizi facevano parte di qualche Corp o di qualche gruppo di supporto che possedeva strumentazione
    militare. Non che la stupisse. Spesso quelli con cui avevano a che fare utilizzavano questo tipo di strumentazioni.
    Ma che voleva un gruppo del genere da Mr. Fabulous, il VIP degli Spammers?

    Di sicuro non erano stati pagati dall'\emph{Associazione degli uomini senza problemi d'erezione}, annoiati dalle
    continue lettere che promettevano il ritorno alla gloria dei vecchi tempi, o dal gruppo \emph{Giocatori Sfigati},
    il quale si sentiva truffato dalle continue offerte di soldi facili attraverso il Poker online.

    Giusto? Non potevano essere loro. Spulciò i files presenti sulla macchina. Magari qualcosa c'era. Il PC, per fortuna
    sua, era lindo. Fresco d'installazione, con solo gli applicativi necessari e con una cartella dei documenti
    meravigliosamente pulita. Lì dentro non trovò nulla. Ma quella non gliela raccontava mica giusta. Era impossibile
    che non avessero intelligence per una missione così bene organizzata. Doveva esserci qualche socket aperta, qualche
    file con foto. Qualcosa. Decise di lanciare il programma per il recupero di dati cancellati. La tizia che aveva
    fatto svenire non poteva aver avuto abbastanza tempo per lanciare un file shredder (uno di quei programmi per
    la sovrascrittura dei file più e più volte con dati random per evitarne il recupero), perciò lo mise a lavorare ed
    intanto lei tentò di scansionare il sistema in cerca di socket aperte. Magari sarebbe riuscita ad intercettare
    qualche comunicazione.

    E infatti...

  \section*{Arsenicos}

    Arsenicos guardò la ragazza volare da un tetto all'altro, mentre intorno a sè l'acqua bagnò la strada. Vide alcune
    delle guardie alle quali erano andate in tilt le tute ritirarsi verso l'interno del piano. Non avrebbe potuto
    inseguirli in quelle condizioni. Sperava che Sofia sapesse cosa stava facendo, visto che lui non poteva aiutarla
    senza distruggere i palazzi intorno. Si girò intorno per dare un'occhiata alla strada. Non sarebbe finita in quella
    maniera. Se veramente volevano uccidere il VIP quello non bastava, sarebbero dovuti passare oltre ATHENA.

    Arsenicos si aspettò altri colpi di lanciamissili, perciò strappò una paratia in metallo montate sulle costruzioni
    ai lati, utilizzate per proteggerle da incidenti o cose come le battaglie di questa sera. Dopo poco arrivò una
    comunicazione da parte di Sofia.

    ``Arsenicos Online.'' disse lui, utilizzando la sua solita frase ``Coso, ascolta. Ho appena intercettato delle
    comunicazioni attraverso la radio di questi quà. E ti consiglierei di tenere d'occhio i tetti. Sta per arrivare
    un'altra delle loro unità. Non ho capito di che stiano parlando. Potrebbe essere un gruppo di soldati armati di
    lanciamissili, un elicottero oppure...'' Arsenicos portò in primo piano il radar e notò, effettivamente, che
    qualcosa si stava avvicinando a quella posizione ad alta velocità, giusto trenta metri sopra la sua posizione.
    Significava che andava in giro per i tetti. Non erano soldati, non venivano indicati con un triangolo, normalmente
    veniva usato una specie di nube per indicare il luogo più probabile. Non era facile usare dei radar a lungo raggio
    per scovare dei nemici così piccoli. ``Ho capito, Signorina Zen. Vado.'' e mise giù la comunicazione.

    Quello che stava per arrivare era un Armoured Frame d'assalto a giudicare dalla velocità. Dei piloti normali non
    avrebbero mai fatto volare un elicottero a quell'altitudine. Con un Armoured Frame, invece, si potevano sfruttare i
    tetti per muoversi. Agganciò il fucile d'assalto sulla schiena ed estrasse la Sub Machine Gun, Gae Bolg. Con quella
    avrebbe potuto combattere usando la piastra come scudo. Attivò i thrusters verticali e quelli secondari per saltare
    sui tetti. Atterrò accucciato, sbalzando via uno dei nemici, il quale si trovava là per sbaglio, facendolo finire
    per strada. Scansionò il suo campo di vista per vedere se riusciva a prendere la mira sul nemico. Purtroppo il
    nemico riuscì a sparare per primo.

    Prima di riuscire a vedere per intero la figura dell'avversario dei colpi di cannone da spalla colpirono il suo
    scudo, sfondandolo completamente, proiettando delle schegge contro la spalla sinistra. I sistemi di diagnostica
    portarono la sua attenzione ai danni subiti, indicando danni alla corazza esterna e lacerazioni al 20\% dei nanotubi
    al carbonio. Quel livello di danni era ancora accettabile. Non avrebbe lanciato in giro macchine, ma avrebbe ancora
    potuto assestare dei pugni. Il nemico stava arrivando sospinto dal suo main thruster a tutta velocità. Non a caso
    l'aveva listato come d'assalto. Probabilmente aveva un solo colpo per il cannone da spalla, tanto che l'aveva
    sganciato poco prima, facendolo incassare nel tetto di una casa. Era armato di tirapugni a pistoni idraulici da
    sfondamento. Uno di quei colpi ad un Frame del genere ed addio Arsenicos. Niente più fan urlanti.

    Doveva fargli cambiare traiettoia. Impugnò rapidamente la SMG e mirò ad uno dei thrusters secondari. Se riusciva a
    farglielo saltare il sitema avrebbe avuto dei problemi a controbilanciare la perdita così in fretta. Aspettò fino
    all'ultimo momento, quando il sistema di pilotaggio dell'Armoured Frame riuscì a calibrare la mira, e poi premette
    il grilletto. L'SMG sparò tre colpi, di cui uno finì all'orizzonte (sperò di non aver colpito nessuno), uno su una
    piastra laterale ed il terzo colpì in pieno il thruster, facendo cambiare la traiettoria dell'Armoured Frame nemico
    di dieci gradi verso la sua sinistra, visto che il thruster sinistro del nemico non riuscì a spegnersi in tempo.

    Appena il veicolo avversario fu abbastanza vicino Arsenicos tirò un calcio rotante con l'Armoured Frame attivando i
    thrusters posti sulla parte posteriore della gamba destra, andando ad impattare contro il torso del Mecha nemico,
    facendolo schiantare contro il palazzo sul quale si trovava poco prima. Anche lì il sistema si lamentò per i danni
    subiti, ma almeno non erano dati da un cannone a lungo raggio. Appena toccò terra girò ATHENA verso il veicolo
    nemico ed attivò il main thruster, venendo proiettato a tutta velocità contro di esso, mentre stava ancora cadendo
    verso la strada.
    
    Utilizzando tutta l'energia cinetica accumulata tirò un pugno alla testa del Frame nemico, facendolo volare contro
    il palazzo alla fine della strada. Arsenicos non si fermò lì e continuò l'asssalto, andandogli dietro. I danni alla
    testa erano ingenti, ma non bastava, doveva staccarla per disattivare completamente il sistema avversario. Purtroppo
    il nemico era riuscito a riprendersi e, appena Arsenicos fu abbastanza vicino, gli tirò un colpo di pistone diretto
    al petto. Il pilota riuscì a spostare il braccio sinistro abbastanza in fretta da parare il colpo. Questo distrusse
    completamente l'arto, ma almeno lui era vivo ed integro.

    Tirò una testata al Frame avversario, mandandogli in tilt i sistemi per quel tanto che bastò per prendere l'SMG e
    sparare una raffica ravvicinata alla testa, distruggendola. Istantaneamente lanciò in terra l'SMG e disarmò
    l'avversario. Non poteva rischiare un altro colpo da quell'arma, per quanto alla cieca.

    Fatta. Se gli avversari non disponevano più della loro arma principale non erano più un pericolo, giusto? Era l'eroe
    della giornata. La giustizia aveva prevalso ed il VIP era ancora vivo. Ha.

    ``Che fai, Noob?'' si sentì dire, attraverso la radio, da Sofia. Prima ancora di poter rispondere vide la ragazza
    volare, usando il suo artiglio elettromagnetico, sul Frame avversario. Stava frugando intorno ad un punto preciso
    finchè, dopo qualche secondo, aprì uno sportellino e ci collegò un cavetto che le usciva dal collo. La ragazza
    sembrò come immobilizzata per una decina di secondi e poi, ritirando il cavetto, fece ``Fiu. Fatta.'' ``Fatta
    cosa?'' chiese Arsenicos attraverso la radio ``Ho disabilitato il sistema di autodistruzione. Lo sai quanta roba ci
    possiamo razziare da un veicolo come questo?'' La vide esaminare il Frame del nemico ``Sei stato bravo a distruggere
    solo i sensori della testa.''

    La ragazza frugò dietro uno dei pannelli rinforzati presenti sul petto e, dopo un po', ritirò la mano. Si sentirono
    dei rumori di pistoni idraulici, mentre sul retro del Frame alcune delle piastre di protezione si spostavano,
    lasciando libero l'accesso alla cabina di pilotaggio. Arsenicos, a sua volta, aprì la sua cabina, bloccò il sistema
    di controllo di ATHENA e scese per strada. Era inutile, in quel momento, stare dentro. Quello era l'unico Armoured
    Frame nemico, e combattere unità a piedi era scomodo con un veicolo alto quindici metri. Vide la ragazza andare dietro
    il Frame nemico, probabilmente per immobilizzare il pilota. Si avvicinò a sua volta quando, dopo pochissimo, spuntò
    Sofia.

    Tenuta ferma con una presa al collo ed una pistola puntata alla testa dalla pilota del Frame. ``Ok stronzo,'' fece, la donna
    ``un solo movimento sbagliato e puoi dire addio alla ragazza.'' Arsenicos mise le mani avanti ``Va bene, va bene.
    Non ho armi, ok?'' fece lui ``Se vuoi andartene vattene, ma non penso che tu possa richiedere molto altro.''
    ``Voglio il datacube che sta trasportando Mr. Fabulous.'' ``Il che?'' ``Non fare il simpatico con me, pilota.''
    Pilota, non Pivello. Visto?

    MA A CHE STAVA PENSANDO? ``No, sul serio. Non so di cosa stai parlando.'' ``Non ci tieni alla ragazza?'' fece la
    donna, premendo più forte la canna della pistola sulla tempia di Sofia. ``Ascolta, la mia pazienza è limitata. Ora
    conterò fino a dieci, se continuerai con la recita del pilota ignorante farò fuori la ragazza.'' ``Aspetta!'' Cosa
    doveva dirle?

    ``Uno...''

  \section*{Valentine}

    Non aveva trovato molto dentro nella costruzione. E per strada c'era stato rumore di combattimento tra Armoured
    Frames fino a poco prima. Doveva andare a controllare. Uscì dall'appartamento nella quale era entrata e scese le
    scale. In fretta, perchè le tute termo-ottiche funzioneranno bene, ma la loro autonomia fa schifo, tanto che decise
    di disattivare la mimetizzazione, tanto aveva controllato fino a quel punto.

    Ignorò Mr. Fabulous che pareva essersi risvegliato nel suo ripostiglio. L'avrebbe recuperato poi. Scattò fuori dalla
    porta per strada ed attivò di nuovo la tuta. Corse fino all'angolo della costruzione e sbirciò fuori. C'era ATHENA
    con un braccio distrutto inginocchiato di fronte ad un Frame d'assalto con la testa demolita. Per terra c'era il
    Pivello che si guardava intorno, ad un tratto, da dietro l'altro Frame comparì Sofia presa in ostaggio dal pilota
    dell'altro frame.

    Ma saranno stati cretini quei due? Sofia si mise a correre per raggiungere ATHENA. Riuscì a non fare troppo rumore.
    Intanto la discussione tra i due stava arrivando ad uno stallo. Che significava ``datacube''? Valentine era troppo
    preoccuppata all'indennità dei membri del team per ragionare su come lo stronzo poteva averli assoldati per il
    lavoro sbagliato.

    Scivolò fuori dalla copertura del Frame dal lato opposto di dove si stava consumando la trattativa. Si mosse verso
    le gambe dell'altro Frame. ``Uno...'' fece il pilota avversario. Valentine ripose l'arma nella fondina. ``Due'' fece il giro
    della gamba. ``Tre'' Corse in avanti per arrivare dietro l'altra. ``Quattro'' arrivò dietro al pilota. ``Cinque''
    appena la donna finì di dire il numero Valentine le prese la pistola, con una torsione improvvisa nella direzione
    del dito sul grilletto la obbligò a mollare l'arma, mentre con la mano sinistra le mise in leva il braccio
    dell'arma, obbligandola a lasciare BJ. La tuta, con una scarica statica, si spese.

    ``Chi cazz...?'' provò a chiedere la pilota, guardando in cagnesco Valentine ``Che? Ti aspettavi che ti lasciassi
    arrivare a 10?'' rispose lei, per poi spararle alle ginocchia. La donna urlò di dolore mentre si contorceva per
    terra.

    ``Signorina Arthur!'' fece il pivello. Valentine si girò, incazzata, verso il loro pilota e, dopo aver lanciato la
    pistola a Sofia, vi avvicinò con passo minaccioso al Pivello. ``Ah, io sto bene, grazie.'' fece lui, non capendo.

    Valentine lo prese per il colletto e, dopo esserselo trascinato vicino al volto gli urlò ``MA CHE CAZZ...''
