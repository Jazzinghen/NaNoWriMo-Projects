\chapter{Se togliessimo in brutto tempo alla Gran Bretagna...}

  \section*{Sofia}
    
    Ed eccola lì. Sui tetti delle costruzioni Londinesi. Ovviamente era una serata nuvolosa e l'aria era carica di qualla
    strana nebbia che ti permette di vedere ma, appena ti muovi, ti bagni completamente. Le nuvole, puttosto basse,
    riflettevano le luci della città, quindi non sembrava neanche così tanto buio. Sofia era accucciata sul
    cornicione di una casa mentre indossava una di quelle tute tattiche che avevano rubato. Attualmente non c'era
    bisogno di attivarla, visto che nessuno l'avrebbe vista. E, anche fosse, si sarebbe potuta allontanare. A nessuno
    frega di una ragazza che non vuole buttarsi giù, no?

    Sofia invocò l'orologio di sistema per vedere che ore fossero. Nel suo campo visvio, a lato della mappa GPS, comparì
    una rappresentazione di uno di quei orologi da stazioni dei treni, il quale indicava le 19:40. Ora di partire. Sofia
    si alzò, fece dello strecching rapidamente e poi si mise a correre in direzione della struttura di ricerca. Saltò di
    tetto in tetto, poi dovette saltare su una scala antincendio, rischiando pure di farsi un male impressionante.
    ``Scusate!'' urlò a due ragazzi che si stavano godendo la propria serata in una delle camere adiacenti alla scala.

    Poco prima di arrivare sul tetto di quella casa attivò la mimetizzazione termo ottica, la quale fece il suo
    familiare rumore elettrico, e quindi riprese a correre. Dopo un paio di case arrivò in prossimità del centro di
    ricerca. E, come Valentine aveva previsto, poteva vedere la finestra aperta due piani sotto. ``Ora della verità!''
    si disse per incoraggiarsi, mentre si accingeva a raggiungere il ciglio del palazzo ed a saltare. Raggiunse il
    limite, sparò un colpo del suo Hook sul cornicione, così da avere un backup in caso di mancato bersaglio, e quindi
    saltò. Dopo un breve volo finì esattamente dentro nella finestra. Prima di rischiare di slogarsi la spalla sinistra
    sgangiò l'Hook. Atterrò nel corridoio del bagno, solo che la velocità che aveva era troppo alta, perciò rotolò per
    qualche metro, tentando di attutire la cadura, andando però a sbattere contro una delle paratie dei bagni,
    provocandosi una contusione al braccio sinistro.

    Sofia soffocò un impreco. Avrebbe avuto tutto il tempo per quello DOPO, una volta fuori dal palazzo pieno di guardie
    cattivissime. Ora doveva entrare in Sam Fisher mode. O Big Boss mode. Vabbeh, doveva arrivare al bocchettone
    d'areazione, il quale si trovava lungo i corridoi. La batteria della tuta stava per finire, quindi avrebbe avuto non
    moltissimo tempo. Sperava solo che il pavimento del bagno avesse asciugato l'acqua che aveva tirato su mentre
    correva per i tetti. Il braccio le faceva un po' male. Sperava non fosse niente di serio, anche perchè le sarebbe
    servito per scendere giù per i sistemi di areazione. Si avvicinò alla porta del bagno e la aprì lentamente. Sbirciò
    fuori per controllare la posizone delle tenecamere. Le poteva servire per quando sarebbe dovuta uscire più tardi, in
    caso non avesse trovato planimetrie specifiche all'interno del sistema di sicurezza.

    Scivolò fuori dal bagno e si mise a percorrere il più in fretta possibile i corridioi cercando dei grossi
    bocchettoni per l'aria posizionati sui muri lungo il corridoio. Dopo poco vide una specie di grata in legno che
    sorreggeva delle piante rampicanti. Una cosa come gelsomini selvatici. le foglie della pianta sembravano venir
    attirate verso il muro da una qualche forza invisibile. Si avvicinò per guardare meglio. La struttura di legno
    serviva per nascondere la grata di metallo. Sofia immaginò che fosse stato fatto per mantenere un certo stile lungo
    i corridoi. Un'ottima idea, di sicuro, ma adesso avrebbe dovuto spostare la pianta.

    Si mise a spingere il vaso dal fondo, per non farlo cadere. Emise dei versi per lo sforzo ma, dopo un po', riuscì a
    spostare di quel tanto il vaso per fare in modo di poter accedere alla grata. La prese dal basso e la sollevò,
    facendola uscire dalla sua locazione, e tirò verso di sè per spostarlo dal muro. La batteria era quasi finita,
    perciò strisciò con i piedi dentro lo spazio che aveva creato. La tuta si disattivò proprio quando Sofia fu dentro
    nel condotto. Fortunatamente la prima parte era orizzontale. Avrebbe potuto spostare il vaso al suo posto usando
    l'hook e poi chiudere la grata.

    Ora era dentro. Bisognava solo riuscire ad arrivare al locale macchine, il quale, ovviamente, era verso il basso.
    Non poteva sbagliarsi. Proseguì lungo il condotto di areazione finchè non arrivò al fatidico punto d'incontro con
    altri tre condotti. In mezzo all'incrociò c'era un condotto cilindrico (mentre gli altri erano a sezione quadrata),
    largo più o meno cinque metri di diametro, che scendeva verticalmente, dopo aver recuperato l'aria anche dai piani
    più alti.

    Sofia sbuffò. Sarebbe stato un casino. Avrebbe dovuto usare l'Hook senza fare rumore. ``Mmmh... Un lavoretto da
    ragazzi, no?'' si disse, ironicamente, guardando in basso. Attaccò manualmente l'Hook senza spararlo ed iniziò a
    scendere.

  \section*{Valentine}

    Valentine controllò l'orologio della macchina. Il furgoncino delle pulizie doveva passare a breve. Questo
    significava che Sofia doveva essere entrata da poco nella costruzione. Ora il piano era quello di fare a finta di
    aver avuto un incidente con la macchina e farsi aiutare da due aitanti professionisti del pulito. Avevano
    posizionato la Jaguar di Valentine in mezzo alla strada ed aveva appoggiato il segnalatore di guasto.

    Sullo specchietto vide dei fanali in lontananza. Scese dalla macchina. Era vestita con un tailleur, in modo da
    sembrare una donna d'affari che non aveva idea di dove mettere le mani nella sua costosissima macchina. Quando il
    furgone fu abbastanza vicino la donna iniziò a sventolare le braccia, in modo da farsi vedere. Il furgone, come
    previsto, si fermò. Dal lato del guidatore scese un uomo con una tuta azzurra e si avvicinò, chiedendo ``Tutto a
    posto signora, serve una mano?'' ``Oh, sì, grazie!'' rispose lei, falsando un tono preoccupato ``La mia macchina non
    vuole saperne di partire. Può darmi una mano?'' ``Mh? Ma certo. Però sono un po' di fretta, quindi in caso se non
    riesco a risolvere è meglio se chiama un carroatrezzi.'' ``Ah, ma certo. Grazie mille. Mi sta proprio salvando!''
    ``Non si preoccupi.'' E poi si avviò verso la macchina, seguito da Valentine.

    L'allocco aveva abboccato. ``Allora, può aprire il cofano dell'auto?'' chiese l'uomo, di fronte alla macchina,
    ancora con i fanali accesi. ``Ah!'' esclamò lei, per poi avviarsi verso il posto di guida, ``Ma certo.'' Tirata la
    leva del cruscotto, questo scattò leggermente in alto, permettendo all'uomo di sollevarlo completamente. Il cofano
    aveva un sistema idraulico per il posizionamento, perciò non aveva bisogno di venir bloccato. ``Allora?'' chiese
    lei, avvicinandosi. ``Mah.'' fece lui, muovendo le mani in giro per il motore. ``Non lo so. La posizione dei
    componenti è un po' strana, perciò mi ci vorrà un secondo.'' Era ovvio che non ci stava capendo molto. Il motore era
    stato modificato dal Doc, probabilmente non avrebbe saputo metterci mano neanche lei.

    Mentre l'uomo era piegato sul motore il Pivello si era avvicinato al furgone, dal lato del passeggero, per prendersi
    cura dell'altro addetto alle pulizie. Scivolò verso la porta e bussò sulla porta. L'uomo all'interno si avvicinò al
    finestrino per guardare ma non vide nulla nè a destra nè a sinistra, perciò tirò giù il finestrino per guardare
    meglio. Slacciò la cintura di sicurezza e si sporse per guardare meglio. Appena uscì abbastanza con la testa
    Arsenicos scattò verso l'alto da dove si era posizionato e l'aggrappò con entrambe le mani tirando verso il basso.
    L'uomo andò a sbattere con la testa contro la portiera, stordendolo per qualche secondo.

    In quel momento Valentine prese il cofano dell'auto con entrambe le mani e lo tirò vero il basso il più forte che
    potè, facendo svenire l'uomo che stava ancora provando a capire come funzionasse quel motore. Arsenicos si alzò,
    prese di nuovo la testa dell'altro tizio con entrambe le mani e gli diede una testata che lo fece svenire. L'uomo si
    accasciò sul finestrino. Il Pivello spinse l'uomo all'interno e poi aprì la portiera. Valentine fece scivolare a
    terra l'adetto che provò a darle una mano, per poi controllare di non aver lasciato segni sul cofano. Passò le dita
    sul punto d'impatto e constatò che non c'erano stati danni. Il fatto che la carrozzeria era stata rinforzata aveva
    aiutato. Probabilmente senza di quella avrebbe lasciato un'ammaccatura.

    Valentine prese l'uomo a terra e lo trascinò fino al retro del furgone dell'impresa di pulizie. Il Pivello fece lo
    stesso con il suo, mentre da uno dei vicoli laterali uscì Catherine con un borsone in spalla. ``Boss'' iniziò,
    avviandosi verso la macchina, ``sei sicura di non averlo ucciso? Mi sa che ci hai dato dentro troppo.'' ``Dici?''
    chiese lei, guardando l'uomo, poi gli tastò la giugulare per vedere se aveva ancora polso ``Oh, no. È ancora vivo,
    non ti preoccupare. Una volta legato, magari, dagli un'occhiata te, ok?'' disse, sollevata ``Ok, sicuro. Tanto sono
    il medico, no?'' rispose Catherine, buttando la borsa nella XKR-S di Valentine. ``ODDIO!'' urlò Valentine ``Occhio
    con quel borsone! Non rovinarmi gli interni.'' ``Haha.'' rise Catherine ``Sìsì, mi prenderò cura di questa bambina
    per un po'.''

    Valentine ed il Pivello svestirono i due uomini dell'impresa di pulizie e poi indossarono i loro vestiti. Legarono i
    due e li buttarono nel retro del furgone, nel piccolo corridoio che si formava tra i due scaffali ricolmi di automi
    per le pulizie. Catherine diede un'occhiata ai due addetti. Quello che Valentine aveva picchiato aveva bisogno di un
    po' di trattamenti, perciò andò a prendere una siringa ipodermica di nanomacchine mediche general purpose ed il suo
    portatile. Dopo aver iniettato lo sciame nel collo dell'uomo controllò lo stato attraverso l'applicativo di diagnosi
    e gestione. L'uomo aveva un principio di ematoma extra durale, il quale poteva portare al coma. Valentine non era
    una che non riusciva a controllare la sua forza, però colpire alla testa era sempre un rischio. Catherine mandò lo
    swarm in quella posizione in modo che chiudessero la rottura e facessero defluire il sangue.

    ``Ok.'' disse lei, dopo aver chiuso il portatile, alzandosi ``Tutto a posto. Vuoi che me li trascini nella
    macchina?'' ``No, non ci stanno. Piuttosto li teniamo quì e li leghiamo in modo che non possano muoversi.'' ``Ok,
    come vuoi. Io vado.'' ``Ci vediamo dopo.'' ``Sì, in bocca al lupo.'' 

    Valentine salì al posto del guidatore e prese il volante. Aspettò che il Pivello si sedesse al suo lato e poi,
    mettendo in moto, fece ``Pronto?'' ``Speriamo.'' ``Mmmh.'' mugugnò lei, irritata ``Devi mettere a posto questo tuo
    modo di fare, Pivello.''

    ``Ma veramente...''

  \section*{Sofia}
    
    Quel canale d'areazione era sembrato più lungo del previsto. Aveva dovuto attraversare sensori laser, grate e pale
    secondarie larghe quando tutto il condotto che servivano per mantenere l'aria in moto. Era arrivata in fondo nera
    a causa di tutta la polvere accumulata sulle pareti dei condotti, raccolta da ognuno di quegli uffici e laboratori.
    Sperava solo che non ci fossero anche scarti di qualche esperimento. Tipo. Tipo in \emph{The Host}. Maledetto film.
    Non aveva ancora capito se fosse comico o meno. Sicuramente le scene lo erano, ma la storia era TRISTE. Se c'era una
    cosa che non capiva della produzione artistica giapponese erano i... Oh, wait. Ma quel film era giapponese o
    coreano? Avrebbe dovuto controllare una volta fuori da quel posto. Adesso doveva continuare.

    Guardò le uscite che erano presenti in fondo al condotto principale ce n'erano quattro di dimensione simile a quella
    dalla quale era entrata, mentre un condotto era dannatamente più grande, probabilmente il lato era grande il doppio
    di quelli \emph{standard}. Siccome non voleva finire dentro una qualche stanza delle guardie decise di prendere un
    foglietto da una delle sue stasche e liberarlo nell'aria. Il foglietto fluttuò per qualche secondo e poi iniziò a
    muoversi verso l'uscita più grande. Era abbastanza naturale, però non si è mai troppo sicuri in queste cose. Catturò
    il pezzo di carta prima che si allontanasse troppo, se lo rimise in tasca e continuò con la sua esplorazione. Attivò
    la modalità di visione notturna integrata nei suoi occhi e s'incamminò.

    Proseguì per una ventina di metri finchè non arrivò ad una biforcazione. Il tunnel che continuava dritto arrivava a
    quello che sembrava un vicolo cieco, mentre quello che continuava a destra... Anche? Sofia iniziò a rivedere quali
    direzioni aveva preso. Dopo due o tre cicli di controllo capì che quella era l'unica strada possibile. Perciò prese
    fuori di nuovo il fogliettino e lo liberò in aria. Questo andò a destra. Sofia lo riprese e poi proseguì in quella
    direzione. Una volta arrivata in fondo toccò la parete la quale non era in alluminio o acciaio o quello con cui
    erano fatte le altre pareti. Questo era morbido e spugnoso. E, dopo un po', sembrava pizzicare. Ritirò le mani
    d'istinto. Doveva essere un qualche tipo di spugna a nanomacchine per la pulizia dell'aria. Si avvicinò tendendo
    l'orecchio e riuscì a sentire il rumore soffocato di aspiratori.

    Questo significava che non era quella la direzione da prendere, in realtà, a meno che non volesse trovarsi sparsa
    su tutte le pareti della camera d'aspirazione del sistema di ventilazione centralizzato del palazzo. Tornò indietro
    e proseguì per l'altra direzione. La parete di quel vicolo cieco di cisuro sembrava più solida che non il muro
    spugnoso che aveva incontrato andando verso la camera d'aspirazione. Iniziò ad analizzare i contorni della parete.
    Se aveva ragione quello era il canale d'accesso per la manutenzione, il quale dava direttamente alla sala macchine.
    Gli addetti passavano attraverso quel portellone quando dovevano mettere a posto delle pale bloccate oppure dovevano
    liberarsi di qualche cadavere di animale morto nei condotti.

    Dopo un po' riuscì a notare uno sportellino mimetizzato con la parete, in uno degli angoli tra l'usicta ed il
    corridoio. Non era facile vederlo perchè i sistemi di visione notturna non aiutavano particolarmente a distinguere i
    colori. È vero che quelli di ultima generazione facevano interpolazione di dati per migliorare l'immagine. Ma quando
    non c'è luce per fare sampling dei colori... Non ce n'è, non ci sono cazzi. Perciò dovette far collaborare la sua
    vista col tatto per capire che quello era uno sportello per la manutenzione elettronica e non semplicemente del
    rumore nei suoi occhi generato dall'amplificazione del segnale visivo.

    Rimosse la piccola lamina di metallo usando uno dei cacciaviti che aveva nello zainetto, la quale rivelò un piccolo
    monitor LCD ed un connettore universale. Era un pannello d'accesso per la diagnosi dei sistemi elettronici della
    porta. Aveva delle routine automatiche che gli addetti alle riparazioni potevano usare per controllare se la porta
    funzionava correttamente. Avrebbe dovuto hackare il sistema, visto che, normalmente, questi sistemi avevano un ciclo
    di tests che prevedevano l'apertura e la chiusura dei Lock magnetici in rapida successione. In pratica si sarebbe
    dovuta infiltrare nel sistema di test e bloccarlo a metà di un ciclo, così da avere la porta aperta.

    Estrasse il portatile dallo zaino, lo appoggiò a terra e lo collegò al pannello di controllo. Avrebbe potuto
    connettersi direttamente al portatile o al pannello, ma non voleva rischiare in qualche Active Firewall. Quei cazzo
    di sistemi di protezione friggevano i sistemi elettronici. E se era un cervello Cybernetico... Se l'era cercata.
    Avrebbero dovuto rendere illegale quel tipo di software, sul serio. Era per quello che andava sempre in giro con il
    suo Laptop Rugged Military Grade. Era un po' lenta quando lo usava come proxy. Ma lenta era meglio di morta o
    lobotomizzata.

    Lanciò il sistema che faceva lo sniffing dei pacchetti per vedere quali comandi venivano mandati alla porta. Il
    trucco era lasciare la porta aperta senza far scattare un allarme diagnostico. Perchè, in quel caso, i tizi alla
    reception avrebbero ricevuto il segnle che qualcosa non andava giù nella sala macchine. Quindi fece partire un ciclo
    diagnostico. Venti aperture seguite, ognuna, da una chiusura. Sofia controllò il log dei pacchetti. Ogni richiesta
    di apertura e di chiusura aveva allegato un ID, probabilmente generato a random utilizzando l'ora di sistema precisa
    al nanosecondo. Ogni volta che la serratura veniva aperta o chiusa la porta mandava in risposta un pacchetto con
    un ID simile. Quindi il modo migliore sarebbe stato, in quel caso, di intercettare un pacchetto di chiusura,
    scartarlo e quindi forgiare ed iniettare nel sistema un pacchetto di avvenuta chiusura durante l'ultimo ciclo di test. Oppure mandare una
    richiesta di apertura forgiata. Ma non sapeva quale fosse il seed di generazione dei numeri random, perciò era
    meglio farselo produrre direttamente dal sistema stesso.

    Sofia si divertiva a fare quel tipo di cazzatine. Erano degli hacks super fuffa ma comunque la eccitavano. Perchè
    non importa quanto un hack sia difficile, stai sempre fregando dei Noobs.Agirò le dita sopra la tastiera come i
    maghi prima di tirare fuori i conigli dal loro cappello. Si era sempre chiesto dove li tenessero. Voglio dire, non è
    che avevano uno scompartimento segreto nella tuba e si tenevano un coniglietto dentro tutto il tempo, no? Poteva
    essere che, in uno dei momenti nel qule si buttavano nell'acqua o queste cose così si facessero cambiare il cappello
    apposta con dentro un coniglietto, no?

    Sofia ebbe un sussulto come quando ci si sveglia di scatto subito dopo essersi addormentati sui banchi di scuola.
    Doveva hackare la porta, non è che avesse molto tempo. Finì di scrivere un programma semplicissimo che faceva il
    conteggio dei pacchetti e che ne forgiasse uno utilizzando i dati di quello catturato e poi fece partire un altro
    test run della porta. Questa volta, dopo diciannove \emph{clack} la porta rimase aperta, proprio come previsto.
    Sofia mise il computer in sleep e quindi, scollegato dal pannello di controllo, lo rimise nello zainetto. Appoggiò
    la mano sul portellone ed iniziò a spingere lentamente. Appena ci fu abbastanza spazio per guardare all'esterno
    diede un'occhiata, ma non prima di aver disattivato il sistema di amplificazione visiva.

    La stanza era la solita sala macchine nera con neon e tubi dappertutto di colori differenti per sapere esattamente
    che cosa trasportassero e dove andassero. Passerelle di grata metallica collegavano l'entrata ai vari macchinari.
    C'era un terminale vicino all'entrata. Quello serviva per il controllo delle macchine in caso non fosse servito un
    approccio più \emph{diretto}. Scivolà fuori dal condotto di manutenzione e quindi appoggiò il portellone in modo che
    sembrasse chiuso. Visto che era un portellone di manutenzione probabilmente non aveva il sistema di chiusura
    automatica, così da non chiudere gli addetti alle riparazioni all'interno. Inoltre la pressione dell'aria su quella
    porta era troppo bassa per poterla muovere, quindi no problem.

    Iniziò a camminare sulle grate per dirigersi verso l'altro portellone per la manutenzione, quello che aspirava
    l'aria dalla sala server e quindi la spingeva verso l'esterno. Ci sarebbe voluto un pochino per arrivare fino a lì,
    ddoveva muoversi silenziosamente, il che significava appoggiare prima il tallone per poi spostare tutto il peso sul
    piede. L'aveva imparato giocando la prima volta a Metal Gear Solid 1 e non se lo sarebbe mai dimenticato. Quante
    cazzo di volte aveva dovuto strisciare per lunghissimi tratti di gioco soltanto perchè non avevano un controller
    analogico al tempo? Maledetto Kojima. Lui e le sue trollate.

    Arrivò all'altro portellone. Da quel lato la cosa era più facile. Bastava aprire la porta. La cosa buona di quei
    sistemi era che non disattivavano la pompa d'aria quando qualcuno apriva i condotti di manutenzione, a meno che non
    fossero quelli che davano direttamente sul motore. In quel caso veniva triggerato un allarme (che poi fosse
    necessario o meno farne scattare uno vero stava al sistema di controllo, ovviamente, il quale poteva bypassarlo se
    veniva avvisato che si sarebbe andati a lavorare sul motore), il quale disattivava la pompa per l'aria. Questo non
    accadeva, ovviamente, se c'era qualcuno dentro nel sistema e tentava di entrarci attraverso qualche altra strada,
    come stava per fare lei prima.

    Premette il tasto di rilascio magnetico della porta, la quale, con un rumore mettallico, la informò che era aperta.
    Entrò rapidamente e poi si tirò dietro la porta. Da lì era abbastanza facile. C'erano solo due strade. Una portava
    al solito motore, l'altra ad una grata che dava sulla sala server. Arrivò alla grata ed, usando uno dei suoi
    strumenti, iniziò a svitare le viti della grata dall'interno. Poi si immobilizzò. E se ci fosse stata una telecamera, come
    probabile, all'interno della sala server? Erano cazzi. Avrebbero visto subito la grata muoversi se si fosse trovata
    nel campo d'azione. Si appiattì sulla grata per sbirciare attraverso le fenditure lasciate dalle lamine in alluminio
    che formavano la grata. Non riuscì a vedere nessuna telecamera. Questo significava che, probabilmente, era fuori del
    suo campo visivo. Estrasse uno dei gadget che Valentine le aveva dato prima di partire per la missione. Era una
    telecamera ad alta definizione collegata ad una serie di fible ottiche installate all'interno di un tentacolo
    meccanico. C'era pure una porta universale, questo significava che poteva attaccarla ad una delle sue entrate
    installate nel retro del collo per poter ricevere il feed direttamente sul suo campo visivo. Un po' retrò come
    tecnologia. Adesso la gente figa usava degli swarm di nanomacchine che, liberate nella stanza, inviavano un'immagine
    3D realtime della stanza. Però anche la telecamerina tentacolsa era meglio dello specchietto che avrebbe tirato
    fuori se non l'avesse avuta...

    Attaccò la telecamera e, lentamente, fece uscire il tentacolo meccanico. Iniziò a guardarsi in giro. Riuscì a vedere
    i vari server, posti all'interno di colonne posizionate in ordine regolare all'interno della stanza, l'uscita,
    probabilmente, era nascosta dietro quelle colonne. Girò il tentacolo in modo che riprendesse il soffitto. Capì
    perchè non riusciva a vedere nessuna telecamera. Ce n'era una posizionata esattamente sopra il condotto d'areazione.
    A giudicare dalla direzione aveva una zona cieca posta subito sotto di essa. Avranno deciso che era più importante
    tenere d'occhio quella direzione. Probabilmente c'era l'usita da quella parte. Ritirò il tentacolo dopo aver
    constatato che non ci fossero altre telecamere visibili. Finì di svitare le viti che tenevano chiusa la grata e
    quindi, dopo averla spostata il minimo indispensabile, uscì.

    Era ora di praticare un po' di magia sulla telecamera. La telecamera era collegata alla corrente elettrica ed alla
    linea interna del sistema. A giudicare dall'equipaggiamento era una telecamera multi funzione. Doveva avere anche la
    possibilità di registrare tracce termiche e quant'altro. Per trasferire quel tipo di dati serviva una linea
    digitale. Poteva fare in modo che il sistema interno della telecamera producesse sempre la solita immagine, magari
    utilizzando più di un sample temporale, così da fintare una ripresa realistica. Cercò un connettore di
    programmazione. Uno di quei connettori strani da programmatori di basso livello, tipo forme geometriche impossibili
    con un numero primo di connettori, solo per essere più Hipsters.

    Ed, infatti, eccolo lì. Un connettore a undici pin formato da un esagono con sei connettori ed un rettangolo con
    cinque connettori in fila. Estrasse la sua \emph{Scatola Magica}, che era uno strumento da Mercato Nero al di fuori
    di ogni regolamentazione. Era una specie di convertitore di segnale. Bene o male tutti i connettori per la
    programmazione a basso livello passavano dati nello stesso modo, bisognava solo sapere come venivano impacchettati e
    quale pin trasportava quale informazione. Inoltre avevano sempre un datarate più basso dei connettori universali,
    quindi non c'erano problemi di interleaving o di allineamento mancato dei dati. Bastava fare bit stuffing (riempire
    lo spazio extra di zeri) ed eri a posto. La scatola magica o \emph{ArKn Box} era un sistema creato dai
    Retro-Engineers il quale utilizzava un connettore multiplo. Ti bastava attaccare le prese corrette e poi il database
    interno riusciva a riconoscere il connettore. Grazie a questo, poi, veniva eseguita una conversione della
    comunicazione da connettore esoterico ad un più mainstream Connettore Universale, grazie al quale potevi lavorare
    sul tuo PC senza periferiche costruite da chissà quale Corp in India.

    Sofia cercò i connettori corretti e li attaccò alla telecamera stando attenta a non fare troppa forza per non
    muovere la telecamera. L'ArKn Box lavorò per un secondo e poi il led che indicava lo stato divenne verde. Agganciò
    la Scatola al suo laptop ed iniziò a lavorare. Dopo una serie di tentativi riuscì a creare un applicativo che
    generava delle immagini per la telecamera, aggiungendoci anche altri dati sfruttando il metodo utilizzato con la
    porta, aiutata da una serie di samples che aveva recuperato dalla telecamera durante il suo funzionamento normale.

    Fatto questo potè andare verso il terminale principale e mettersi a lavorare per iniettare le informazioni di
    Valentine e di Arsenicos. Ah, ovviamente attaccò anche la tuta alla presa elettrica per ricaricarla. Non si poteva
    mai sapere.

  \section*{Valentine}

    Era passato abbastanza tempo perchè Sofia avesse fatto il suo lavoro, quindi Valentine guidò il furgone fino alla
    sede della Missing Link. Parcheggiò davanti e, aiutata dal Pivello, scaricò i robot per la pulizie su un carrello.
    Una volta che finirono chiuse il furgone e si avviò verso il palazzo.

    Entrò nella hall e si diresse direttamente dalla guardia alla reception, seguita dal Pivello che stava spingendo il
    carrello. ``Salve.''  fece lei, tirandosi su il cappello dell'impresa di pulizie ``Salve.'' rispose la guardia,
    alzando la testa dal suo tablet.  ``Siamo venuti per fare le pulizie.'' ``Ah, certo, posso vedere le vostre badge?''
    ``Ah, certamente.'' rispose lei e dalla sua tasca estrasse un documento d'identificazione perfettamente
    contraffatto da Sofia. Lo porse alla guardia e poi disse al Pivello ``Oi, dai il documento alla guardia, su.'' Il
    ragazzo si mise a frugare nelle tasche finchè non trovò il tesserino, lo porse alla donna, la quale lo diede alla
    guardia.

    ``Allora, com'è la serata?'' chiese Valentine all'uomo ``Mah.'' rispose lui annoiato, dopo aver passato i badge al lettore
    ``Niente di che. Te?'' ``Due palle. Non vedo l'ora di finire per tornare a casa.'' ``Haha!'' l'uomo scoppiò in una
    risata ``Ce l'avete dura voi delle imprese di pulizia eh?'' ``Ah, beh, almeno non dobbiamo sparare alla gente.''
    ``Vero, vero. O picchiarli. Sai quanto si sporcano i vestiti?'' ``E lo dici a me?'' E scoppiarono entrambi in una
    risata. ``Ma...'' fece la guardia, indicando con la testa il ragazzo ``...il ragazzo che ha che non parla?'' ``Chi?
    Il Pivello? Che ci vuoi fare. È nuovo.'' rispose lei, naturale ``Ah! Haha! Piacere di conoscerti \emph{Pivello}.''
    ``Ma veramente...'' rispose lui ``Mi chiamerei Archibald.''

    Era incredibile la naturalezza con la quale rispose a quella provocazione, come se, ormai, ci si fosse abituato da
    tanto tempo. ``Ma certo ma certo.'' poi guardò al terminale ``Oh, ma quanto ci mette questo coso?'' Ci fu una pausa
    da parte di tutti e tre e poi Valentine se ne uscì con un ``È incredibile, eh? Si cintinua ad andare avanti con 'ste
    cose e ancora non riescono a farle andare in fretta.'' ``Sì, veramente, sul serio.''

    Ma che diavolo stava facendo Sofia?

  \section*{Sofia}

    ``Questo wallpaper fa cagare, cazzo. Ma chi l'ha messo su?''

  \section*{Valentine}

    ``Ah. Ok, perfetto. Potete andare.'' fece la guardia, sorridente, riconsegnando le tessere a Valentine. ``Ottimo!''
    fece lei, rispondendo al sorriso, senza fare una piega. Il Pivello sembrava aver perso colore. ``Ti senti bene
    Pivello?'' chiese la guardia, per poi ridere ``Oh, lascialo stare.'' commentò Valentine ``Questi studenti di oggi.
    Non fanno un cazzo dalla mattina alla sera e poi guarda come si riducono a scaricare una ventina di robot.'' ``Haha!
    Insegnagli, allora.'' ``Ok, allora è meglio se andiamo, altrimenti non finiamo più stasera.'' ``Certo. Buon
    lavoro.'' ``Anche a te. Buona serata.''

    Valentine passò vicino al Pivello e se lo portò via, facendogli spingere il carrello. Arrivarono al montacarichi, ci
    inserirono il carrello e poi entrarono a loro volta. Una volta che il montacarichi iniziò a salire Valentine guardò
    il Pivello e gli fece, sottovoce ``Ma che cazzo fai?'' ``Scusi, signorina Arthur, avevo paura che le tessere non
    funzionassero.'' rispose lui, imbarazzato ``Anche fosse non è quello il modo di reagire, maledizione. Potevamo
    comunque uscirne.'' ``Capito.'' ``Perfetto. Perchè adesso è ora di fare sul serio.''

    Passarono dei secondi di silenzio, in sottofondo giravano le note di una versione easy listening di Fly Me To The
    Moon, e poi Valentine fece ``Ah, sì, pivello.'' ed estrasse dalla borsa che aveva appoggiato sul carrello assieme ai
    robot una H\&K UCP MKII verde militare con mirino laser e gliela lanciò ``Ecco, tieni.''

    Era la prima volta che Valentine dava qualcosa di personalizzato ad Arsenicos, oltre allo stipendio ovviamente.
    ``S...Sono onorato di questo, signorina Arthur!'' fece lui, con gli occhi che sembravano brillargli, tenendo con
    entrambe le mani l'arma. ``Non gli ho fatto montare uno smart pack perchè so che non hai nessun HUD.'' continuò lei
    ``Comunque non abituarti. L'ho fatto solo perchè non hai un'arma, ok?'' ``Sìsì!'' il Pivello maneggiò l'arma per un
    po' ``Comunque, vedi di non usarla in questa missione.'' lo ammonì Valentine, secca ``Sìssignora, signorina
    Arthur.'' ``Non hai il silenziatore. Potrebbe essere un problema. Non sono arrivata a trovarne uno. Avevano solo
    quei pezzi di plastica che durano un caricatore.'' ``Capito.'' ``Tieni.'' fece lei, lanciandogli anche una fondina
    da schiena ``Vedi di nasconderla, ok? Non voglio che vada tutto a puttane.'' ``Farò tutto il necessario.'' e si legò
    la fondina sotto la tuta da addetto, per poi inserirci l'arma da fuoco.

    ``Ok.'' fece quindi la donna ``Tra poco arriviamo al piano subito sotto quello dove tengono la ragazza.'' si fece
    improvvisamente più seria ``Aspettati resistenza anche su questo piano.'' poi lo guardò ``Ricordati. Evita di
    uccidere le guardie, non siamo quì per ammazzare nessuno. Meglio stordirle e lasciarle legate dentro un armadietto o
    cose del genere. Ma se provano ad ucciderti non tirarti indietro, ok?'' ``Certo.'' ``Ottimo.''

    Il montacarichi rallentò gradualmente fino a fermarsi al diciassettesimo piano con un rumore di campanello.
    dall'altra parte della grata c'erano due uomini equipaggiati con SMG e giacca antiproiettile, sulla schiena
    Valentine riuscì a leggere \emph{Adrian's Wall} scritta con una tinta rinfrangente. Doveva essere il gruppo di
    suppporto della Missing Link. I due, mentre Valentine apriva il cancello del montacarichi, si girarono per guardare
    chi fosse salito. Quello vicino a Valentine le si avvicinò e le intimò ``Questo piano è off-limits. Andatevene o
    dovremo usare la forza.'' Valentine, tentando di mantenere la facciata, rispose ``Siamo dell'impresa di pulizia. Ci
    hanno dato da pulire questi piani. È sicuro che non possiamo fare un giro di pulizie?'' ``Impossibile. Ci hanno
    detto che nessuno, al di fuori dei ricercatori, è stato autorizzato.'' intanto la seconda guardia si era avvicinata
    al Pivello ``Eppure io ho quì dei documenti che testimoniano il contrario.'' continuò lei, iniziando a frugare nei
    vestiti. Estrasse un foglio di carta, una ricevuta che aveva stampato da Amazon.co.fruk l'ultima volta che aveva
    ordinato qualcosa e, dopo averlo aperto, coprendo il logo dell'azienda, lo mostrò all'uomo il più vicino al suo
    corpo. ``Visto?'' continuò lei, sicura. L'uomo, per niente convinto, si avvicinò alla donna, per osservare meglio il
    pezzo di carta. Appena fu vicino abbastanza a Valentine lei gli scaricò contro una ginocchiata in volto, subito
    seguita da un rabbit punch alla nuca. Nello stesso momento il Pivello prese il braccio destro dell'altra guardia,
    prima che potesse fare qualsiasi cosa, glielo mise in leva dietro la schiena, gli bloccò la gamba sinistra ed iniziò
    a strangolarlo. Valentine appoggiò a terra la sua guardia e poi si avvicinò agli altri due. Tirò un pugno diretto al
    setto nasale della guardia per accelerare il processo. Il Pivello la appoggiò a terra e poi li legarono usando dei
    lacci per i pollici e del nastro adesivo da riparazioni per la bocca.

    Li appoggiarono sul carrello assieme ai robot e poi lo spinsero nel corridoio. In quel momento Sofia conosceva la
    loro posizione e poteva controllare le telecamere prima che comparissero nel loro campo visivo, in modo che non
    venissero nemmeno mostrati sui monitor della sicurezza. Arrivarono alla porta di uno degli stanzini della
    manutenzione. In qualche secondo il LED a lato della porta che indicava lo stato della serratura divenne verde con
    un leggero \emph{clack}. Valentine entrò nella stanza ed aprì due degli armadietti per gli atrezzi. Aiutata dal
    Pivello mise le due guardie inerti all'interno degli armadietti e, dopo aver attaccato le loro gambe e le braccia al
    fondo dell'armadio con il nastro da riparazioni, li chiuse. Dalla borsa sul bancale estrasse due cinture tattiche
    con delle tasche per tenere l'equipaggiamento. Ne lanciò una al Pivello ed indossò l'altra. Dopo che anche il
    ragazzo finì d'indossare la cintura Valentine uscì dalla stanza e si appiattò contro il muro dall'altro lato del
    corridoio. Una volta che anche il Pivello fu uscito dalla stanza ed ebbe chiuso la porta iniziarono a muoversi in
    direzione della rampa di scale d'emergenza.

    Continuarono per un po', fino ad arrivare poco prima di un incrocio. Valentine riuscì a sentire i passi di una
    guardia che stava per attraversare il corridoio che tagliava quello che stavano percorrendo, perciò si andò a
    schiacciare dietro un mobile posto lungo il muro sulla destra. Il Pivello, visto questo, si mosse per andare a
    posizionarsi nello spazio incavato nel muro a sinistra che portava alla porta di un laboratorio. Valentine si sforzò
    per rallentare il suo respiro. Ci volle più di un minuto perchè la guardia attraversasse l'incrocio e continuasse
    per la sua strada. Valentine uscì dal suo nascondiglio e si avvicinò all'angolo formato dal muro sinistro con
    l'altro corridoio. Sbirciò rapidamente per vedere cosa stesse facendo la guardia, la quale stava continuando a
    camminare lungo il corridoio. Appena questa girò l'angolo Valentine scattò dall'altra parte seguita dal Pivello.

    Arrivarono alla porta che dava sulle scale dopo un paio di cambi di corridoio. Aprirono la porta metallica spingendo
    sul maniglione antipanico e si trovarono in una tromba di scale in cemento illuminata da dei neon posizionati ad
    ogni mezzo piano. Valentine iniziò a salire le scale. Il loro obiettivo si trovava in una delle stanze del piano
    subito sopra a quello. ``Signorina.'' attirò l'attenzione il Pivello, a bassa voce ``Mh?'' chiese lei, non
    fermandosi ``Senta, io ho dei dubbi sulla ragazza.'' ``Ma se non l'hai ancora vista. Già ti fai problemi da
    sottotrama romantica?'' gli chiese lei, ironica ``Aspetti. In che senso?'' poi, dopo aver capito la battuta fece
    ``No, aspetti, intendo in un altro modo.'' ``Su.'' lo incalzò lei, sbrigativa ``Non abbiamo tutto questo tempo. Ma
    non potevi parlarmene prima?'' ``È solo una sensazione ma... Non le sembra un po' troppo perfetta?'' ``Ho detto
    niente sottotrama romantica, Pivallo.'' ``Mh.'' mugugnò lui, abbattuto ``Va bene, ne parleremo dopo.'' ``Perfetto.''
    chiuse il discorso lei, avvicinandosi alla porta che portava al diciottesimo piano.

    Valentine avrebbe voluto controllare cosa ci fosse al di là della porta ma, purtroppo, era una taglia fuoco, quindi
    era sigillata su tutti i lati. Doveva andare alla vecchia maniera. Girò lentamente la maniglia della porta, finchè
    non raggiunse la fine della corsa, poi attivò la tuta mimetica che gli stronzi che li avevano attaccati una
    settimana prima avevano così gentilmente regalato loro. Il Pivello fece lo stesso. Non avevano ancora attivato le
    tute perchè la tecnologia era abbastanza nuova e quindi consumavano una quantità impressionante di energia
    eltettrica. Magari non quando stavano fermi visto che non c'era da aggiornare lo stato dei proiettori che
    ricoprivano la tuta, ma quando si muovevano o, peggio, quando combattevano, la quantità di aggiornamenti richiesti
    bruciava la batteria in pochi minuti.

    Aprì leggermente la porta per vedere se c'era qualcuno nelle sue vicinanze e quindi, constatato che la strada fosse
    libera, creò abbastanza spazio per passare e si mosse nel corridoio, seguita a breve distanza dal Pivello. Il
    ragazzo accompagnò la porta mentre lei andò avanti per contollare come fosse messa la situazione. In quella zona la
    sicurezza non sembrava molto presente, in realtà. Probabilmente avevano concentrato tutte le forze vicino alla
    stanza dove tenevano la ragazza. Oppure... Poteva essere che mister Bella Giacca avesse veramente fatto qualcosa per
    rendergliela più facile.

    La cosa sembrava abbastanza impossibile, quindi Valentine allontanò quel pensiero. Dopo un po' si sentì toccare la
    spalla. Significava che il Pivello l'aveva raggiunta. Non avrebbero capito in altra maniera dove si trovassero l'un
    l'altro se non avessero usato questo metodo, in quanto erano andati in missione disattivando qualunque
    comunicazione. Avevano dei beacon criptati con i quali sapere più o meno la posizione di ognuno. Era grazie a quello
    che Sofia conosceva la loro locazione. Interpolando quei dati con la planimetria della struttura era possibile avere
    un'idea ben precisa di dove si trovassero in ogni momento.

    Disattivarono le tute per risparmiare energia una volta arrivati in una zona calda e quindi proseguirono lungo i
    corridoi. Superarono stando accucciati alcune vetrate che davano su dei laboratori vuoti. Pareva che in quella
    sezione della Missing Link facessero ricerca di robotica e cybernetica. Era un mercato che fruttava. Valentine non
    riusciva a capire perchè tenessero la figlia del CEO quì, ma probabilmente la teoria del \emph{dove meno te lo
    aspetti} poteva funzionare.

    Mentre vagavano per il piano lei ed il Pivello riuscirono a passare oltre quattro o cinque guardie, le quali
    sembravano semplicemente girare il piano in perlustrazione. Le guardie dovevano essere quì da tanto, visto che non
    parevano molto concentrate. Una volta arrivati vicino all'entrata della stanza dove tenevano sotto protezione la
    ragazza notarono due guardie in assetto d'assalto di fronte alla porta. Erano senza elmetti corazzati, ma avevano
    giubbotti che rendevano inutili i colpi al corpo. Valentine attivò il sistema mimetico, subito seguita dal Pivello.
    
    I due avanzarono rapidamente lungo il corridoio, pronti ad attaccare le due guardie. Valentine si precipitò sul
    primo, mentre il pivello assaltò il secondo. La donna attaccò la guardia tirandole un colpo diretto al collo con la
    mano posizionata come se volesse strangolarlo. Questo stordì l'avversario il quale, prima di poter reagire, subì un
    colpo con il palmo della mano dal basso verso l'alto sotto il mento, mettendolo KO. Il Pivello andò per un più
    classico Thunderclap alle orecchie del nemico, per poi sbilanciarlo spazzando via un piede e spingerlo a terra
    usando tutto il suo peso. Appoggiò la mano destra sul volto della guardia e lo spinse verso il pavimento, facendolo
    svenire con un colpo. Valentine non controllò neppure lo stato della guardia che aveva colpito, era in piedi davanti
    alla porta, aspettando per il solito rumore che ne indicava l'apertura da parte di Sofia.

    Non dovette aspettare molto perchè avesse accesso alla stanza. Aprì la porta e trascinò dentro la sua guardia, cosa
    che fece anche il Pivello. La stanza era un grande laboratorio con tutta la strumentazione installata sulle pareti.
    Al centro era presente una specie di lettino a forma di capsula, con gli esterni metallici e gli interni foderati.
    Valentine, prima di cercare la ragazza, chiuse la porta, buttò in un angolo la guardia ed iniziò a spogliarla.
    Recuperò munizioni, SMG, pistola, equipaggiamento, poi la legò. ``Pivello.'' fece, senza neanche distogliere
    l'attenzione dal suo lavoro ``Sìssignora.'' rispose lui, facendo la stessa cosa. Una volta recuperato tutto il
    possibile Valentine iniziò a dare un'occhiata alla strumentazione lungo i muri, mentre il Pivello andò direttamente
    verso la capsula.

  \section*{Arsenicos}

    Il capo era sempre lanciato per il recupero di materiale sul campo. \emph{È gratis}, diceva. Finì di legare la
    guardia che aveva stordito e poi si mise a cercare in giro. Siccome Valentine stava già lavorando sui computer e la
    strumentazione scientifica installati lungo i muri decise di avviarsi verso la capsula, che pareva essere l'unico
    posto dove tenere una ragazza. A meno che questa non fosse uscita ed allora tutta quella missione avrebbe perso
    improvvisamente di senso, no?

    S'immobilizzò appena riuscì a vedere chi c'era dentro. Era Zoè, nessun dubbio. Ragazza sui diciotto-venti, capelli
    neri di media lunghezza mossi. Un metro e sessanta. Ok. Ora, il problema era: perchè diavolo fosse svestita. Non che
    non fosse un bel vedere ma...

    ``Hem, signorina Arthur?'' chiese lui, imbarazzato ``Immagino tu abbia trovato la ragazza, Pivello.''  commentò
    Valentine, senza distogliere lo sguardo dai sistemi, per poi continuare a commentare ``Beh, non era molto difficile
    scoprire dove fosse.'' ``Sì, ma.'' ``Non eccitarti coso, ti vedo sai?'' ``Agli ordini!'' Iniziò a controllare che la
    ragazza stesse bene e che non l'avessero messa su quel lettino perchè le servivano cure specifiche. Anche se avrebbe
    potuto essere un buon motivo per portarla da quelle parti. Guardò attentamente la ragazza, senza toccarla, per poi
    avvicinarsi alla testa, la quale sembrava leggermente sollevata, ma non riusciva a capire da cosa. Avvicinò la mano
    e la fece scivolare gentilmente sotto la testa. Arrivò fino al collo dove parevano esserci dei connettori abbbastanza robusti
    che le entravano nel collo.

    ``Signorina Arthur, credo che la ragazza possa avere un cervello cybernetico.'' commentò lui, senza spostare la
    mano, come per permetterle di appoggiare la testa a qualcosa. Doveva essere faticoso stare in quella posizione.
    ``Mh. Ottimo. Non solo dobbiamo portare via una ragazza, ma questa sta pure tutto il suo tempo in Internet.''
    commentò lei, continuando a cercare. Probabilmente stava tentando di trovare il terminale per disattivare la
    connessione. ``Mh, non lo trova strano?'' ``Che?'' ``Questo. Voglio dire.'' ``Mah. Secondo te se BJ avesse i soldi
    non si costruirebbe un terminale d'accesso così?''

    Arsenicos non riusciva ad immaginarsi una ragazza attiva come Sofia farsi costruire una postazione di connessione
    diretta che la obbligava a stare ferma. Probabilmente avrebbe speso tutti quei soldi per avere uno dei portatili più
    potenti della storia. ``No.'' ``Ah, davvero?'' ``Sì, ne sono abbastanza certo. Inoltre. Perchè la ragazza dovrebbe
    essere svestita?'' ``Beh, perchè.'' Valentine si fermò per un momento per pensare a quello che le era stato detto
    ``SVESTITA?'' ``Affermativo.'' ``Cosa le hai fatto?'' ``Ma che sta dicendo?'' chiese, quasi indignato, Arsenicos e
    poi, girandosi, continuò ``Non mi crederà persona tale da fare una cosa simile, spero.'' ``Ok, ok, scusa. Era una
    battuta.'' poi, tornando seria, ``Comunque, se non riesco a trovare un modo per staccarla non possiamo portarcela
    via. E allora sono cazzi.'' e tornò a cercare tra le strumentazioni.

    Arsenicos notò che anche il resto del corpo sembrava sospeso. Fece scivolare l'altra mano lungo il braccio destro e
    sentì altri connettori, come se avesse avuto delle entrate lungo il braccio. Che fosse stato un modo per avere una
    connessione migliore?

    Dopo qualche minuto si accesero i monitor da computer sparsi in giro per la stanza. Erano presenti una grande
    quantità di dati biometrici, grafici, mappe dei vasi sanguigni. In uno dei monitor vicini a Valentine comparì una
    schermata di un programma di messaggistica istantanea. L'altro utente aveva un nome anonimo. ``Hei!'' c'era scritto
    poi, dopo un secondo, comparì una nuova linea, preceduta dal timestamp, che diceva ``Problems?'' Valentine sembrava
    non sapesse che fare. Che li avessero scoperti? Prima che la donna potesse avvicinarsi alla tastiera di quel
    computer per rispondere sulla chat comparì una nuova linea con una emoticon. L'emoticon era una faccia sorridente
    distorta. Era un meme che arrivava direttamente da 4chan: la cool face, la versione degli Old Fags.

    ``Per DIO, BJ!'' rispose sulla chat Valentine ``Che?'' fu la risposta ``Non farmi prendere un colpo in questa
    maniera.'' ``Ho visto che eravate fermi e mi sono preoccupata.'' ``Ho capito, ma non comparire in quella maniera,
    senza neanche il nome.'' ``Vuoi che mi logghi QUÌ con le mie info di login?'' nuova linea ``Ma sei matta? °\_o''
    ``Hai ragione. Senti, riesci a darmi una mano?'' ``Per la gnocchina alla quale il Pivello sembra essere
    interessato?'' nuova linea di nuovo ``Scusa, ma non mi piacciono particolarmente le Visual Noverls.'' Si stava
    riferendo ai videogames di simulazione d'appuntamento giapponesi ``Mmmh.'' scrisse Valentine, per poi aggiungere in
    una nuova linea ``Intendo sconnetterla. Non riesco a capire dove siano i sistemi.'' nuova linea ``Quì è un casino.''

    ``Ok, NEWFAG. :3'' rispose Sofia, scherzosa ``Adesso mi connetto via SSH...'' ``SSH...'' le fece l'eco Valenine
    fuori dalla chat, sospirando.

    Passarono un paio di minuti senza nessuna attività nè sulla chat nè sugli altri monitor e poi le varie finestre
    contententi i dati iniziarono a chiudersi a cascata, i connettori si ritrassero, le luci nel laboratorio si fecero
    più soffuse, probabilmente per lasciare che gli occhi della ragazza si abituassero alla luce, ed i computer nella
    sala si misero a mancinare dati.
