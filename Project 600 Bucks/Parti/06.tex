\chapter{La preparazione è il 90\% della battaglia. Il resto sono botte di culo.}

  \section*{Sofia}

    Quella sera Sofia aveva dormito stranamente bene, anche se era lontana dalla sua stanza, dalla sua collezione, per
    non parlare di essere in mezzo alla faida di tre Corps. Per non parlare del Bazaar! Oddio.

    In quel momento stava correndo per Parigi alla volta del Bazaar. Fortunatamente lei non era stata tirata dentro
    tutto quel casino. A quanto pareva il piano di Valentine prevedeva l'uso di un Cervello Cybernetico di ultima
    generazione e di pagarlo con parte dei soldi della tessera di credito che le avevano consegnato l'altra sera. Non
    sapeva cosa il capo avesse in mente, ma sembrava figo, quindi decise di seguirla fino in fondo.
    Il piano generale prevedeva di fare il \emph{dump} del cervello cybernetico di Zoè nel caso qualcosa andasse storto.
    Inoltre, a quanto aveva capito, durante il dump avrebbe dovuto agganciare un \emph{demone} al cervello della ragazza
    per fare in modo che continuasse ad uploadare il suo \emph{stream di coscienza} in uno spazio dedicato, affittato
    ad Amazon con un'altra parte dei soldi della tessera. Questo demone doveva avere anche un qualche sistema che facesse lo
    switch dei sistemi del cervello, trasferendo l'intera coscienza ad un altro cervello. Sofia aveva sentito parlare di
    sistemi che potevano fare al caso loro, ma erano ai limiti della legge persino nei Bazaar. Inoltre non sapeva se
    funzionassero veramente. Era vero che era un po' come copiare un programma, farlo partire su un altra macchina, non
    fargli fare nulla e poi, ad un certo punto, fare il \emph{dump} completo dello stato del programma e della CPU,
    pipeline, cache, registri e tutto e trasferirlo su quell'altra macchina, riattivando le operazioni. Era una cosa che
    si faceva con i programmi. Solo che con un Cervello Cybernetico... C'era da sperare che non servisse.

    Inoltre il Newb e Catherine, che non erano stati visti la sera prima, sarebbero dovuti andare alla base a
    controllare quale fosse lo stato e se fosse possibile accedere al Frame. In quel caso sarebbero dovuti entrare a
    mettere a rendere operativo il più in fretta possibile ATHENA e poi... Lasciarlo lì. Ovviamente non potevano
    girarsene con un veicolo metallico di quindici metri di giorno sperando di essere stealth. Sarebbe servito per
    l'ultima parte del piano. Durante la giornata Valentine avrebbe \emph{spammato} i capi di Missing Link, Chevalier e
    Silicon Knights che, in realtà, il Team avesse già rubato la tecnologia che tutti volevano e che l'avrebbe scambiata
    per Zoè (quello solo con gli agenti dei Silicon Knights) e, soprattutto, per venir lasciati in pace il giorno dopo
    al Bazaar, di nuovo. Era un piano dai tempi così limitati che Sofia non riusciva a capire come avrebbe fatto a
    funzionare, soprattutto sapendo che il Bazaar poteva mettersi in mezzo, visto che, probabilmente, sicome il Newb
    faceva parte del gruppo, avevano coniserato tutti responsabili.

    Comunque quello era l'unico piano che avevano in quel momento, quindi stare lì a farsi dei dubbi era inutile. Sofia
    arrivò al Bazaar, nel quale c'era un'atmosfera molto più rilassata, soprattutto senza tutti gli Armoured Frames
    nelle strade circostanti. Ne aveva visti uno o due dei gruppi di ieri sera, in realtà, andando in giro per la città,
    ma non vicino al Mercato. Entrata iniziò a vagare per le strade. Lei era sempre passata solo nella sezione
    elettronica ed in quella degli Otaku, non era mai andata in quella dei componenti Cybernetici, che era quella anche
    dei medici usati dai gruppi senza soldi che non volevano andare negli ospedali pubblici, visto che sapeva già di
    avere l'ultima versione installata, quindi non era proprio spinta ad andare a vedere quali fossero le ultimissime
    offerte.

    Il settore medico era, probabilmente, uno dei più strani. Le decorazioni e l'architettura era un vero melting pot,
    con le facciate dei negozi che cambiavano radicalmente stile una dall'altra, anche se erano attigue. C'erano
    praticanti di Ayurveda, Chiroprati, Streghe Wiccane, Agopunturisti, Antichi Maestri del Feng-Shui, Druidi,
    Sciamani. Ed ogni negozio, naturalmente, aveva il suo stile perchè tutto influisce nella cura del paziente,
    ovviamente. In mezzo a tutto questo marasma, comunque, un buon cinquanta percento delle attività erano di dottori
    standard che non avevano ricevuto un posto da qualche Corp medica, perchè potevano fare più soldi in quell'ambiente
    o perchè non erano proprio limitati dalla morale del sistema medico internazionale. I dottori, tra l'altro, potevano
    essere anche esperti d'ingegneria Cybernetica o delle Nanomacchine, vista la quantità di persone con impianti
    Cybernetici (che non fossero il cervello o altre parti di questo tipo). Sarebbe stata dura trovare un Cervello
    Cybernetico di ultima generazione che non fosse contraffatto. Iniziò a chiedere in giro, dicendo che era interessata
    ad un'installazione e che voleva sapere i prezzi e dove poter reperire la componentistica.

    Le ci vollero due ore per iniziare ad avere delle informazioni serie. E la cosa incredibile era che gli erano state
    fornite da una Strega Wiccana \emph{riformata}, la quale credeva che la tecnologia era entrata a far parte della
    natura durante le ultime Guerre, soprattutto in Europa, dove le nanomacchine sparse nell'atmosfera avevano
    contribuito al ritorno alla natura delle città, trasformando appunto il vecchio continente nel nuovo polmone verde
    del Mondo. La \emph{Strega} le aveva detto che avrebbe potuto trovare quel tipo di protesi da Macchiavelli-dono.
    \emph{Macchiavelli}. Macchiavelli! Cazzo. Gente che poteva cambiarsi il cognome. Ma non poteva sceglierne un altro
    se voleva fare quel lavoro? Fatto stava che il signor Macchiavelli era gran sacerdote Shintoista che, oltre a
    praticare qualche tipo di massaggio giapponese commerciava in componentistoca Cybernetica di alta qualità. Sofia non
    riusciva a capire benissimo cosa c'entrassero le due cose.

    Non che avesse capito la storia del Wiccanesimo riformato, in realtà.

    Comunque decise di seguire le indicazioni, le quali portavano oltre l'ultimo piano di uno dei palazzi della sezione
    medica del Bazaar. Dopo essere arrivata all'ultimo livello di negozi doveva prendere una rampa di scale in legno che
    salivano sul tetto, sotto la cupola che chiudeva il mercato, le quali passavano sotto un Tori di legno rosso
    laccato. Sul tetto era piazzato uno...

    ``Stramadelettissimo tempio Shintoista.'' commentò lei, una volta finita la rampa di scale. Ok, quella era la cosa
    più epica che avesse visto nell'ultimo periodo. Anche più di Gurren Lagann. Era un tempio shintoista. Tradizionale.
    Con il giardino intorno. Sul tetto di una costruzione commerciale. MIND BLOWN. Soprattutto per lei che era fan del
    Giappone, in realtà. Tra l'altro c'era un foro nella cupola subito sopra il tempio, proprio come sulle piazze, per
    far vedere il cielo. Cose mistiche.

    No, aspetta. Sarà una cosa figa, ma non è che è una cosa così tipicamente giapponese, no? Voglio dire. Mont Saint
    Michel. Una chiesa in cima ad una specie di collina fatta di negozi in mezzo al mare. E ci avevano ambientato una
    parte di Onimusha 3. MIND BLOWN. Tra l'altro Mont Saint Michel, dopo la Terza Guerra Mondiale era diventata uno
    degli HQ per la Corp della Chiesa Cattolica. Posto strano.

    A Sofia piacevano le architetture strane. Si fermò per registrare delle immagini nella sua memoria esterna, come una
    chiavetta dati per tenerci dento i ricordi che vuoi essere sicuro di poter rivedere. ``Nice.'' commentò dopo aver
    registrato una serie d'immagini, per poi avviarsi verso l'entrata del tempio. Salì la corta rampa di scale che dava
    sulla passatoia che circondava la costruzione e poi si avviò verso l'entrata. Superò l'entrata e, nello spazio
    subito attiguo alla porta, si tolse le scarpe prima di salire sul pavimento in legno lucido. Non ci volle molto
    prima che arrivasse un uomo alto quasi due metri, capelli corti brizzolati e Van-Dyke ben tenuto, che indossava una tenuta da sacerdote shintoista ad accoglierla.
    ``Salve.'' fece lui ``Salve.'' ripose Sofia, stupita. Si sarebbe quasi aspettata che il tizio parlasse in
    giapponese. ``Cosa la porta quì, signorina?'' chiese lui, andando al punto ``Ah. Hem.'' lei si guardò in giro,
    sperduta ``De-Devo chiamarla in qualche modo particolare? Prostrarmi a terra o cose così?'' ``Oh, no, non ti
    preoccupare. Non sono mica così formale.'' ripose lui, sorridendo. ``Starei cercando hem. Macchiavelli-dono, sa mica
    dove...'' chiese lei, non sapendo bene come riferisi all'uomo ``Sono io.'' rispose lui, senza che Sofia andasse
    avanti. ``Ah.'' ripose lei, un po' delusa. Come faceva uno del genere ad essere diventato un sacerdote Shintoista e
    non un... Sacerdote Asgardiano? Uno degli svariati esempi del perchè il detto \emph{Non giudicare il libro dalla
    Copertina} funzionasse sul serio.

    ``Di cosa hai bisogno, ragazza? Non mi pari qualcuno che ha bisogno d'aiuto per entrare in contatto con le divinità
    tecnologiche.'' Oh. Cazzo. ``Hem...'' mugugnò lei, non sapendo come rispondere, poi decise di dire, semplicemente,
    la verità. Non era brava con le contrattazioni. Non sapeva raggirare le persone come Valentine ``...Io sarei venuta
    perchè ho sentito che lei può aiutarmi a trovare un cervello Cybernetico di ultima generazione.'' l'uomo la guardò
    per qualche secondo e poi chiese ``Perchè una ragazza che dispone di un Lem-Tec Synapse versione 13 ha bisogno di un
    cervello Cybernetico? Di sicuro il tuo è così bleeding edge che quelli che puoi trovare quì sembrano preistorici.''

    Ok. Come minchia aveva fatto a scoprire il modello del suo cervello? Era stata hackata? Istintivamente andò a
    passarsi le dita dietro il collo per vedere che non ci fosse nulla di attaccato. Il sacerdote notò la cosa e
    commentò ``Non ti preoccupare, ragazza vicina alle divinità tecnologiche.'' lei alzò lo sguardo per guardarlo in
    faccia ``Non mi serve hackarti per scoprire che modello monti. Tutto ciò che ci circonda parla del divino.'' indicò
    i suoi occhi ``Come la leggerissima vibrazione che hanno i tuoi occhi quando tentano di adattarsi alla luce. È un
    sistema che hanno aggiunto nei nuovi prototipi della Lem-Tec per alleviare gli effetti dell'adattamento alla luce.
    Serve per fare multisampling con differenti angoli d'impatto della luce, per ridurre l'irradiamento, e poi fanno il
    merging con tecniche HDR. Ho letto un paper su questa tecnica. E siccome non penso che ci sia nessuno di così folle da girare con un cervello che è
    ancora in fase di Beta allora rimane solo il Synapse 13.''

    Ok, doveva ammeterlo. Il tizio era stradannatamente bravo. Riconoscere un cervello dalle features non era cosa da
    poco. Perchè diavolo si trovava lì invece che essere in una qualche azienda? ``Ma vieni. Siediti quì, arrivo subito
    con qualcosa, così che possiamo parlare di ciò che cerchi.'' Disse lui, indicandole la strada. Lei lo seguì fino a
    che non arrivarono in una stanza che dava sul giardino esterno con un tavolino basso. ``Grazie.'' fece lei, per poi
    andare a sedersi, inginocchiata, al tavolo. Lo guardò perplessa andare via.

    Non riusciva ancora a capacitarsi della persona che aveva incontrato. Poi la sua mente vagò verso un'altra cosa che
    l'aveva lasciata stupita, ovvero come l'avesse zittita Valentine l'altra sera. Non aveva senso...

  \section*{Sofia - 3 capitoli fa}

    La richiesta di Valentine di cercare informazioni sulla ragazza e sull'impiegato della Missing Link l'aveva
    distolta dalla lettura de \emph{The Rollers}, ma ora aveva risolto e poteva tornare a farsi un po' i fatti suoi.
    Avrebbe dovuto ripassare ancora il piano ed ad uploadare le mappe sulla memoria esterna, però per ora avrebbe solo
    letto quel manga e bevuto un drink.

    Arrivò oltre la metà. A quel punto si viene a scoprire che i Rollers ancora vivi si sono separati, inseguendo ciò
    che avevano sempre voluto fare, però sempre di più c'erano segni che, in realtà, al gruppo mancava l'azione di un
    tempo. Intanto Hax era arrivato al cospetto di Eclipse, la quale era un personaggio abbastanza strano.
    ``Mihihihi...'' rise sotto i baffi Sofia, leggendo la parte dove Hax stava per uscire dalla sala di Eclipse,
    dubbioso, e la dea gli disse ``Fuori di quì, dai.''

    Fu a quel punto che rivcevette un'altra chiamata, sempre da Sofia. Non un'altra ricerca. ``Sì?'' rispose lei ``Ciao
    Sofia. Senti, ho degli ordini aggiuntivi solo per te per la missione a Londra. Non dir nulla agli altri.'' ``Eh?''
    la ragazza si guardò il giro per vedere che non ci fosse nessuno. Come se potessero sentire la comunicazione, visto
    che passava tutto attraverso circuiti interni ``Dimmi Capo.'' ``Allora, ascolta. Voglio che tu, prima di andartene
    dalla Missing Link, lasci delle tracce mal nascoste che ci riconducono al luogo ed all'ora dello scambio.
    Nessun'altra informazione su di noi. Solo quello.''

    Ma che richiesta era? Era una missione sotto copertura! ``Scusa, Valentine. Temo di non aver capito. Mi stai
    veramente chiedendo di lasciare delle tracce in modo che sappiano dove andiamo a consegnare la ragazza che salviamo
    da loro?'' ``Sofia, la scena è cambiata. Non posso spiegarti moltissimo, ma ora ci paga la Missink Link.'' ``COSA?''
    Sofia urlò al comunicatore. Da quando facevano doppio gioco? ``Ok, calma. Il problema è che non so neanche se sia
    veramente la Missing Link.'' ``Tutto questo è ridicolo. È perchè la ragazza non è della Chevalier?'' ``Ascolta.
    Proprio per tentare di capire ti ho chiesto di fare quello. Se quelli che ci pagano veramente sono della Missing
    Link allora non ci sono problemi se trovano dei nostri dati, no?'' ``Mh. Sì.'' ``Se, invece, sono degli altri allora
    la Missing Link aggiungerà un fattore di disturbo extra.'' ``Scusa. Mi stai dicendo che stiamo andando a fare un
    lavoro per un gruppo che non sappiamo neanche chi sia?'' ``No, stiamo andando lì a capire che diavolo sta
    succedendo.''

    ``Per la giustizia.'' aggiunse Valentine, prima di salutarla e metterle giù.

  \section*{Sofia}

    Solo che ieri se l'era presa a morte. Che diavolo aveva in testa?

    Dopo una decina di minuti il sacerdote tornò con un vassoio che trasportava del tè e dei dolci. Lo appoggiò sul
    tavolino e si mise al lato opposto rispetto a Sofia. ``Allora.'' iniziò lui, passandole una tazza con entrambe le
    mani ``Parlami di questo artefatto divino che stai cercando.'' ``Artefatto?'' chiese lei, sempre più confusa da
    tutto quello che aveva visto in poco tempo ``Sì, esatto.'' ``Hem. Credo non non riuscire a seguire.'' ``Cosa non
    riesci a seguire, mia cara?'' ``Tipo. Che c'entra lo Shintoismo con le protesi cybernetiche, esattamente?''

    Macchiavelli-dono prese la sua tazza dal vassoio e l'appoggiò di fronte a se. Poi iniziò a spiegare, con calma
    ``Vedi, per lo Shintoismo tutto ha una propria divinità dedicata. Un giardino pubblico, una casa, i suoi mobili,
    probabilmente ogni utensile. Ogni volta che viene creata una nuova tecnologia si generano una moltitudine di nuove
    divinità. Nel caso delle protesi Cybernetiche, visto il livello di integrazione che questi oggetti hanno con il
    corpo umano, bisogna fare in modo che le persone che ricevono questi oggetti siano in sincronia con le divinità che
    inabitano la componentistica. Per questo il mio tempio è dedicato specificatamente a quello: a far passare
    attraverso dei riti di purificazione coloro che riceveranno una protesi Cybernetica. Ed, incidentalmente, vendere
    loro i migliori componenti, scegliendo quelli che più sono in risonanza con la loro anima.''

    Oh, minchia. Come le NanoMacchine per le Wiccane. ``Fate calibrazioni pre installazione e fisioterapia?'' chiese
    lei, tentando di riportare il discorso ad un livello terreno ``Quello che voi chiamate calibrazione noi lo chiamiamo
    purificazione.'' rispose lui, senza fare una piega, per poi aggiungere ``Anche se qualcosa di differente c'è. Fatto
    sta che il ratio di installazioni andate a buon fine di questo tempio è del 99.95\%'' Cazzo. Quel livello di
    precisione, se era vera, era quasi più alto della possibilità che l'installazione di un nuovo cervello su un
    paziente come lei andasse a buon fine. Normalmente c'era una possibilità del 75\% circa di rifiuto da parte del
    corpo.

    ``Tutto questo è molto interessante. Ma... Io sto cercando un cervello per quacuno che non sono io.'' commentò lei
    ``Sì, come ti ho detto prima non me ne stupisco particolarmente. Una come te non ha bisogno di nuove protesi.''
    ``Già. Purtroppo non posso far venire quì la ragazza, perciò...'' ``Basterà qualche informazione, tenterò di trovare
    il componente con le divinità più adatte.'' ``Se lo dice lei.''

    Quella trattativa sarebbe andata avanti un bel po'.

  \section*{Catherine}

    Catherine aveva preso Arsenicos ed erano andati alla base per mettere a posto l'Armoured Frame e per recuperare il
    materiale necessario per lo scambio che si sarebbe tenuto il giorno successivo. Dopo aver parcheggiato il monovolume
    nel solito parcheggio la donna scese, seguita dal ragazzo. Pareva che non avessero ancora trovato quel posto, il che
    era un bene. Però comunque, entrando, controllò la presenza di trappole. Potevano aver minato il luogo per
    eliminarli. Non sembrava una tattica strana per quella gente.
    
    Ci misero un po' vista la cura con la quale Catherine controllò ogni angolo ed ogni punto dove sarebbero dovuti
    passare tra due oggetti abbastanza vicini tra di loro. Inoltre la donna, prima di passare al lavoro sull'Armoured
    Frame, controllò anche l'entrata negli uffici, anche se sembrava improbabile che avessero aggiunto trappole lì senza
    metterne neanche una finta per distrarli. Una volta constatato che nessuno avesse piazzato qualche sorpresa
    all'interno della base Catherine si rivolse ad Arsenicos ``Ok ragazzo. Io mi metto a finire le calibrazioni. Tu vai
    a recuperare quello che ti hanno richiesto che poi ho bisogno che tu faccia un giro di controlli diagnostici mentre
    stai all'interno.'' ``Sissignora, signora Mac an Bahird.'' ripose lui, immediatamente ``Ottimo.'' e quindi scese le
    scale.

    Superò i vari pezzi di Armoured Frames che avevano razziato dalla missione precedente e si avvicinò alla
    strumentazione. Il braccio nuovo di ATHENA aveva bisogno di una riverniciata, anche perchè in quel momento era un
    patchwork di componenti presi da veicoli differenti. Era ancora appeso ai cavi che lo reggevano al soffitto per
    poterlo collegare senza dover far sdraiare il Frame. Comunque l'importante era di finire le procedure per l'aggancio
    e quindi fare le calibrazioni finali. Aveva già fatto girare delle simulazioni che avevano dimostrato che tutti i
    tensori interni funzionavano correttamente. Ora doveva vedere come la prendevano i microprocessori dedicati una
    volta ricevuti i comandi diretti da parte del processore principale.

  \section*{Arsenicos}

    Arsenicos entrò nella sezione degli uffici. Quello che doveva recuperare erano altre munizioni, delle granate
    fumogene per la signorina Arthur, un sacco di gadget. Decise di recuperare uno dei borsoni in tessuto nero e di
    buttarci dentro la roba. Recuperò anche il fucile di precisione di Valentine, probabilmente le sarebbe servito. Oh.
    Ed il fucile a pompa d'assalto di Catherine, che aveva lasciato lì. Si era presa solo il set di lanciamissili per
    la sera precedente. Saggiò il peso della borsa. Faceva ancora un po' di fatica a causa della scarica elettrica.
    Doveva avergli bruciato un tot di fibra muscolare in giro per il corpo. Avrebbe dovuto fare dell'attività extra per
    recuperare.

    Poi si ricordò che doveva recuperare anche l'arma di Sofia. Non sapeva ne avesse una. Andò a controllare dove gli
    aveva detto. Si avvicinò al divano, per poi accucciarsi e frugare sotto di esso con il braccio. Ad un tratto toccò
    qualcosa di metallico. Lo afferrò e tirò verso di sè. L'oggetto si staccò con un rumore di scotch strappato.
    Arsenicos estrasse l'arma e la sollevò per vedere di cosa si trattasse.

    Un piede di porco. Un piede di porco con un'estremità dipinta d'arancione. ``Eh?'' si chiese, perplesso. Sofia? Con
    un'arma del genere? Si sarebbe aspettato qualche taser, qualche arma a distanza. Non un piede di porco. Decise di
    non pensarci su troppo e lo gettò dentro nella borsa assieme al resto.

    Poi gli venne in mente. Sofia gli aveva chiesto un'altra arma, in effetti. Solo che i pacchi non vengono lasciati
    davanti al capannone, ma in una specie di discarica abbandonata sul retro. Decise di andare a controllare. Lasciò lì
    la borsa e si diresse verso l'uscita secondaria degli uffici che dava su di una scala metallica installata sul lato
    del magazzino. Uscì e notò che c'era una cassa gigandesca, della dimensione di una macchina. Non poteva essere
    l'arma che aveva richiesto. Era per soldati a piedi, come faceva ad occupare tutto quello spazio? Poteva essere anche
    solo spazzatura, in realtà, visto che spesso la gente la usava esattamente per il suo scopo di copertura, però non
    gli ci voleva nulla per controllare.

    Scese le scale metalliche in grata di corsa, per poi avvicinarsi alla cassa. Mentre scendeva gli era venuta paura
    che potesse essere una trappola. Una cassa con quella quantità di esplosivo poteva tirare giù quasi tutte le
    costruzioni attigue. Si avvicinò con circospezione e controllò che non ci fossero segni che indicassero la presenza
    di una trappola. Niente cavi, niente fori sospetti da cui potessero uscire fasci laser per sensori di distanza,
    niente sensori di vibrazione.

    Anche perchè, in quel caso, sarebbe già morto.

    Su uno dei lati c'era attaccata una busta di plastica contenente dei documenti. Sopra erano impresse proprio le
    direzioni per portare la cassa quì, non c'era dubbio. Probabilmente era qualcuno che conosceva il posto. Bussò sulla
    parete di metallo che dava sull'hangar. Dopo qualche secondo si aprì una porta dalla quale uscì la signora Mac an
    Bahird. ``Che c'è?'' chiese lei, per poi esclamare ``Oh, buon Dio!'' si avvicinò alla cassa e, dopo averla
    esaminata, esclamò ``E così il Doc ce l'ha fatta a mandare \emph{Fragarch}. Adoro quell'uomo.'' e poi tornò dentro.

    ``Fragarch?'' chiese Arsenicos, che non aveva la minima idea di che fosse. Da dentro si sentirono dei rumori di
    macchinari. Dopo qualche minuto si aprirono i portelloni alti come tutto l'hangar e ne uscì la donna a bordo di un
    montacarichi. Recuperò la cassa e, dopo aver fatto manovra, lo portò all'interno. Arsenicos fece per seguirla
    all'interno ma la sua attenzione fu catturata da una cassa metallica incassata nel ciarpame che era presente a
    terra. Era grande anche quella: lunga due metri con una sezione di uno per due. Sopra era attaccata una lettera
    messa in una busta di plastica. Corse dentro. ``Signora Mac an Bahird.'' disse lui, una volta vicino abbastanza
    ``Sì?''

    La donna stava iniziando a smontare la cassa, dopo aver staccato la busta di plastica contenenti i documenti. ``Ho
    trovato un'altra cassa. Questa volta è in metallo.'' ``Ce la fai a portarla dentro?'' ``Dipende da quanto pesa.''
    ``Ok, tu vai. E controlla in giro, che magari ti sei perso qualcos'altro.'' e poi, quando lui fu qualche metro più
    in là ``Ma che è oggi? Natale?''
    
    Arsenicos tornò nello spiazzo pieno di rottami e guardò meglio. Dopo una decina di minuti arrivò in uno degli
    angoli. Mimetizzato tra i rifiuti c'era una valigia mimetica in metallo. Quella non aveva nessuna indicazione, però
    era tenuta troppo bene per essere della spazzatura vera. E anche non fosse stato l'avrebbe potuta utilizzare per
    portare in giro le armi. È sempre molto professionare avere la propria valigia quando si va in missione. La
    trasportò vicino alla cassa metallica impiantata per terra, probabilmente spinta giù da quella più grande, e tentò
    di tirarla fuori. Dopo averla guardata bene si mise a frugare su uno dei lati più sottili in cerca di una maniglia
    per sollevarla. Riuscì ad aggrappare qualcosa, solo che non poteva fare forza mentre era sdraiato. Aveva bisogno di
    qualcos'altro per fare leva. Recuperò un tubo di metallo che spuntava da uno dei cumuli di rottami lì intorno.

    Tornò alla scatola e fece scivolare il tubo nello spazio tra il \emph{terreno} ed il contenitore, andandolo ad
    incastrare ad una delle manglie. Una volta trovato un punto d'appoggio tentò di fare forza verso il basso sul tubo.
    Qualcosa si stava smuovendo, però non era ancora abbastanza. Iniziò allora a spingere in maniera ritmica, sperando
    di smuovere i rottami presenti dall'altro lato della scatola. Dopo una decina di queste spinte sentì che la scatola
    si muoveva abbastanza per provare a farla girara. Caricò le gambe e strinse il palo, per poi spiccare un leggero
    salto. Una volta atterrato spinse verso il basso con tutta la forza che aveva. Il tubo si piegò leggermente per lo
    sforzo, però la scatola sembrava muoversi.

    Ad un certo punto la scatola scatto verso l'alto ed il tubo si spaccò a metà, finendo in testa ad Arsenicos. Lui,
    signorilmente, imprecò in silenzio tenendosi il punto d'impatto con entrambe le mani. Rimase accucciato per un po',
    per rendersi conto della cazzata che aveva fatto, per poi alzarsi e girare per un po' lamentandosi ed inspirando tra
    i denti per il male. ``Che stai facendo, ragazzo?'' chiese Catherine, che era andata a controllare cosa fosse stato
    quel rumore ``Ah...'' rispose lui, sofferente ``Nulla.'' ``Mh?'' non sembrava convinta ``Ho sbattuto il ginocchio.''
    ``Capito. Stai più attento, la prossima volta.'' ``Hai. Sìsì, grazie.'' Non poteva dirle che aveva fatto una
    stupidaggine del genere.

    Una volta ripreso dal dolore si avvicinò alla scatola, che ora aveva la maniglia in bella vista, la prese ed
    trascinò fuori dal buco la cassa. La cassa aveva un lock elettronico a metà, il quale permetteva di aprirla a metà sul lato
    lungo. Siccome pesava un tot andò a prendere un carrello e le la roveciò delicatamente sopra con la serratura verso
    l'alto. Ci appoggiò sopra anche la valigia mimetica e poi spinse il carrello nel punto dove stava lavorando la
    signora Mac an Bahird.

    ``Ecco quì.'' disse lui, ancora con le mani sul carrello ``Ottimo.'' rispose lei, che intanto aveva quasi finito di
    aprire la cassa. All'interno si poteva vedere la silouette di una pistola per Frames. La donna si avvicinò alla
    valigia. ``Mmmmh...'' mugugnò, guardandola ``Questa non ha neanche un'etichetta. Sei sicuro sia per noi?'' ``Non
    proprio, ma sembrava messa troppo bene per essere un rifiuto.'' ``Ok, darci un'occhiata non costa nulla.''

    Catherine ruotò la valigia in modo da avere le serrature rivolte verso di lei e le fece scattare. Sollevò il
    coperchio, il quale si fermò a formare un angolo retto con la parte sottostante, e si fermò ad osservare il
    contenuto. ``Mh'' disse, sorpresa ``Un gioiellino.'' poi, girando la valigia, chiese ad Arsenicos ``Tu ne sai
    nulla?''

    All'interno era presente un fucile squadrato quasi completamente in metallo, con il corpo principale formato da due
    parti che dividevano il corpo per il lungo, le quali si congiungevano nell'ultimo quarto. Sulla parte superiore
    dell'arma era locata una valvola con vite e sul lato superiore dell'arma era montata un accessory rail. L'impungatura
    sporgeva nella parte finale, mentre più o meno a due terzi del corpo verso la fine della canna, se così
    poteva chiamarsi, c'era un'impugnatura per appoggiare l'altra mano. Il design dell'arma faceva in modo che fosse
    utilizzabile sia da destri che da mancini. Nella parte superiore della valigia c'erano tre caricatori, due più
    grandi, lunghi all'incirca una spanna, in metallo color antracite, mentre il terzo era largo la metà, sempre in
    metallo scuro, ma decorato con bande diagonali azzurre. Oltre a questo era presente un computer balistico da
    collegare alla Porta Universale locata su entrambe i lati dell'arma, un mirino telescopico anch'esso collegabile
    attraverso Porta Universale e due bombolette azzurre con una banda sulla circonferenza che indicava la presenza di
    azoto liquido nei contenitori. L'arma era abbastanza pesante, probabilmente per aumentare la stabilità, però il
    design faceva in modo che non fosse assurdamente lunga, visto che gran parte della canna era integrata nel corpo
    dell'arma.

    Era il WA 2100 che Sofia aveva richiesto. Dentro c'era un biglietto che riportava semplicemente <<Combattere per la
    Giustizia richiede forza di volontà. Ed equipaggiamenti per sostenerla.>> Arsenicos sorrise. Il suo Mentore durante
    gli anni di accademia era sempre il solito. Il ragazzo provò a maneggiare l'arma senza installare nulla, però notò
    che non faceva per lui. Era abituato alle pistole, non ai fucili di precisione. Si girò per rimettere l'arma a posto
    quando notò che Catherine, la quale aveva aperto la cassa metallica, era ferma, con gli occhi pieni di stupore.

  \section*{Il Doc - 3 capitoli fa}

    Catherine se n'era andata e lui poteva tornare a fare quello che stava facendo prima.

    Finiire quel livello di Ninja Gaiden 3 in Master Ninja. Fuck Yeah. Oppure vedere se riusciva a prendere A o S (GASP)
    ad Ikaruga. Mentre camminava in giro per il suo laboratorio
    continuando a pensare che, in effetti, avrebbe dovuto cambiare l'architettura kitsch di quel posto, sghignazzò sotto
    i baffi.

    Ah, quella tipa era così tsundere per lui. Oh, wait. Non è che era lui quello tsundere? Lì c'era sempre il problema
    di riuscire a discernere da punto di vista in prima persona e terza persona. In una serie anime riesci sempre a
    riconoscere quale tipo di archetipo segue un personaggio in quanto vedi la scena dall'esterno, ma se sei dentro
    l'azione.... Magari sei TE quel tipo di personaggio. Come fai ad essere sicuro? ``ODDIO!'' urlò in mezzo ad uno dei
    corridoi guardando in aria ``Metti che lei è una delle puttane psicopatiche!'' ipotesi temibile ``Nonono.
    Probabilmente è più tipo Hitagi di Bakemonogatari. Sìsì.'' si era perso via nei suoi discorsi.

    Poi si accorse di essere arrivato al laboratorio. Doveva mandare l'arma. Mh. Fargarn... Frang... Fanagr... Cazzo.
    Fortunatamente aveva scritto il nome su un pezzo di carta. Salì fino al piano con il prototipo del Tekko e recuperò
    il pezzetto di carta che aveva lasciato appoggiato al ripiano con l'arma. ``Ah. Fragarch! Ottimo.'' esclamò lui,
    sollevando il biglietto. Poi si girò ad osservare l'arma.

    Perchè Catherine se l'era presa, poi? Avrebbe provato l'arma in uno o due giorni, alla fine.

    Poi andò ad uno dei terminali, fece il login, aprì il browser internet e navigò fino alla compagnia di trasporti che
    usava per mandare le casse al Team. Lì inserì le informazioni di connessione ed aprì lo stato della spedizione del
    Tekko con munizioni. Era ancora nei magazzini. Premette il tasto per aggiornare la pagina.

    Nessun cambiamento. Ed ora? Niente.

    Sarebbe potuto stare lì tutto il giorno ad osservare.

    Aggiorna. Mh.

    ``Oh, beh. Ora di Dark Souls...'' commentò lui, allontanandosi, dimenticandosi di Ninja Gaiden ed Ikaruga.

  \section*{Catherine}

    ``Ok.'' commentò tra se e se Catherine, dopo essersi ripresa, continuando a guardare il Tekko e le munizioni
    all'interno della cassa ``E con questo temo di doverlo veramente invitare fuori a cena adesso. Mi pare il minimo.''
    ``Mh?'' chiese Arsenicos, mentre rimetteva a posto il fucile. ``Oh, nulla, Arsenicos.'' rispose lei, richiudendo la
    scatola ``Ora è meglio se torniamo a fare quello per il quale siamo quì così possiamo andare a portare questa roba
    al The Rose and the Crown. Non voglio nulla venga lasciato al caso per domani.'' ``Sissignora, signora Mac an
    Bahird!'' rispose lui, prontamente, per poi dirigersi verso le scale.

    Catherine chiuse i portelloni che davano sullo spiazzo dietro il magazzino e tornò a lavorare al Frame. Ora era
    ancora più lanciata.

  \section*{Sofia}

    Ma che diavolo ci faceva lì?

    Dopo aver preso il tè ed il dolcetti con il sacerdote erano andati nella costruzione dietro il tempio, la quale
    conteneva un magazzino di componentistica cybernetica, solo che ogni singolo componente era contenuto in un
    cassettone in legno laccato. Dentro il componente era appoggiato su una specie di cuscino in velluto rosso. Aveva
    passato l'ultima mezz'ora a descrivere la ragazza. Macchiavelli-dono aveva ascoltato e poi si era messo a pensare.
    Alla fine l'aveva portata là per andare a recuperare il componente di cui lei aveva bisogno.

    Non quello che voleva.

    ``Mh, a giudicare dalla descrizione che mi hai fatto, ragazza...'' disse l'uomo, chiudendo gli occhi. Se adesso
    faceva la scena dell'indicare un cassettone ad occhi chiusi dicendo che quello era il componente che lo stava
    chiamando se ne sarebbe andata. ``...quello è il componente adatto.'' indicando un cassettone.

    Ok, basta. Lei adorava il Giappone. Lo Shintoismo era qualcosa di magnifico nella sua idea. Ma questo era un
    ciarlatano. Si girò per andarsene. ``Aspetta.'' disse lui, senza muoversi ``Che accade? Credi che stia fingendo?''
    Lei si girò, alterata ``Ascolti. Sono stata quì a seguire i suoi discorsi e tutto il resto. Rispetto le sue
    ideologie e credo che lei sia un mostro per quanto riguarda la conoscenza della tecnologia Cybernetica, ma quando è
    troppo è troppo. Cos'è questa scenata?'' ``Beh, in quanto sacerdote devo ascoltare ciò che mi dicono le divinità
    nei componenti, per darti il migliore che tu possa trovare.'' ``Ok, mi dica perchè devo continuare a star quì ad
    ascoltarla? Probabilmente ha solo indicato il cassettone con dentro il cervello cybernetico più recente che ha.''
    ``Mh, non direi. Prova ad aprirlo.''

    Sofia, senza dire null'altro andò verso il cassettone. Sapeva che era così. Lei ne aveva chiesto uno avanzato e
    quello gli stava per vendere qullo più nuovo e costoso, solo che ci aggiungeva la scenata. Probabilmente le sarebbe
    costato di più che non da altre parti proprio per quel motivo. Si sollevò sulla punta dei piedi per raggiungere il
    cassettone e, presa la maniglia, lo estrasse dalla sede. Una volta riuscito a prenderlo con entrambe le mani lo calò
    fino a terra. ``Ecco, visto?'' disse lei, senza neanche guardare ``È  un modello di ultimiss...'' s'immobilizzò
    quando vide ciò che c'era all'interno.

    Era un vecchio modello di cervello Cybernetico della Pisa Technologies Inc. Un QPBrain, Mk.23 a giudicare dal numero
    impresso su uno degli emisferi. QP stava per
    Quantuum-Positronic. Avevano sempre avuto il gusto del pacchiano da quelle parti, doveva ammetterlo. Era un
    componente sperimentale che non aveva mai raggiunto il mercato in quanto era troppo avanzato. Il costo di produzione
    era troppo alto ed il sistema di computazione era completamente rivoluzionato. Addirittura non sapeva neppure dove
    metterlo sulla scala di prestazioni, in realtà.

    ``Ma... Ma.'' tentò di commentare, senza sapere veramente da che parte iniziare ``Sì?'' chiese l'uomo, girandosi
    ``Ma chi diavolo sei?'' chiese lei, quasi impaurita, per poi aggiungere senza aspettare una risposta ``Come fai ad
    avere questo tipo di componentistica? Come fai a sapere che la ragazza non andrà fuori di testa entro le prime 48
    ore come gli altri che l'hanno testato? Nonono, aspetta. Quanto costa, tra l'altro? Ma non è importante. Com...'' Il
    sacerdote le appoggiò la mano sulla testa, per calmarma ``Calma. Questo è quello di cui hai bisogno.''

    Ok, lo sapeva anche lei. Ma non poteva riversare la coscienza di Zoè in un cervello che aveva piegato la volontà di
    altre ventidue per... ASPETTA UN SECONDO, CAZZO. ``Se questo è il Mark 23 significa che...'' ``Non diventi uno come
    me se non sei in sincronia con gli spiriti.'' ``Non era questo che intendevo.'' ``Haha!'' rise lui, sfregandole la
    testa ``Mi hai scoperto, eh?'' e poi, togliendo la mano ``Allora, Sofia, come va con le tue protesi?''

    Macchiavelli-dono era uno dei progettisti capo della Pisa Technologies Inc. solo che lei non lo sapeva. Sapeva che
    quello che progettava i componenti che lei \emph{testava} sulla propria pelle era uno in gamba. Ma non pensava si
    fosse anche bevuto il cervello. ``Quindi tutta la storia dello Shintoismo era una finta?'' ``Non offendermi.'' era
    veramente offeso ``Eh?'' ``Quando progetto la componentistica per te pensi che non faccia purificazioni specifiche?
    Gli spiriti che hai dentro i tuoi componenti sono stati fatti \emph{traslocare} ogni volta.'' ``È un gergo tecnico o
    ci credi sul serio?'' ``Perchè non dovrei crederci?'' Ok. Ora stava iniziando a crederci. Se qualcuno del calibro di
    quell'uomo aveva una convinzione del genere doveva esserci qualcosa di vero, no?

    ``Perchè non hai dato un QP anche a me?'' chiese, quindi, cambiando completamente discorso ``Beh, due cose. La prima
    è che, secondo me, potrebbe ancora saltarti il cervello. La seconda è che devi abbandonare l'egoismo.'' ``Ah.'' ``Ma
    non ti preoccupare, ragazza delle macchine. Un giorno ne progetterò uno per te. Ed allora dovrò fare ben più che un
    trasloco. Ci sarà una comunità lì dentro.'' e si mise a ridere. ``Quanto... Quanto le devo per questo?'' chiese lei,
    sollevando il cervello dalla custodia. ``Mh. Che ne dici di centomila?'' stava già per urlare che era troppo. Poi si
    fermò ``Centomila? Ne è sicuro?'' ``Sì.'' Si era bevuto il cervello. ``O-ok.''  estrasse la carta con i soldi da una
    delle sue tasche ``Accettate Carte?'' ``Certo. Ma prima...'' ``Ma prima?''

    ``Devi purificarti se vuoi trasportare il cervello senza rovinarlo.''

    Una lunghissima giornata.

  \section*{Valentine}

    Era sera, ormai. L'aveva passata tutta a tentare di buttare giù un piano sensato per mettere tutto a posto, ma la
    quantità di basse probabilità, combinate, ne generavano una più bassa, ovviamente. Non aveva mai amato il calcolo
    delle probabilità all'accademia, ma che ci volevi fare? La realtà ti dimostra sempre che, se vuoi tirare sei sei di
    fila con un dado a sei facce devi avere sei volte di seguito la botta di culo. E quante volte capitava? Certamente
    non pensavi ad una volta ogni $6^6 = 46656$ volte, ma era sicuramente raro.

    Mancavano equipaggiamenti. E siccome i gruppi avevano quella quantità di Frames schierati erano cazzi. Loro ne
    avevano uno. E Catherine aveva solo una piccola quantità di armi anticarro. Non poteva sperare nell'\emph{One
    Shot-One Kill}, anche perchè lei era la prima a non assicurarglielo. Anche sperare che quelli del Bazaar aiutassero
    era difficile. Non avevano abbastanza impatto per far saltare tutti quei Frames. Anche perchè, normalmente, non
    dovevano preoccuparsi di danni al Mercato da parte di Frames, visto che le Corp non andavano a disturbarli, per paura
    di ripercussioni. I mercanti del Bazaar sapevano come fare per farla pagare alla gente. Le Corps, comunque, avevano
    abboccato. Avrebbero mandato i loro agenti al Bazaar, il giorno dopo, per lo scambio.

    Valentine prese un sorso di cosmopolitan mentre nell'aria del locale aleggiavano le note di una canzone jazz.
    Controllò l'ora sull'orologio attaccato dietro il bancone, un vecchio orologio analogico con ventiquattro ore e due
    sotto quadranti per i minuti ed i secondi. QUEL MALEDETTO OROLOGIO. Era come se fosse sempre presente nel campo
    visivo. Gli altri ci stavano mettendo molto di più di quello che
    avesse calcolato. Va bene Catherine ed il Pivello, che dovevano mettere a posto un Frame. Ma Sofia dove diavolo era?
    Doeva solo comprare un cervello cybernetico e dello spazio nella rete. Comunque, finchè non chiamavano, non era un
    problema. Aveva detto loro di non usare comunicazioni se non in caso di pericolo. Non voleva che venissero
    intercettati.

    Verso le nove, però, il campanello della porta indicò che qualcuno era entrato nel locale. Siccome aveva chiesto al
    proprietario di indicare che era chiuso per quella giornata, c'erano poche probabilità che fosse qualcuno che non
    faceva parte del Team. ``Oh, alla buon ora Sofia dove...'' iniziò, girandosi, per poi interrompersi per correggersi
    ``Ah, Pivello. Dov'è Catherine?'' All'entrata c'era il ragazzo con un borsone nero a tracolla stracolmo che doveva pesare un bel
    po', vista l'espressione, che aveva appena appoggiato a terra una cassa metallica con sopra una valigia mimetica.

    ``Veramente...'' ansimò per la fatica ``...mi chiamerei Arsenicos.'' ``Immagino.'' lo liquidò lei, per poi passare a
    cose più importanti ``Ma cos'è tutta quella roba?'' ``Eh? Questo?'' prese il borsone e glielo mostrò ``È
    l'equipaggiamento che mi ha chiesto.'' ``Immaginavo, volevo sapere cosa c'è nei contenitori che non ti avevo chiesto
    ``Ah, posta.'' ``Posta.'' ``Sì, no, aspetti.'' il ragazzo mise la mano avanti mentre riprendeva fiato, dopo aver
    appoggiato il borsone a terra ``Ma come fa?'' si chiese lui, ancora piegato ``Mh?'' ``Ah, no. Dicevo. Questi
    scatoloni erano nella posta.'' ``Nel \emph{giardino}?'' ``Esatto.'' ``Ok, chi l'ha ordinata?'' ``Io'' battendo sulla
    valigia ``E... Boh?'' ``Boh. Ok, ora passiamo alla parte importante. Quanto?'' ``In che senso, scusi?'' ``Quanto ci è
    costato tutto questo?'' ``Ah? Oddio! Non lo so. La mia valigia è un favore.'' fece una pausa ``Per la Giustizia. Per
    lo scatolone metallico... Non lo so. Non so neppure chi l'abbia mandato, se è per quello.''

    Ma si stavano perdendo via ``Aspetta un attimo. Abbiamo perso il filo del discorso?'' ``Quale?'' ``Appunto.'' sarebbe
    durato ancora un po'. Prese un altro sorso di cosmopolitan dal bicchiere da cocktail ``Ok, reiniziamo. Ciao
    Pivello.'' ``Ma veramente...'' ``Sì, saltiamo le parti poco importanti.'' ``Dov'è Catherine?'' ``Hem. Penso che sia a
    cercare parcheggio.'' ``Parcheggio.'' ``Sì, ha detto che quì sono tutti posti a pagamento.'' ``Mmmh... Che palle.'' 

  \section*{Catherine}

    Dieci minuti per trovare parcheggio. Non era possibile. Fortuna che aveva lasciato lì il ragazzo, così che andasse
    da Valentine a spiegarli come fosse messo ATHENA e come mai lei ci stesse mettendo così tanto. Aprì la porta del
    locale aiutandosi col corpo e si fermò senza entrare. Tre metri dopo la porta c'erano Valentine ed Arsenicos che
    stavano discutendo dello sviluppo urbano di Parigi dalla fine della Terza Guerra Mondiale.

    ``Nono.'' commentò il ragazzo ``Deve capire che il tipico schema di sviluppo urbano a griglia è sì pratico, ma non
    aiuta particolarmente con l'impronta ambientale che vogliamo dare alla città. Dovrebbe svilupparsi più a raggiera. O
    qualcosa di frattale.'' ``Assolutamente no, Pivello.'' rispose seccamente Valentine ``Ascolta, probabilmente non hai
    mai guidato una macchina, ma una lunga via retta è spesso il miglior modo per ridurre i contumi da trasporto.''
    ``Questo perchè stiamo ancora parlando di trasporti a combustibile fossile.'' ``No, beh, anche con le celle ad
    idrogeno.'' ``Beh, ma quelle emettono vapore acqueo.'' ``Sè. E come fai a mettere l'idrogeno nelle celle? Hai
    bisogno di impianti apposta.'' ``Beh, generiamo l'elettricità con risorse rinnovabili per quelle.'' ``Esattamente.
    Ma quì ti sei perso un passaggio. Sto parlando di efficenza energetica, non di emissioni nell'atmosfera.''

    Se avesse lasciato andare sarebbero stati capaci di parlare per ore. Magari un'altra volta non se ne sarebbe
    preoccupata, ma quella aveva bisogno che il Boss si concentrasse. ``Ok.'' s'intromise ``Mi spiegate che ci fate
    quì?'' ``Signora Mac an Bahird. La signorina Arthur crede che lo sviluppo urbano basato sui concetti usati nelle
    metropoli Panamericane sia migliore di quello che è stato usato nelle città Europee per centinaia di anni.'' ``Come
    la periferia di Barcellona?'' lo punzecchiò lei ``Quello non vale.'' disse Arsenicos, girandosi verso la donna
    ``Oi! Almeno hai fatto rapporto?'' chiese Catherine ``Ah? Oddio! No, mi perdoni. Rimedierò.''

    ``Capo. Abbiamo finito di riparare, calibrare e sincronizzare Fragarch ad ATHENA. Possiamo usarlo nella missione di
    domani.'' descrisse lui, conciso. ``Mh.'' Valentine perse un po' del suo tono. Non che non fosse una buona notizia.
    Ma aveva già tenuto conto della possibilità di utilizzare il Frame. Sarebbero stati molto più nei cazzi se non
    fosse stato disponibile. ``E ora. Vuoi dirmi CHE CAZZO c'è in questa cassa?'' chiese lei, subito dopo ``Ah. Doveva
    dirmelo subito. C'è. Hem...'' ripose il ragazzo, per poi girarsi in difficoltà a guardare Catherine.

    ``C'è un Tekko con cariche.'' rispose lei, concisa ``Un che?'' rispose il Boss ``Un lanciamissili termobarico a stadio triplo con
    sistema di locking multiplo. I missili possono raggiungere una velocità di crociera di Mach 22, il che significa che
    può oltrepassare gli scudi elettromagnetici montati nelle ultime versioni di Frames.'' Valentine sembrò pensare al
    significato di quella cosa. Catherine conosceva le competenze del Boss ed esplosivi e sistemi di propulsione non
    erano per lei. ``Guarda, se ti stai chiedendo se faccia danni posso dirti che, con un missile di questi penso di
    poter tirare giù due Frame in un colpo.

    A Valentine sembrò illuminarsi il volto ``Tu... COSA?'' chiese, eccitata come una ragazzina dopo essersi messa con
    il suo primo ragazzo ``Se hanno degli scudi magnetici penso di far entrare almeno un colpo ogni tre su due target
    differenti. A quanto ho capito questi missili sparano sei sub missili.'' era sicura delle specifiche, ma non del
    funzionamento corretto. Era pur sempre il Doc ``E, se conosco bene il tizio, deve averci messo qualche tipo di trano
    eplosivo termobarico. Con la sola velocità d'impatto dei missili posso tirarci giù una palazzina.''

    Catherine riusciva a vedere una serie di nuovi piani che frullavano in testa al Boss solo guardando come muoveva gli
    occhi. La sentiva anche mugugnare, in effetti. ``E non è tutto, tra l'altro.'' aggiunse Catherine, per farla
    esplodere ``In che senso?'' Valentine li guardò ansiosa ``Ragazzo.'' disse Catherine ad Arsenicos facendo un cenno
    con la testa.

    ``Eh? Ah, sì, subito!'' rispose lui, sollevando da terra la valigia ed appoggiandola sopra la cassa contenendo il
    Tekko. Aprì i lucchetti, sollevò il coperchio e la fece dirare in direzione di Valentine. Ok, ora aveva
    un'espressione tale che diceva che, se il giorno dopo non avesse dovuto salvare la pelle a lei ed al suo Team,
    sarebbe andata a cercarsi qualcuno con cui passare la notte. ``E questo cos'è?'' chiese, ancora più eccitata ``È un
    Walther WA2100.'' rispose lui ``Oddio. Un fucile della serie WA...'' Catherine poteva giurare di vedere un brillio
    negli occhi del Boss ``Lo conosce? È una versione limitata di fucile Gauss. Può sparare sia paletti di acciaio
    ricoperti di Tungsteno amorfo oppure proiettili di mercurio raffreddati ad azoto liquido.'' ``Sì. SÌ! E vedo anche
    un computer balistico ed un mirino 2.5-30.'' commentò lei. Senza distogliere lo sguardo. ``Sì, ma me l'ha chiesta
    Sofia, quindi non so se posso dargliela, signorina.''

    Valentine si fermò. Sembrava aver perso parte dell'entusiasmo, ma Catherine poteva capire che stava stravolgendo i
    piani. ``Ok, chiederò a lei, allora. Meglio avere il miglior assetto possibile per il team.'' commentò lei, per poi
    aggiungere, a voce un po' più bassa ``Però, che strano. Mi era parso di capire che non fosse per lei il
    combattimento a distanza... Quest'arma avrà una gittata media di uno o due chilometri.'' Lasciò stare il discorso e
    si diresse al bancone. ``Ottimo, ragazzi. Con questo abbiamo una possibilità in più per domani. Qualcuno sa che
    abbiamo questa roba?'' ``Non penso.'' rispose suibito Catherine ``Il Tekko è da parte del Doc, mentre il WA2100 pare
    sia un equipaggiamento fornito dal mentore di Arsenicos.'' ``Grande.'' poi, dopo essersi fermata un attimo a
    pensare, chiese ``E che è Fragarch?'' ``È una pistola per ATHENA.'' ``AH! Giusto.'' ``Ma non sappiamo chi l'abbia
    chiesta.'' ``Beh, l'ho chiesta io.'' ``Eh?'' fecero, in coro, Catherine ed Arsenicos ``Mi sembra ovvio.'' rispose
    lei ``A quanto ho capito il Pivello è abile nel combattimento con pugnale da combattimento e pistola in corpo a
    corpo. Volevo che avesse le stesse potenzialità sull'Armoured Frame. Però. Sono stupita che il Doc ci abbia messo
    così poco...''

    Arsenicos stava per esplodere dalla gioia, ma si tratteneva al solito, per sembrare un uomo composto.

    Ora mancava solo Sofia.

  \section*{Sofia}

    Purificazioni. Maledetto. Sofia si guardò. Aveva dovuto lasciare da \emph{MACCHIAVELLI-DONO} i suoi vestiti ed
    andare in giro con quegli abiti da sacerdotessa Shintoista fantascientifica perchè ``Questi non comprometteranno
    l'equilibrio all'interno del componente''. E aveva anche dovuto togliere la biancheria intima. Poi bagni, rituali.
    Guardò la scatola in legno laccato, contenente il componente, che stava tenendo con entrambe le mani. Almeno il tipo
    le aveva fornito anche il sistema per trasportarlo e fare la copia on the fly. Doveva trovare il modo per metterselo
    in spalla, o a tracolla. Aspetta un secondo.

    Non doveva mica andare in giro vestita così il giorno dopo, vero? Si guardò in giro. I ragazzi la guardavano con uno
    sguardo da lupi affamati. Odio. Ce la faceva a farli scappare con lo sguardo? Ci provò ma non ci furono reazioni.
    Maledetti. Ci mise un paio di ore per arrivare fino a casa. Non poteva correre in quello stato, non poteva usare
    l'Hook e non aveva i soldi per un taxi. Appoggiò la cassa al corrimano vicino all'entrata del pub, che scendeva nel
    seminterrato, tenendola su con il corpo, ed aprì la porta. ``Oi.'' fece, stanca ``Ragazzi sono...'' Non finì la
    frase perchè, tanto, nessuno l'ascoltava.

    Valentine era al bancone, farfugliante, mentre buttava giù piani, mentre il Newb e Catherine stavano facendo il
    check dell'equipaggiamento su di un tavolo. Beh. Erano proprio preoccupati che fosse così in ritardo. Poi, per caso,
    Valentine si girò ``AH!'' esclamò, vedendola ``Sofia, finalm...'' poi scoppiò a ridere ``Che cos'è quello?'' `` Il
    cervello.'' ``No, intendo, il vestito? Sei andata ad una gara di cosplay?'' ``Haha.'' fece la ragazza, per niente
    divertita ``Sono stata da un matto.'' ``Mh. E così...'' aggiunse Catherine, la quale si era girata, incuriosita dal
    discorso ``...è questo ciò che cerchi da un uomo.'' Non ci poteva credere. Erano loro quelle che dovevano fare quel
    tipo di battute? Non riusciva a capire. ``No, oi, ragazze. Aspettate.'' tentò di calmarle, lei, senza sapere dove
    mettere la scatola ``Voglio dire.'' aggiunse Valentine, alzandosi dal suo sgabello al bancone ed avvicinandosi a
    lei. Forse le avrebbe dato una mano. Valentine la guardò bene ``È fatto bene per essere un costume.'' poi si abbassò
    al livello dei fianchi ``Mh. Non vedo neanche la linea degli slip.'' Valentine alzò lo sguardo ``Come hai fatto?''
    Non li aveva. Ma come erano finite a fare quel discorso? Voleva solo andare a cambiarsi e bere un bicchiere di
    qualcosa ``N...Non...'' iniziò lei, che non ce la faceva più ``Mh?'' ``NON LE HO, OK?!?'' urlò lei, imbarazzata.

    Il Newb scattò in piedi sbattendo le mani sul tavolo. E basta. Tutte lo fissarono, aspettando qualche reazione.
    ``Che accade?'' gli chiese Catherine ``Stai stai sudando.'' aggiunse, ironica ``Ni-Niente.'' rispose lui, nervoso
    ``Sicuro?'' ``Sìsì, signora Cath... Mac an Bahird. Mi sono solo punto con il cacciavite per le riparazioni. Ha.
    Haha.'' risatina nervosa per lui, poi si rimise seduto.

    Dopo aver spiegato come fosse andata la spedizione al Bazaar per recuperare il componente andò nella stanza che
    avevano affittato a cambiarsi. Finalmente poteva mettersi dei vestiti normali. Tornò giù mentre gli altri componenti
    del Team si erano rimessi a lavorare. Si diresse dal Capo e le chiese ``Ok, Vale. Cos c'è da fare?'' ``Mh. In realtà
    la tua parte l'hai fatta. Ora devi solo lasciarmi schiarire le idee e poi vi spiegherò tutto.'' ``Capito.'' ora di
    andare a recuperare un drink ed a buttarsi giù su uno dei divanetti. ``Ah.'' disse Valentine, per attirare la sua
    attenzione ``Sì, BJ, il Pivello e Catherine hanno riportato in dietro dell'equipaggiamento. Vai a vedere che ti
    serve prima di buttarti giù, ok?'' ``No problem.''  rispose Sofia. Se continuava così sarebbe impazzita. Aveva
    bisogno di un po' di relax. Non che tutti i \emph{cicli di purificazione} non fossero stati rilassanti, però
    l'andare in giro per città conciata in quel modo l'aveva messa un po' sulle spine. Quel tipo di cosplay non faceva
    per lei.

    ``Ok.'' disse, una volta arrivata al tavolo dove i due stavano ancora dando l'ultima controllata agli
    equipaggiamenti ``Che avete per me?'' ``Ah, giusto. Allora.'' ripose il Newb, che si diresse verso la cassa di
    metallo con sopra la valigia, dopo essersi alzato ``È arrivato quello che mi hai chiesto.'' e le porse la valigia.
    ``Quello che ho chiesto?'' chiees lei, piuttosto confusa. Cos'era? Il contenitore non era proprio uno di quelli con
    i quali spedivano le cose che ordinava da Internet. Aprì la valigia e rimase a guardare il contenuto per un po', non
    capendo. ``E questo che cos'è?'' chiese, indicando il fucile ``Cioè. Posso capire che sia un fucile, ma quando te
    l'ho chiesto?'' ``Eh? È l'\emph{arma demoniaca}. Scusa se ne ho recuperata una sola.'' ``WAT'' rispose lei,
    sorpresa. Quando aveva detto una cosa del genere? ``Beh, grazie. Ma io non so usare queste armi.'' rispose lei,
    chiudendo la valigia. Al bancone Valentine scattò in piedi.

    ``Ah... EH?'' chiese lui ``Mh, eppure sono sicuro...'' ``Lascia, lascia, Pivello...'' s'intromise Valentine ``Ma,
    veramente...'' ``Allora,'' continuò Valentine, ignorando completamente il ragazzo, mettendo il braccio intorno alle
    spalle di Sofia ``non è uno spreco. Se non lo usi te lo prendo io, BJ.'' ``Ah, certo, non c'è problema.'' ``Visto,
    Pivello? Non hai fatto una cazzata. Lo userò io.'' commentò il Capo, rivolgendosi al Newb ``O-ok.'' rispose lui, non
    capendo bene ``Ah, sì, tra l'altro. Ho anche preso questo...'' continuò lui, frugando nel borsone nero che si era
    portato dalla base. Ne estrasse il piede di porco della ragazza e glielo porse ``...questo ti serve, no?'' Gli occhi
    di Sofia brillarono ``E CERTO!'' prese lo strumento e lo accarezzò ``Grande, Newb. Questo sì che mi sarà utile
    domani.''

    Valentine si allontanò con la valigia. Sofia si mise a controllare un po' di equipaggiamenti. Avrebbe avuto bisogno
    solo del minimo indispensabile. Non voleva venir rallentata durante la corsa. Alla fine, dopo che aveva selto ciò
    che le serviva Catherine la guardò e, allungandole una granata EMP, le disse ``Magari non ti serve. Ma non si sa
    mai.'' ``Vuoi uccidermi?'' chiese lei, prendendo comunque la granata ``Voglio che impari ad andare in giro con una
    di queste senza impazzire.'' e poi tornò a lavorare.

    ``Mh. Ok.'' non era molto convinta ``Beh, ora sono distrutta. Vado a dormire.'' e si alzò dal tavolo. ``'Notte.'' le
    fece Catherine distrattamente. ``Buonanotte.'' disse il Newb che poi aggiunse ``Ah, non ti preoccupare. A me col
    vestito di prima piacevi.''

    ``Eh?'' chiese lei, per poi andarsene. Ma sarà stato idiota?
