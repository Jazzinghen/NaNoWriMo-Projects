\chapter{Chi offre di più?}

  \section*{Valentine}

    Valentine montò sulla propria macchina, una Jaguar XKR-S color verde Jaguar. Adorava quella macchina: sprizzava
    Britannicità da tutti i pori, inoltre era magnifica da guidare. Ed a Valentine guidare piaceva moltissimo. Poteva
    essere considerata una macchina d'antiquariato, ormai, visto che l'aveva
    comprata usata al Bazaar da un uomo che l'aveva comprata prima della Terza Guerra mondiale. Da quando ers in
    possesso di Valentine, però, il veicolo aveva subito vari upgrades. Molti di questi upgrades erano stati progettati
    dal Doc, l'uomo che stavano andando ad incontrare quella sera, e montati e calibrati da Catherine. Tra questi
    potenziamenti c'erano un HUD installato su tutto il parabrezza (ovviamente), un motore ibrido idrogeno/batteria
    elettrica da 540 cavalli (Una modifica proprio dell'AJ-V8 GEN III R originale), dei sistemi di contromisure elettroniche anti missili e, immancabile, un sistema
    MercuryWings custom di thrusters, come quelli secondari montati sui Frames, che poteva venir utilizzato come turbo.
    Quest'ultimo non l'aveva ancora usato, ma non poi mai essere sicura di quando devi fuggire da qualche stronzo
    impazzito o cose del genere. Inoltre l'estetica della macchina non era stata rovinata per nessuna customizzazione. I
    thrusters, ad esempio, erano stati installati all'interno della carrozzeria e nel bagagliaio. Quando vengono
    attivati il retro del veicolo cambia di geometria per fare posto ai thrusters, oltre a far comparire degli alettoni
    per mantenere il veicolo incollato alla strada.

    Catherine salì a sua volta dal lato passeggeri, a sinistra, e, una volta chiusa la portiera, disse ``Boss, senti,
    quei ragazzi hanno delle ottime capacità, non devi...'' Valentine mise in moto la macchina utilizzando il tasto per
    l'accensione posto a lato del volante. La chiave magnetica che aveva in tasca avrebbe dato il permesso per far
    partire l'auto. Il lettore musicale integrato nel computer di bordo fece automaticamente partire un album di musica
    jazz. La musica era l'altra cosa che serviva a Valentine per guidare, dopo il carburante. ``Catherine, so che cosa
    vuoi dire...'' rispose lei, mentre si mise a far manovra per uscire dal
    parcheggio ``...ho capito anche io che sia Sofia che Arsenicos hanno le capacità per rendere questo gruppo migliore
    di come non era prima. Addirittura potrebbero portarlo avanti senza di noi un giorno, quando decideremo di andare a
    ritirarci alle Canarie...'' ``Me ne tornerò in Irlanda, sai?'' ``Oh.'' commentò, lei, leggermente stupita,
    rimettendosi in posizione frontale, pronta a partire. ''Ok, dicevo. Il fatto è che le nostre finanze sono strette,
    un sacco, purtroppo. Mi piacerebbe anche a me dare ai due una possibilità per addestrarsi e per diventare migliori
    prima di andare sul campo.'' Aspettò all'imbocco sulla strada che non ci fossero macchine e poi continuò a guidare
    ``Ma bisogna essere duri per far capire loro cosa devono imparare.''

    ``Mah.'' obiettò Catherine, perplessa ``Secondo me c'è qualcos'altro dietro.'' ``Ancora con questa storia?'' ``Non
    nasconderlo. Sei diventata fissata con questa storia dei bilanci e della quantità di soldi spesi da quando Nicolas
    se n'è andato. Una volta l'importante era il risultato, non il profitto finale.'' ``Senti, il discorso è facile. Da
    quando non c'è Nicolas le spese per l'Armoured Frame sono schizzate. Quell'uomo era un mago.'' ``Le spese
    dell'Armoured Frame saranno anche state più basse ma dovevo scendere in campo io. E lo sai che significa, vero?''

    Valentine, quando Catherine le disse ciò, iniziò a scorrere le varie missioni dove avevano dovuto mettere in campo
    la meccanica del gruppo. Erano il tipo di missioni che facevano spendere soldi al gruppo in quel momento. Solo che
    al tempo non c'era un pilota ben allenato ma poco esperto che faceva saltare in aria case e si buttava nel
    combattimento a testa bassa con la giustizia che ardeva nel cuore. C'era una ex meccanica esperta d'infiltrazioni
    in territorio nemico, meastra delle tattiche di guerriglia anti-Frames. Queste tattiche più e più volte avevano
    previsto l'uso di esplosivi costosi, demolizioni deliberate di palazzi, messa in campo di veicoli di terra armati di
    array di lanciarazzi anti corazzati. Catherine era una forza della natura. Una volta sul campo significava trovarsi
    mezzo isolato raso al suolo o inabitabile per mesi. Il numero di missioni portate a termine con lei erano
    impressionanti.
    
    Questo, però, significava che le spese per le strumentazioni erano enormi. Una volta era riuscita a spendere quello
    che avevano usato per l'ultima missione da sola. Certo, aveva distrutto quattro Armoured Frames senza doverne usare uno,
    però comprare 180 Lanciamissili Anti-Carro AT-6 solo per poterli disseminare all'interno di un intero quartiere era
    costoso.

    ``Ok, ora che mi ci fai pensare anche le spese che avevamo un tempo erano alte.'' ammise Valentine, con un po' di
    rammarico. Il viaggio continuò per un po' senza che nessuna delle due disse una parola. Poi, mentre si trovavano ad
    un semaforo, Valentine sbattè le mani sul volante. ``Ok.'' fece, irritata, ''Va bene, lo ammetto. Sono incazzata.''
    ``Ti manca?'' ``No. Ok, no, quello no. Sul serio.'' Era seria. Non era un fatto di relazione romantica, quello,
    ``Figurati. Semmai sono cazzi suoi se ha lasciato una come me.'' Valentine era dannatamente sicura quando si
    trattava del suo carisma.

    ``Il problema è che quello stronzo, dopo averci parlato per tanto tempo di come si combatta per un fine più alto,
    dopo aver detto che non erano i soldi, l'importante, ma che era il risultato, che fa? Se ne va per lavorare per un
    gruppo paramilitare che lo paga fior fiore di soldoni per uccidere ricercatori, rapire gente e cose così?'' ``Oh,
    dai. In quei gruppi fanno un sacco di altri lavori come i nostri. Non è che sono tutti stronzi, Valentine.'' ``Sì,
    ma di sicuro non disdegnano fare lavori simili quando vengono pagati una badilata di soldi.'' Valentine era
    piuttosto alterata. Non le capitava spesso di sfogarsi riguardo a quel discorso. L'ultima volta che l'aveva fatto
    doveva essere ubriaca. Sembrava più facile al tempo. ``Beh. E questo è un buon motivo per prendersela con due
    ragazzi che hanno deciso per lavorare per te invece che vendere i propri servigi a qualche gruppo più
    \emph{remunerativo}?'' ``Ho bisogno che imparino.'' ``Hai paura che in una di queste missioni potremmo incontrare
    Nicolas e il suo nuovo gruppetto di amici e che ci freghi, vero?'' ``No, è solo...''

    Catherine girò la testa per guardarla, facendole capire che non avrebbe funzionato. ``Sì.'' erano di nuovo ferme ad
    un semaforo. Valentine girò la testa verso Catherine per guardarla e continuò con il discorso ``Voglio fargliela
    pagare, ok? Ti ricordi che cosa ha detto quando se n'è andato?'' Catherine si girò in avanti e fece un cenno con la
    testa per indicare che il semaforo era diventato verde ``Sì, certo. Che senza di lui non saremmo andate molto
    lontane. È per quello che hai mandato il messaggio in giro per le accademie.'' ``Esatto. Quello stronzo.'' ``E
    invece che beccarti un qualche istruttore di pilotaggio di Frame stanco della vita d'accademia ti sei trovata uno
    studente appena uscito dagli studi.'' ``Ed una Otaku.'' ``Ed una hacker con le contropalle che conosce il Parkour ed
    ha vissuto per quasi tutta la sua vita con un cervello e due arti Cybernetici.'' ``Sì, ma converrai con me che è
    strana.'' ``Mah. Non più di quei ragazzi che spendono i loro finesettimana in discoteche a strafarsi di droghe e poi
    fanno a finta di essere dei bravi ragazzi durante la settimana. Almeno lei non nasconde le sue passioni. E non sono
    dannose.'' ``Legge fumetti porno. Tra ragazzi.'' ``Scusa? E da quando è un problema?'' ``Eh?''

    Ok, il discorso era arrivato ad un punto un po' particolare, in effetti. Com'erano arrivate alla discussione su
    Sofia? ``E poi si fa chiamare BJ!'' continuò Valentine ``Ah, sì. Mi aveva spiegato il perchè tempo fa, ma non me lo
    ricordo molto bene, dovresti fartelo spiegare una volta che abbiamo tempo.'' ``Beh, non che ci sia molto da
    spiegare.'' ``Che intendi?'' ``Beh, dai, insomma, non mi pare una buona idea avere una pratica del genere come
    soprannome.'' ``Hahaha!'' Catherine scoppiò a ridere ``No, guarda, mi sa che hai capito male.'' ``Ah.'' La hacker
    era improvvisamente diventata ancora più strana agli occhi di Valentine.

    ``Comunque sia.'' finì il discorso Catherine ``Vedrai che quei due hanno un sacco di assi nelle maniche da
    giocare.'' ``E chi te lo dice?'' ``Una come me se le sente queste cose.'' ``Beh, la veterana sei te.'' ``Dillo
    un'altra volta e ti faccio vedere chi è la \emph{veterana}, miss Naval Intelligence.'' rispose Catherine, con tono
    ironico ``Haha! Lo vedremo la prossima volta che andiamo a bere, Miss Braccio di Ferro.''

  \section*{Catherine}

    Dopo il viaggio in macchina erano arrivate finalmente al Bazaar. Avevano dovuto passare un posto di controllo dove
    applicano dei sigilli alle armi (oltre a farti togliere il caricatore per poi mettere un sigillo anche
    alll'imboccatura del compartimento per i caricatori delle armi). Se ne trovavano alcune con il sigillo rotto finivi in grossi, grossissimi guai. I
    mercanti erano gli unici che potevano maneggiare armi e quelle che vendevano non erano cariche, quindi la gente
    poteva controllare quanto fossero buone prima di comprare. 

    Valentine parcheggiò la macchina in uno degli autosilo all'entrata dalla quale erano arrivate e scesero per strada.
    Il Bazaar ricordava moltissimo quelli del passato. Erano degli enormi mercati permanenti locati all'intero di
    intricate reti di gallerie. Solo che, in questo caso, le gallerie non erano create apposta. Venivano generate dai
    piani intermedi costruiti tra i palazzi ai lati delle case, trasformando così un intero quartiere in un Bazaar a più
    piani. In possimità degli incroci e delle piazze i piani intermedi venivano interrotti, permettendo così di vedere
    il cielo dalla strada.

    Le etnie all'interno del Bazaar erano estremamente eterogenee. In pratica quasi tutte le culture si mescolavano per
    le sue strade. Questo faceva in modo che venissero organizzate spesso feste in occasione di chissà quale tradizione
    nelle varie piazze. Per non parlare della scelta che era possibile trovare in giro per i vari negozi. Catherine
    aveva sempre trovato quello che cercava. E se non c'era te lo facevano arrivare in meno di una settimana. Meglio di
    qualunque corporazione. E la cosa che faceva impazzire Valentine, pareva, era che non c'era altro modo per andarsene
    via con quello che volevi senza contrattare. Per la meccanica era un problema. Non era molto brava, quindi finiva
    per pagare sempre quel qualcosina in più. Ma se era per tenere in piedi una comunità economica indipendente da
    qualunque Corp lo pagava più che volentieri.

    Normalmente lei e Valentine andavano al Bazaar per recuperare parti per il Frame, munizioni, armi custom,
    hardware per i vari PC del team o gadget per le missioni. Oggi Catherine era là per andare a trovare il Doc, il
    quale le aveva detto che voleva farle provare un nuovo <<gadget>> che aveva progettato. Non sapeva esattamente cosa
    doveva prendere il Boss, ma sicuramente era equipaggiamento per la missione a Londra. Probabilmente Sofia non aveva
    esattamente idea su cosa avrebbe avuto bisogno per infiltrarsi, perciò Valentine stava andando a prendere anche
    quello. Sotto sotto i due nuovi acquisti le stavano simpatici.

    Si divisero alla prima piazza. Lei doveva seguire una delle strade principali e poi girare in uno dei vicoletti
    laterali per poi entrare in un palazzo che doveva essere un ex parcheggio pubblico. Il Doc, l'uomo che stava andando
    a trovare, era uno di quei strani personaggi che potevi trovare nei romanzi polizieschi. Era il ricercatore cacciato
    dall'università perchè aveva idee troppo strane, troppo contro gli interessi delle aziende che finanziavano la
    ricerca universitaria.
    
    E, come tutti i personaggi di quel genere, viveva all'interno di un laboratorio gigantesco. Non riusciva minimamente
    a capire come fosse possibile che quell'uomo potesse permettersi tutto quello. Appena entrò dentro una porta in
    metallo che doveva essere una vecchia via di fuga antincendio si trovò in un corridoio che staccava completamente
    con il Bazaar, addirittura staccava dal palazzo. Il corridoio era formato da piastre in metallo percorse fori, posti
    in maniera regolare ogni cinquanta centimetri, ed era illuminato da dei sistemi a led installati sul soffitto.
    Catherine non si sentiva mai al sicruro all'interno di quella stanza. Sembravano quelle per la depressurizzazione
    per uscire nello spazio. Non che ne avesse mai vista una dal vivo, però quando le guardava in fotografia erano
    esattamente così. Percorse tutto il corridoio fino ad arrivare ad un portone in acciaio. A lato del portone c'era
    uno schermo con una telecamera attaccata. Lo schermo si accese, mostrando una giovane ragazza con i capelli neri
    e... Delle orecchie da gatto?

    ``Benvenuto a casa del Doc. Chi sei-nya?'' fece la ragazza al monitor con tono squillante. Nya. ``Sono Catherine, il Doc mi ha chiamata
    quì in quanto doveva mostrarmi una cosa.'' rispose lei, facendo a finta d'ignorare la ragazza ``Ah, ma certo!'' da
    sotto il monitor uscì una periferica che ricordava uno dei periscopi dei sottomarini con due semisfere ai lati
    ``Gentile ospite, le devo chiedere di confermare i suoi dati biometrici-nya. Se vuole guardare all'interno dello
    scanner per iridi mentre appoggia le mani sui lettori d'impronte e sistemi venosi, grazie-nya!''

    Catherine fece ciò che le era stato richiesto. Era sempre così. Il tizio non riusciva a fidarsi della gente. Quante
    volte avrà controllato i suoi dati? Era anche vero che, visto il suo lavoro, non era sicuro fidarsi troppo della
    gente. I ricercatori sono merce preziosa per le Corps, le quali non si sentono mai abbastanza avanti nell'R\&D, le
    quali fanno di tutto pur di mettere le mani su qualcuno di geniale. E questo quì era uno di quelli. Il laser
    tricromatico a bassa intensità le scansionò le iridi mentre le sue mani venivano controllate per vedere se la coppia
    impronte digitali / vasi sanguigni corrispondeva.

    ``Ora può allontanarsi dalla strumentazione, signorina Mac an Bahird.'' fece la ragazza al monitor ``Ora, se può
    attendere un altro momento, verrà decontaminata.'' Il \emph{decontaminata} che usavano da quelle parti significava
    rimuovere tutta la polvere, la pelle morta ed i capelli caduti dai vestiti della persona. Infatti, dopo pochi
    secondi, la porta che dava sull'esterno si sigillò, dei portelloni sotto la grata che faceva da pavimento si
    aprirono ed i fori iniziarono ad espellere aria a tutta potenza. Quel procedimento durò due minuti. Poi, una volta
    che le valvole nei muri ed i portelloni si richiusero, la ragazza al monitor fece ``Grazie per aver pazientato.
    Prego, entri pure. Ah, un'ultima cosa. Se fosse possibile le chiederemmo di togliersi le scarpe una volta entrata.
    Grazie.''

    La porta di fronte a lei si aprì emettendo un sibilo, mostrando un corridoio di un interno in stile giapponese
    tradizionale. Catherine fece un passo sul pavimento in legno lucido e quindi, una volta entrata con entrambe i
    piedi, si tolse le scarpe, lasciandole ai lati della porta. Non riusciva mai ad abituarsi allo stile di quella casa.
    Camminò lungo il corridoio fino ad arrivare a quello che doveva essere un giardino interno con terme all'aperto,
    solo che, sul fondo della vasca, c'erano le solite piastre in acciaio che rovinavano un po' l'atmosfera. Proseguì
    lungo la passatoia che faceva il giro al giardino per arrivare ad una porta in carta di riso e legno. Quando provò
    ad aprirla questa scattò di lato, mostrando l'interno di un laboratorio asettico, con tutti i mobili in acciaio,
    rifiniti di bianco, blu, arancione, rosso, verde o viola. Tutte le rifiniture erano comunque minimaliste, tipo
    strisce oppure poligoni: i colori sembravano annegare nel grigio dell'acciaio. Su uno dei lati della stanza c'era un uomo in piedi, piegato su della
    componentistica, che sembrava stesse per scoppiare dall'incazzatura. Nell'aria si potevano sentire le note di una
    canzone come quelle dei vecchi, vecchissimi film di James Bond. Sarà stata <<Snake Eater>>, al solito.

    ``Doc.'' fece Catherine, per salutarlo ``FFFFFFF'' sibilò alterato l'uomo, stringendo i pugni, per poi alzarli al
    cielo ed urlare ``FUCK!'' poi, abbassando le mani ed aprendole per indicare la pila di parti elettroniche, continuò
    ``Cazzo. Ma ti pare possibile che questi stronzi ti vendano della roba che dovrebbe, IN TEORIA, essere progettata
    per lavorare assieme e invece...?'' non stava parlando a lei. Faceva sempre così. Incazzarsi lo faceva lavorare
    meglio, diceva ``...e invece?'' chiese Catherine, semplicemente per farlo continuare ``E INVECE UN CAZZO! Guarda,
    guarda!'' il Doc corse verso un monitor montato sul muro dove erano presenti delle linee scritte su un terminale con
    lo sfondo viola trasparente, che lasciava intravedere uno sfondo in stile cartone giapponese che rappresentava una
    ragazza. ``Errori. Errori di comunicazione dappertutto. Queste merde. Anni che lavorano a questa roba ed ancora non
    riescono a fare del fottutissimo software decente.'' il Doc si mosse verso la scrivania, prese un sorso dalla sua
    tazza di tè e poi riprese ``Voglio dire, ma saranno stronzi questi coglioni! GGGRARGH!'' ringhiò, mettendosi le mani
    nei capelli, per poi tornare ad inveire ``Voglio dire. Fai un sistema del genere, ma mi farai una struttura dati
    seria, magari che possa comprendere? NO! Che fanno? Offuscano il codice. LO OFFUSCANO. Cioè. Ci devo lavorare con
    'sta merda, lo sapete, stronzi? L'ho paga...'' si fermò, sembrava più calmo ``Ah, no. Non l'ho pagata, vero. Vabbeh.
    Fatto sta che questi quì dovrebbero scrivere meglio il loro codice.''

    Si guardò in giro, come per vedere se non aveva distrutto nulla, poi commentò ``Vabbeh. Vorrà dire che dovrò darci
    dentro. Alla fine, se è gratis, è solo un fatto di hackare e patchare tutto. Che vuoi che sia.'' ``Già.'' provò a
    commentare Catherine, senza sapere di che stesse parlando. Il Doc si girò e, andandole incontro, le fece ``Ah!
    Catherine! Ciao.'' e poi l'abbracciò ``Com'è?'' allontanandosi ``Bene, Doc, grazie. Te?'' ``Ah...'' fece, sospirando
    ed agitando la mano destra in aria, come per indicare il <<tutto questo>> ``...solita fuffa. Recupero del materiale
    che sembra figo e poi dopo mi ritrovo con del ciarpame che non funziona. Ma lo farò funzionare, vedrai. A bestemmie,
    ma ce la farò.''

    Catherine saltò subito al dunque, avrebbe fatto due chiacchiere con lui se ci fosse stato il tempo dopo ``Doc, ho
    sentito che mi hai chiamato. Volevi farmi vedere qualcosa?'' ``Oh? Ah, sì!'' il Doc mise le mani avanti ``Ok,
    preparati, perchè ho preparato una di quelle cose che fanno bagnare le ragazze.'' Catherine alzò il sopracciglio
    destro ``Se mi hai chiamato quì per farmi vedere un giocattolino sessuale farò in modo che tu non possa più pensare
    alle donne per un po'.'' ``Nono.'' rispose subito lui, quasi offeso ``Lo sai che se volessi farti provare qualcosa
    del genere prima t'inviterei a cena.'' ``Credi di essere abbastanza, tappo?'' il Doc era effettivamente più basso di
    lei di almeno una decina di centimetri ``Oh? Vuoi provarci? Lo sai che adoro le ragazze più alte.''

    A Catherine piacevano gli scambi di battute che avevano ogni tanto loro due. ``Oh? Ma la ragazza che gira da queste
    parti non se la prenderebbe? Sembrava carina.'' ``CHI?'' ``Quella con le orecchie da gatto, scemo.'' ``HAHA! LOL.
    Quella? È un'intelligenza artificiale che ho creato perchè mi stavo rompendo le palle di rispondere al citofono. Te
    l'ho sempre detto. Potranno piacermi le ragazze su un monitor, ma niente batte toccare i seni dal vivo!'' era serio,
    come lo è sempre quando parla di tette, poteva quasi vedere le fiamme della sua passione in fondo agli occhi ``Ah,
    ok. Allora potrei decidere dopo che ho visto il tuo \emph{gadget}.''

    Il Doc mise la mano sulla zip dei pantaloni, abbassando lo sguardo. Christine non disse nulla. Il Doc alzò quindi lo
    sguardo e, con tono ironico, fece, ``Ah, non intendevi questo?'' iniziò a camminare verso l'uscita del laboratorio e
    sottolineò ``Peccato.''

    Una volta fuori iniziarono a camminare lungo la passatoria e poi su per delle scale ``Ma parliamo d'altro. Come sta
    andando con il gruppo, cara?'' ``Valentine è sbottata oggi, Doc.'' ``Ha!'' sembrava divertito ``Che ha? Le brucia il
    sederino per Nicolas?'' ``Mh, non so se apprezzerebbe.'' ``Oh, sai che non userei questi termini con lei.'' Arrivati
    in cima alle scale girarono a sinistra ``Vabbeh, senti. Dille che la smetta. Sono sicuro che la farà pagare al
    cazzone.'' ``Sì, lo so anche io, ma sai come vanno le cose.'' ``Oh, certo che lo so. È per quello che le dico di
    darci dentro. Le nove reclute?'' ``Hanno fatto faville nell'ultima missione.'' ``Tipo?'' ``Distruzione di serbatio
    d'acqua per distribuzione pubblica con una Limo'' ``HAHA!'' il Doc scoppiò a ridere di gusto ``Grandi, cazzo.'' Si
    girò per guardare in faccia Catherine ``Devi portarmeli quì un giorno, voglio conoscerli.'' ``No, che poi diventano
    come te.'' ``Geniali?'' ``Perversi.''

  \section*{Valentine}
    
    Valentine andava sempre al Bazaar prima di ogni missione. Era un rito, ormai. Per non parlare del fatto che spesso
    doveva recuperare dell'equipaggiamento mancante all'ultimo momento. Non era un fatto di disorganizzazione, era un
    fatto di mentalità umana. Spesso è quando si arriva all'ultimo minuto che viene in mente esattamente quale sia
    l'oggetto più utile per un dato incarico. Quella volta, però, oltre agli acquisti dell'ultimo minuto, doveva anche
    pensare agli ammanchi generati dall'ultima missione.

    Il Pivello sarà anche abile, ma è un po' troppo Trigger-Happy per i suoi gusti. Ed il conto del gruppo. Percorse
    la via dedicata alla vendita di materiale da Otaku, la casa di Sofia quando non aveva da fare per l'università, per
    il team o non aveva da giocare a qualche videogioco oscuro. La via sfociava nella piazza più grande del Bazaar
    parigino, abbastanza grande per poter fare da mercato per la componentistica per Frames.

    L'effetto di uscire dalla via degli Otaku per entrare in quello dei venditori di componentistica era piuttosto
    strano. Dopo aver camminato per una decina di minuti tra negozi che vendevano videogames, modellini di personaggi di
    serie animate e simulatori d'appuntamento erotici, cuscini alti un metro e sessanta con sopra stampati altri
    personaggi e volumi di serie manga, leggermente costretti dall'altezza dei piani intermedi, si arrivava in una
    piazza gigantesca, con due o tre Frames in esposizione (dipendeva dalla disponibilità), armi, c'erano addirittura
    dei generatori sparsi lungo la strada. Tanto chi vuoi che te li rubi? Non si può entrare con le macchine, quindi non
    c'è neanche la possibilità che qualcuno arrivi, lo carichi in fretta e scappi.

    La componentistica sparsa in giro per la piazza non era tutta proprietà di un singolo negozio, ovviamente, anche
    visto che la piazza era completamente circondata di una miriade differenti di essi. Il fatto, tra l'altro, che ci
    fossero altri piani rialzati attorno alla piazza aumentava il numero possibile di attività commerciali. Addirittura
    i negozi sui piani rialzati attaccavano i componenti a delle impalcature che collegavano ogni lato della piazza a
    quello opposto, facendo così in modo che ci fossero un'enorme quantità di componenti \emph{stesi} per tutta
    l'altezza dei palazzi che formavano i lati della piazza.

    La gente che cercava lei, comunque, non aveva un negozio nè al piano terra nè ai lati della piazza. Prese una delle
    rampe di scale ed iniziò a salire. Arrivò in cima anche all'ultima rampa, superò i negozi che si trovavano su quel
    piano, i quali, essendo il più in alto di tutti, erano anche quelli più da fighetti, finchè non arrivò ad una
    balaustra. Controllò che ci fosse ancora la passatoia in metallo per poi scavalcare il parapetto e continuare lungo
    la passatoia. Alla fine di questa c'era una scala in metallo che portava sul tetto della costruzione, attraverso una
    botola.

    Arrivata sul tetto seguì una passerella che portava ad una specie di grosso container color Digital Urban Camo, il
    quale aveva attaccato sul retro un tendone dello stesso colore. Continuò lungo la passerella finchè non arrivò alla
    porta del container. Bussò. Dopo qualche secondo qualcuno dagli speaker installati ai lati della porta chiese ``Chi
    è?'' ``Arthur'' rispose lei, scocciata ``AH!'' lo spioncino della porta si aprì di scatto poi, in un battito di
    ciglia, si richiuse ``Tenente Colonnello!'' sentì dire dalle casse, mentre i vari lucchetti della porta venivano
    aperti. La porta si aprì, rivelando un uomo sui trentacinque che impugnava un MP-9, diretta evoluzione dell'H\&K
    MP-7, con mirino ottico, mirino laser e smart pack. L'uomo abbassò l'arma e, facendole cenno con la mano sinistra,
    le disse ``Entra, prego. Scusa per la scenata. Ma sai, la sicurezza.''. Valentine entrò nel container, superando
    l'uomo, il quale chiuse la porta. ``Quante volte ti ho detto che non serve chiamarmi così? Sai che non mi piace
    particolarmente.'' commentò lei, senza girarsi ``Ah, scusa. Hai ragione, ogni tanto mi sfugge, lo sai.''

    Quell'uomo era Neil Myles. Lavorava per il Ministero della Difesa Franco-Britannico, prima che venisse scoperto a vendere
    armi a gruppi esterni. Venne posto sotto corte marziale e quindi allontanato con disonore. Questo non gli impedì di
    continuare a vendere equipaggiamento di livello militare nel Bazaar. Anzi, questo gli diede ancora più spazio, in
    realtà. Era diventato un po' più difficile poter accedere alla mercanzia, ma ora aveva qualcuno che rischiava la
    carriera che lo faceva per lui, quindi era ancora meglio. Durante la Quarta Guerra mondiale faceva parte del gruppo
    d'infiltrazione di Valentine, era per quello che la chiamava in quel modo. Non le aveva mai parlato in terza persona neanche
    quando erano in guerra assieme, non era l'uomo. E a lei andava bene così.

    ``Allora. Sei venuta per fare il solito rifonrimento post-missione oppure c'è dell'altro?'' chiese lui, mentre si
    stavano dirigendo verso la stanza che fungeva da soggiorno, oltre che da luogo per le trattative ``In realtà sono
    quì sia per fare rifornimento che per comprarti dell'equipaggiamento extra, sto per andare a fare un lavoro.'' gli
    rispose lei, mentre superavano una porta che dava sul laboratorio dell'\emph{ufficio} di Neil. Entrati nel soggiorno
    Neil la superò, appoggiò l'MP-9 sul tavolo e poi andò verso la cucinetta. ``Allora.'' fece l'uomo dall'altra stanza
    ``Prenditi pure una sedia o un posto sul divano. Lo sai che puoi fare come se fosse casa tua, no?'' ``Sì, grazie.''
    rispose lei. Era comunque titubante dal fare come se fosse a casa propria nell'ufficio/appartamento di un mercante
    di materiale di contrabbando.

    Si sentì il rumore di bicchieri che venivano lavati a mano ``Ho sentito di casini lungo le strade tre giorni fa.''
    inizò Neil ``Tra la gente del \emph{settore} si parla di uno scontro tra Armoured Frames customizzati e di agenti
    Panamericani.'' Valentine s'irrigidì leggermente sulla sua sedia ``Pfff. Sembra una di quelle spy movie.'' la testa
    di Neil spuntò dalla porta che dava verso la cucinetta e fece ``In che casini ti eri ficcata?'' continuò, divertito
    ``Chi ti dice che eravamo noi?'' ``Haha!'' Neil rise e ritornò a lavare i bicchieri ``Ma che, scherzi? Potrei
    riconoscere il lavoro del Doc anche da una foto a bassa risoluzione.'' spense l'acqua ed asciugò i bicchieri con una
    pezza. ``Abbiamo avuto dei problemi con uno dei nostri clienti.'' rispose Valentine, tentando di nscondere
    l'imbarazzamento. Avere tutti quegli anni alle spalle e farsi fregare da uno come Mr. Fabulous...

    Neil arrivò dalla cucina portando due bicchieri, uno con dentro del ghiaccio ed uno vuoto, ed una bottiglia di Rum
    Jamaicano. Appoggiò i bicchieri sul tavolo, ci versò dentro del Rum e quindi spinse quello con ghiaccio verso la
    donna. ``Come piace a te.'' fece lui, sedendosi sulla sedia di fronte a quella di Valentine e poi, alzando il
    bicchiere, fece ``Cin!'' Valentine andò a battere il fondo del suo bicchiere sul lato di quello dell'uomo e rispose
    ``Cin.''

    Il mercante prese un sorso di alcoolico e poi appoggiò il bicchiere. ``Che tipo di lavoro?'' chiese lui, curioso
    ``Sai che non parlo dei miei lavori, Niel.'' rispose lei, secca ``Ah, giusto. Hai parua che ti venda?'' ``No, ho
    paura che vengano quì e te lo estorcano. Meno sai meglio è per entrambi.'' Neil sbuffò diivertito e poi commentò
    ``Mah, dovrebbero solo provarci a venire a disturbarmi. Mi manca un po' d'azione.'' ``Pensavo che lavorare come
    mercante di equipaggiamento militare rubato fosse abbastanza pericoloso per te.'' ``Mah, sì. Però la gente
    preferisce rifornirsi giù ai piani bassi, quindi non ho molto giro. Dicono che vendo roba \emph{troppo professionale} per
    loro. La gente non capisce nulla di qualità dei prodotti. Ogni volta che devo recuperare materiale per te,
    comunque, ho un sacco di casini. Se sono ancora aperto è perchè lavori. Non so come farei senza.'' ``Haha.'' questa
    volta era lei che rideva per quell'affermazione ``Devi ringraziare le mie due nuove reclute. Sono matti.'' ``Ah. In
    effetti ho sentito dalle news che la Ville-Lumière Holdings non è stata molto contenta del piano che i due hanno
    portato a termine.'' ``E non hai ancora sentito la lista di merce che devo comprarti.''

  \section*{Catherine}

    Arrivarono di fronte ad un portone d'acciaio a due ante proprio alla fine del corridoio in stile giapponese. Un
    pugno nell'occhio. Il Doc iniziò ad inserire una password su una tastiera montata sul muro a lato della porta.
    ``Ma parlando d'altro...'' fece Christine, guardandosi intorno ``Dimmi...'' rispose lui, mentre pigiava tasti
    ``Cazzo.'' sbagliato password ``...quando dai una sistemata al posto?'' ``Dici eh?'' commentò lui ``Sì, comunque,
    hai ragione. Questo posto sembrava figo all'inizio, ma ora...'' la porta iniziò ad aprirsi, facendo scorrere le due
    ante in direzioni opposte ``...è pacchiano. Pacchianissimo. Dovrei trasformarla in una base più...'' ``Più?''
    ``Cyberpunk.'' ``Eh?'' ``Sì, dai. Più schermi in giro, più roba high tech. Magari lascerei le terme. Quelle sono
    fighe. Magari le monto all'ultimo piano con una vista sul Bazaar. Adoro questo posto, sai?'' ``Sì, lo so, me l'hai
    detto un casino di volte.'' la porta finì di aprirsi, rivelando una stanza gigantesca.

    Quello era il VERO laboratorio del Doc. Dal soffitto pendevano dei componenti per Armoured Frames, mentre ai lati
    della sala, grande abbastanza per contenere un Frames in posizione eretta, c'erano vari soppalchi, tutti collegati
    da scale. I lati esterni della stanza erano ricoperti di vetrate che permettevano di vedere in una sola direzione,
    così che il Doc potesse godere del Bazaar senza che occhi indiscreti lo disturbassero. Non è che odiasse la gente,
    al contrario. Spesso non vedeva l'ora di uscire per andare a farsi un giro con gli altri matti del quartiere, però
    il suo lavoro attirava troppe attenzioni indesiderate. Anche quì, comunque, i mobili erano in acciaio rifinito con
    decorazioni minimaliste. A quanto pare ogni colore indicava uno specifico tipo di funzione per il piano di lavoro.

    ``Ok,'' fece il Doc, iniziando a salire una rampa di scale, ``è da questa parte.'' e quindi
