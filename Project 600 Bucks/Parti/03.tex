\chapter{Chi offre di più?}

  \section*{Valentine}

    Valentine montò sulla propria macchina, una Jaguar XKR-S color verde Jaguar. Adorava quella macchina: sprizzava
    Britannicità da tutti i pori, inoltre era magnifica da guidare. Ed a Valentine guidare piaceva moltissimo. Poteva
    essere considerata una macchina d'antiquariato, ormai, visto che l'aveva
    comprata usata al Bazaar da un uomo che l'aveva comprata prima della Terza Guerra mondiale. Da quando ers in
    possesso di Valentine, però, il veicolo aveva subito vari upgrades. Molti di questi upgrades erano stati progettati
    dal Doc, l'uomo che stavano andando ad incontrare quella sera, e montati e calibrati da Catherine. Tra questi
    potenziamenti c'erano un HUD installato su tutto il parabrezza (ovviamente), un motore ibrido idrogeno/batteria
    elettrica da 540 cavalli (Una modifica proprio dell'AJ-V8 GEN III R originale), dei sistemi di contromisure elettroniche anti missili e, immancabile, un sistema
    MercuryWings custom di thrusters, come quelli secondari montati sui Frames, che poteva venir utilizzato come turbo.
    Quest'ultimo non l'aveva ancora usato, ma non poi mai essere sicura di quando devi fuggire da qualche stronzo
    impazzito o cose del genere. Inoltre l'estetica della macchina non era stata rovinata per nessuna customizzazione. I
    thrusters, ad esempio, erano stati installati all'interno della carrozzeria e nel bagagliaio. Quando vengono
    attivati il retro del veicolo cambia di geometria per fare posto ai thrusters, oltre a far comparire degli alettoni
    per mantenere il veicolo incollato alla strada.

    Catherine salì a sua volta dal lato passeggeri, a sinistra, e, una volta chiusa la portiera, disse ``Boss, senti,
    quei ragazzi hanno delle ottime capacità, non devi...'' Valentine mise in moto la macchina utilizzando il tasto per
    l'accensione posto a lato del volante. La chiave magnetica che aveva in tasca avrebbe dato il permesso per far
    partire l'auto. Il lettore musicale integrato nel computer di bordo fece automaticamente partire un album di musica
    jazz. La musica era l'altra cosa che serviva a Valentine per guidare, dopo il carburante. ``Catherine, so che cosa
    vuoi dire...'' rispose lei, mentre si mise a far manovra per uscire dal
    parcheggio ``...ho capito anche io che sia Sofia che Arsenicos hanno le capacità per rendere questo gruppo migliore
    di come non era prima. Addirittura potrebbero portarlo avanti senza di noi un giorno, quando decideremo di andare a
    ritirarci alle Canarie...'' ``Me ne tornerò in Irlanda, sai?'' ``Oh.'' commentò, lei, leggermente stupita,
    rimettendosi in posizione frontale, pronta a partire. ''Ok, dicevo. Il fatto è che le nostre finanze sono strette,
    un sacco, purtroppo. Mi piacerebbe anche a me dare ai due una possibilità per addestrarsi e per diventare migliori
    prima di andare sul campo.'' Aspettò all'imbocco sulla strada che non ci fossero macchine e poi continuò a guidare
    ``Ma bisogna essere duri per far capire loro cosa devono imparare.''

    ``Mah.'' obiettò Catherine, perplessa ``Secondo me c'è qualcos'altro dietro.'' ``Ancora con questa storia?'' ``Non
    nasconderlo. Sei diventata fissata con questa storia dei bilanci e della quantità di soldi spesi da quando Nicolas
    se n'è andato. Una volta l'importante era il risultato, non il profitto finale.'' ``Senti, il discorso è facile. Da
    quando non c'è Nicolas le spese per l'Armoured Frame sono schizzate. Quell'uomo era un mago.'' ``Le spese
    dell'Armoured Frame saranno anche state più basse ma dovevo scendere in campo io. E lo sai che significa, vero?''

    Valentine, quando Catherine le disse ciò, iniziò a scorrere le varie missioni dove avevano dovuto mettere in campo
    la meccanica del gruppo. Erano il tipo di missioni che facevano spendere soldi al gruppo in quel momento. Solo che
    al tempo non c'era un pilota ben allenato ma poco esperto che faceva saltare in aria case e si buttava nel
    combattimento a testa bassa con la giustizia che ardeva nel cuore. C'era una ex meccanica esperta d'infiltrazioni
    in territorio nemico, meastra delle tattiche di guerriglia anti-Frames. Queste tattiche più e più volte avevano
    previsto l'uso di esplosivi costosi, demolizioni deliberate di palazzi, messa in campo di veicoli di terra armati di
    array di lanciarazzi anti corazzati. Catherine era una forza della natura. Una volta sul campo significava trovarsi
    mezzo isolato raso al suolo o inabitabile per mesi. Il numero di missioni portate a termine con lei erano
    impressionanti.
    
    Questo, però, significava che le spese per le strumentazioni erano enormi. Una volta era riuscita a spendere quello
    che avevano usato per l'ultima missione da sola. Certo, aveva distrutto quattro Armoured Frames senza doverne usare uno,
    però comprare 180 Lanciamissili Anti-Carro AT-6 solo per poterli disseminare all'interno di un intero quartiere era
    costoso.

    ``Ok, ora che mi ci fai pensare anche le spese che avevamo un tempo erano alte.'' ammise Valentine, con un po' di
    rammarico. Il viaggio continuò per un po' senza che nessuna delle due disse una parola. Poi, mentre si trovavano ad
    un semaforo, Valentine sbattè le mani sul volante. ``Ok.'' fece, irritata, ''Va bene, lo ammetto. Sono incazzata.''
    ``Ti manca?'' ``No. Ok, no, quello no. Sul serio.'' Era seria. Non era un fatto di relazione romantica, quello,
    ``Figurati. Semmai sono cazzi suoi se ha lasciato una come me.'' Valentine era dannatamente sicura quando si
    trattava del suo carisma.

    ``Il problema è che quello stronzo, dopo averci parlato per tanto tempo di come si combatta per un fine più alto,
    dopo aver detto che non erano i soldi, l'importante, ma che era il risultato, che fa? Se ne va per lavorare per un
    gruppo paramilitare che lo paga fior fiore di soldoni per uccidere ricercatori, rapire gente e cose così?'' ``Oh,
    dai. In quei gruppi fanno un sacco di altri lavori come i nostri. Non è che sono tutti stronzi, Valentine.'' ``Sì,
    ma di sicuro non disdegnano fare lavori simili quando vengono pagati una badilata di soldi.'' Valentine era
    piuttosto alterata. Non le capitava spesso di sfogarsi riguardo a quel discorso. L'ultima volta che l'aveva fatto
    doveva essere ubriaca. Sembrava più facile al tempo. ``Beh. E questo è un buon motivo per prendersela con due
    ragazzi che hanno deciso per lavorare per te invece che vendere i propri servigi a qualche gruppo più
    \emph{remunerativo}?'' ``Ho bisogno che imparino.'' ``Hai paura che in una di queste missioni potremmo incontrare
    Nicolas e il suo nuovo gruppetto di amici e che ci freghi, vero?'' ``No, è solo...''

    Catherine girò la testa per guardarla, facendole capire che non avrebbe funzionato. ``Sì.'' erano di nuovo ferme ad
    un semaforo. Valentine girò la testa verso Catherine per guardarla e continuò con il discorso ``Voglio fargliela
    pagare, ok? Ti ricordi che cosa ha detto quando se n'è andato?'' Catherine si girò in avanti e fece un cenno con la
    testa per indicare che il semaforo era diventato verde ``Sì, certo. Che senza di lui non saremmo andate molto
    lontane. È per quello che hai mandato il messaggio in giro per le accademie.'' ``Esatto. Quello stronzo.'' ``E
    invece che beccarti un qualche istruttore di pilotaggio di Frame stanco della vita d'accademia ti sei trovata uno
    studente appena uscito dagli studi.'' ``Ed una Otaku.'' ``Ed una hacker con le contropalle che conosce il Parkour ed
    ha vissuto per quasi tutta la sua vita con un cervello e due arti Cybernetici.'' ``Sì, ma converrai con me che è
    strana.'' ``Mah. Non più di quei ragazzi che spendono i loro finesettimana in discoteche a strafarsi di droghe e poi
    fanno a finta di essere dei bravi ragazzi durante la settimana. Almeno lei non nasconde le sue passioni. E non sono
    dannose.'' ``Legge fumetti porno. Tra ragazzi.'' ``Scusa? E da quando è un problema?'' ``Eh?''

    Ok, il discorso era arrivato ad un punto un po' particolare, in effetti. Com'erano arrivate alla discussione su
    Sofia? ``E poi si fa chiamare BJ!'' continuò Valentine ``Ah, sì. Mi aveva spiegato il perchè tempo fa, ma non me lo
    ricordo molto bene, dovresti fartelo spiegare una volta che abbiamo tempo.'' ``Beh, non che ci sia molto da
    spiegare.'' ``Che intendi?'' ``Beh, dai, insomma, non mi pare una buona idea avere una pratica del genere come
    soprannome.'' ``Hahaha!'' Catherine scoppiò a ridere ``No, guarda, mi sa che hai capito male.'' ``Ah.'' La hacker
    era improvvisamente diventata ancora più strana agli occhi di Valentine.

    ``Comunque sia.'' finì il discorso Catherine ``Vedrai che quei due hanno un sacco di assi nelle maniche da
    giocare.'' ``E chi te lo dice?'' ``Una come me se le sente queste cose.'' ``Beh, la veterana sei te.'' ``Dillo
    un'altra volta e ti faccio vedere chi è la \emph{veterana}, miss Naval Intelligence.'' rispose Catherine, con tono
    ironico ``Haha! Lo vedremo la prossima volta che andiamo a bere, Miss Braccio di Ferro.''

  \section*{Catherine}

    Dopo il viaggio in macchina erano arrivate finalmente al Bazaar. Avevano dovuto passare un posto di controllo dove
    applicano dei sigilli alle armi (oltre a farti togliere il caricatore per poi mettere un sigillo anche
    alll'imboccatura del compartimento per i caricatori delle armi). Se ne trovavano alcune con il sigillo rotto finivi in grossi, grossissimi guai. I
    mercanti erano gli unici che potevano maneggiare armi e quelle che vendevano non erano cariche, quindi la gente
    poteva controllare quanto fossero buone prima di comprare. 

    Valentine parcheggiò la macchina in uno degli autosilo all'entrata dalla quale erano arrivate e scesero per strada.
    Il Bazaar ricordava moltissimo quelli del passato. Erano degli enormi mercati permanenti locati all'intero di
    intricate reti di gallerie. Solo che, in questo caso, le gallerie non erano create apposta. Venivano generate dai
    piani intermedi costruiti tra i palazzi ai lati delle case, trasformando così un intero quartiere in un Bazaar a più
    piani. In possimità degli incroci e delle piazze i piani intermedi venivano interrotti, permettendo così di vedere
    il cielo dalla strada.

    Le etnie all'interno del Bazaar erano estremamente eterogenee. In pratica quasi tutte le culture si mescolavano per
    le sue strade. Questo faceva in modo che venissero organizzate spesso feste in occasione di chissà quale tradizione
    nelle varie piazze. Per non parlare della scelta che era possibile trovare in giro per i vari negozi. Catherine
    aveva sempre trovato quello che cercava. E se non c'era te lo facevano arrivare in meno di una settimana. Meglio di
    qualunque corporazione. E la cosa che faceva impazzire Valentine, pareva, era che non c'era altro modo per andarsene
    via con quello che volevi senza contrattare. Per la meccanica era un problema. Non era molto brava, quindi finiva
    per pagare sempre quel qualcosina in più. Ma se era per tenere in piedi una comunità economica indipendente da
    qualunque Corp lo pagava più che volentieri.

    Normalmente lei e Valentine andavano al Bazaar per recuperare parti per il Frame, munizioni, armi custom,
    hardware per i vari PC del team o gadget per le missioni. Oggi Catherine era là per andare a trovare il Doc, il
    quale le aveva detto che voleva farle provare una nuova arma che aveva progettato. Non sapeva esattamente cosa
    doveva prendere il Boss, ma sicuramente era equipaggiamento per la missione a Londra. Probabilmente Sofia non aveva
    esattamente idea su cosa avrebbe avuto bisogno per infiltrarsi, perciò Valentine stava andando a prendere anche
    quello. Sotto sotto i due nuovi acquisti le stavano simpatici.

    Si divisero alla prima piazza. Lei doveva seguire una delle strade principali e poi girare in uno dei vicoletti
    laterali per poi entrare in un palazzo che doveva essere un ex parcheggio pubblico. Il Doc, l'uomo che stava andando
    a trovare, era uno di quei strani personaggi che potevi trovare nei romanzi polizieschi. Era il ricercatore cacciato
    dall'università perchè aveva idee troppo strane, troppo contro gli interessi delle aziende che finanziavano la
    ricerca universitaria
