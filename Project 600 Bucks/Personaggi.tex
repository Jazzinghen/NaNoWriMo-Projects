\documentclass[9pt,a4paper,cleardoubleempty]{scrbook}

\usepackage{}
\usepackage[]{classicthesis-ldpkg}
\usepackage[parts,linedheaders,pdfspacing]{classicthesis}

\usepackage[T1]{fontenc}    % imposta la codifica dei font
\usepackage[utf8]{inputenc} % lettere accentate da tastiera
\usepackage[italian]{babel} % per scrivere in italiano
% \usepackage{layaureo}       % imposta i margini di pagina
\usepackage{lipsum}         % genera testo fittizio
\usepackage{indentfirst}
\usepackage{url}            % per scrivere gli indirizzi Internet
\hypersetup{pdfborder={0 0 0}}
\usepackage{allrunes}
\usepackage{phoenician}

\usepackage{color}
\definecolor{Brown}{cmyk}{0,0.81,1,0.60}
\definecolor{OliveGreen}{cmyk}{0.64,0,0.95,0.40}
\definecolor{CadetBlue}{cmyk}{0.70,0.57,0.23,0}

\usepackage[final]{pdfpages}

% args: image path
%       caption
%       width
%       label
\newcommand{\image}[4]{%
    \begin{figure}[hbt]%
    \centering%
    \includegraphics[width=#3\columnwidth]{#1}%
    \caption{\emph{#2}}%
    \label{#4}%
    \end{figure}%
}

% args: image path
%       caption
%       label
\newcommand{\imageFullWidth}[3]{\image{#1}{#2}{0.8}{#3}}

\newcommand{\picref}[1]{figura \ref{#1}}

\newcommand{\website}[2]{Sito web di \techname{#1}: \\ \url{#2}}
\newcommand{\paper}[3]{#1: \emph{#2} \\ \url{#3}}

\newcommand{\techname}[1]{{\sc #1}}
\newcommand{\cypher}[1]{{\tt #1}}
\newcommand{\syntax}[1]{{\tt #1}}
\newcommand{\codeconst}[1]{\lstinline{#1}}
\newcommand{\newterm}[1]{\emph{#1}}
\newcommand{\foreignword}[1]{\emph{#1}}

% Vari dettagli
\newcommand{\BW}{\emph{Burning Wheel}}

% Entità, NPC/PC e Luoghi importanti del gioco
\newcommand{\sciencegroup}{\techname{Athena Collective}}
\newcommand{\magicgroup}{\techname{Burning Universities}}
\newcommand{\theogroup}{\techname{Credo dei Dodici Divini}}

\newcommand{\theworld}{\emph{Xeresia}}
\newcommand{\theincident}{\emph{Shynthaian Pathfinder Incident}}
\newcommand{\firstvalley}{\emph{Valle di Ribia}}

\newcommand{\weapongoddess}{\emph{La II: Farcaster}}

\newcommand{\dacav}{\emph{Il ``Giovane'' Vecchio Dacav}}
\newcommand{\gods}{\emph{Nonoriko}}
\newcommand{\mitch}{\emph{Niyu-Rin}}
\newcommand{\ranger}{\emph{Alastor}}

\usepackage{}
\author{Michele "Jazzinghen" Bianchi}
\title{Project 600 Bucks: Characters}
\begin{document}
    
  I personaggi sono:
  \begin{itemize}
    \item Valentine Arthur (UK), il capo del team. Esperta di armi da fuoco
      oltre che di contrattazione. Ogni tanto si perde nei propri pensieri.
      Ma almeno non sono strani come quelli di Sofia
    \item Sofia ``BJ'' Zen (Italy), la hacker del team. Siccome in Italia
      non avevamo soldi per i mecha abbiamo sviluppato hacking ed impianti
      cybernetici. Sofia è una dei pochi milioni che ha un cervello
      cybernetico ed alcuni impianti. È stata salvata in questa maniera da
      un'incidente avuto durante l'infanzia, dopo che alcuni parenti hanno
      accetto a farla partecipare ad una sperimentazione Italiana. Lavora
      con il team di Valentine perchè deve pagarsi l'università. Il suo
      Hardware viene aggiornato gratuitamente dall'azienda italiana leader nella
      produzione di componentistica cyborg, in quanto era parte
      dell'accordo, perciò ha sempre l'ultima versione. È una nerd paurosa,
      fan del passato, dei vecchi videogames, serie anime, ecc.
    \item Catherine Mac an Bahird (France - Ireland), il meccanico del
      team. Veterana della terza e quarta guerra mondiale, sulla
      quarantina. Un sacco di cicatrici in giro per il corpo e sempre
      incazzata. 
    \item Arsenios ``Il Pivello'' Trejo, il pilota del gruppo. Un ragazzo
      che combatte per la giustizia. Questo porta a casini, perdite di
      denaro da parte del team (per riparazioni e munizioni), ecc. È
      dannatamente bravo, comunque.
  \end{itemize}
\end{document}
